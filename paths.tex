We have seen in the previous chapter that the encode-decode method can be used
in a variety of cases when we want to make statements about the equality
types -- or \emph{path spaces} -- of higher inductive types.
Going through all necessary steps of such a proof can be somewhat tedious, but
part of it is very mechanical work.
One main goal of this chapter is to present a different method to
directly work with equality types of coequalizers and pushouts
(and construction based on these):
Since elimination rules such as one for coequalizers characterize only
the points  of the type, but in the constructors we create points
and equalities simultaeously, we believe that it is natural to hope for
an ``induction principle for equalities'' which is reminiscent of an elimination
rule.
More concretely, for our case of a coequalizer $\quot : \UU$ of a type $A : \UU$
and typal relation $\_\sim\_ : A \to A \to \UU$,
let us assume we are given a type family
\begin{equation*}
Q: \{a, b : A\} \to [a] = [b] \to \UU \text{.}
\end{equation*}
Is it possible to have simple-to-check conditions which are sufficient to
conclude $Q(q)$ for a general $q$ (instead of just $glue(s)$ for some $s : a \sim b$)?












