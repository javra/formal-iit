We have seen in the previous chapter that the encode-decode method can be used
in a variety of cases when we want to make statements about the equality
types -- or \emph{path spaces} -- of higher inductive types.
Going through all necessary steps of such a proof can be somewhat tedious, but
part of it is very mechanical work.
One main goal of this chapter is to present a different method to
directly work with equality types of coequalizers and pushouts
(and construction based on these):
Since elimination rules such as one for coequalizers characterize only
the points  of the type, but in the constructors we create points
and equalities simultaeously, we believe that it is natural to hope for
an ``induction principle for equalities'' which is reminiscent of an elimination
rule.
More concretely, for our case of a coequalizer $\quot : \UU$ of a type $A : \UU$
and typal relation $\_\sim\_ : A \to A \to \UU$,
let us assume we are given a type family
\begin{equation*}
Q: \{a, b : A\} \to [a] = [b] \to \UU \text{.}
\end{equation*}
Is it possible to have simple-to-check conditions which are sufficient to
conclude $Q(q)$ for a general $q$ (instead of just $glue(s)$ for some $s : a \sim b$)?

\begin{remark}
Note that $Q$ above quantifies over two elements of $A$ and an equality of $\quot$.
If instead we asked the same question for a type family
\begin{equation*}
S: (x, y : \quot) \to x = y \to \UU \text{,}
\end{equation*}
the answer would be that we could the J-rule to populate this family by giving
$S(\refl_x)$.
The principle we want for the applications we presented is the version
where endpoints are ``restricted'' as above.
\end{remark}

It turns out that, like for the J-rule, there is a generalization of the above question.
We get this generalization by switching from an \emph{unbased} (or \emph{global}) type family
to a \emph{based} (or \emph{local}) one:
We can fix one of the two endpoints to be $[a_0] : \quot$ and replace $Q$
by a family which is indexed only \emph{once} over $A$:
\begin{equation}
P : (b : A) \to [a_0] = [b] \to \UU \text{.}
\end{equation}
Like for the two versions of the J-rule, a principle answering the based version
of the question also answers the unbased one, which is why we will focus
exclusively on the former.

In order to get some intuition for the subtleties of the equality types,
let us first look at a hypothetical principle which turns out to be wrong.
Usually, induction principles contain one case for every constructor,
the standard equality constructor is $\refl$ and with \textsc{Coeq-Intro2}, we
have one further path constructor $\glue$.
Thus, we might try whether it is sufficient to assume terms
\begin{align*}
r &: P(\refl_{[a_0]}) \text{ and}\\
p &: (b : A)(s : a_0 \sim b) \to P(\glue(s))
\end{align*}
to conclude that $(b : A)(q : [a_0] = [b]) \to P(q)$?
It turns out that this attempt fails:
Consider the relation $\sim$ on the natural numbers defined by
\begin{equation*}
(m \sim n) :\equiv m + 1 = n \text{.}
\end{equation*}
We can look at the coequalizer $\specialquot{\N}$.
Let us take $1 : \N$ as the base point and $P : (n : \N) \to ([0] = [n]) \to \UU$
defined by $P(n,q) :\equiv (n \geq 1)$.
The terms $r$ and $p$ are constructed easily, but at the same time, it is clear
that $P(0, \glue(k)\inv)$ is empty (where $k$ is a proof for $0 + 1 = 1$).

The above naïve suggestion was easy to disprove, but let us try to understand
why it was insufficient.
Equalities that come from $A$ can, by the J-rule, be assumed to be $\refl$;
these are sufficiently covered.
However this is not true for equalities that are generated using the $\glue$ constructor.
The counterexample uses the fact that we have not explicitely closed them under symmetry
and similarity -- we could have also used that we have not closed them
under transitivity.

How could we fix this? Given an equality $q$ in $\quot$, we can compose it
with $\glue(s)$ assuming the endpoints match.
This suggests that the induction principle we are looking for should assume
$Q(q) \to Q(q \ct \glue(s))$.
But we can also compose with $\glue(s)\inv$,
suggesting that we also need a function $Q(q)  \to Q(q \ct \glue(s)\inv)$.
The operations of composing with $\glue(s)$ and its inverse should furthermore
be inverse to each other,
wich motivates us to ask for only \emph{one} of them and require this one to be
an equivalence, i.\,e. $Q(q) \simeq Q(q \ct \glue(s))$.
This finally leads us to a valid induction principle, which is short, useful, and
comes with two $\beta$-rules.
Proving this princple is the main result of this chapter:
\begin{thm}[Induction for Coequalizer Equality]
Assume $A$ and $\sim$ as before, a point $a_0 : A$, and we are further given
a type family
\begin{equation*}
P : (b : A) \to [a_0] = [b] \to \UU \text{,}
\end{equation*}
together with terms
\begin{align*}
r &: P(\refl_{[a_0]}) \text{ and} \\
e &: \{b, c : A\}(q : [a_0] = [b])(s : b \sim c) \to P(q) \simeq P(q \ct \glue(s)) \text{.}
\end{align*}
The we can counstruct a dependent function
\begin{equation*}
\ind_{r, e} : \{b : A\}(q : [a_0] = [b]) \to P(q)
\end{equation*}
with the following equalities reminiscent of $\beta$-rules:
\begin{align}
\ind_{r,e} (\refl_{[a_0]}) &= r \label{eq:paths-thm-based-first-beta} \\
\ind_{r,e}(q \ct \glue(s)) &= e (q,s, \mathsf{ind}_{r,e}(q)) \label{eq:paths-thm-based-second-beta} \text{.}
\end{align}
\end{thm}

\begin{remark}
The theorem can be proved in a way which makes the first $\beta$-rule hold judgmentally.
This is what we have done in our formalization, but we will refrain from checking
whether equalities hold strictly in this chapter.
\end{remark}

In the following sections we will first prove this main result (Chapter~\ref{sec:paths-main}),
then modify it to obtain a version for pushouts (Chapter~\ref{sec:paths-pushout}),
and present a few smaller applications (Chapter~\ref{sec:paths-applications})
by characterizing the loop space of the circle and proving that
pushouts preserve embeddings,
before applying the approach to state a version of the Seifert-van Kampen theorem
which instead of groupoids refers to \emph{higher} fundamental groupoids
(Chapter~\ref{sec:paths-svk}).

\section{The Main Theorem: Path Spaces in Coequalizers}\label{sec:paths-main}

\section{Equality in Pushouts}\label{sec:paths-pushout}

\section{First Applications}\label{sec:paths-applications}

\section{Free Groupoids and a Higher Seifert-van Kampen Theorem}\label{sec:paths-svk}







