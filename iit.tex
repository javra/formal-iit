As we have mentioned in \Cref{sec:intro-iit}, we want to make a sharp pivot at
this point of the thesis and want to explore \emph{inductive-inductive types}.
This chapter will first introduce inductive-inductive types by a few illustrative
examples, before we will start with its main purpose:
Giving a formal specification of what inductive-inductive types are.

The treatment of inductive-inductive types, which includes
all the subsequent chapters, will happen in a type theory which
admits the K-rule (cf. \Cref{rmk:tt-k}) and thus we will not have any meaningful
higher equalities in types.
While some of the constructions are easily transferable to homotopy type theory,
others rely on the K-rule and it is not obvious how to make them \emph{coherent}
with respect to higher equalities.
Further than having the K-rule we will also take the liberties in regarding some
equalities as \emph{strict}.
We will use $\equiv$ instead of $=$ to denote strict equality.

Inductive-inductive types are a feature of type theories which comes naturally
in provers like Agda, which are based on dependent pattern matching.
A formal treatment of their syntax and semantics is still interesting for
several reasons:
We like to give
grounds to their soundness, we might want to implement them in theorem provers
which are not based on dependent pattern matching but on a derivation of elimination rules,
or we might use them as a tool to prove the consistency of concepts like
dependent pattern matching.

The use case for inductive-inductive types is always of the following nature:
We want to define some data by the means of an inductive types, but we want
its point constructors to refer to data from another type family which is indexed
over the very type we are just defining.
In the following, we will take a look at a few examples which we are going to
revisit at various steps throughout this presentation:

\begin{example}[Type Theory Syntax]\label{ex:ttintt}
\cite{ttintt} showed how to internalise the syntax of type theory inside type
theory itself, using a quotient inductive-inductive type.
Leaving out terms and substitutions, we arrive at a fragment of type theoretical
syntax specifying a type of contexts and a type of types over a certain contexts,
together with type formers for a unit type and a $\Pi$-type:
We want $\mathop{Con} : \UU$ to be inductively defined by
\begin{align*}
\mathop{nil}	&: \mathop{Con} \text{and } \\
\mathop{ext}	&: (\Gamma : \mathop{Con})(A : \mathop{Ty}(\Gamma)) \to \mathop{Con} \text{,}
\end{align*}
while simultaneously defining a family $\mathop{Ty} : \mathop{Con} \to \UU$ with constructors
\begin{align*}
\mathop{unit}	&: (\Gamma : \mathop{Con}) \to \mathop{Ty}(\Gamma) \text{ and} \\
\mathop{pi}	&: (\Gamma : \mathop{Con})(A : \mathop{Ty}(\Gamma))(B : \mathop{Ty}(\mathop{ext}(\Gamma, A))) \to \mathop{Ty}(\Gamma) \text{.}
\end{align*}
\end{example}

\begin{example}[Free Dense Completion]
\citet{nordvallinductive} proposed the example of a ``free dense completion'' of
an order (or, more general, any relation) which for any type $A : \UU$ and
any type valued relation $\_<\_ : A \to A \to \UU$ on $A$ freely adds midpoints
to all pairs of related elements by $\_<\_$.
It does so by introducing a new type $A' : \UU$ inductively generated by the
original points and their midpoints:
\begin{align*}
\iota_A		&: A \to A' \text{ and} \\
\mathop{mid}	&: \{x, y : A'\}(p : x <' y) \to A' \text{.}
\end{align*}
But since our relation was only defined on $A$, we have to extend it to $A'$ by
postulating
\begin{align*}
\iota_<		&: \{a, b : A\}(p : a < b) \to \iota_A(a) <' \iota_A(b) \text{,} \\
\mathop{mid}_l	&: \{x, y : A'\}(p : x <' y) \to x < \mathop{mid}(p) \text{, and} \\
\mathop{mid}_r	&: \{x, y : A'\}(p : x <' y) \to \mathop{mid}(p) < y \text{.}
\end{align*}
\end{example}


\section{Signatures for Inductive-Inductive Types}\label{sec:ii-syntax}

%TODO add waay more explanation
Inductive-Inductive Types are specified by giving a context in  a small type
theoretic syntax which we will refer to as \emph{source type theory}.
This idea originates from the work by \citet{ambrussyntax} on the syntax of \emph{higher}
inductive-inductive types, which we adapt and rid of equality
constructors to only allow for inductive-inductive types.
In contrast to their presentation we will leave the context of the ambient type
theory implicit and, instead of highlighting syntax of the ambient type theory,
mark elements of the source type theory in \gr{green}.
The only technical difference between our type theory and this proposed source
type theory is that we want an explicit separation of sort and point constructors,
which prohibit some of the possible types which are possible in \cite{ambrussyntax}.
This separation is achieved by introducing \emph{kinds} (see below).

We assume that the source type theory makes use of the standard syntax of type
theory, using contexts, types, terms, and variables.
We regard the presentation to be \emph{intrinsic} meaning that
we will only ever consider wellformed contexts, types, and terms.
We will nevertheless use the turnstile notation of syntax, and, for example
write \grm{\vdash \Gamma}, to say that \grm{\Gamma} is a context -- instead of
writing something like \grm{\Gamma : Con}.
We hope that this makes it less confusing to the reader since we will also have
type theoretical syntax as one of our \emph{main examples} for inductive-inductive
types.

Types and terms are uniquely ascribed to one of two \emph{kinds}:
Either their kind is \grm{\Sc} which indicates that the type contains sort
constructors, or their kind is \grm{\Pc} because elements of it describe
point constructors.
We will write \grm{\Gamma \vdash A :: k} to say that \grm{A} is a type of kind
\grm{k} and \grm{\Gamma \vdash t : A :: k} to state that \grm{t} is a term of the
type \grm{A} which in turn has kind \grm{k}.
Often, we will omit the annotation of the sort, meaning that a judgment is to
hold true for both \grm{\Sc} and \grm{\Pc}, or that the kind of a term's type
has already been specified.

It's important that contexts can be extended by sort and point types in any order
to be able to capture sorts which depend on previously
defined point constructors.
So we have the usual two rules for context formation:
\begin{equation*}
\inferrule{}{\grm{\vdash \cdot}}
\qquad
\inferrule{\grm{\Gamma \vdash A :: k}}
  {\grm{\vdash \Gamma, A}}
\end{equation*}

We need one atomic building block for sort types:
For plain types and the codomain of function types we need a type \grm{\UU} which serves as a
token for the \emph{universe}.
We will call terms of this universe ``small types''.
Positiviy requires that these are the only (internal) types which are allowed in
the domain of functions.
An operation \grm{\El} reifies these small types to big types, making our version
of universe what is commonly referred to as ``Tarski-style universe'' (cf. ~\cite{luotarski}):
\begin{equation*}
\inferrule{\grm{\vdash \Gamma}}{\grm{\Gamma \vdash \UU :: \Sc}}
\qquad
\inferrule{\grm{\Gamma \vdash a : \UU}}{\grm{\Gamma \vdash \El(a) :: \Pc}}
\end{equation*}

For sorts which are type families over other sorts that we seek to define, and for
constructors which recursively refer to other constructors, we need $\Pi$-types
which have a small type as their codomain.
To distinguish them from the other function types, which we will introduce below,
we will often refer to them as
\emph{recursive $\Pi$-types}.
Note that whether we want to build a sort or a point type only depends on the
kind of the \emph{codomain} of such a $\Pi$-type.
To eliminate from $\Pi$-types we want a rule for its \emph{application} which
turns a term of a $\Pi$-type into a term of its codomain.
Note that there is no introduction rule in the form of $\lambda$ terms for these
functions, since they are not needed in the description of inductive-inductive types.
\begin{equation*}
\inferrule{\grm{\Gamma \vdash a : \UU} \\
  \grm{\Gamma, \El(a) \vdash B :: k}}
  {\grm{\Gamma \vdash \Pi(a, B) :: k}}
\qquad
\inferrule{\grm{\Gamma \vdash f : \Pi(a, B)}}
  {\grm{\Gamma, \El(a) \vdash \IIapp(f) : B}}
\end{equation*}

Additionally, we want sorts to be able to be parametrised by previously defined
types which are not part of the signature itself.
The same goes for point constructors.
Since this cannot be captured using the previous $\Pi$-type, we will do the obvious
and just introduce another type former for this occasion.
We will usually call it \emph{external} or \emph{non-recursive} function type.
Note that external functions must have a fixed kind.
This is to prevent a function which, depending on the input returns sometimes
a sort and sometimes a point constructor.
\begin{equation*}
\inferrule{T : \UU \\
  (\bltau : T) \to \grm{\Gamma \vdash B(\bltau) :: k}}
  {\grm{\Gamma \vdash \ExtPi{T}{B} :: k}} \\
\qquad
\inferrule{\grm{\Gamma \vdash f : \ExtPi{T}{B}} \\
  \tau : T}
  {\grm{\Gamma \vdash f(\bltau) : B(\bltau)}}
\end{equation*}

Since we are working with explicit substitutions, we need to postulate a calculus
for substitutions \grm{\IISub{\sigma}{\Gamma}{\Delta}} between any two contexts
\grm{\Gamma} and \grm{\Delta}.
The substitutions should form a category as postulated by the following rules:
\begin{equation*}
\inferrule{\grm{\vdash \Gamma}}
  {\grm{\IISub{\id}{\Gamma}{\Gamma}}}
\qquad
\inferrule{\grm{\IISub{\sigma}{\Delta}{\Sigma}} \\
  \grm{\IISub{\delta}{\Gamma}{\Delta}}}
  {\grm{\IISub{\sigma \circ \delta}{\Gamma}{\Sigma}}}
\end{equation*}
\begin{align*}
\grm{\id \circ \sigma} &= \grm{\sigma} \\ %TODO change id typesetting
\grm{\sigma \circ \id} &= \grm{\sigma} \\
\grm{(\sigma \circ \delta) \circ \gamma} &= \grm{\sigma \circ (\delta \circ \gamma)}
\end{align*}

We can pull back types and terms along substitutions, and these pullbacks are functorial
in the categorical structure:
\begin{equation*}
\inferrule{\grm{\Delta \vdash A :: k} \\
  \grm{\IISub{\sigma}{\Gamma}{\Delta}}}
  {\grm{\Gamma \vdash A[\sigma] :: k}}
\qquad
\inferrule{\grm{\Delta \vdash t : A} \\
  \grm{\IISub{\sigma}{\Gamma}{\Delta}}}
  {\grm{\Gamma \vdash t[\sigma] : A[\sigma]}}
\end{equation*}
\begin{align*}
\grm{A[\id]} &= \grm{A} \\
\grm{A[\sigma \circ \delta]} &= \grm{A[\sigma][\delta]} \\
\grm{t[\id]} &= \grm{t} \\
\grm{t[\sigma \circ \delta]} &= \grm{t[\sigma][\delta]}
\end{align*}

We have a canonical substitution into the empty context and we can extend substitutions
by giving a term in a type pulled back to their domain.
\begin{equation*}
\inferrule{\grm{\vdash \Gamma}}
  {\grm{\IISub{\epsilon}{\Gamma}{\cdot}}}
\qquad
\inferrule{\grm{\IISub{\sigma}{\Gamma}{\Delta}} \\
  \grm{\Delta \vdash A} \\
  \grm{\Gamma \vdash t : A[\sigma]}}
  {\grm{\IISub{\sigma, t}{\Gamma, A}{\Delta}}}
\end{equation*}
Empty substituition and extension simplify by the following laws:
\begin{align*}
\grm{\sigma} &= \grm{\epsilon}
  &\text{ for all \grm{\IISub{\sigma}{\Gamma}{\cdot}}, and} \\
\grm{(\delta , t) \circ \sigma} &= \grm{(\delta \circ \sigma), t[\sigma]}
  &\text{ for \grm{\IISub{\delta}{\Gamma}{\Delta}}, \grm{\IISub{\sigma}{\Sigma}{\Gamma}}.}
\end{align*}
A substitution into an extended context allows us to project out
a ``shorter'' substitution and a term as a terminal component:
\begin{equation*}
\inferrule{\grm{\IISub{\sigma}{\Gamma}{\Delta,A}}}
  {\grm{\IISub{\pi_1(\sigma)}{\Gamma}{\Delta}}}
\qquad
\inferrule{\grm{\IISub{\sigma}{\Gamma}{\Delta,A}}}
  {\grm{\Gamma \vdash \pi_2(\sigma) : A[\pi_1(\sigma)]}} \text{, with}
\end{equation*}
\begin{align*}
\grm{\pi_1(\sigma, t)} &= \grm{\sigma} \text{,} \\
\grm{\pi_2(\sigma, t)} &= \grm{t} \text{, and} \\
\grm{(\pi_1(\sigma), \pi_2(\sigma))} &= \grm{\sigma} \text{.}
\end{align*}
Finally, we also need rules that tell us, how the constructors of the universe
and the $\Pi$-types behave when pulled back along an arbitrary substitution
\grm{\IISub{\sigma}{\Gamma}{\Delta}}:
\begin{align*}
\grm{\UU[\sigma]} &= \grm{\UU} \text{,} \\
\grm{\El(a)[\sigma]} &= \grm{\El(a[\sigma])} \text{,} \\
\grm{\Pi(a, B)[\sigma]} &= \grm{\Pi(a[\sigma],B[\sigma \wedge \El(a)])} \text{,} \\
\grm{\IIapp(f)[\sigma \wedge El(a)]} &= \grm{\IIapp(f[\sigma])} \text{,} \\ %TODO use a better symbol than wedge
\grm{\ExtPi{T}{B}[\sigma]} &= \grm{\ExtPi{T}{\blm{\lambda \tau. \grm{B(\bltau)[\sigma]}}}} \text{, and} \\
\grm{f(\bltau)[\sigma]} &= \grm{f[\sigma](\bltau)} \text{.}
\end{align*}
This concludes the specification of the syntax for inductive-inductive types.

\begin{defn}
Above, \grm{\sigma \wedge A} is one of several auxiliary constructions on the syntax
which are helpful when dealing with substitutions and which can be derived from
the other rules.

The first one is the operation known as \emph{weakening} which for any \grm{\Gamma \vdash A}
gives a substition \grm{\IISub{\wk}{\Gamma, A}{\Gamma}} from the extended into
the original context by \grm{\wk := \pi_1(\id)}.

Likewise, we can apply the second projection to the identity substitution such that
whenever \grm{\Gamma \vdash A}, we have the first \emph{variable} of the context
$\grm{\vz} := \grm{\pi_2(\id)}$ with \grm{\Gamma, A \vdash \vz : A[\wk]}.
Transporting a term \grm{\Gamma \vdash t : A} along the weakening substitution
defined above for any \grm{\Gamma \vdash B}, we get \grm{\Gamma, B : t[\wk] : A[\wk]}.
We will write $\grm{\vs(t)} := \grm{t[\wk]}$ for this variable.
Together, \grm{\vz} and \grm{\vs} form \emph{typed de Bruijn indices} to select
variables from a context via numbering them with a zero (\grm{\vz}) and a
successor (\grm{\vs}).

For \grm{\IISub{\sigma}{\Gamma}{\Delta}} and \grm{\Gamma \vdash A} we can
``lift'' \grm{\sigma} along \grm{A} to get a substitution
\grm{\IISub{\sigma \wedge A}{\Gamma, A[\sigma]}{\Delta, A}}.
This operation can be defined by
\begin{equation*}
\grm{\sigma \wedge A} := \grm{\sigma \circ \wk, \vz(A[\sigma])} \text{.}
\end{equation*}
At last, every term \grm{\Gamma \vdash t : A} gives rise to a substitution
\grm{\IISub{\langle t \rangle}{\Gamma}{\Gamma, A}}, representing the extension
of \grm{\Gamma} by \grm{t}, via
\begin{equation*}
\grm{\langle t \rangle} := \grm{id, t} \text{.}
\end{equation*}
\end{defn}

\begin{defn}[Application]
Most often, we want to use the application of the inductive function type in the
form where we give the function term and its input separately, as in the following
rule:
\begin{equation*}
\inferrule{\grm{\Gamma \vdash f : \Pi(a, B)} \\
  \grm{\Gamma \vdash u : \El(a)}}
  {\grm{\Gamma \vdash f(u) : B[\sigma]}}
\text{,}
\end{equation*}
for some substitution \grm{\sigma}.
Now we know that we can define this substitution by
\begin{equation*}
\grm{f(u)} := \grm{\IIapp(f)[\langle u \rangle]} \text{,}
\end{equation*}
and $\grm{\sigma} = \grm{\langle u \rangle}$.
\end{defn}

\begin{remark}\label{rmk:iit-qit}
The syntax with its postulated equations itself on the one hand shows features
of induction-induction itself
-- types are dependent on contexts, etc. --
but the equations turn it what is called a \emph{quotient inductive-inductive type},
as \citet{constructingqiits} define it.
Quotient inductive-inductive types are a very useful generalization of
inductive-inductive types,
featuring \emph{path constructors} beside sort and point constructors.
They can be used, for example
to represent the type of real numbers in type theory \citep{hottbook}.
\end{remark}

We will now look at a few example to make it clearer on how to encode inductive-inductive
declarations in the syntax presented above. %TODO more fluff
To see what the different function types are used for, consider the following two
examples:

\begin{example}[Natural numbers]\label{ex:ii-syntax-nat}
The encoding of the natural numbers as would correspond to the
following source type theory context using the first :
\begin{equation*}
\grm{
\cdot,\, \UU,\, \El(\vz),\, \Pi(\vs(\vz), \El(\vs(\vs(\vz))))
}\text{.}
\end{equation*}
Often, we will, instead of denoting variables using de~Bruijn indices, use names
as binders in contexts and domains of $\Pi$-types to make example contexts more
legible.
Assuming we always use fresh names, this is not any more imprecise than restricting
ourselves to use \grm{\vz} and \grm{\vz} instead:
\begin{equation*}
\grm{
\cdot,\, \N : \UU,\, 0 : \El(\N),\, \mathop{S} : \Pi(n : \N, \El(\N))
}\text{.}
\end{equation*}
\end{example}

\begin{example}[Vectors]\label{ex:ii-syntax-vec}
The type of vectors over a type $A : \UU$ can be represented using the ``external''
natural numbers $\N$.
In the following, the constructor \grm{\mathop{cons}} uses both the non-inductive
and the inductive function type:
\begin{equation*}
\begin{gathered}
\grm{ \cdot,\, \mathop{vec} : \ExtPi{\N}{\lambda n.\, \UU},\, \mathop{nil} : \El(\mathop{vec}(\blm{0})),} \\
\grm{ \mathop{cons} : \ExtPi{A}{\blm{\lambda a.\,
  \grm{\ExtPi{\N}{\blm{\lambda n.\, \grm{\Pi(v : \mathop{vec}(\blm{n}),\, \El(\mathop{vec}(\blm{n + 1})))}}}}}}}
\end{gathered}
\end{equation*}
With de-Bruijn indices instead of names the signature \grm{\Gamma_{vec}} would be
\begin{equation*}
\begin{gathered}
\grm{ \cdot,\, \ExtPi{\N}{\lambda n.\, \UU},\, \El(\vz(\blm{0})),} \\
\grm{ \ExtPi{A}{\blm{\lambda a.\,
  \grm{\ExtPi{\N}{\blm{\lambda n.\, \grm{\Pi(\vs(\vz)(\blm{n}),\, \El(\vs(\vs(\vz))(\blm{n + 1})))}}}}}}}
\end{gathered}
\end{equation*}
%An example for a type which uses the third, small function type is the following
%definition of full $\omega$-ary rooted trees:
%\begin{equation*}
%\grm{
%\cdot,\, T : \UU,\, \mathop{leaf} : \underline{T},\,
%	\mathop{node} : (\blm{\N} \to T) \to \underline{T}
%} \text{.}
%\end{equation*}
\end{example}

\begin{example}[Type Theory Syntax]
The example of the syntax of type theory~\ref{ex:ttintt} is represented by the
following signature \grm{\Gamma_{ConTy}}:
\begin{equation*}
\begin{gathered}
\grm{\cdot,\, \mathop{Con} : \UU,\, \mathop{Ty} : \Pi(\Gamma : \mathop{Con},\, \UU),} \\
\grm{\mathop{nil} : \El(\mathop{Con}),} \\
\grm{\mathop{ext} : \Pi(\Gamma : \mathop{Con},\, \Pi(A : \mathop{Ty} ,\, \El(\mathop{Con}))),} \\
\grm{\mathop{unit} : \Pi(\Gamma : \mathop{Con},\, \El(\IIapp(\mathop{Ty}))),} \\
\grm{\mathop{pi} : \Pi(\Gamma : \mathop{Con},\, \Pi(A : \IIapp(\mathop{Ty}),\,
   \Pi(B : ?,\, ?)))} %TODO complete this after doing it in agda
\end{gathered}
\end{equation*}

\end{example}

\section{Algebras of Inductive-Inductive Types}

To give meaning to the codes expressed in the source type theory, we need to
interpret the contexts as a type in the ambient type theory whose elements are
the algebras of of the specified inductive-inductive type.
This means that the interpretation of our contexts must give the types of the
sort and point constructors they specify.

\begin{defn}[Algebra Operator] %TODO name?
By structural recursion over the source syntax, we define an operation $-^\CC$
which assigns types to source contexts, fibrations over those to types, sections of
these fibrations to terms, and maps between types to substitutions:
\begin{equation*}
\begin{gathered}
\inferrule{\grm{\vdash \Gamma}}{\grm{\Gamma}^\CC : \UU_1}
\qquad
\inferrule{\grm{\Gamma \vdash A :: \Sc}}{\grm{A}^\CC : \grm{\Gamma}^\CC \to \UU_1}
\qquad
\inferrule{\grm{\Gamma \vdash A :: \Pc}}{\grm{A}^\CC : \grm{\Gamma}^\CC \to \UU_0}
\\[.7em]
\inferrule{\grm{\Gamma \vdash t : A}}
  {\grm{t}^\CC : (\gamma : \grm{\Gamma}^\CC) \to \grm{A}^\CC(\gamma)}
\qquad
\inferrule{\grm{\IISub{\sigma}{\Gamma}{\Delta}}}
  {\grm{\sigma}^\CC : \grm{\Gamma}^\CC \to \grm{\Delta}^\CC}
\end{gathered}
\end{equation*}

We will give the construction on contexts, types, subsitutions and terms in the
same order as there were presented in Section~\ref{sec:ii-syntax}.
On contexts, the operation is defined by iterated $\Sigma$-types:
\begin{align*}
\grm{\cdot}^\CC &:\equiv \unit \text{ and} \\
\grm{(\Gamma,\,A)}^\CC &:\equiv (\gamma : \grm{\Gamma}^\CC) \times \grm{A}^\CC(\gamma)
\end{align*}

The universe in the syntax needs of course be mapped to the metatheoretic universe.
We chose $\UU_1$ as a target for context interpretation to make sure $\UU_0$ fits
in there.
The operation \grm{\El} is just there to make the conversation between small and
big types, which in turn is needed to ensure positivity of the constructors.
Since this distinction doesn't have any semantic meaning, \grm{\El} will just be
ignored by the algebra operator:
\begin{align*}
\grm{\UU}^\CC(\gamma) 			&:\equiv \UU_0 \\
\grm{(\El(a))}^\CC(\gamma)		&:\equiv \grm{a}^\CC(\gamma)
\end{align*}
Recursive $\Pi$-types become metatheoretic dependent function spaces, with \grm{\IIapp} the
usual function application:
\begin{align*}
\grm{\Pi(a, B)}^\CC(\gamma)		&:\equiv (\alpha : \grm{a}^\CC(\gamma)) \to \grm{B}^\CC(\gamma, \alpha) \\
\grm{\IIapp(t)}^\CC(\gamma, \alpha)	&:\equiv \grm{t}^\CC(\gamma, \alpha)
\end{align*}
Non-recursive $\Pi$-types also become functions, but here we have to apply the
external argument to the codomain to be able to evaluate its interpretation:
\begin{align*}
\grm{\ExtPi{T}{B}}^\CC(\gamma)		&:\equiv (\tau : T) \to B(\bltau)^\CC(\gamma) \\
\grm{f(\bltau)}^\CC(\gamma)		&:\equiv \grm{f}^\CC(\gamma, \tau)
\end{align*}

Unsurprisingly the category structure of substitution is achieved by interpreting
it into the one of metatheoretic functions between context interpretations:
\begin{align*}
\grm{\id}^\CC(\gamma)			&:\equiv \gamma \\
\grm{(\sigma \circ \delta)}^\CC(\gamma)	&:\equiv \grm{\sigma}^\CC(\grm{\delta}^\CC(\gamma))
\end{align*}
Pulling back a type or a term along a substitution means interpreting it after
applying the function which we get from interpreting the substitution:
\begin{align*}
\grm{A[\sigma]}^\CC(\gamma)		&:\equiv \grm{A}^\CC(\grm{\sigma}^\CC(\gamma)) \\
\grm{t[\sigma]}^\CC(\gamma)		&:\equiv \grm{t}^\CC(\grm{\sigma}^\CC(\gamma))
\end{align*}
The interpretation of the empty substitution is the unique map into the interpretation
of the empty context.
Extension of and projections from a substitution now justify their name by being
interpreted as the extension of a function and the projections of a $\sigma$-type:
\begin{align*}
\grm{\epsilon}^\CC(\gamma)		&:\equiv \star \\
\grm{(\sigma, t)}^\CC(\gamma)		&:\equiv (\grm{\sigma}^\CC(\gamma), \grm{t}^\CC(\gamma)) \\
\grm{\pi_1(\sigma)}^\CC(\gamma)		&:\equiv \pr_1(\grm{\sigma}^\CC(\gamma)) \\
\grm{\pi_2(\sigma)}^\CC(\gamma)		&:\equiv \pr_2(\grm{\sigma}^\CC(\gamma))
\end{align*}

All the rules of the substitution calculus which were mentioned in Section~\ref{sec:ii-syntax}
are preserved \emph{definitionally}, which, in the end, is due to types and
functions forming a \emph{strict} category. %TODO more explanation here

%\begin{align*}
%\grm{(t \app \blm{u})}^\CC(\gamma)	&:\equiv (\grm{t}^\CC(\gamma))(u) \text{.}
%\end{align*}

\end{defn}

\begin{example}[Natural numbers]\label{ex:iit-nat-alg}
Strictly speaking, the algebra interpretation of the
context from our example of natural numbers (Example~\ref{ex:ii-syntax-nat})
would compute to the following iterated $\Sigma$-type:
\begin{equation*}
\grm{\left(N' : (N : \top \times \UU) \times \pr_2(N)\right)
  \times \left(\pr_2(\pr_1(N')) \to \pr_2(\pr_1(N')) \right) } \text{.}
\end{equation*}
Obviously, this is unnecessarily complicated and we can easily transform this
type using equivalences to see that we can also express the algebras as being
elements of the type
\begin{equation*}
\grm{(N : \UU) \times N \times (N \to N)} \text{.}
\end{equation*}
\end{example}

%TODO more examples

\section{Morphisms of Algebras}

When we talk about the questions whether a language support inductive-inductive
types, we obviously don't only want their constructors --
in fact there are many algebras which don't meet our expectation of what the
``realization'' of an inductive-inductive definition should be.
Referring to \Cref{ex:iit-nat-alg} above, any type $N$ with a point $z : N$ and
a function $s : N \to N$ is an algebra for the definition of the natural numbers,
even types which are either ``too large'' and contain much more points,
think for example of the real numbers,
or too ``too small'' -- the unit type $\unit$ is an algebra for the natural numbers
as well.
What we want is an algebra which is just large enough that all
constructors are injective.
We will express this as a categorical property:
We will equip the algebras over a context with a notion of a \emph{morphism of algebra}
that turns them into a category.
The fact that the type is equipped with a non-dependent eliminator will then
correspond to the fact that we can find a morphism from its
realization to any other algebra -- the type is weakly initial.
The fact that it also comes with $\eta$-rules is expressed by the fact
that this morphism is unique, making the realization the \emph{initial algebra}.
In this section we first give the definition of morphisms and then define
initiality.

\begin{defn}[Morphism Operator]
Like for the algebra operator we again perform structural recursion over the
syntax to define, what the type of morphisms between two of its algebras should
be.
Apart from contexts we will also give the operation on types, terms and substitutions,
consisting of data indexed over two algebras:
\begin{equation*}
\begin{gathered}
\inferrule{\grm{\vdash \Gamma} \\
  \gamma, \delta : \grm{\Gamma}^\CC }
  {\grm{\Gamma}^\MM(\gamma, \delta) : \UU } \qquad
\inferrule{\grm{\Gamma \vdash A :: k} \\
  \gamma, \delta : \grm{\Gamma}^\CC \\
  \mu : \grm{\Gamma}^\MM(\gamma, \delta) }
  {\grm{A}^\MM(\mu) : \grm{A}^\CC(\gamma) \to \grm{A}^\CC(\delta) \to \UU}\\[.7em]
\inferrule{\grm{\Gamma \vdash t : A :: k} \\
  \gamma, \delta : \grm{\Gamma}^\CC }
  {\grm{t}^\MM : (\mu : \grm{\Gamma}^\MM(\gamma, \delta))
    \to \grm{A}^\MM(\mu, \grm{t}^\CC(\gamma), \grm{t}^\CC(\delta))}\\[.7em]
\inferrule{\grm{\IISub{\sigma}{\Gamma}{\Delta}} \\
  \gamma, \delta : \grm{\Gamma}^\CC }
  {\grm{\sigma}^\MM : \grm{\Gamma}^\MM(\gamma, \delta)
    \to \grm{\Delta}^\MM(\grm{\sigma}^\CC(\gamma), \grm{\sigma}^\CC(\delta))}
\end{gathered}
\end{equation*}

We will again proceed in the order we presented the type theory in
\Cref{sec:ii-syntax}, starting with the different ways we can form contexts.
There are no surprises here since we can define it by recursion and because
the empty context does not contain any information:
\begin{align*}
\grm{\cdot}^\MM(\gamma, \delta)
  &:\equiv \unit \text{ and} \\
\grm{(\Gamma,\,A)}^\MM((\gamma, \alpha), (\delta, \beta))
  &:\equiv (\mu : \grm{\Gamma}^\MM(\gamma,\delta)) \times \grm{A}^\MM(\mu, \alpha, \beta) \text{.}
\end{align*}

For the universe, the definition makes sure that morphisms between sorts
are indeed functions between their corresponding realizations.
The element operator requires a proof that the two ways we can
obtain an algebra of the term in the codomain of the morphism match:
\begin{align*}
\grm{\UU}^\MM(\mu, \gamma, \delta)
  &:\equiv \grm{\gamma}^\CC \to \grm{\delta}^\CC \text{ and} \\
\grm{\El(a)}^\MM(\mu, \alpha, \beta)
  &:\equiv \left( \grm{a}^\MM(\mu, \alpha) = \beta \right) \text{.}
\end{align*}
\end{defn}

Recursive $\Pi$-types are mapped to dependent function spaces on the
set which is the algebra of the universe term.
For the application we can perform induction on the equality proof and
thus we can assume it to be $\refl$.
\begin{align*}
\grm{\Pi(a, B)}^\MM(\mu, \pi, \phi)
  &:\equiv \left(\alpha : \grm{a}^\CC(\gamma)\right)
    \to \grm{B}^\MM\left((\mu, \refl), \pi(\alpha), \phi(\grm{a}^\MM(\mu, \alpha))\right) \\
\grm{\IIapp(f)}^\MM\{\gamma, \alpha\}(\mu, \refl)
  &:\equiv \grm{f}^\MM(\mu, \alpha)
\end{align*}

Morphisms between interpretations of non-recursive $\Pi$-types are functions over
the external parameter as well:
\begin{align*}
\grm{\ExtPi{T}{B}}^\MM(\mu, \pi, \phi)
  &:\equiv (\tau : T) \to \grm{B(\bltau)}^\MM(\mu, \pi(\tau), \phi(\tau)) \\
\grm{f(\bltau)}^\MM(\mu)
  &:\equiv \grm{f}^\MM(\mu, \tau)
\end{align*}

The treatment of substitutions looks almost identical to the one in the algebra
operator in that the cateogorical structure of subsitution gets interpreted
as the strict category of functions in the outer type theory:
\begin{align*}
\grm{\id}^\MM(\mu)
  &:\equiv \mu \\
\grm{\sigma \circ \delta}^\MM(\mu)
  &:\equiv \grm{\sigma}^\MM(\grm{\delta}^\MM(\mu)) \\[.7em]
\grm{A[\sigma]}^\MM(\mu, \alpha, \beta)
  &:\equiv \grm{A}^\MM(\grm{\sigma}^\MM(\mu), \alpha, \beta) \\
\grm{t[\sigma]}^\MM(\mu)
  &:\equiv \grm{t}^\MM(\grm{\sigma}^\MM(\mu)) \\[.7em]
\grm{\epsilon}^\MM(\mu)
  &:\equiv \star \\
\grm{(\sigma,\,t)}^\MM(\mu)
  &:\equiv \left(\grm{\sigma}^\MM(\mu), \grm{t}^\MM(\mu)\right) \\
\grm{\pi_1(\sigma)}^\MM(\mu)
  &:\equiv \pr_1(\grm{\sigma}^\MM(\mu)) \\
\grm{\pi_2(\sigma)}^\MM(\mu)
  &:\equiv \pr_2(\grm{\sigma}^\MM(\mu))
\end{align*}
Substitution laws hold strictly again.

The definition of morphisms doesn't yet make the algebras over a given
signature \grm{\Gamma} a category: For this we would still need to define
identity and composition and prove the category laws.
But these constructions are trivial: In last consequence
they are identity and composition of functions in the outer type theory.
Besides this, they are not necessary to state the most important categorical attribute
is definable without identity and composition:
Initiality.

\begin{defn}[Initial Algebra]\label{def:ii-initiality}
An algebra $\gamma : \grm{\Gamma}^\CC$ is called \textbf{weakly initial} if
for every other algebra $\delta : \grm{\Gamma}^\CC$ we can find a morphism
\begin{equation*}
\mu : \grm{\Gamma}^\MM(\gamma, \delta) \text{.}
\end{equation*}
It is (strongly) \textbf{initial}, if this morphism is unique for every $\delta$.
\end{defn}

The existence of initial algebras for every possible context
will be the main topic of \Cref{chp:red}.

\begin{remark}
We now defined what it means for an algebra to admit a non-dependent eliminator
and its $\beta$-rules, but what about the dependent eliminator?
It is usually defined as the sections of an \emph{dependent morphism}
or a \emph{displayed algebra},
over the inital algebra -- we will see this notion for the inductive families in the
next chapter.
But, as \citet{constructingqiits} have proven for their version of syntax for higher
inductive-inductive
types, the dependent version can be recovered from the non-dependent one,
representing the dependent morphism as a non-dependent one by
taking its $\Sigma$-type.
\end{remark}

%TODO

%\section{Displayed Algebras and Sections}

%TODO

%\section{Existence of Inductive-Inductive Types}

%TODO

%\section{Initiality implies Elimination}

%TODO: use sigma construction for this, cite whoever's done this first








