\documentclass[12pt,headings=optiontohead,openany,oneside,a4paper]{book}
%TODO: page layout
\usepackage[backref=page,
unicode,
pdfauthor={Jakob von Raumer},
pdftitle={Higher Inductive Types, Inductive Families, and Inductive-Inductive Types},
pdfsubject={Mathematics, Computer Science},
pdfkeywords={interactive theorem proving}]{hyperref}

%code listings
\usepackage{mathpartir}
\usepackage{newunicodechar}
%\renewcommand{\MintedPygmentize}{./pygments-main/pygmentize}
\usepackage{fontspec,xunicode}
\usepackage[osf]{mathpazo}
%\usepackage{shellesc}
\usepackage{minted}
\setmainfont[Ligatures=TeX]{TeX Gyre Pagella}
\setmonofont{Iosevka}
%\newfontfamily{\freeserif}{DejaVu Sans Mono}
%\newunicodechar{⦃}{\ensuremath{\texttt{\{}\mkern-7mu\texttt{|}}}
	%{\makebox[.5em]{\freeserif⦃}}
%\newunicodechar{⦄}{\ensuremath{\texttt{|}\mkern-7mu\texttt{\}}}}
	%{\makebox[.5em]{\freeserif⦄}}
%\newunicodechar{→}{\texttt{\freeserif{→}}}
%\newunicodechar{⟶}{\texttt{\freeserif{⟶}}}
%\newunicodechar{⁻}{\texttt{\freeserif{⁻}}}
%\newunicodechar{▹}{\freeserif{▹}}
%\newunicodechar{ℕ}{\texttt{\freeserif{ℕ}}}
%\newunicodechar{⟨}{\texttt{\freeserif{⟨}}}
%\newunicodechar{⟩}{\texttt{\freeserif{⟩}}}
%\newunicodechar{⬝}{\ensuremath{\ct}}
%\newunicodechar{∼}{\freeserif{∼}}
%\newunicodechar{≃}{\freeserif{≃}}
%\newunicodechar{≅}{\freeserif{≅}}
%\newunicodechar{∘}{\freeserif{∘}}
%\newunicodechar{ʰ}{\freeserif{ʰ}}
%\newunicodechar{ᵍ}{\freeserif{ᵍ}}
%\newunicodechar{⇒}{\freeserif{⇒}}
%\newunicodechar{⋆}{\freeserif{⋆}}
%\newunicodechar{∘}{\freeserif{∘}}
%\newunicodechar{←}{\freeserif{←}}
%\newunicodechar{∙}{\freeserif{∙}}
%\newunicodechar{▶}{\freeserif{▶}}
%\newunicodechar{∀}{\freeserif{∀}}

%workaround to remove red boxes
\AtBeginEnvironment{minted}{\renewcommand{\fcolorbox}[4][]{#4}}
\usemintedstyle{default}
\newminted{lean}{
	linenos,
	numbersep=5pt,
	frame=single,
	fontsize=\footnotesize,
	framesep=1mm,
	samepage,
	autogobble}
\newminted[leancodebr]{lean}{
	linenos,
	numbersep=5pt,
	frame=single,
	fontsize=\footnotesize,
	framesep=1mm,
	autogobble}
\newcommand{\leani}[1]{\mintinline{lean}{#1}}
\newminted{agda}{
	linenos,
	numbersep=5pt,
	frame=single,
	fontsize=\footnotesize,
	framesep=1mm,
	samepage,
	autogobble}
\newcommand{\agdai}[1]{\mintinline{agda}{#1}}
\linespread{1.07}
\usepackage[dvipsnames]{xcolor}
\usepackage{graphicx}
\usepackage{array}
\usepackage[all,2cell,cmtip]{xy}
\usepackage{tikz}
\usetikzlibrary{decorations.pathmorphing,arrows}
\usepackage[numbered]{bookmark}
\usepackage{fancyhdr}
\usepackage{amssymb,amsmath,amsthm,mathrsfs,wasysym}
\usepackage{enumitem,mathtools,xspace}
\usepackage[nottoc]{tocbibind}
\usepackage{cleveref}
\usepackage{aliascnt}
\usepackage{natbib}
\usepackage{mathtools}
\usepackage{footnote}
\usepackage{booktabs}
\usepackage{xifthen}
%\usepackage[top=1in, bottom=1.25in, left=1.25in, right=1.25in]{geometry} %TODO decide
\usepackage{rotating}
\usepackage{makecell}
\usepackage{stmaryrd}
%\usepackage{thmtools} this makes tex take forever
%\usepackage{thm-restate} this as well

\fancyhead[LO]{\leftmark}
\fancyhead[RE]{\rightmark}
\fancyhead[LE,RO]{\thepage}
\cfoot{}
\pagestyle{fancy}

\def\defthm#1#2#3{
	\newaliascnt{#1}{thm}
	\newtheorem{#1}[#1]{#2}
	\aliascntresetthe{#1}
	\crefname{#1}{#2}{#3}
}

\newtheorem{thm}{Theorem}[section]
\crefname{thm}{Theorem}{Theorems}
\defthm{lemma}{Lemma}{Lemmas}
\defthm{axiom}{Axiom}{Axioms}
\defthm{corollary}{Corollary}{Corollaries}
\theoremstyle{definition}
\defthm{defn}{Definition}{Definitions}
\defthm{example}{Example}{Examples}
\defthm{remark}{Remark}{Remarks}

\newcommand{\inv}{^{-1}}
\newcommand{\twotype}{\mathbf{2}}
\newcommand{\unit}{\mathbf{1}}
\newcommand{\emptytype}{\mathbf{0}}
\newcommand{\UU}{\mathcal{U}}
\newcommand{\isProp}{\mathsf{isProp}}
\newcommand{\isSet}{\mathsf{isSet}}
\newcommand{\PropU}{\mathsf{Prop}}
\newcommand{\SetU}{\mathsf{Set}}
\newcommand{\seg}{\mathsf{seg}}
\newcommand{\isContr}{\mathsf{isContr}}
\newcommand{\isIso}{\mathsf{isIso}}
\newcommand{\idtoiso}{\mathsf{idtoiso}}
\newcommand{\idtoeqv}{\mathsf{idtoeqv}}
\newcommand{\ua}{\mathsf{ua}}
\newcommand{\alliso}{\mathsf{alliso}}
\newcommand{\refl}{\mathsf{refl}}
\newcommand{\ap}{\mathsf{ap}}
\newcommand{\coe}{\mathsf{coe}}
\newcommand{\apd}{\mathsf{apd}}
\newcommand{\happly}{\mathsf{happly}}
\newcommand{\ind}{\mathsf{ind}}
\newcommand{\rec}{\mathsf{rec}}
\newcommand{\pr}{\mathsf{pr}}
\newcommand{\inl}{\mathsf{inl}}
\newcommand{\inr}{\mathsf{inr}}
\newcommand{\ishae}{\mathsf{ishae}}
\newcommand{\apiop}{\ap_{\iota'}}
\DeclareMathOperator{\comp}{comp}
\newcommand{\id}{\mathsf{id}}
\DeclareMathOperator{\invv}{inv_1}
\DeclareMathOperator{\invh}{inv_2}
\DeclareMathOperator{\obj}{obj}
\DeclareMathOperator{\EI}{EI}
\DeclareMathOperator{\IE}{IE}
\DeclareMathOperator{\op}{op}
\DeclareMathOperator{\inj}{inj}
\newcommand{\assoc}{\mathsf{assoc}}
\newcommand{\assocv}{\mathsf{assoc}_1}
\newcommand{\assoch}{\mathsf{assoc}_2}
\newcommand{\idLeft}{\mathsf{idLeft}}
\newcommand{\idLeftv}{\mathsf{idLeft}_1}
\newcommand{\idLefth}{\mathsf{idLeft}_2}
\newcommand{\idRight}{\mathsf{idRight}}
\newcommand{\idRightv}{\mathsf{idRight}_1}
\newcommand{\idRighth}{\mathsf{idRight}_2}
\newcommand{\leftInv}{\mathsf{leftInv}}
\newcommand{\leftInvv}{\mathsf{leftInv}_1}
\newcommand{\leftInvh}{\mathsf{leftInv}_2}
\newcommand{\rightInv}{\mathsf{rightInv}}
\newcommand{\rightInvv}{\mathsf{rightInv}_1}
\newcommand{\rightInvh}{\mathsf{rightInv}_2}
\newcommand{\respectId}{\mathsf{respectId}}
\newcommand{\respectIdv}{\mathsf{respectId}_1}
\newcommand{\respectIdh}{\mathsf{respectId}_2}
\newcommand{\respectComp}{\mathsf{respectComp}}
\newcommand{\respectCompv}{\mathsf{respectComp}_1}
\newcommand{\respectComph}{\mathsf{respectComp}_2}
\newcommand{\isequiv}{\mathsf{isequiv}}
\newcommand{\isodd}{\mathsf{isodd}}
\newcommand{\thin}{\mathsf{thin}}
\newcommand{\Sbase}{\mathsf{base}}
\newcommand{\Sloop}{\mathsf{loop}}
\newcommand{\Smerid}{\mathsf{merid}}
\newcommand{\Ssurf}{\mathsf{surf}}
\newcommand{\vecty}{\mathsf{Vec}}
\newcommand{\nil}{\mathsf{nil}}
\newcommand{\cons}{\mathsf{cons}}
\newcommand{\ct}{
	\mathchoice{\mathbin{\raisebox{0.5ex}{$\displaystyle\centerdot$}}}
		{\mathbin{\raisebox{0.5ex}{$\centerdot$}}}
		{\mathbin{\raisebox{0.25ex}{$\scriptstyle\,\centerdot\,$}}}
		{\mathbin{\raisebox{0.1ex}{$\scriptscriptstyle\,\centerdot\,$}}}
}
\newcommand{\trunc}[2]{\mathopen{}\left\Vert #2\right\Vert_{#1}\mathclose{}}
\newcommand{\tproj}[3][]{\mathopen{}\left|#3\right|_{#2}^{#1}\mathclose{}}
\newcommand{\squash}[1]{\trunc{}{#1}}
\newcommand{\isntype}[1]{\mathsf{is}\mbox{-}{#1}\mbox{-}\mathsf{type}}
\newcommand{\N}{\mathbb{N}}
\newcommand{\Sph}{\mathbb{S}}
\newcommand{\Set}{\mathbf{Set}}
\newcommand{\Cat}{\mathbf{Cat}}
\newcommand{\Fam}{\mathbf{Fam}}
\newcommand{\Alg}{\mathbf{Alg}}
\newcommand{\CT}{\mathbf{CT}}
\newcommand{\CTw}{\mathbf{CTw}}
\newcommand{\old}[1]{\overline{#1}}
\newcommand{\oldold}[1]{\overline{\overline{#1}}}
\newcommand{\gr}[1]{{\color{ForestGreen}#1}}
\newcommand{\grm}[1]{\ensuremath{\gr{#1}}}
\newcommand{\blm}[1]{\ensuremath{{\color{Black}#1}}}
\newcommand{\tqm}[1]{\ensuremath{{\color{Blue}#1}}}
%\newcommand{\grinfer}[2]{\grm{\inferrule{{#1}}{{#2}}}}
\newcommand{\Sc}{\mathsf{S}}
\newcommand{\Pc}{\mathsf{P}}
\newcommand{\CC}{\mathsf{A}} %TODO maybe jus replace
\renewcommand{\AA}{\mathsf{A}}
\newcommand{\EE}{\mathsf{E}}
\newcommand{\PP}{\mathsf{P}}
\newcommand{\mm}{\mathsf{m}}
\newcommand{\MM}{\mathsf{M}}
\newcommand{\DD}{\mathsf{D}}
\newcommand{\WW}{\mathsf{W}}
\newcommand{\RR}{\mathsf{R}}
\newcommand{\Susp}{\mathsf{Susp}}
\newcommand{\SuspN}{\mathsf{N}}
\newcommand{\SuspS}{\mathsf{S}}
\newcommand{\Suspmerid}{\mathsf{merid}}
\renewcommand{\SS}{\mathsf{S}}
\newcommand{\Sg}{\mathsf{\Sigma}}
\newcommand{\con}[1]{\mathsf{con}(\grm{#1})}
\newcommand{\contwo}[2]{\mathsf{con}(#1,\grm{#2})}
\newcommand{\conthree}[3]{\mathsf{con}(#1, #2, \grm{#3})}
\newcommand{\IFconS}[1]{\mathsf{con_\Sc}(\tqm{#1})}
\newcommand{\IFcon}[1]{\mathsf{con}(\tqm{#1})}
\newcommand{\IFelimS}[2]{\ifthenelse{\isempty{#2}}{\mathsf{elim_\Sc}(\tqm{#1})}{\mathsf{elim_\Sc}(\tqm{#1}, #2)}}
\newcommand{\IFelim}[2]{\ifthenelse{\isempty{#2}}{\mathsf{elim}(\tqm{#1})}{\mathsf{elim}(\tqm{#1}, #2)}}
\newcommand{\elim}{\mathsf{elim}}
\newcommand{\flatten}[1]{\blm{\mathsf{flatten}(\grm{#1})}}
\newcommand{\annotate}[1]{\blm{\mathsf{an}(\grm{#1})}}
\newcommand{\anntwo}[2]{\blm{\mathsf{an}(#1, \grm{#2})}}
\newcommand{\app}{\mathrel{@}}
\newcommand{\SCon}{\vdash_\Sc}
\newcommand{\var}{\mathsf{var}}
\newcommand{\vz}{\mathsf{vz}}
\newcommand{\vs}{\mathsf{vs}}
\newcommand{\El}{\mathsf{El}} %TODO change to operator?
\newcommand{\IW}[4]{\mathsf{IW}_{#1, #2}^{#3,#4}}
\newcommand{\IWsup}[2]{\mathsf{sup}(#1,#2)}
\newcommand{\LSub}[3]{\mathsf{LSub}_{\tqm{#1}}(\tqm{#2}, \tqm{#3})}
\newcommand{\wk}{\mathsf{wk}}
\newcommand{\IFSub}[3]{#2 \overset{#1}{\longrightarrow} #3}
\newcommand{\IISub}[3]{#2 \overset{#1}{\longrightarrow} #3}
\newcommand{\IIapp}{\mathsf{app}}
\newcommand{\ExtPi}[2]{\hat{\Pi}(\blm{#1},\, #2)}
\newcommand{\ExtPiS}[2]{\hat{\Pi}_\Sc(\blm{#1},\, #2)}
\newcommand{\ExtPiP}[2]{\hat{\Pi}_\Pc(\blm{#1},\, #2)}
\newcommand{\bltau}{\blm{\tau}}
\newcommand{\blalpha}{\blm{\alpha}}
\newcommand{\blalphaCC}{\blm{\alpha^\CC}}
\newcommand{\blgamma}{\blm{\gamma}}
\newcommand{\blgammaCC}{\blm{\gamma^\CC}}
\newcommand{\blpi}{\blm{\pi}}
\newcommand{\blphi}{\blm{\phi}}
\newcommand{\blphiCC}{\blm{\phi^\CC}}
\newdimen\diagx
\setlength{\diagx}{2cm}
\newdimen\diagy
\setlength{\diagy}{1.5cm}
\newcommand{\pout}[3]{#2 \! \sqcup^{#1} \! #3}
\newcommand{\glue}{\ensuremath{\mathsf{glue}}}

%stuff from the lics paper
\newcommand{\quot}{A \! \sslash \! {\scriptstyle \sim}}
\newcommand{\Z}{\mathbb Z}
\newcommand{\specialquot}[1]{#1 \! \sslash \! {\scriptstyle \sim}}
\newcommand{\specialtwoquot}[2]{#1 \! \sslash \! #2}
\newcommand{\CatA}{\mathcal A}
\newcommand{\CatC}{\mathcal C}
\newcommand{\CatD}{\mathcal D}
\newcommand{\CatP}{\mathcal P}
\newcommand{\bproj}[1]{\tproj{}{#1}}
\newcommand{\inp}{\ensuremath{\mathsf{in}}}
\newcommand{\Sone}{\mathbb{S}^1}
\newcommand{\code}{\mathsf{code}}
\newcommand{\glconstr}{\mathsf{gl}}

\renewcommand{\backrefalt}[4]{
	\ifcase #1
		(No citations.)
	\or
		(Cited on page\ #2.)
	\else
		(Cited on pages\ #2.)
	\fi
}
\renewcommand{\arraystretch}{2.2} %TODO maybe change this locally

%overwrite citation styles
\defcitealias{hottbook}{Homotopy Type Theory, 2013}

\begin{document}

%\title{On Path Spaces of Higher Inductive Types and a Formal Treatment of Type Erasure on Inductive-Inductive Types}
%\title{On Path Spaces of Higher Inductive Types and the Reduction of Inductive-Inductive Types to Inductive Families}
\title{Higher Inductive Types, Inductive Families, and Inductive-Inductive Types}
\author{Jakob von Raumer}

\frontmatter

\maketitle

\chapter{Abstract}

Martin-Löf type theory is a formal language which is used both as a
foundation for mathematics and a the theoretical basis of a range of
functional programming languages.
Inductive types are an important part of type theory which is necessary to express
data types by giving a list of rules stating how to form this data.
In this thesis we we tackle several questions about different classes of
inductive types.

In the setting of homotopy type theory, we will take a look at higher inductive
types based on homotopy coequalizers and characterize their path spaces with
a recursive rule which looks like an induction principle.
This encapsulates a proof technique known as ``encode-decode method''.

In an extensional meta-theory we will then explore the phenomenon of induction-induction,
specify inductice families and discuss how we can reduce each instance of an
inductive-inductive type to an inductive family.
Our result suggests a way to show that each type theory which encompasses
inductive families can also express all inductive-inductive types.

\tableofcontents

\mainmatter

\chapter{Introduction}\label{chp:intro}

\section{Examples of Inductive-Inductive Types}

In the following, we will take a look at a few examples which we are going to
revisit at various steps throughout this presentation:

\begin{example}[Type Theory Syntax]\label{ex:ttintt}
\cite{ttintt} showed how to internalise the syntax of type theory inside type
theory itself, using a quotient inductive-inductive type.
Leaving out terms and substitutions, we arrive at a fragment of type theoretical
syntax specifying a type of contexts and a type of types over a certain contexts,
together with type formers for a unit type and a $\Pi$-type:
We want $\mathop{Con} : \UU$ to be inductively defined by
\begin{align*}
\mathop{nil}	&: \mathop{Con} \text{and } \\
\mathop{ext}	&: (\Gamma : \mathop{Con})(A : \mathop{Ty}(\Gamma)) \to \mathop{Con} \text{,}
\end{align*}
while simultaneously defining a family $\mathop{Ty} : \mathop{Con} \to \UU$ with constructors
\begin{align*}
\mathop{unit}	&: (\Gamma : \mathop{Con}) \to \mathop{Ty}(\Gamma) \text{ and} \\
\mathop{pi}	&: (\Gamma : \mathop{Con})(A : \mathop{Ty}(\Gamma))(B : \mathop{Ty}(\mathop{ext}(\Gamma, A))) \to \mathop{Ty}(\Gamma) \text{.}
\end{align*}
\end{example}

\begin{example}[Free Dense Completion]
\cite{nordvallinductive} proposed the example of a ``free dense completion'' of
an order (or, more general, any relation) which for any type $A : \UU$ and
any type valued relation $\_<\_ : A \to A \to \UU$ on $A$ freely adds midpoints
to all pairs of related elements by $\_<\_$.
It does so by introducing a new type $A' : \UU$ inductively generated by the
original points and their midpoints:
\begin{align*}
\iota_A		&: A \to A' \text{ and} \\
\mathop{mid}	&: \{x, y : A'\}(p : x <' y) \to A' \text{.}
\end{align*}
But since our relation was only defined on $A$, we have to extend it to $A'$ by
postulating
\begin{align*}
\iota_<		&: \{a, b : A\}(p : a < b) \to \iota_A(a) <' \iota_A(b) \text{,} \\
\mathop{mid}_l	&: \{x, y : A'\}(p : x <' y) \to x < \mathop{mid}(p) \text{, and} \\
\mathop{mid}_r	&: \{x, y : A'\}(p : x <' y) \to \mathop{mid}(p) < y \text{.}
\end{align*}
\end{example}


\chapter{Basic Type Theory}\label{chp:tt}

This chapter shall serve to introduce the basic notions of type theory which
we will need for the subsequent content of this thesis.
At first (Chapter~\ref{sec:tt-dtt}), we will have a general look on dependent type theory, its use and
how it differs from a set theoretic foundation,
afterwards (Chapter~\ref{sec:tt-w}) we want to unify the notion of an inductive
type using the concept of indexed W-types.
Then (Chapter~\ref{sec:tt-hott}), we will explain how homotopy type theory was created to have a suitable
language to reason about higher equalities and how it provides a synthetic way to
formalize topological insights.
Finally, in Chapter~\ref{sec:tt-provers}, we will give examples of two theorem
provers based on dependent types, Agda and Lean, and point out some of their differences.

\section{Dependent Type Theory}\label{sec:tt-dtt}

The term ``type theory'' stems from the early nineteenth centry, when
Bert\-rand Russell sought to lay out an alternative form of set theory which did not
suffer from the paradox which Russell discovered.
Todays versions of type theory have little in common with Russell's attempts
but rather rely on the considerations of Per Martin-Löf (\cite{martin-lof2, martin-lof1})
who, starting in the 1970's, built a new mathematical foundation based on the
$\lambda$-calculus, himself drawing inspiration from previous logicians and
mathematicians like Alonzo Church and Haskell Curry.
Often, dependent type theory is also referred to as Martin-Löf type theory (MLTT).

Based on a phenomenon known as the ``Curry-Howard corres\-pon\-dence'', type theory
can serve both as a theoretical foundation for a formal representation of mathematics,
as well as a principle for the specification for strongly typed functional
programming languages.
It was implemented in computer languages for programming and theorem proving which
are massively used in the field of formal verification and in the formalization of
mathematics.
Among the most commonly used implementations are the theorem prover Coq (\cite{coq})
which notably has a lot of users in the field of hardware verification,
the prover Agda (\cite{agda}), which is popular amongst type theorists themselves,
and the Microsoft Research based project Lean(\cite{mouracade}), which has
drawn considerable attention from researching mathematicians as a tool to
formally verify their proofs.

Type theory differs from a set theoretic mathematical foundation (let us, as a
point of reference, consider set theories based on first-order predicate logic
like Zermelo-Fraenkel set theory) in several
important aspects:
\begin{itemize}
\item Type theory follows a paradigm called \emph{``propositions-as-types''}.
This means that statements like theorems and conjectures are represented
using the same class of objects as other data like sets or (algebraic) structures.
In contrast to this, most set theoretic foundations are built on a \emph{dichotomy between
the propositions and the objects} they describe:
They first start out with a logical framework on which axiomatically a theory
of sets is introduced.
The coherence between these two levels must then be created using an axiom like
the comprehension axiom in Zermelo-Fraenkel set theory.
\item Type theory is \emph{typed} while set theory is \emph{untyped}.
While in set theory, objects can be an element of different sets --
consider the number two which is an element both of the set of even integers
as well as the set of all integers --
type theory is based on the principle that every piece of data (every term)
is assigned a unique type which is known at the point of the creation of the data.
This assignment, called typing, is decidable, and we consider it a judgment
rather than a provable proposition that a given term $t$ has type $A$.
\item Type theory is inherently \emph{constructive} while many
set theoretic foundations, such as Zermelo-Fraenkel, are \emph{non-constructive}.
This has great consequences for the computational use of the mathematics represented:
Every type theoretic function the codomain of which are the natural numbers, can compute a
numeral for any given input.
\item While in set theory, \emph{sets are the only primitives} and all data is
encoded as sets, type theory
provides often-used objects like \emph{functions and inductively defined types
as primitives}.
This means that we can use these without having to care about how to assemble them
from other primitives. 
%It also offers modularity in the following sense: %TODO find a better word than modularity
%In contrast to a function in set theory, a function in type theory only reveals
%its constituing information:
%What is the resulting output for a given input?
\end{itemize} %TODO maybe say something about sizes

Let us now fix some notation and some basic constructions which will occur
throughout this text.
As mentioned, we will, whenever we talk about a certain piece of type theoretic
data, accompany it with its type,
we need a notation for this kind of \textbf{type judgment}:
We write $t : A$ to state that the term $t$ is of type $A$.
This is similar to the element relation of a set, but it also represents the fact
that $t$ is a proof for a proposition $A$.
Sometimes, we want to express that two terms $s$ and $t$ of the same type $A$
only differ by an unfolding or folding of a definition, or that the application of
a reduction rule to $s$ results in $t$.
We will notate this by $s \equiv t$.
Our type theory will then make no distinction between $s$ and $t$.
This means that if $A$ and $B$ are types with $A \equiv B$, then
$s : A$ implies $s : B$.
To keep type checking decidable it is easy to see that it is important
to keep the question whether two terms are what we call \textbf{definitionally equal}
decidable itself.

We said above that each piece of data, so each term,
has a unique type (up to definitional equality).
This statement also holds true for types itself.
Types which themselves contain types, are called \textbf{universes} and we will
denote them using $\UU$.
The universe itself also needs a type, but
assuming $\UU : \UU$ is inconsistent.
This is often referred to as
``Girard's paradox'' \citep{girard72,hurkens95} which can be seen as the type theoretic equivalent
to Russell's paradox.
The solution we assume for the remainder of this text is
to assume that we have an infinite chain of universes
\begin{equation*}
\UU_0 : \UU_1,~~ \UU_1 : \UU_2,~~ \UU_2 : \UU_3,~~ \ldots \text{,}
\end{equation*}
each contained in the next one.
Most often, we will chose to leave the index implicit and regard our constructions
as being \emph{universe polymorphic}, meaning that they are valid in any chosen
universe.
Some type theories are constructed to be \emph{cumulative} in the sense that whenever
we have a type $A : \UU_i$, it is a type in the succeeding universe, so $A : \UU_{i+1}$.
Our constructions will not rely on cumulativity and they are formalized in
non-cumulative type theories.

Until now, we have not talked about any way to form types.
In the following, we will get to know the non-dependent version of some basic types,
some of which will later be generalized to a dependent form.
These type formers will be presented in a very similar way:
We will give a rule on how to form the type itself (called formation rule),
rules on how to construct elements of the type (called introduction rules),
rules on how to use the elements of the type (called elimination rules),
and some of them will be followed by some reduction rules offering definitional
equalities used to simplify terms.
As we have mentioned in the comparison to set theory, functions are basic building
blocks in type theory.
The non-dependent functions form what is known as \emph{simply typed $\lambda$-calculus}.
The formation, introduction, and elimination rules correspond to the fact that
we can form the function type of any type for its domain and any type for its
codomain, the fact that we can build a function by
specifying its output for any given input, and the fact that we can apply
a function to any term of its domain:
\begin{equation*}
\begin{gathered}
\inferrule*[left=$\to$-Form]{A,B : \UU}{A \to B : \UU} \qquad
\inferrule*[left=$\to$-Intro]{a : A \vdash \Phi[a/x] : B}{(\lambda x. \Phi) : A \to B} \\[.7em]
\inferrule*[left=$\to$-Elim]{f : A \to B \\ a : A}{f(a) : B}
\end{gathered}
\end{equation*}
Here, $\Phi$ is a term that may have $x$ as a free variable.
$\Phi[a/x]$ denotes the replacement of every free appearance of $x$ by $a$.
Additional to these rules we also have $\eta$-conversion and $\beta$-reduction:
\begin{equation*} \label{eq:eta-beta}
\begin{aligned}
\inferrule*[left=$\eta$]{f : A \to B}{(\lambda x. f(x)) \equiv f} \qquad
\inferrule*[left=$\beta$]{(\lambda x. \Phi) : A \to B \\ a : A}
	{(\lambda x. \Phi)(a) \equiv \Phi[a/x]}
\end{aligned}
\end{equation*}
In type theory, these functions are used to represent
both functions between sets as well as implications between propositions.
A special kind of function are those of the form $B : A \to \UU$.
These so called \textbf{type families} are used to represent set-valued
functions as well as propositions with a free variable in $A$.

\begin{remark}\label{rmk:tt-syntax}
When giving inference rules as the ones above, we always assume an arbitrary
\emph{context} of variables for their premises as well as their conclusion.
Often, when presenting the syntax of a type theory, rules on how
to form these contexts $\vdash \Gamma$ are given together with rules for the
formation of types $\Gamma \vdash A$ in a given context $\Gamma$ and the
treatment of terms $\Gamma \vdash t : A$ of a given type $A$ in context $\Gamma$.
We will choose to be implicit about the variable context in this chapter, while
being more explicit about them when we introduce other syntaxes in later chapter,
for which a more formal treatment is appropriate.
\end{remark}

Since we want a representation of logics in type theory, and so far the only logical
connective we introduced are implications, we might next ask for types representing
the propositions ``true'' and ``false''.
Since the elements of this type should correspond to the proofs of ``true'' and
``false'', we can conclude that the type corresponding to ``true'' should contain
one canonical element, while the one corresponding to ``false'' should contain
none.
Thus, we will call these two types the \textbf{empty type} and the \textbf{unit type},
respectively and denote them with $\emptytype$ and $\unit$.
The formation and introduction rules for these are not very surprising:
\begin{equation*}
\begin{gathered}
\inferrule*[left=$\emptytype$-Form]{ }{\emptytype : \UU} \qquad
\inferrule*[left=$\unit$-Form]{ }{\unit : \UU} \qquad
\inferrule*[left=$\unit$-Intro]{ }{\star : \unit} \\[.7em]
\end{gathered}
\end{equation*}
The elimination rule for $\emptytype$ says the we can derive a proof for any statement,
given a proof of ``false'' (``ex falso quodlibet''), while the elimination rule
for $\unit$ specifies that when showing a statement about the elements of the
unit type, it suffices to consider its canonical element $\star$:
\begin{equation*}
\inferrule*[left=$\emptytype$-Elim]{C : \emptytype \to \UU \\ x : \emptytype}
	{\elim_\emptytype(C,x) : C(x)} \qquad
\inferrule*[left=$\unit$-Elim]{C : \unit \to \UU \\ p : C(\star) \\ x : \unit}
	{\elim_\unit(C,p,x) : C(x)}
\end{equation*}
with the reduction rule $\elim_\unit(C,p,\star) \equiv p$.
Some type theories assume so called $\eta$-rules for types like $\unit$, which
say that for every $x : \unit$, we have $x \equiv \star$.
We will \emph{not} assume these rules to hold, but instead we will later be able to
express the fact that all instances of $\unit$ are equal to $\star$ using propositional
equality.

To give an example for a type which contains more than just one element, we can
consider the type of booleans $\twotype$, containing elements $0_\twotype$
and $1_\twotype$.
This time, we have more than just one constructor, and the elimination rule
requires us to give proofs for both of the two elements:
\begin{equation*}
\begin{gathered}
\inferrule*[left=$\twotype$-Form]{ }{\twotype : \UU} \qquad
\inferrule*[left=$\twotype$-Intro1]{ }{0_\twotype : \twotype} \qquad
\inferrule*[left=$\twotype$-Intro2]{ }{1_\twotype : \twotype} \\[.7em]
\inferrule*[left=$\twotype$-Elim]{C : \twotype \to \UU \\ p_0 : C(0_\twotype) \\ p_1 : C(1_\twotype) \\ x : \twotype}
  {\elim_\twotype(C, p_0, p_1, x) : C(x) }
\end{gathered}
\end{equation*}
with the reduction rules $\elim_\twotype(C, p_0, p_1, 0_\twotype) \equiv p_0$
and $\elim_\twotype(C, p_0, p_1, 1_\twotype) \equiv p_1$.

As a minimal example of an infinite type, the set theoretic equivalent of which
would be introduced axiomatically, we can take a look at the type of
natural numbers.
It is generated inductively by the zero element and the successor function.
The eliminator provides the principle of induction on the natural numbers, which
also makes sure that, when proving properties about the natural numbers,
we can assume that every element of the natural numbers is equal to a repeated
applying the successor function to zero:
\begin{equation*}
\begin{gathered}
\inferrule*[left=$\N$-Form]{ }{\N : \UU} \qquad
\inferrule*[left=$\N$-Intro]{ }{0 : \UU \\ S : \N \to \N} \\[.7em]
\inferrule*[left=$\N$-Elim]
	{C : \N \to \UU \\ p : C(0) \\ q_n : C(n) \to C(S(n)) \text{ for all $n : \N$} \\
		x : \N}
	{\elim_\N(C, p, q, x) : C(x)} %TODO find a better solution to replace the pi type
\end{gathered}
\end{equation*}
with the reduction rules
\begin{align*}
\elim_\N(C, p, q, 0) &\equiv p \text{ and} \\
\elim_\N(C, p, q, S(x)) &\equiv q(x, \ind_\N(C, p, q, x)) \text{.}
\end{align*}
Note that we regain, by restricting $\elim_N$ to constant type families,
the \emph{non-dependent eliminator} or \emph{recursor},
corresponding to the usual way to recursively define functions $f : \N \to A$ for
a type $A : \UU$ by providing the value $f(0)$ and, for each $n : \N$ the
recursive definition $f(n) \to f(S(n))$.
It has the following type:
\begin{equation*}
\rec_\N(A) \equiv \ind(\lambda x. A) : A \to (\N \to A \to A) \to \N \to A \text{.}
\end{equation*}

So far we did not do the word ``dependent'' in the name of the type theory
much justice,
but now we will move on to introduce some dependent type formers.
The first of these will be \textbf{$\Pi$-types} or \text{dependent function types}.
When considering non-dependent functions, the codomain was a fixed type $B$
such that for all inputs $a : A$, the output $f(a)$ is an element of $B$.
For dependent functions, however, the codomain is a type family $B : A \to \UU$,
and for each input $a : A$ the output $f(a)$ is of type $B(a)$.
Considering that we want to use type families to represent propositions depending on
a free variable, these functions represent the universal quantification over this
free variable!
While for application and $\lambda$-abstraction we don't introduce new notation
for dependent functions, we will denote the type of all dependent functions
on a type family $B : A \to \UU$ as $\prod_{(a : A)} B(a)$, or, alternatively,
as $\Pi(B)$ (as a shorter variant) or $(a : A) \to B(a)$ (often called
``Agda-notation'' in referral to the theorem prover of that name).
The rules to form the type of $\Pi$-types and to introduce and apply dependent
functions generalize the rules for non-dependent functions as follows:
\begin{equation*}
\begin{gathered}
\inferrule*[left=$\Pi$-Form]{A : \UU_i \\ B : A \to \UU_j}
	{\textstyle{(\prod_{(a : A)} B(a))} : \UU_{\max\{i,j\}}} \qquad
\inferrule*[left=$\Pi$-Intro]{a : A \vdash \Phi[a/x] : B(a)}
	{(\lambda x. \Phi) : \textstyle{\prod_{(a : A)} B(a)}} \\[.7em]
\inferrule*[left=$\Pi$-Elim]{f : \textstyle{\prod_{(a : A)} B(a)} \\ a : A}
	{f(a) : B(a)}
\end{gathered}
\end{equation*}
Again, we have the rules for $\beta$-reduction and $\eta$-conversion like in
the non-dependent case, yielding reduction rules in the form of judgmental equalities
$(\lambda x. f (x)) \equiv f$ and $(\lambda x. \Phi)(a) \equiv \Phi[a/x]$.
Having $\Pi$-types at our disposal allows us to state the rules governing
a couple of further essential type formers.
Note that, once we have added dependent functions, we can rediscover
non-dependent functions as the special case of dependent functions over
a constant type family.
When iterating $\Pi$-types,
we will often find that the argument of a dependent function is already
determined by an earlier argument, as in
$f : (a : A)(b : B(a)) \to C(a, b)$.
In this case, we borrow the notation used by many theorem provers and
use curly brackets to denote arguments which will be left implicit:
If $f :\{a : A\}(b : B(a)) \to C(a, b)$ and $b : B(a)$,
we write $f(b) : C(a, b)$.
If we later want to state these explicitly we will re-use curly brackets
to denote that we reintroduce them: $f\{a\}(b) : C(a, b)$.

Two important logical connectives are still missing:
Conjunction and disjunction. In type theory these coincide with
the product and disjoint union (sum) of types.
The rules for the product type are as follows:
\begin{equation*}
\begin{gathered}
\inferrule*[left=$\times$-Form]{A, B : \UU}{A \times B : \UU} \qquad
\inferrule*[left=$\times$-Intro]{a : A \\ b : B}{(a, b) : A \times B} \\[.7em]
\inferrule*[left=$\times$-Elim]{C : A \times B \to \UU \\
	p: (a : A)(b : B) \to C(a,b) \\ x : A \times B}
	{\elim_{A \times B}(C, p, x) : C(x)}
\end{gathered}
\end{equation*}
with the reduction rule $\elim_{A \times B}(C, p, (a, b)) \equiv p(a, b)$.
The projections of an instance $x : A \times B$ are then
defined by induction:
\begin{align*}
\pr_1(x) &:\equiv \elim_{A \times B}((\lambda y. A), (\lambda a. \lambda b. a), x) : A \text{ and}\\
\pr_2(x) &:\equiv \elim_{A \times B}((\lambda y. A), (\lambda a. \lambda b. b), x) : B \text{,}
\end{align*}
yielding $\pr_1((a, b)) \equiv a$ and $\pr_2((a, b)) \equiv b$ judgmentally.
The type representing disjunction has two constructors determining whether
we provide proof for its left or its right type:
\begin{equation*}
\begin{gathered}
\inferrule*[left=$+$-Form]{A : \UU \\ B : \UU}{A + B : \UU} \\[.7em]
\inferrule*[left=$+$-Intro1]{a : A}{\inl(a) : A + B} \qquad
\inferrule*[left=$+$-Intro2]{b : B}{\inr(b) : A + B} \\[.7em]
\inferrule*[left=$+$-Elim]
	{C : (A + B) \to \UU \\  p : (a : A) \to C(\inl(a))
		\\ q : (b : B) \to C(\inr(b)) \\ x : A + B}
	{\elim_{A + B}(C, p, q, x) : C(x)}
\end{gathered}
\end{equation*}
For the sum type, we have the reduction rules
\begin{align*}
\elim_{A + B}(C, p, q, \inl(a)) &\equiv p(a) \text{ and} \\
\elim_{A + B}(C, p, q, \inr(b)) &\equiv q(b) \text{.}
\end{align*}

Looking at the product type, we can find a generalization which is very useful
when we want to model existential quantification of a type family $B : A \to \UU$
in type theory:
A version of the product type where the type of the second component of a pair $(a, b)$
may depend on $a : A$ via the type family, i.\,e. $b : B(a)$.
The type holding this kind of pair for a fixed type family $B : A \to \UU$
is called the \textbf{$\Sigma$-type} over $B$.
We will again have three different notations for this type, of which in this text
we will mostly prefer the latter one:
The type is usually \citepalias{hottbook} denoted by
$\sum_{(a : A)} B(a)$, mimicing mathematical notation for sums.
Furthermore there is the short variant of writing $\Sigma(B)$ and
an Agda-inspired notation \mbox{$(a : A) \times B(a)$}.
The inference rules for this type are just a slight generalization of the
rules we have already seen for the non-dependent product type:
\begin{equation*}
\begin{gathered}
\inferrule*[left=$\Sigma$-Form]{A : \UU \\ B : A \to \UU}
	{(a : A) \times  B(a) : \UU} \qquad
\inferrule*[left=$\Sigma$-Intro]{a : A \\ b : B(a)}
	{(a, b) : (a : A) \times  B(a)} \\[.7em]
\inferrule*[left=$\Sigma$-Elim]
	{C : (a : A) \times  B(a) \to \UU \\
		p : (a : A)(b : B(a)) \to C((a,b)) \\
		x :  (a : A) \times B(a) }
	{\elim_{(a : A) \times B(a)}(C, p, x) : C(x)}
\end{gathered}
\end{equation*}
As our notation already suggests, we can view the product $A \times B$ as a special
case of a $\Sigma$-type where $B$ is a constant type family.
Note that also the sum type $A + B$ can be defined as a special case of the sigma
type over $C : \twotype \to \UU$ with $C(0_\twotype) :\equiv A$
and $C(1_\twotype) :\equiv B$.
This is why $\Sigma$-types are sometimes also referred to as
\emph{dependent sum types}.
By induction we can define the \textbf{projections}
\begin{align*}
\pr_1 &: ((a : A) \times B(a)) \to A \text{ and} \\
\pr_2 &: (x : (a : A) \times B(a)) \to B(\pr_1(x)) \text{.}
\end{align*}
which return the components of the dependent pairs.
When we iterate $\Sigma$-types an products we will be liberal with the notation
and allow notation like $\pr_3, \ldots$ as well.

With the type formers we met so far, we can already represent a lot of mathematical
definitions and knowledge.
For example, if we wanted to define what it means for a natural number to be odd,
we could set
\begin{equation*}
\isodd :\equiv \rec_\N(\UU, \emptytype, (\lambda n. \lambda A. A \to \emptytype)) : \N \to \UU \text{.}
\end{equation*}
For example, this gives us the statement that the number one is odd, witnessed by
the following term:
\begin{align*}
(\lambda x. x) :~ & \emptytype \to \emptytype \\
 &\equiv \elim_\N((\lambda x. \UU), \emptytype, (\lambda n. \lambda A. A \to \emptytype), 0)
  \to \emptytype\\
 &\equiv \elim_\N((\lambda x. \UU), \emptytype, (\lambda n. \lambda A. A \to \emptytype), S(0))\\
 &\equiv \isodd(S(0)) \text{.}
\end{align*}

\section{Inductive Types}\label{sec:tt-w}

So far, the presented type formers may seem like a zoo of unrelated,
random examples.
Some are generalization of others (like $\Sigma$-types generalize sums),
but the question one might ask is if there is any overarching principle behind
the choice of those type formers.
The feature that all of them have in common is that, rather than by an enumeration
of their elements,
they are defined by their formation, introduction, and elimination rules -- their
elements are those which are generated \emph{inductively} by their introduction
rules, also called \emph{constructors}.

Some of the type formers we have seen were \emph{parameterized} by other types,
but more than that, dependent types allow us to specify \emph{indexed} inductive
types.
One example for this is the type of \emph{vectors} of a type $A$.
Vectors are a variant of lists, where we use the typing to keep track of the
length of its elements:
The type of vectors on $A$ is not a type but a type family:
\begin{equation*}
\vecty_{A} : \N \to \UU \text{.}
\end{equation*}
It has two constructors and an eliminator of the following form:
\begin{equation*}
\begin{gathered}
\inferrule{ }
  {\nil : \vecty(0)}
\qquad
\inferrule{n : \N \\ a : A \\ v : \vecty(n)}
  {\cons(a, v) : \vecty(n + 1)} \\[.7em]
\inferrule{C : \{n : \N\} \to \vecty(n) \to \UU \\
  p_\nil : C(\nil) \\
  p_\cons : \{n : \N\}(a : A)(v : \vecty(n)) \to C(v) \to C(\cons(a,v)) \\
  n : \N \\ v : \vecty(n)}
  {\elim_\vecty(C, p_\nil, p_\cons, v) : C(v)}
\end{gathered}
\end{equation*}
with two reduction rules
\begin{align*}
\elim_\vecty(C, p_\nil, p_\cons, \nil) &\equiv p_\nil \text{ and} \\
\elim_\vecty(C, p_\nil, p_\cons, \cons(a, v)) &\equiv p_\cons(a, v, \elim_\vecty(C, p_\nil, p_\cons, v)) \text{.}
\end{align*}
In natural language, the eliminator says that to show a statement about vectors
it is sufficient to prove it about the empty vector and that the statement is stable
under extending an arbitrary vector.
Note that the natural number in the conclusion of both introduction rules
is not a variable, and the $\cons$ even modifies the natural number.
As such, we call the dependency on the length in the type of vectors
an \emph{index} instead of a parameter.

While dependent and non-dependent functions were primitives needed to even talk
about other type formers,
all other examples which we have seen, can be captured by the concept of
\emph{indexed inductive type}.
But since we want to be more formal than to just assume the presence of indexed
inductive types based on giving a few examples,
we will present different attempts to make precise a definition of what
an indexed inductive type is.
One schematic approach was made by \cite{dybjer94}, while the approach which
we want to introduce here requires an encoding of the constructors in a very specific
way to be able to present the inductive type as an instance of a very general
form of a \emph{tree}.
The types covered by this approach are called \emph{indexed W-Types}, and they
cover a bigger fragment of inductive types than the non-indexed version,
which are also known as ``Petersson-Synek trees'' \citep{petersson1989set},
and which are in a topos theoretic setting, discussed by \cite{moerdijkwellfounded}.
The generalization to indexed W-Types, as presented here,
was found by \cite{indexedcontainers}.

\begin{defn}
Assume that we are given the following data:
\begin{itemize}
\item A type $I : \UU$ of \emph{indices},
\item a type $A : \UU$ encoding the number and non-recursive input \emph{data} or \emph{shapes} for contructors,
\item a function $o : A \to I$ assigning to each piece of input data the corresponding \emph{output index} of
the constructor,
\item a type $B : A \to \UU$ of \emph{recursive occurrences} or \emph{positions} for each
bit of input data, and
\item a function $r : (a : A) \to B(a) \to I$ of indices of \emph{recursive occurrences}.
\end{itemize}
Then, we assume that we have
a type $\IW{A}{B}{o}{r} : I \to \UU$ with the following introduction and elimination
rules:
\begin{equation*}
\begin{gathered}
\inferrule*[left=IW-Intro]{a : A \\
  c : (b : B(a)) \to \IW{A}{B}{o}{r}(r(a, b))}
  {\IWsup{a}{c} : \IW{A}{B}{o}{r}(o(a))} \\[.7em]
\inferrule*[left=IW-Elim]{C : \{i : I\} \to \IW{A}{B}{o}{r}(i) \to \UU \\
  \makecell{p :  (a : A)
      \left (c : (b : B(a)) \to \IW{A}{B}{o}{r}(r(a, b))\right) \\
      \qquad \qquad \to \Bigl((b : B(a)) \to  C(c(b))\Bigr)
      \to C(\IWsup{a}{c})  } }
  {\elim_\mathsf{IW}(C, p) : (i : I)(w : \IW{A}{B}{o}{r}(i)) \to C(w) }
\end{gathered}
\end{equation*}
We furthermore assume that we are provided the reduction rule
\begin{equation*}
\elim_\mathsf{IW}(C, p, o(a), \IWsup{a}{c})
  \equiv p(a, c, \lambda b : B(a).\, \elim_\mathsf{IW}(C, p, r(a,b), c(b))) \text{.}
\end{equation*}
\end{defn}

This definition may seem very confusing and overly complicated, but this is necessary
to capture all possible indexed inductive types in full generality.
In words, the constructor describes that we chose a constructor and giving possible
non-recursive input data by providing $a : A$,
and then, based on this data, give data for all recursive occurrences of the type
in a constructor in the form of $c$, we get a new element $\IWsup{a}{c}$
which is reminiscent of the \emph{supremum} of the given data.
Conversely the elimination rule describes that to proof a statement $C$ about this
tree like structure it is enough to prove $C$ for $\IWsup{a}{c}$
under the hypothesis that it holds for all recursive input data $c(b)$.

Assuming that we already have $\emptytype$, $\unit$, $\twotype$ and $\Sigma$-types
at our disposal (allowing us to define $A + B$),
indexed W-types live up to our expectations of capturing
all previously mentioned inductive types.
To provide a translation of these, including indexed inductive types still to
be defined in the next chapter, we will provide all the
arguments for the respective indexed W-types in Table~\ref{tbl:tt-iws},
while often, in the case of $I \equiv \unit$ describe
$\unit \to A$ in place of a type $A$ itself.

\begin{remark}[Plain W-Types]
Indexed W-types can be reduced to the simpler class of \emph{W-types},
which only are specified by giving a type $A : \UU$ and a family $B : A \to \UU$,
without having a type of indices $I : \UU$, together with $o$ and $r$.
We can think of them as the class of indexed W-types where $I \equiv \unit$.

The fact that we can present each indexed W-types by a non-indexed one was shown
by \citet{indexedcontainers} using the K-rule,
and by \citet{Sattler:indexedW} without presence of the K-rule and under assumption
of functional extensionality.
\end{remark}

In Chapters~\ref{chp:iit} and \ref{chp:if} we will introduce two more calculi to
replace indexed W-types.

\begin{sidewaystable}
\centering
{$\begin{array}{r|lllll}
\text{Type Former} & I : \UU & A : \UU & B : A \to \UU & o : A \to I & r : (a : A) \to B(a) \to I \\
\hline
\N
  & \unit
  & \makecell[cl]{\unit \\ + \unit}
  & \makecell[cl]{\inl(\star) \mapsto \emptytype \\ \inr(\star) \mapsto \unit}
  & \_ \mapsto \star
  & \_ \mapsto \star \\
\vecty_A
  & \N
  & \makecell[cl]{\unit \\ + A \times \N}
  & \makecell[cl]{\inl(\star) \mapsto \emptytype \\ \inr(a, n) \mapsto \unit}
  & \makecell[cl]{\inl(\star) \mapsto 0          \\ \inr(a, n) \mapsto n + 1}
  & \makecell[cl]{ - \\ \inr(a, n) \mapsto (\star) \mapsto n} \\
x =_A \_
  & A
  & \unit
  & \star \mapsto \emptytype
  & \star \mapsto x
  & -
\end{array}$}
\caption{The input data for the indexed W-types corresponding to the type formers
given in Chapter~\ref{sec:tt-dtt}.}\label{tbl:tt-iws}
\end{sidewaystable}

\section{Typal Equality and Homotopy Type Theory}\label{sec:tt-hott}

Until now, the only equations we encountered were judgmental equalities which
aren't ``visible'' internally in the type theory.
But since we want to follow the paradigm of propositions-as-types, we also want to
have a way to represent the statement that two terms of a type $A : \UU$ are
equal \emph{inside} the type theory.
To correct this,
we introduce what is usually called \textbf{propositional equality} or
\textbf{typal equality}.
For each type $A : \UU$ and each two elements $a, b : A$ we want to have
a type $(a =_A b) : \UU$ of proofs that $a$ and $b$ are equal.
We can view this as an inductive definition, giving only the witness for this relation
to be reflexive as a constructor:
\begin{equation*}
\begin{gathered}
\inferrule*[left={$=$-Form}]{A : \UU \\ a, b : A}{a =_A b : \UU} \qquad
\inferrule*[left={$=$-Intro}]{A : \UU \\ a : A}{\refl_a : a =_A a} \\[.7em]
\inferrule*[left={$=$-Elim}]
	{a : A \\ C : (b : A) \to a =_A b \to \UU \\
		c : C(a, \refl_a) \\
		b : A \\ p : a =_A b}
	{\elim_{=_A}(a, C, c, b, p) : C(b, p)}
\end{gathered}
\end{equation*}
with the reduction rule $\elim_{=_A}(C, c, a, p, \refl_a) \equiv c(a)$.
It can be defined as an indexed W-type as per \Cref{tbl:tt-iws}.
The elimination rule says that to prove a statement indexed over varying right
hand sides of an equality and corresponding equality proofs,
we can assume it to be the reflexivity witness $\refl$.

\begin{remark}\label{rmk:tt-k}
Apart from the above elimination rule, which is sometimes referred to as
\textbf{J-rule}, some type theories also assume the so called
\textbf{K-rule}, which in proofs also allow us to simplify a given equality proof
to $\refl$ if the type family which we want to inhabit varies both endpoints
of the equality simultaneously, as in the following:
\begin{equation*}
\inferrule*[left={$=$-K}]{C : (a : A) \to a =_A a \to \UU \\
    c : (a : A) \to C(a, \refl_a) \\
    a : A \\ p : a =_A a}
  {\elim'_{=_A}(C, c, a) : C(a) }
\end{equation*}
We will assume this rule to hold in Chapter~\ref{chp:iit} and later chapters,
but explicitly assume its absence in this and the following two chapters.

In literature, our version of \textsc{$=$-Elim} is often called the
\emph{based} J-rule, because the equation's left hand side is fixed,
or Paulin-Mohring J-rule, after \cite{Moh93}.
An alternative, but provably equivalent version is the so-called
\textbf{unbased J-rule} in which the type family varies over both sides
of the equation:
\begin{equation*}
\inferrule*[left={$=$-Elim'}]
	{C : (a, b : A) \to a =_A b \to \UU \\
		c : C(a, a, \refl_a) \\
		a, b : A \\ p : a =_A b}
	{\elim'_{=_A}(C, c, a, b, p) : C(b, p)}
\end{equation*}
\end{remark}

While we introduced types as a means to represent sets and propositions,
passing on the K-rule also gives rise to a further interpretation:
We can use types to model the homotopy types of topological spaces.
This principle, together with the univalence axiom which we will introduce
in the next chapter, is the underlying insight of the field of \emph{homotopy type theory}.
This correspondence was first explored by \cite{awodey2009homotopy}
and made precise by \cite{kapulkinlumsdaine},
who, after an idea by Vladimir Voevodsky,
modelled homotopy type theory using simplicial sets.
In this correspondence -- a succinct list of which can be found in Table~\ref{tbl:tt-hott} --
equality proofs correspond to \textbf{paths}, a name which we will use as a synonym
from using from now on.

\begin{sidewaystable}
{\small
\begin{tabular}{r|ccc}
  & Logical Interpretation & Set Interpretation & Homotopical Interpretation \\
\hline
(closed) Type $A : \UU$
  & Proposition
  & Set
  & Topological Space \\
Function $f : A \to B$
  & Proof of implication
  & Function between sets
  & Continuous map \\
Type family $B : A \to \UU$
  & Proposition with free variable in $A$
  & Family of sets indexed by $A$
  & Fibration over base $A$ \\
Pair $(a, b) : A \times B$
  & Proof of conjunction
  & Pair in cartesian product
  & Point in product space \\
Element $\inl(a) : A + B$
  & Proof of disjunction
  & Element of disjoint union
  & Point in disjoint union \\
Dep. fn. $f : \Pi(B)$
  & Proof of universal quantification
  & Dep. set valued function
  & Section of fibration \\
Dep. pair $(a, b) : \Sigma(B)$
  & Proof of existential quantification
  & Dependent product
  & Point in total space of fibration \\
Equality $p : a =_A b$
  & --
  & Equality in $A$
  & Path from $x$ to $y$ \\
Equivalence $f : A \simeq B$
  & Proof of biconditional
  & Bijection
  & Homotopy equivalence
\end{tabular}}
\caption{Three interpretations}\label{tbl:tt-hott}
\end{sidewaystable}

But what does equality, as defined here, entail?
The most important observation is, that two equal objects are \emph{indiscernable}
by any attribute in the sense that for any type family $C : A \to \UU$ if
$C(a)$ and $a = b$, then we can prove $C(b)$, as we can prove in the following
lemma which defines an operation to achieve this:
\begin{lemma}[Transport]
Let $A : \UU$ and $C : A \to \UU$.
For any two elements $a, b : A$ with $p : a = b$ we can define the
\textbf{transport} operation on $p$:
\begin{equation*}
p^* : C(a) \to C(b)
\end{equation*}
\end{lemma}

\begin{proof}
By induction, we can assume $p$ to be $\refl_a$, on which we can define
$(\refl_a)^*(c) :\equiv c$.
\end{proof}

While it is obvious that equality should be an equivalence relation,
we only assume a witness for reflexivity.
But what about its symmetry and transitivity?
It turns out that these can be proven by induction without the need
to add them as constructors:
\begin{lemma}
Let again be $A : \UU$, $a, b : A$, and $p : a = b$.
Then, there is an equality proof $p\inv : b = a$, called the \textbf{inverse}
of $p$.
\end{lemma}

\begin{proof}
Applying the eliminator to $p$, we can reduce this problem to finding a path
$(\refl_a)\inv : a = a$, which we can define to be $\refl_a$ itself.
\end{proof}

\begin{lemma}
For $A : \UU$, elements $a, b, c : A$ and paths $p : a = b$ and $q : b = c$,
there is a path $p \ct q : a = c$, proving that propositional equality is
transitive.
\end{lemma}

\begin{proof}
Here it suffices to apply induction on the second path and give the definition
\begin{equation*}
p \ct \refl_b :\equiv p \text{.}
\end{equation*}
\end{proof}

Having provided the proof that equality is an equivalence relation,
we can already conclude that it makes each type a \emph{setoid}.
But more than that we can also prove that it carries the structure of a
\emph{groupoid}:
\begin{lemma}[Groupoid laws]
Let $A : \UU$, $a, b, c, d : A$ and $p : a = b$, $q : b = c$ and $r : c = d$.
Then,
\begin{itemize}
\item $p = p \ct \refl_b = \refl_a \ct p$,
\item $p\inv \ct p = \refl_b$, $p \ct p\inv = \refl_a$,
\item $(p\inv)\inv = p$ and
\item $p \ct (q \ct r) = (p \ct q) \ct r$.
\end{itemize}
\end{lemma}

\begin{proof}
The first three laws can be proven by induction on the path $p$, the last one
by induction on $r$.
\end{proof}

As a suitable notion of equality, propositional equality is respected by
functions:
\begin{lemma}
Let $A, B : \UU$, $f : A \to B$, and $a, b : A$. Then, there is a function
$\ap_f : (a = b) \to (f(a) = f(b))$ such that $\ap_f(\refl_a) \equiv \refl_{f(a)}$.
\hfill $\square$
\end{lemma}

Under this notion $\ap_f$, every function is functorial with respect to
equality.
Besides this, it is also functorial with respect to function composition.
The following laws can be proved by induction on the paths involved:
\begin{lemma}
Let $A, B, C : \UU$, $f : A \to B$ and $g : B \to C$. For paths $p : a =_A b$ and
$q : b =_A c$ we have equalities
\begin{itemize}
\item $\ap_f(p \ct q) = \ap_f(p) \ct \ap_f(q)$,
\item $\ap_f(p\inv) = \ap_f(p)\inv$,
\item $\ap_g(\ap_f(p)) = \ap_{g \circ f}(p)$, and
\item $\ap_{\id_A}(p) = p$. \hfill $\square$
\end{itemize}
\end{lemma}

If in the above considerations $f$ is instead a dependent function, we can not
necessarily consider the type $f(a) = f(b)$ since the left and right hand side
need not have the same type.
For this situation, there is a dependent version of the above defined function
$\ap_f$:
\begin{lemma}\label{thm:apd-hott}
Let $A : \UU$, $B : A \to \UU$ and $f : (a : A) B(a)$. Then, we can construct
a dependent function
\begin{equation*}
\apd_f : \{a, b\}(p : a =_A b) \to p_*(f(a)) =_{B(b)} f(b) \text{,}
\end{equation*}
such that $\apd_f(a, a, \refl_a) \equiv \refl_{f(a)}$. \hfill $\square$
\end{lemma}

We continue by considering one of the representation of one of the most import
notions of homotopy theory:
Homotopies between functions.
\begin{defn}[Homotopy of functions]\label{def:tt-htpy}
Two maps $f, g : A \to B$ are called \textbf{homotopic} or \textbf{pointwise equal}
if for all $a : A$ we have
$f(a) = g(a)$.
We define the notation
\begin{equation*}
f \sim g :\equiv (a : A) \to f(a) = g(a) \text{.}
\end{equation*}
We can define the same for two dependent functions $f, g : \prod_{(a : A)} B(a)$
over the same type family.
\end{defn}

\begin{remark}[Function Extensionality]\label{rmk:tt-funext}
Since in homotopy theory, we can find a path $p : f = g$ in the space of
functions, whenever $f \sim g$,
we might ask if in homotopy type theory this holds as well --
are two functions equal as soon as they are pointwise equal?
In the setting of homotopy type theory we will be able to derive this from
univalence, while in settings where we assume the K-rule, we will usually
assume this property of the function space, which is called \textbf{function extensionality},
axiomatically.
\end{remark}

So far, we have not considered what it means for two \emph{types to be equal}.
Of course, our definition of equality is valid for the universe itself as well,
but how does it relate to how we represent
biconditionals of propositions,
bijections between sets,
and homotopy equivalences between spaces?
One common representation of all of these three concepts is the notion of
equivalence of types:
\begin{defn}[Equivalences]\label{def:tt-hae}
Let $A, B : \UU$. A function $f : A \to B$ is called an \textbf{equivalence}
between $A$ and $B$, if there is a $g : B \to A$ such that
$\eta : g \circ f \sim \id_A$ and $\epsilon : f \circ g \sim \id_B$ and furthermore
\begin{equation*}
\tau : \prod_{a : A} \ap_f(\eta(a)) =_{f(g(f(a)))=a} \epsilon(f(a)) 
 \equiv \ap_f \circ \eta \sim \epsilon \circ f \text{.}
\end{equation*}
We need $\tau$ to make sure that each two witnesses for the fact that $f$ is an equivalence are
equal.
Since $\tau$ is one of the two commutativity conditions for pairs
of adjoint functors, this kind of equivalence is also called a \textbf{half
adjoint equivalence}.
%The type of witnesses for $f$ to be an equivalence shall be
%\begin{equation*}
%\isequiv(f) :\equiv \sum_{g : B \to A} ~ \sum_{\eta : g \circ f \sim \id_A} ~
%\sum_{\epsilon : f \circ g \sim \id_B} \ap_f \circ \eta \sim \epsilon \circ f \text{.}
%\end{equation*}
The type of all equivalences between two types $A, B : \UU$ is denoted by
\begin{equation*}
A \simeq B :\equiv \sum_{f : A \to B} \isequiv(f) \text{.}
\end{equation*}
\end{defn}
It is easy to show that the equivalence of types is indeed an equivalence relation
and that it behaves as we expect it to on easy examples such as
$(\unit \to A) \simeq A$, $(\emptytype + A) \simeq A$, and
$(\emptytype \times A) \simeq \emptytype$.

But how can we now compare the equivalence and equality on types?
By induction the reflexivity proof for equivalence is enough to show that
equality implies equivalence:
For each $A, B : \UU$ there is
\begin{equation*}
\idtoeqv_{A, B} : (A =_{\UU} B) \to (A \simeq B) \text{.}
\end{equation*}
But how about the other direction?
It turns out that there is no way to obtain equality from equivalence,
so one solution
is to assume axiomatically, that $\idtoeqv_{A, B}$ has an inverse function.
The consistency of this axiom has been shown by \cite{kapulkinlumsdaine},
after an idea by Vladimir Voevodsky.
\begin{axiom}[Univalence]
For every $A, B: \UU$ we asume that
$\idtoeqv_{A, B}$ is itself an equivalence.
This implies that
\begin{equation*}
(A =_\UU B) \simeq (A \simeq B)
\end{equation*}
and yields an inverse to $\idtoeqv_{A, B}$ which we call
\begin{equation*}
\ua_{A, B} : (A \simeq B) \to (A =_\UU B) \text{.}
\end{equation*}
\end{axiom}

\begin{remark}
Postulating axioms is avoided as often as possible in type theory, since it is detrimental
to one of the desirable properties of the type theory: normalization.
To give an example, with an axiom like $\ua$, not every closed term of the natural numbers
can be reduced to a numeral any more.
It is for this reason that type theorists have been looking for ways to achieve univalence
without postulating it as an axiom.
One approach is to not consider equality as an inductive type any more but to have it as
a primitive in the language.
This approach is used in a variety of type theory called ``cubical type theory''
\citep{cohen2016cubical}.
\end{remark}

Another instance of equalities we have not explored so far are \emph{iterated equalities}:
Equalities between equality proofs.
Since for types $A : \UU$ representing propositions we might only care about whether
or not we have a proof for it, it might not make sense to
consider multiple elements, so we might want that for all $a, b : A$ we have $a = b$.
For types that should represent sets we \emph{do} want multiple elements, but
we might not be interested in having different equality proofs --
so for $a, b : A$ and $p, q : a =_A b$ we might want to have a proof for $p =_{a = b} q$.
This effect, that at a certain level of iteration, all equality proofs should become
equal is called \textbf{truncation}:
\begin{defn}[$n$-types]
For a type $A : \UU$ we set
\begin{align*}
\isntype{(-1)}{(A)} \equiv \isProp(A) &:\equiv (a, b : A) \to a = b \text{,} \\
\isntype{0}{(A)} \equiv \isSet(A)  &:\equiv (a, b : A)(p, q : a = b) \to p = q \text{, and} \\
\isntype{(n + 1)}{(A)} &:\equiv (a, b : A) \to \isntype{n}{(a = b)} \text{ for $n : \N$.}
\end{align*}
and call $A$ a \textbf{(mere) proposition} if $\isProp(A)$, a \textbf{set} if $\isSet(A)$ and more
general an
\textbf{$n$-type} or $n$-truncated if $\isntype{n}{(A)}$.
We also extend this definition to include what it means for a type to represent a
contractible space, a singleton, or a true proposition:
\begin{equation*}
\isntype{(-2)}{(A)} \equiv \isContr(A) :\equiv (a : A) \times ((b : B) \to a = b) \text{.}
\end{equation*}
\end{defn}

There are some important, but easy to prove facts about truncatedness:
\begin{lemma}
\begin{itemize}
\item If $A : \UU$ is an $n$-type, then $A$ is an $(n+1)$-type.
\item For each $n \geq -2$ and $A : \UU$, the type $\isntype{n}(A)$ is a mere
proposition.
\item If $A : \UU$ is an $n$-type and $B : A \to \UU$ such that for each $a : A$,
$B(a)$ is an $n$-type, then $(a : A) \times B(a)$ is an $n$-type as well.
\item If $A : \UU$ and $B : A \to \UU$ are such that $B(a)$ is an $n$-type for each
$a : A$, then $(a : A) \to B(a)$ is an $n$-type as well.
\item Let $n \geq -1$. Then, $A : \UU$ is an $(n+1)$-type if and only if for all $a : A$,
the equality type $a =_A a$ is an $n$-type.
\newpage
\item Let $n \geq 0$. Then, $A : \UU$ is an $n$-type, if and only if for all $a : A$,
the \emph{$n$-fold iterated loop space} $\Omega^{n+1}(A, a)$ is contractible.
The $n$-fold iterated loop space is defined by recursion on $n$ by
\begin{align*}
\Omega^1(A, a) &:\equiv (a =_A a) \text{ and}\\
\Omega^{n+1}(A, a) &:\equiv (\refl_{\ddots_a} =_{\Omega^n(A,a)} \refl_{\ddots_a}) \text{.}
\end{align*}
\hfill $\square$
\end{itemize}
\end{lemma}

\begin{remark}[Uniqueness of Identity Proofs]\label{rmk:tt-uip}
Truncation levels are a notion which behaves quite differently
depending on whether we assume presence of the K-rule or not.

If, as in homotopy type theory, we assume univalence and only the J-rule,
it is an easy exercise to find types which are not sets.
In fact the universe $\UU$ itself is not a set \citepalias[Example 3.1.9]{hottbook}:
We can construct
two different automorphisms of $\twotype$: The identity function
and the map $f : \twotype \to \twotype$ with $f(0_\twotype) :\equiv 1_\twotype$
and $f(1_\twotype) :\equiv 0_\twotype$.
It is easy to prove that these are non-equal and as such, using univalence,
yield different instances of the type $\twotype = \twotype$.

If on the other hand we \emph{do} assume the K-rule, then it is immediate
-- by first applying the J-rule to $p$ and then the K-rule to $q$ -- that
for every type $A$ we can use the rule to show the inhabitedness of the
type family
\begin{equation*}
(a, b : A) (p, q : a = b) \to p = q \text{,}
\end{equation*}
and so $A$ is a set.
\end{remark}


\section{Theorem Provers Based on Type Theory}\label{sec:tt-provers}

As remarked at the beginning of \Cref{chp:tt}, we value type theory not only
for its capability to represent mathematics and logics, but also for its good
computational behaviour.
This suggests to have a look at how type theory is implemented in different languages
the focus of which can be either to be a functional programming language based
on dependent types or a theorem proving tool, or to be useful in both of these cases.
With regards to this thesis, introducing different implementations is relevant
in two ways:
\begin{enumerate}
\item A good part of the new results in the form of definitions, lemmas,
and theorems, has been formalized in a one of these languages, and
\item both the higher Seifert-van Kampen theorem presented in~\Cref{sec:paths-svk}
as well as the attempted reduction from inductive-inductive type to inductive
families in ~\Cref{chp:red} suggest useful new features or extensions of implementations
of type theory.
\end{enumerate}
We will thus concentrate on the two languages we have used in this respect:
Lean~\citep{mouracade} and Agda~\citep{agda}.
This does not mean that the content discussed is irrelevant to other implementations
or that these are less important.
Au contraire -- the theorem prover Coq~\citep{coq} is not only the implementation
with the biggest user base but also has a lot of distinguishing features.

Before we delve into the differences between Lean and Agda, we will first
take a look at what these provers have \emph{in common}.
The following list contains both fundamental design principles as well as
several features which are important to improve the usability of the
implementation:
\begin{itemize}
\item Implementation of \emph{Martin-Löf type theory} with an unlimited chain of universes.
Construction can be made \emph{universe polymorphic} by the use of
universe variables which are used as variable indices for the unvierses.
\item Function types distinguish between explicit and \emph{implicit arguments}.
The latter are useful to make definitions more succinct.
\item Both languages are equipped with a system of \emph{modules} or \emph{name spaces}
used to organize definitions and their scope.
\item There is a way to customize the implementation by the means of \emph{options}.
This enables us to switch between a set-truncated setting and
homotopy type theory.
\item The implementations have an source code which is open and thus reviewable and
customizable by everybody.
\item There are a one or several text editors which support an interactive
and well-integrated use of the language.
\end{itemize}

Let us next concentrate on Lean as our first specimen.

\subsection{Lean}\label{sec:tt-lean}

The development of the theorem prover Lean was started by
Leonardo de Moura at Microsoft Research in 2013.
It aims to be both a useful tool for the formalization of mathematics and
the verification of programs written in other languages, as well as to
be a useful and performant programming language itself.
Another goal of Lean is to make automated theorem proving (such as the solution
for SMT-style problems) available in a dependently typed theorem prover.
As of now, Lean is in its third major version, after a lot of the system was
overhauled in the transition from Lean 2.
The fourth version is under development but not yet ready to be used as a prover.

Lean -- as its name suggests -- relies on a relatively small trusted kernel to which
the user code is compiled to.
This makes it easier to make claims about the consistency of the prover.
Let us first take a look at some of the specifics of Lean's type theory.

Lean is based on \emph{inductive families} as described by \citet{dybjer94}.
Whenever the user writes down the definition of such an inductive family,
Lean generates its constructors and its dependent eliminator and makes
it available to the user.
For example, the following snippet shows a definition of the natural numbers:
\begin{leancodebr}
inductive nat : Type 0
| zero : nat
| succ : nat → nat

#check nat.succ nat.zero -- Prints nat.succ nat.zero : nat
#check @nat.rec_on -- Π {C : nat → Type u} (n : nat),
                   --   C nat.zero →
                   --   (Π (a : nat), C a → C (nat.succ a)) → C n
\end{leancodebr}

A widely used subclass of inductive types are \textbf{structures}, which are
similar to what is often referred to as ``records''.
Structures are inductive types which are non-recursive and only have one constructor.
That means that they are equivalent to iterated sigma types but, among other advantages,
have named projections.
Structures provide a basic inheritance mechanism as they can extend other structures:
\begin{leancode}
structure graph (V : Type) :=
  (E : V → V → Type)

structure refl_graph (V : Type) extends graph V :=
  (refl : Π (v : V), E v v)

structure trans_graph (V : Type) extends graph V :=
  (trans : Π (u v w : V), E u v → E v w → E u w)
\end{leancode}

Lean features a strict and impredicative universe \leani{Prop} of propositions.
Since this universe is incompatible with homotopy type theory, we have to be
careful to exclude its use whenever we want formalizations to hold in
a setting of homotopy type theory instead of truncated type theory.
\emph{Strict} here means that if we have a proposition \leani{p : Prop} and two
proofs \leani{a, b : P}, these are considered judgmentally equal, thus for
every type family \leani{P → Type}, the fibers \leani{P a} and \leani{P b}
are judgmentally equal as well.
\emph{Impredicative} means, that for every type family valued in \leani{Prop},
its universal quantification \leani{∀ a, P a} is in \leani{Prop} as well, even
if \leani{A} is not.

Lean makes heavy use of \emph{tactics}.
These are commands which can modify a given proof goal or a series of proof
goals.
They are accesible once the user has switched from a mode where she can
enter literal proof terms to a \emph{tactic-mode},
which allows her to give proofs and definitions as a sequence of tactic applications,
which are then desugared as the do-notation of a \emph{tactic monad}.
Here is an example for a tactic proof in lean, using a simplification tactic,
a rewriting tactic, and a tactic calling previously defined theorems:
\begin{leancodebr}
example : ∀ (n : ℕ), n = 0 ∨ n > 0 :=
begin
  intro n,
  induction n with n' IH,
  simp,
  right,
  cases IH,
  rw IH, constructor,
  apply nat.zero_lt_succ
end
\end{leancodebr}

Lean has its own \emph{meta-language} which allows the user to define new
language features like commands and tactics.
It does so by providing a way to reflect arbitrary terms into an inductive \emph{type of
expressions} \leani{expr},
such that by recursion on this type, structural recursion on the term can 
be emulated.

Lean has several big libraries, the biggest one, which provides basic data types,
tactics, and formalization in different fields of mathematics, is called
\emph{Mathlib} (\url{https://github.com/leanprover-community/mathlib}).
It is not recommended to use Lean without using Mathlib.
Another big formalization effort it Lean's homotopy type theory library,
the development of which has been described by \citet{leanhott}.
It was originally written in Lean 2 and then ported to Lean 3.
A good introduction to the language can be found online \citep{avigad2015theorem}.

\subsection{Agda}

Agda is a theorem prover. Its first version was written by Catarina Coquand, while
the development of the current major version, Agda 2, was initiated by
Andreas Abel and Ulf Norell in 2005.
There are biannual ``Agda Implementors Meetings'' to discuss new ideas on how to
improve Agda.

Compared to Lean, Agda implements a wider range of inductive types,
also allowing \emph{inductive-inductive types}.
In contrast to Lean it does not automatically produce the eliminator for an
inductive definition but instead provides an induction principle based
on \emph{dependent pattern matching}:
The user can write (dependent) functions with an inductive domain, by pattern
matching on its input.
In the following example, an indexed type of vectors and its head function are
defined.
Agda's pattern matching algorithm recognizes that it can exclude the case of the first
constructor in the definition of the head function.
\begin{agdacode}
data Vec (A : Set) : N → Set where
  nil : Vec A Z
  cons : ∀{n} → A → Vec A n → Vec A (S n)

hd : ∀{A n} → Vec A (S n) → A
hd (cons a v) = a
\end{agdacode}

Agda has several options which make it possible to use it with a range of different
type theories:
Homotopy type theory is supported by declaring via an option \agdai{--without-K}
that it should reject any use of the K-rule.
Agda also has an option \agdai{--rewriting} whic makes it possible to declare
many equalities as strict by declaring their proofs to be \emph{rewrite rules}
\citep{cockxrewrite}, giving the type theory the flavour of extensional type
theory.

Agda has also been modified to support a type theory which provides
a constructive, non-axiomatic version of univalence.
Based on the geometry of its identity types, this type theory is called
\emph{cubical type theory}, and the modified version has the name
``Cubical Agda'' \citep{andrea:cubicalagda}.

%TODO
% ** ttintt
% 










\chapter{Higher Inductive Types}\label{chp:hit}

\section{Examples of Higher Inductive Types}

The propositional equality discussed in the last chapter works just fine if we want
to \emph{prove} things to be equal.
But in mathematics as well as computer science, we also often want to
\emph{make things equal}, in the sense that we might want to consider a type
$A$ and two of its elements $a, b : A$ and want to obtain
another type which only differs from $A$ in that $a$ and $b$ are equal.
In short, we want to take \emph{a quotient}.
In this chapter we will present different ways to achieve quotients in the setting
of homotopy type theory.
Afterwards we will show how we can derive all of these from the basic notion
of a \emph{homotopy coequalizer}.

The general way to achieve quotients is to go beyond the inductive types
which we encountered in the last chapters -- and which were all examples of
indexed W-types --
and also allow for constructors which,
in contrast to the \emph{point constructors} which we have seen so far,
are \emph{path constructors}.
Instead of adding new elements to the inductive type,
these constructors are there to make instances of other constructors equal.
In a setting where the K-rule is present, and every type is a set (see Remark~\ref{rmk:tt-uip}),
this bigger class of inductive types is called \textbf{quotient inductive types},
while, as we will see, in homotopy type theory it is more
fitting to call them \textbf{higher inductive types} (HITs) given the fact
that equality types carry not only proofs but data.
Higher inductive types
allow the development of a synthetic version of homotopy theory inside HoTT (cf.~\cite{Buchholtz2018,Buchholtz2018CellularCI,buchholtz2016cayley,BuchRijke_projectiveSpaces,favonia:SvK,licataFinster_Eilenberg,licataBrunerie_s1again,Brunerie2017,rijke:join}).
A main objective of this line of research is to describe, classify, and compare
path spaces (i.\,e.\ equality types) or homotopy groups (i.\,e.\ truncated path spaces)
of higher inductive types such as circles and spheres.

A minimal example for a higher inductive type could be the following definition of would be a
type theoretic representation of the topological space of the interval $I$.
Its constructors are two points, but furthermore also a path which
connects these two points:
\begin{equation*}
\begin{gathered}
%\inferrule*[left=$I$-Form]{ }{I : \UU} \qquad
\inferrule*[left=$I$-Intro1]{ }{0_I : I} \qquad
\inferrule*[left=$I$-Intro2]{ }{1_I : I} \qquad
\inferrule*[left=$I$-Intro3]{ }{\seg : 0_I = 1_I} \\[.7em]
\inferrule*[left=$I$-Elim]
	{C : I \to \UU \\ c_0 : C(0_I) \\ c_1 : C(1_I) \\ p : \seg_*(c_0) = c_1 \\ x : I}
	{\elim_I(C, c_0, c_1, p, x) : C(x)}
\end{gathered}
\end{equation*}
It is easy to check that the unique map $I \to \unit$ is an equivalence,
and so $I$ is contractible and thus a set.
But what if instead of two point constructors we only had one,
as in the higher inductive types goverened by the following rules?
\begin{equation*}
\begin{gathered}
%\inferrule*[left=$\Sph^1$-Form]{ }{\Sph^1 : \UU} \qquad
\inferrule*[left=$\Sph^1$-Intro1]{ }{\Sbase : \Sph^1} \qquad
\inferrule*[left=$\Sph^1$-Intro2]{ }{\Sloop : \Sbase =_{\Sph^1} \Sbase} \\[.7em]
\inferrule*[left=$\Sph^1$-Elim]
	{C : \Sph^1 \to \UU \\ c : C(\Sbase) \\ p : \Sloop^*(c) = c \\ x : \Sph^1}
	{\elim_{\Sph^1}(C,c,p,x) : C(x)}
\end{gathered}
\end{equation*}
We can see that $\Sloop$ introduces a new path from $\Sbase$ to itself which
cannot be reduced to $\refl_\Sbase$.
This means that we can interpret it as a loop which is not homotopic
to the identity, and so the type represents the topological space of the circle
(cf. \Cref{fig:hit-circle}).
The fact that $\Sloop$ is not equal to $\refl_\Sbase$ also makes it clear that
this type is not a set.

In the same way in which we can add arbitrary paths between constructors,
we can also use iterated equality types to express the addition
of arbitrary higher dimensional cells (surfaces, volumes).
An example for this is the definition of a twodimensional sphere, where we have
one basepoint and one surface:
\begin{equation*}
\begin{gathered}
%\inferrule*[left=$\Sph^1$-Form]{ }{\Sph^1 : \UU} \qquad
\inferrule*[left=$\Sph^2$-Intro1]{ }{\Sbase : \Sph^2} \qquad
\inferrule*[left=$\Sph^2$-Intro2]{ }{\Ssurf : \refl_{\Sbase} =_{\Sbase = \Sbase} \refl_{\Sbase}} \\[.7em]
%\inferrule*[left=$\Sph^2$-Elim]{ TODO?
%	{C : \Sph^2 \to \UU \\ c : C(\Sbase) \\ p : \mathsf{}(c) = c \\ x : \Sph^1}
%	{\elim_{\Sph^1}(C,c,p,x) : C(x)}
\end{gathered}
\end{equation*}

\begin{figure}
\centering
\begin{tikzpicture}[x=\diagx, y=\diagx]
 \draw[black] (.5,0) circle (.5);
 \draw[black,fill=black] (0,0) circle (.6ex);
 \node at (-.3,.0) {$\mathsf{base}$};
 \node at (1.2,.15) {$\mathsf{loop}$};
\end{tikzpicture}
\caption{A circle with a base point.}\label{fig:hit-circle}
\end{figure}

Besides these closed examples for higher inductive types, we can also
have important constructions which are parametrized over an arbitrary type.
One important operation in topology, especially in the field of homologies,
is the one of the \textbf{suspension} $\Susp(A)$ of spaces $A$ which turns an $n$-dimensional
type into a $(n+1)$-dimensional type.
Suspensions will also provide a way to conveniently define all higher-dimensional
spheres by setting $\Sph^{n + 1} :\equiv \Susp(\Sph^n)$.
The suspension has two point constructors (sometimes called the north and
south pole) and for each point in $A$ a path between those points:
\begin{equation*}
\begin{gathered}
\inferrule{ }
  {\SuspN : \Susp(A)} \qquad
\inferrule{ }
  {\SuspS : \Susp(A)} \qquad
\inferrule{a : A}
  {\Suspmerid(a) : \SuspN = \SuspS} \\[.7em]
\inferrule{C : \Susp(A) \to \UU \\
  c_\SuspN : C(\SuspN) \\
  c_\SuspS : C(\SuspS) \\
  c_\Suspmerid : (a : A) \to \Suspmerid(a)^*(\SuspN) = \SuspS}
  {\elim_{\Susp(A)}(C, c_\SuspN, c_\SuspS, c_\Suspmerid) : (x : \Susp(A)) \to C(x) }
\end{gathered}
\end{equation*}

Another very general construction is the \textbf{pushout of types}.
It does not only depend on one a single
type but instead has as input a whole \emph{span} of types, meaning
three types $L$, $M$, and $N$, and functions $f : L \to M$ and $g : L \to N$.
It consists of a type $P \equiv \pout{L}{M}{N}$ the point constructors of which are to functions
$\inl$ and $\inr$ as in the following diagram:
\begin{center}
   \begin{tikzpicture}[x=\diagx,y=-\diagy]
   \node (A) at (0,0) {$L$};
   \node (C) at (1,0) {$N$};
   \node (B) at (0,1) {$M$};
   \node (D) at (1,1) {$P$}; %{$\pout L M N$};
  
   \draw[->] (A) to node [left] {$f$} (B);
   \draw[->] (A) to node [above] {$ g$} (C);
   \draw[->, dashed] (B) to node [below] {$\inl$} (D);
   \draw[->, dashed] (C) to node [right] {$\inr$} (D);
  \end{tikzpicture}
\end{center}
The introduction rules are the same as for the sum type, but each instance  of $L$
contributes a new path in the resulting type:
\begin{equation*}
\begin{gathered}
\inferrule{m : M}
  {\inl(m) : \pout{L}{M}{N}} \qquad
\inferrule{n : N}
  {\inr(n) : \pout{L}{M}{N}} \\[.7em]
\inferrule{l : L}
  {\glue(l) : \inl(f(l)) = \inr(g(l)) }
\end{gathered}
\end{equation*}
One can try to visualize the resulting type as the sum of $M$ and $N$ which
was glued along an $L$-shaped overlapping.
The pushout is a very general construction.
In fact, it is easy to check that all the previous higher inductive types we
presented were just special cases of a pushout,
for example the suspension $\Susp(A)$ can also be defined
as the pushout where both $f$ and $g$ are the unique map $A \to \unit$.

An example for a higher inductive types, the importance of which is easily recognizable
even we only care for propositions and sets is the one of the \textbf{truncation operator}.
It is a tool to bring types to the desired level of truncation,
as we can see in the following example:
Remember that we represented the disjunctive logical connective with
the sum type.
But the sum of two propositional types is not necessarily a proposition itself:
For example we have $\unit + \unit \simeq \twotype$.
So how can we make it a proposition?
The solution is to add equations between \emph{all} the points in the sum type
and additionaly make sure that no trivial higher equality proofs exist.
Similar situations can happen for every truncation level.
For example, a collection of sets is generelly not a set itself.
The elimination principle for this operation has to make sure that only
functions on the truncated type, which represent continuous maps, can
be created, and so it has to require that the fibers of the type family which
we want to inhabit are truncated:
\begin{equation*}
\begin{gathered}
\inferrule*[left=Trunc-Intro1]{a : A}
  {\tproj{n}{a}{} : \trunc{n}{A}}
\qquad
\inferrule*[left=Trunc-Intro2]{	}
  {\isntype{n}{\left(\trunc{n}{A}\right)}} \\[.7em]
\inferrule*[left=Trunc-Elim]{C : \trunc{n}{A} \to \UU \\
  p : (x : \trunc{n}{A}) \to \isntype{n}{(C(x))} \\
  q : (a : A) \to C(\tproj{n}{a}{}) }
  {\elim_{\trunc{n}{A}}(C,p,q) : (x : \trunc{n}{A}) \to C(x) }
\end{gathered}
\end{equation*}

Next, we will discuss how -- parallel to how we unified the examples for
indexed inductive types using indexed W-types --
find a general way to express all higher inductive types which we need.

\section{Coequalizers as a Fundamental HIT}

For a long time after homotopy type theory became its own field,
the question about what could be a suitable syntax and sematics for
higher inductive types remained open.
To make up for the lack of core support for higher inductive types
in interactive theorem provers, users of Coq, Agda, and Lean decided to come
up with pragmatic definitions and implementations.
In Agda and Coq, the common solution was to apply a trick first documented by
\cite{licatatrick} which uses \emph{private inductive types} to
hide the elimination principle generated by the prover and replace it by
a manually defined one.
Another strategy which the homotopy type theory formalizations in Lean 2 are
based on, and which are described by \cite{leanhott},
is to find a single higher inductive types which can serve as a \emph{universal}
example which can be used to derive a big range of other instances.
This is similar to the generalized type of trees which make up indexed W-types.

The type we use as such a fundamental higher inductive type is much simpler
than the definition of indexed W-types, and it is a straightforward
generalization of \textbf{quotients} in set-based type theory.
Suppose that we have a type $A : \UU$ and and equivalence relation
$\sim : A \to A \to \UU$.
For a type to be justifiably called the quotient of $A$ by $\sim$,
we want a projection map from $A$ into the quotient and we need to make
sure that any $a, b : A$ with $a \sim b$ are projected to equal elements
of the quotient.
An elimination principle should say that maps out of the quotient are
the same as makes out of $A$ respecting the relation $\sim$.
And these are exactly the features of the type of \textbf{homotopy coequalizers}
which we will now introduce.
The reason why we refrain from calling them quotients
-- even though \cite{leanhott} call them \emph{typal} quotients --
will become apparent in the examples we will study after giving the inference rules
and is due to the fact that we differ from several common assumptions about
quotients:
\begin{itemize}
\item The base type $A : \UU$ does not need to be a set,
\item the relation $\sim : A \to A \to \UU$ does not need to be an equivalence relation, and
\item it does not have to be a relation at all, because for $a, b : A$, the type of relatedness witnesses $a \sim b : \UU$
does not need to be a proposition but can contain multiple elements and even
structure in higher equalities.
(It might be reasonable to speak of a quotient by a higher relation,
which is the approach taken by
\cite{boulierRijkeTab_higherRels}.)
\end{itemize}
The formation, introduction, and elimination rules for the coequalizer
are the following:
\begin{equation*}
\begin{gathered}
\inferrule*[left=Coeq-Form]{A : \UU \\ \_\sim\_ : A \to A \to \UU}
  {\quot : \UU}\\[.7em]
\inferrule*[left=Coeq-Intro1]{a : A}
  {[a] : \quot} \qquad
\inferrule*[left=Coeq-Intro2]{a, b : A \\ s : a \sim b}
  {\glue(s) : [a] = [b] }\\[.7em]
\inferrule*[left=Coeq-Elim]{P : \quot \to \UU \\
  f : (a : A) \to P([a]) \\
  e : \{a, b : A\} (s : a \sim b) \to \glue(s)^*(f(a)) = f(b)  }
  {\elim_\quot(P,f,e) : (a : \quot) \to P(x)}
\end{gathered}
\end{equation*}

\begin{remark}[Pathovers]
Since we will often encounter equality types with a transported term on one
side, it merits its own name and notation:
Whenever we have a type family $B : A \to \UU$, two points $a, a' : A$,
and a path $p : a = a'$, we will for $b : B(a)$ and $b' : B(a')$ define the
type of
\textbf{paths over} or \textbf{dependent paths over} $p$ to be
\begin{equation*}
(b =_p b') :\equiv (p*(b) = b') \text{.}
\end{equation*}
These path-overs have proven to be useful for many formalizations and their
attributes, as well as their generalizations to square-overs and
cube-overs have been described by \cite{licatacubical}.
\end{remark}

\begin{remark}[Coequalizer of Functions]
In category theory, coequalizers are not defined on an object and a relation
but instead are a universal construction on two functions $f, g : X \to A$.
It can be thought as an object in which for each $x : X$, $f(x)$ and $g(x)$ are identified.
This can be expressed in our variant of a coequalizers by setting the relation
on $A$ to be
\begin{equation*}
a \sim b :\equiv (x : X) \times (f(x) = a) \times (g(x) = b) \text{,}
\end{equation*}
or, alternatively, by defining $\_\sim\_$ inductively by $f(x) \sim g(x)$ for each
$x : X$.
It is then easy to see that the coequalizer of $f$ and $g$ is $\quot$.

Vice versa, we can view $\quot$ for an arbitrary $A : \UU$ and
$\sim : A \to A \to \UU$ as the coequalizer on the maps
\begin{equation*}
 \pr_1,\pr_2 : \left(\Sigma(a,b : A).a \sim b\right) \rightrightarrows A \text{.}
\end{equation*}
\end{remark}

Another reason why we might avoid to conflate homotopy coequalizers with
ordinary quotients is, that $\quot$ is not always a set, even if $A$ is:
We can express the circle which we have defined earlier as a coequalizer
in the following simple way:
Set $A :\equiv \unit$ and for $a , b : A$, set $a \sim b :\equiv \unit$.
While this, at first glance, might seem to yield a trivial coequalizer,
it actually add an \emph{additional} path to the unit type, and
it is easy to check that $\quot \simeq \Sph^1$.

To see the importance of not requiring the relation to be propositional,
consider the case of the pushout.
It can be written as a coequalizer on $A :\equiv M + N$, but it is easy that the
following definition for $\_\sim\_$ is not necessarily a proposition,
and in general it is neither reflexive, nor symmetric, nor transitive:
\begin{align*}
(\inl(m) \sim \inl(m')) &:\equiv \emptytype \text{,} \\
(\inl(m) \sim \inr(n)) &:\equiv (l : L) \times (f(l) = m) \times (g(l) = n) \text{, and} \\
(\inr(m) \sim x) &:\equiv  \emptytype \text{.}
\end{align*}

With the pushouts (and sequential colimits which are defined similarly),
and given that there are more intricate constructions giving
propositional truncations \citep{floris_proptrunc,kraustrunc} and higher truncations~\citep{rijke:join},
we have presented ways to encode all the higher inductive types
which we have seen so far as coequalizers.

\begin{remark}[Coequalizers as Pushouts]
Basing all necessary higher inductive type is a somewhat arbitrary decision
based on aesthetical considerations, since we can also derive coequalizers
from pushouts in the following way:
\begin{center}
   \begin{tikzpicture}[x=\diagx,y=-\diagy]
   \node (A) at (0,0) {$A + (a, b : A) \times a \sim b$};
   \node (C) at (2,0) {$A$};
   \node (B) at (0,1) {$A$}; %TODO
   \node (D) at (2,1) {$P$}; %{$\pout L M N$};
  
   \draw[->] (A) to node [left] {$f$} (B);
   \draw[->] (A) to node [above] {$ g$} (C);
   \draw[->, dashed] (B) to node [below] {$\inl$} (D);
   \draw[->, dashed] (C) to node [right] {$\inr$} (D);
  \end{tikzpicture}
\end{center}
with
\begin{align*}
f(\inl(a)) &:\equiv a \text{,} \\
f(\inr(a, b, s)) &:\equiv a \text{,} \\
g(\inl(a)) &:\equiv a \text{, and} \\
g(\inr(a, b, s)) &:\equiv b \text{.}
\end{align*}
It is then easy to check that the pushout $P$ is equivalent to $\quot$.
As a result, we can conclude that pushouts and homotopy coequalizers
are \emph{interderivable} in homotopy type theory.
\end{remark}

\section{Encode-Decode Proofs}\label{sec:hit-encode-decode}

To motivate our results, which we will present in the next chapter,
we will now look at some important problems which occur when
we want to prove facts about higher inductive types.
Often, we want to find out what specific equality types of
higher inductive types look like.
For a very concrete example, one of the most basic results in homotopy theory
is the proof of the statement that \emph{the fundamental group of the circle is
isomorphic to the group $\Z$ of integers}.
It is an example of a proplem which in a classical, set theoretic setting,
is solved in a way which is very different from the approach which is needed
in its type theoretic counterpart.
\cite{licataShulman_circle} were the first to translate this fact into the
setting of homotopy type theory.
Another fundamental theorem in homotopy theory is the \emph{theorem of
Seifert-van Kampen}.
It provides proof that the fundamental groupoid of a pushout of two spaces is
isomorphic to the pushout of the fundamental groupoid of these spaces.
This is an important help when trying to calculate homotopy groups of spaces
which were ``glued'' together from more primitive spaces using pushouts.
\cite{favonia:SvK} documented and formalized in Agda a type theoretic
equivalent of the Seifert-van Kampen thoerem.
In this chapter, we will revisit the proof ideas for both of these results
and discover, what they have in common.

The intuitive way to compare the integers with the loop space
$\Omega(\Sph^1)$ of the sphere is the following:
Each loop can be reached by following the constructor $\Sloop$ a number of
times either in its actual direction or by reversing the direction, i.\,e.\
concatenating with $\Sloop \inv$.
Following an inverted loop after a non-inverted one or vice versa cancels out since
we have
\begin{equation*}
\Sloop \ct \Sloop \inv = \refl = \Sloop \inv \ct \Sloop \text{.}
\end{equation*}
The integer corresponding to a loop would then be the ``exponent'' of the number
of times we concatenate $\Sloop$ (with $\Sloop^0 = \refl$).
But translating this intuition into a formal proof, especially in order
to define a function from the loop space to the integers, requires some tricks.
\begin{thm}[Loop Space of the Circle, \cite{licataShulman_circle}]\label{thm:hit-s1}
The loop space of the circle is equivalent to the
type of the integers: $\Omega(\Sph^1, \Sbase) \simeq \Z$
\end{thm}

\begin{proof}
Instead of directly defining a map $\Omega(\Sph^1, \Sbase) \to \Z$,
we perform the type theoretic equivalent of constructing the
\emph{universal cover} for the circle.
Over each point of the circle, we will have a whole type, i.\,e. a type family
$\code : \Sph^1 \to \UU$
and by the end we will be able to prove that this type family satisfies the
following equivalences:
\begin{equation}\label{eq:hit1}
(\Sbase = x) \simeq \code(x) \simeq \Z \text{,}
\end{equation}
after which we can get the desired result by plugging in $x \equiv \Sbase$.
The definition of this type family is determined by the following two equations:
\begin{align*}
\code(\Sbase) &\equiv \Z \\
\ap_\code(\Sloop) &= \ua(S) \text{,}
\end{align*}
with $S : \Z \to \Z$ being the successor function on the integers, which
is obviously an equivalence.
Formally speaking, we can achieve this by setting
\begin{equation*}
\code :\equiv \elim_{\Sph^1}(\lambda \_.\UU, \Z, \ua(S)) \text{.}
\end{equation*}
We can then calculate that
\begin{equation}\label{eq:hit2}
\Sloop^*(z) \equiv (\ap_\code(\Sloop))^*(z) \equiv (\ua(S))^*(z) \equiv S(z) \text{,}
\end{equation}
and that, by the same reasoning, $(\Sloop\inv)^*(z) = P(z)$ where
$P = S\inv$ is the predecessor function on $\Z$.
After we gave the definition of $\code$, the second equivalence of \eqref{eq:hit1}
is straightforward to prove by induction on $x$, while the first one is more involved.
We call the map into $\code(x)$, which we have to define,
the \emph{encoding} map and the inverse the \emph{decoding}.

The encoding works by, for each path, transporting along this type in the type
family $\code$, starting from zero:
\begin{align*}
\mathsf{encode} &: \{x : \Sph^1\}(p : \Sbase = x) \to \code(x) \\
\mathsf{encode}(p) &:\equiv p^*(0)
\end{align*}
This is the same as defining $\mathsf{encode}$ as $0$ induction on $p$.
It is easy to check via \eqref{eq:hit2} that $\mathsf{encode}(\Sloop)$
calculates to $S(z)$, $\mathsf{encode}(\Sloop \ct \Sloop)$ to
$S(S(z))$ etc.

Defining the decoding function 
$\mathsf{decode}(x) : \code(x) \to (\Sbase = x)$
goes by induction on the circle element
$x$ involved.
For the induction we first have to give the base case
$\mathsf{decode}(\Sbase) : \code(\Sbase) \to (\Sbase = \Sbase)$ which we set to the function which is
constantly $\Sloop$.
We then have to prove that
\begin{equation*}
\Sloop^*(\lambda \_.\, \Sloop) = (\lambda\_.\, \Sloop) \text{,}
\end{equation*}
where the transport is over the type family $\lambda x'.\, \code(x') \to (\Sbase = x')$,
a calculation we don't want to detail here. This defines $\mathsf{decode}$.
It is then left to prove by induction that
\begin{align*}
\mathsf{encode}(\mathsf{decode})(c) &= c \text{ for all $c : \code(x)$ and} \\
\mathsf{decode}(\mathsf{encode})(p) &= p \text{ for all $x : \Sph^1$ and $p : \Sbase = x$.}
\end{align*}
completing the proof of the theorem.
\end{proof}

The proof for the Seifert-van Kampen theorem employs the same method of
providing a type family acting as a universal cover.
It can be seen as a generalization of the above theorem.
Recall that in a classical topological setting the statement of the
theorem is the following:
\begin{quote}
The fundamental groupoid of the pushout of two spaces is equivalent to the pushout of
its fundamental groupoids.
\end{quote}
So while the for the theorem about the circle the ``right-hand side'' of the
equivalence was rather clearly defined, here the
question is whether we find a good definition of the pushout of groupoids.
Since we attempt another example of an encode-decode proof, we
might as well first define the morphisms of the pushout of groupoids as the
code itself.
Intuitively, the morphisms the pushout $P :\equiv \pout{L}{M}{N}$
are ``zig-zags'' consisting of morphisms
alternatingly in ${M}$ and ${N}$, separated by morphisms in $L$.
\begin{defn}[Codes for Groupoid Pushouts, \cite{favonia:SvK}]
We define a type family $\code : P \to P \to \UU$, by recursion on both arguments.
That means that it suffices to give families $M \to M \to \UU$,
$M \to N \to \UU$, $N \to M \to \UU$, and $N \to N \to \UU$, which agree on
$f(l)$ and $g(l)$ for all $l : L$ in either argument.
We will omit proof of this consistency.

For the first case, i.\,e.\ if we have $m, m' : N$, we want $\code(\inr(m), \inr(m'))$ to be
the inductive type where elements are characterized as lists
\begin{equation*}
 (p_0, x_1, q_1, y_1, p_1, x_2, q_2, y_2, p_2, \ldots, y_k, p_k) \text{,}
\end{equation*}
the entries of which (cf. \Cref{fig:hit-svk-zigzag}) have the following type for
each $i : \N$ with $0 < i \leq k$:
\begin{align*}
x_i, y_i &: L \\
p_0 &: \trunc{0}{m = f(x_1)} \\
p_i &: \begin{cases} \trunc{0}{f(y_i) = f(x_{k + 1})} & \text{for $i < k$} \\
  \trunc{0}{f(y_k) = m'} & \text{for $i = k$}  \end{cases} \\
q_i &: \trunc{0}{g(x_k) = g(y_k)} \text{.}
\end{align*}
These lists have to be quotiented by the path constructors
\begin{align*}
(\ldots, q_i, y_i, \refl_{f(y_i)}, y_i, q_{i+1}, \ldots)
  &= (\ldots, q_i \ct q_{i+1}, \ldots) \text{ and} \\
(\ldots, p_i, x_i, \refl_{g(x_i)}, x_i, p_{i+1}, \ldots)
  &= (\ldots, p_i \ct p_{i+1}, \ldots) \text{,}
\end{align*}
and the resulting type is made a set by the means of truncation.
The type families for the three remaining cases are likewise defined
``zig-zags'' in the form of quotiented lists, where the parity of the length
is changed and/or the position of $p_i$ and $q_i$ are swapped.
\end{defn}

\begin{figure}
 \centering
  \begin{tikzpicture}[x=\diagx,y=-\diagy]
   \node (A) at (0,0) {$m$};
   \node (B) at (1,0) {$g(x_1)$};
   \node (C) at (1,1) {$f(x_1)$};
   \node (D) at (2,1) {$f(y_1)$};
   \node (E) at (2,0) {$g(y_1)$};
   \node (F) at (3,0) {$g(x_2)$};
   \node (G) at (3,1) {$f(x_2)$};
   \node (H) at (4,1) {$f(y_2)$};
   \node (I) at (4,0) {$g(y_2)$};
   \node (J) at (5,0) {$m'$};
  
   \draw[->] (A) to node [above] {$p_0$} (B);
%    \draw[->] (B) to node [left] {$\scriptstyle {\glue(x_1)}$} (C);
   \draw[->] (B) to node {} (C);
   \draw[->] (C) to node [above] {$q_1$} (D);
%    \draw[->] (D) to node [left] {$\scriptstyle {\glue(y_1)^{-1}}$} (E);
   \draw[->] (D) to node {} (E);
   \draw[->] (E) to node [above] {$p_1$} (F);
%   \draw[->] (F) to node [left] {$\t {\glue(x_2)}$} (G);
   \draw[->] (F) to node {} (G);
   \draw[->] (G) to node [above] {$q_2$} (H);
%    \draw[->] (H) to node [left] {$\scriptstyle {\glue(y_2)^{-1}}$} (I);
   \draw[->] (H) to node {} (I);
   \draw[->] (I) to node [above] {$p_2$} (J);
  \end{tikzpicture}
\caption{``Zig-zag paths'' in the pushout codes, vertical arrows correspond
to applications of $\glue$, horizontal arrows to applications of
$\ap_\inl$ and $\ap_\inr$.}\label{fig:hit-svk-zigzag}
\end{figure}

With this definition, the statement of the theorem can be stated as the following:
\begin{thm}[Seifert-van Kampen, \cite{favonia:SvK}]\label{thm:paths-svk}
For each $x, y : P$ we have
\begin{equation*}
\trunc{0}{x = y} \simeq \code(x, y) \text{.}
\end{equation*}
\end{thm}

\begin{proof}
Again, we define functions $\mathsf{encode}$ and $\mathsf{decode}$ constituting
the equivalence.
The strategy is generally the same as for the characterization of equalities
in the circle:
The encoding function
\begin{equation*}
\mathsf{encode} : \{u, v : P\} \to \trunc{0}{u = v} \to \code(u, v)
\end{equation*}
can be defined by induction on the truncated type (since we truncated $\code(u,v)$)
then on the path $p : u = v$ involved, and setting the function to be the trivial
element in the respective type of lists:
\begin{equation*}
\mathsf{encode}(\tproj{}{\refl}) :\equiv (\refl) \text{.}
\end{equation*}
This is the same as transporting $p$ along the type family $\lambda v.\, \code(u,v)$.
The decoding function has to take a list and ``glue together'' the paths in the
pushout which it represents, as for example
\begin{align*}
 & \mathsf{decode} (p_0, x_1, q_1, y_1, p_1, \ldots) \\
:\equiv & \ap_\inl(p_0) \ct \glue(x_1) \ct \ap_\inr(q_1) \ct \glue(y_1)\inv \ct \ap_\inl(p_1) \ct \ldots
\end{align*}
The proof that these functions are inverse to each other is again done by induction.
\end{proof}

%TODO by thorsten: explain how this relates to the classical statement
\begin{remark}[The classical Seifert-van Kampen Theorem]
How does the statement of \Cref{thm:paths-svk} relate to the classical
version of the theorem?
The left side of the equivalence clearly correponds to morphisms between
$x$ and $y$ in the fundamental groupoid (i.\,e. the equalities) of the pushout
of the spaces, but what about the right-hand side?
We can show that $\code(x,y)$ is not some arbitrary construction but that
its definition as a quotient of lists can be straightforwardly generalized
to a characterization of morphisms in the pushout of any two groupoids.
\end{remark}










\chapter{Path Spaces of Higher Inductive Types}\label{chp:paths}
\chaptermark{Path Spaces of HITs}

We have seen in the previous chapter that the encode-decode method can be used
in a variety of cases when we want to make statements about the equality
types -- or \emph{path spaces} -- of higher inductive types.
Going through all necessary steps of such a proof can be somewhat tedious, but
part of it is very mechanical work.
One main goal of this chapter is to present a different method to
directly work with equality types of coequalizers and pushouts
(and construction based on these):
Since elimination rules such as one for coequalizers characterize only
the points  of the type, but in the constructors we create points
and equalities simultaeously, we believe that it is natural to hope for
an ``induction principle for equalities'' which is reminiscent of an elimination
rule.
More concretely, for our case of a coequalizer $\quot : \UU$ of a type $A : \UU$
and typal relation $\_\sim\_ : A \to A \to \UU$,
let us assume we are given a type family
\begin{equation*}
Q: \{a, b : A\} \to [a] = [b] \to \UU \text{.}
\end{equation*}
Is it possible to have simple-to-check conditions which are sufficient to
conclude $Q(q)$ for a general $q$ (instead of just $glue(s)$ for some $s : a \sim b$)?














\chapter{Specification of Inductive-Inductive Types}\label{chp:iit}
\chaptermark{Specification of IITs}

Inductive-Inductive Types are specified by giving a context in  a small type
theoretic syntax which we will refer to as \emph{source type theory}.

%TODO add waay more explanation
This idea originates from Ambrus Kaposi's work on the syntax of \emph{higher}
inductive-inductive types (~\cite{ambrussyntax}) which we adapt and rid of equality
constructors to only allow for inductive-inductive types.
In contrast to their presentation we will leave the context of the ambient type
theory implicit and, instead of highlighting syntax of the ambient type theory,
mark elements of the source type theory in \gr{green}.

\section{Signatures for Inductive-Inductive Types}

We assume that the source type theory makes use of the standard syntax of type
theory, using contexts, types, terms, and variables.
Types and terms are uniquely ascribed to one of two \emph{kinds}:
Either their kind is \grm{\Sc} which indicates that the type contains sort
constructors, or their kind is \grm{\Pc} because elements of it describe
point constructors.
We will write \grm{\Gamma \vdash A :: k} to say that \grm{A} is a type of kind
\grm{k} and \grm{\Gamma \vdash t : A :: k} to state that \grm{t} is a term of the
type \grm{A} which in turn has kind \grm{k}.
Often, we will omit the annotation of the sort, meaning that a judgment is to
hold true for both \grm{\Sc} and \grm{\Pc}, or that the kind of a term's type
has already been specified.

It's important that contexts can be extended by sort and point types in any order
(see Example~\ref{ex:tmnil}) to be able to capture sorts which depend on previously
defined point constructors.
So we have the usual two rules for context formation:
\begin{equation*}
\inferrule{}{\grm{\vdash \cdot}}
\qquad
\inferrule{\grm{\Gamma \vdash A :: k}}
  {\grm{\vdash \Gamma, A}}
\end{equation*}

We need one atomic constructor for sort types:
For plain types we and the codomain we need a type \grm{\UU} which serves as a
token for the \emph{universe}.
We will call terms of this universe ``small types''.
Positiviy requires that these are the only (internal) types which are allowed in
the domain of functions.
An operation \grm{\El} reifies these small types to big types, making our version
of universe what is commonly referred to as ``Tarski-style universe'' (cf. ~\cite{luotarski}):
\begin{equation*}
\inferrule{\grm{\vdash \Gamma}}{\grm{\Gamma \vdash \UU :: \Sc}}
\qquad
\inferrule{\grm{\Gamma \vdash a : \UU}}{\grm{\Gamma \vdash \El(a) :: \Pc}}
\end{equation*}

For sorts which are type families over other sorts that we seek to define, and for
constructors which recursively refer to other constructors, we need $\Pi$-types
which have a small type as their codomain.
Note that whether we want to build a sort or a point type only depends on the
kind of the \emph{codomain} of such a $\Pi$-type.
To eliminate from $\Pi$-types we want a rule for its \emph{application} which
turns a term of a $\Pi$-type into a term of its codomain:
\begin{equation*}
\inferrule{\grm{\Gamma \vdash a : \UU} \\ 
  \grm{\Gamma, \El(a) \vdash B :: k}}
  {\grm{\Gamma \vdash \Pi(a, B) :: k}}
\qquad
\inferrule{\grm{\Gamma \vdash t : \Pi(a, B)}}
  {\grm{\Gamma, \El(a) \vdash \IIapp(t) : B}}
\end{equation*}

Since we are working with explicit substitutions, we need to postulate a calculus
for substitutions \grm{\IISub{\sigma}{\Gamma}{\Delta}} between any two contexts
\grm{\Gamma} and \grm{\Delta}.
The substitutions should form a category as postulated by the following rules:
\begin{equation*}
\inferrule{\grm{\vdash \Gamma}}
  {\grm{\IISub{\id}{\Gamma}{\Gamma}}}
\qquad
\inferrule{\grm{\IISub{\sigma}{\Delta}{\Sigma}} \\
  \grm{\IISub{\delta}{\Gamma}{\Delta}}}
  {\grm{\IISub{\sigma \circ \delta}{\Gamma}{\Sigma}}}
\end{equation*}
\begin{align*}
\grm{\id \circ \sigma} &= \grm{\sigma} \\
\grm{\sigma \circ \id} &= \grm{\sigma} \\
\grm{(\sigma \circ \delta) \circ \gamma} &= \grm{\sigma \circ (\delta \circ \gamma)}
\end{align*}










%\section{(OLD) --The Source Type Theory}

Since for some examples of inductive-inductive types, especially ones which are
\emph{infinitely branching}, we need functions with external domain as the domains of
other functions, we will also add another function type which is itself small, and
has external domain and small codomain:
\begin{equation*}
\begin{gathered}
\inferrule{\grm{\vdash \Gamma} \\ A : \UU_i \\ (x : A) \to (\grm{\Gamma \vdash b : \UU})}
	{\grm{\Gamma \vdash ((\blm{x : A}) \to b) : \UU}}
\qquad
\inferrule{\grm{\Gamma \vdash t : \underline{(\blm{x : A}) \to b}} \\ u : A}
	{\grm{\Gamma \vdash (t \app \blm{u}) : \underline{b[\blm{x} \mapsto \blm{u}]}}}
\end{gathered}
\end{equation*}

\section{(OLD) --Motives and Methods}

TODO: Add definition of $\grm{-}^\MM$.

TODO: Add examples.

\section{(OLD) --Recursion and Computation}

TODO: Add definition of $\grm{-}^\EE$.

TODO: Add examples.

\section{(OLD) --Existence of HIITs}

\begin{defn}[Dependent Eliminator]
An algebra $c : \grm{\Gamma}^\CC$ is said to admit \emph{dependent elimination}
if for each motive $m : \grm{\Gamma}^\MM(c)$ there is a dependent eliminator
$\elim_{\grm{\Gamma}}(m) : \grm{\Gamma}^\EE(c, m)$.
\end{defn}

\begin{thm}[Admissibility of Inductive-Inductive Types]
Our type theory admits inductive-inductive types if for each wellformed context
$\grm{\Gamma}$ we can find a constructor $\con{\grm{\Gamma}}$ which admits dependent
elimination $\elim_{\grm{\Gamma}}$.
\end{thm}

\section{(OLD) --Morphisms of Algebras}

In the following, it will be useful to regard the algebras of a given context
\grm{\Gamma} as a category.
To this end, we need to make clear what a morphism between \grm{\Gamma}-algebras
is.
Intutitively a morphism is given by maps between the interpretations of the sorts,
together with evidence that those maps preserve the interpretation of point
constructors.

\begin{defn}[Morphisms of Algebras]
We want to define the following by mutual recursion on contexts, types and terms:
\begin{equation*}
\begin{gathered}
\inferrule{\grm{\vdash \Gamma} \\ \gamma_0, \gamma_1 : \grm{\Gamma}^\AA}
	{\grm{\Gamma}^\mm(\gamma_0,\gamma_1) : \UU} \\[.7em]
\inferrule{\grm{\Gamma \vdash A} %\\ \gamma_0, \gamma_1 : \grm{\Gamma}^\AA TODO decide if show
		\\ g : \grm{\Gamma}^\mm(\gamma_0, \gamma_1) \\
		\\ \alpha_0 : \grm{A}^\AA(\gamma_0) \\ \alpha_1 : \grm{A}^\AA(\gamma_1)}
	{\grm{A}^\mm(g, \alpha_0, \alpha_1) : \UU} \\[.7em]
\inferrule{\grm{\Gamma \vdash t : A} \\ \gamma_0, \gamma_1 : \grm{\Gamma}^\AA 
		\\ g : \grm{\Gamma}^\mm(\gamma_0, \gamma_1)}
	{\grm{t}^\mm(g) : \grm{A}^\mm(g, \grm{t}^\AA(\gamma_0), \grm{t}^\AA(\gamma_1))}
\end{gathered}
\end{equation*}

Like we did for the definition of algebras, we want morphisms of contexts to be
just iterated $\Sigma$-types of the respective interpretation of types:
\begin{align*}
\grm{\cdot}^\mm(\gamma_0, \gamma_1) &:\equiv \unit \text{ and} \\
\grm{(\Gamma, x : A)}^\mm(\gamma_0, \gamma_1) &:\equiv
	(g : \grm{\Gamma}^\mm(\pr_1(\gamma_0), \pr_1(\gamma_1))) \times \grm{A}^\mm(g, \pr_2(\gamma_0), \pr_2(\gamma_1)) \text{.}
\end{align*}
The core of the definition on sort types is that the universe is interpreted as
a function space:
\begin{align*}
\grm{\UU}^\mm(g, \alpha_0, \alpha_1)  			&:\equiv \alpha_0 \to \alpha_1 \text{,} \\
\grm{(\underline{a})}^\mm(g, \alpha_0, \alpha_1)	&:\equiv (\grm{a}^\mm(g, \alpha_0) = \alpha_1) \text{,} \\
\grm{((x : a) \to B)}^\mm(g, \alpha_0, \alpha_1)	&:\equiv (x : \grm{a}^\CC(\gamma_0))
							\to \grm{B}^\mm((g, \refl), \alpha_0(x), \alpha_1(\grm{a}^\mm(g, x))) \text{,} \\
\grm{((\blm{x : A}) \to B)}^\mm(g, \alpha_0, \alpha_1)  &:\equiv (x : A)
							\to \grm{(B(\blm{x}))}^\mm(g, \alpha_0(x), \alpha_1(x)) \text{.}
\end{align*}
On terms, consider the foo
\begin{align*}
\grm{x}^\mm(g) 				&:\equiv g.\grm{x} \qquad \text{for variables \grm{x},} \\ %???
\grm{(t(u))}^\mm(g)			&:\equiv (\grm{u}^\mm(g))_* (\grm{t}^\mm(g)(\grm{u}^\AA(\gamma_0))) \text{,} \\ %TODO check this
\grm{((\blm{x : A}) \to b)}^\mm(g)	&:\equiv (x : A) \to \grm{b}^\mm(g) \text{,} \\
\grm{(t(\blm{u}))}^\mm(g)		&:\equiv \grm{t}^\mm(g)(u) \text{, and} \\
\grm{(t \app \blm{u})}^\mm(g)		&:\equiv (\grm{t}^\mm(g))(u) \text{.} %TODO use happly here sometimes!!! uargh
\end{align*}

\end{defn}

TODO: Add examples.



\chapter{Specification of Inductive Families}\label{chp:if}

\section{Signatures for Inductive Families}

Previous specifications of mutual inductive families have taken different approaches:
Some are based on the notion of a polynomial functor while others...

Applying the same principle as in the case of inductive-inductive types we want
to create a specification based on the contexts of type theory syntax.
We already saw that we can obtain such a specification by just restricting the
syntax for inductive-inductive types to not use the recursive $\Pi$-type for sorts,
but this approach doesn't capture the full %TODO
extent of inductive families being a much simpler concept than inductive-inductive
types.
Given the strategy of our recursion we want the specification to capture at least
the following features of inductive families:
\begin{itemize}
\item Sorts are either types of funcions over existing types.
\item Point constructor can also be indexed over existing (``external'') types.
\item Point constructors can refer to any sort being defined.
\end{itemize}

The first point above says that we want the type of \emph{sort types} \blm{\tqm{\Sc} : \UU}
to be inductively generated by a \emph{universe} token \tqm{\UU : \Sc} and a constructor
of external functions for sorts which are meant to be \emph{type families}:
\tqm{\Pi_\Sc(\blm{T}, B) : \Sc} for a type \blm{T : \UU} and a function
\tqm{B : \blm{T \to \tqm{\Sc}}}.
Note that in contrast to the sort types of inductive-inductive definitions these
do not depend on a context.

Instead, we say that a \emph{sort context} is just a list of sort types without
any interdependencies:
\begin{equation*}
\begin{gathered}
\inferrule{}{\tqm{\SCon \cdot_\Sc}}
\qquad
\inferrule{\tqm{\SCon \Gamma_\Sc} \\ \tqm{B : \Sc}}{\tqm{\SCon \Gamma_\Sc, B}}
\end{gathered}
\end{equation*}

In order to refer to sorts we introduce a simplified term calculus based on typed
de Bruijn indices for bound variables and an application operation for type families:
\begin{equation*}
\begin{gathered}
\inferrule{\tqm{\SCon \Gamma_\Sc} \\ \tqm{B : \Sc}}{\tqm{\Gamma_\Sc, B \SCon \var(\vz) : B}}
\qquad
\inferrule{\tqm{\Gamma_\Sc \SCon \var(v) : B}}{\tqm{\Gamma_\Sc, B' \SCon \var(\vs(v)) : B}}
\\[.7em]
\inferrule{\tqm{\Gamma_\Sc \SCon t : \Pi_\Sc(\blm{T}, B)} \\ \blm{\tau : T}}
  {\tqm{\Gamma_\Sc \SCon t(\blm{\tau}) : B(\blm{\tau})}}
\end{gathered}
\end{equation*}

Point constructors will be represented by \emph{point types} over a given sort
context.
This means that opposite to inductive-inductive types, they cannot depend on
other point types.
The type formers we need are the element type for the universe \tqm{\UU}, an
external, non-recursive function type like the one we have for sorts, and an
internal function type used for recursive point constructors:
\begin{equation*}
\begin{gathered}
\inferrule{\tqm{\Gamma_\Sc \SCon a : \UU}}{\tqm{\Gamma_\Sc \SCon \El(a)}}
\qquad
\inferrule{\blm{T : \UU} \\ \blm{(\tau : T) \to \tqm{\Gamma_\Sc \SCon B(\blm{\tau})}}}
  {\tqm{\Gamma_\Sc \SCon \Pi_\Pc(\blm{T}, B)}}
\\[.7em]
\inferrule{\tqm{\Gamma_\Sc \SCon a : \UU} \\ \tqm{\Gamma_\Sc \SCon A}}
  {\tqm{\Gamma_\Sc \SCon a \Rightarrow_\Pc A}}
\end{gathered}
\end{equation*}

As a last building block of the syntax, we can now form full contexts consisting
of sort and point constructors.
Such a context \tqm{\Gamma} can be formed over a given sort context \tqm{\Gamma_\Sc}
which we will denote as a subscript to the turnstile or omit when inferrable.
The empty context can be formed over the empty sort context, an extension of
a context by a sort constructor happens in parallel to an extension of its sort
context, and an extension by a point constructor leaves the sort context fixed:
\begin{equation*}
\begin{gathered}
\inferrule{}{\tqm{\vdash_{\cdot_\Sc} \cdot}}
\qquad
\inferrule{\tqm{\vdash_{\Gamma_\Sc} \Gamma} \\ \tqm{B : \Sc}}
  {\tqm{\vdash_{\Gamma_\Sc, B} \Gamma, B}}
\qquad
\inferrule{\tqm{\vdash_{\Gamma_\Sc} \Gamma} \\ \tqm{\Gamma_\Sc \SCon A}}
  {\tqm{\vdash_{\Gamma_\Sc} \Gamma, A}}
\end{gathered}
\end{equation*}

\begin{remark}
While for the signatures of inductive-inductive types, contexts, types, and terms
depend on each other we can here define sort types, sort contexts, terms, point
types, and contexts in the presented order without referring to later constructions.
This means that unlike mentioned in Remark~\ref{rmk:iit-syntax}, we  can %TODO cite remark
internalize this syntax just using inductive families.

An Agda formalization of the syntax looks as follows, with variables and terms
separated:
\begin{agdacode}
data TyS : Set₁ where
  U  : TyS
  Π̂S : (T : Set) → (T → TyS) → TyS

data SCon : Set₁ where
  ∙c   : SCon
  _▶c_ : SCon → TyS → SCon

data Var : SCon → TyS → Set₁ where
  vvz : ∀{Γc}{B} → Var (Γc ▶c B) B
  vvs : ∀{Γc}{B}{B'} → Var Γc B → Var (Γc ▶c B') B

data Tm : SCon → TyS → Set₁ where
  var  : ∀{Γc}{A} → Var Γc A → Tm Γc A
  _\$S_ : ∀{Γc}{T}{B} → Tm Γc (Π̂S T B) → (α : T) → Tm Γc (B α)

data TyP : SCon → Set₁ where
  El   : ∀{Γc} → Tm Γc U → TyP Γc
  Π̂P   : ∀{Γc}(T : Set) → (T → TyP Γc) → TyP Γc
  _⇒P_ : ∀{Γc} → Tm Γc U → TyP Γc → TyP Γc

data Con : SCon → Set₁ where
  ∙    : Con ∙c
  _▶S_ : ∀{Γc} → Con Γc → (A : TyS) → Con (Γc ▶c A)
  _▶P_ : ∀{Γc} → Con Γc → (B : TyP Γc) → Con Γc
\end{agdacode}
%TODO remove backslash
\end{remark}




\chapter{Reducing Inductive-Inductive Types to Inductive Families}\label{chp:red}
\chaptermark{Reducing Inductive-Inductive Types}

\section{(OLD) -- Fragments of Inductive-Inductive Types}

As we have seen in the previous sections, inductive-inductive types as specified
allow for a very broad variety of definitions.
We will now see that it is easy to carve out different subsets of specifications
to obtain more restrictive fragments of inductive types.
Starting from the largest of these subsets, we will first see that there is a
straightforward way to restrict inductive-inductive types to those whose constructors
are finitary in the sense that no point constructor depends on an infinite
amount of data: %TODO improve that last sentence

\begin{defn}[Finitary IITs]
Given a specification \grm{\Gamma} for an inductive-inductive types we say that
it is \textbf{finitary} if the infinitary $\Pi$-type is not used.
\end{defn}

One example of a specification which does not meet this requirement are the
infinitely branching trees. %TODO cite example

Instead of only preventing the use of external data in ``small functions''

We want to reduce inductive-inductive types to inductive families.
This means, we postulate that inductive families be admissible in our target
type theory and show that this implies the existence of all inductive-inductive
types.
Since we want to reuse the way of specifying inductive-inductive types to specify
instances which are as well inductive families, we want to rediscover the specifications
of inductive families as a subset of all inductive-inductive specifications:

\begin{defn}[Inductive families]
A context \grm{\Gamma} is said to specify a \textbf{inductive family} if it is
generated without using the inductive function type in the specification of sorts
and thus, no sorts depend on other sorts but are only iterated function depending
on external types.
This means that the formation rule for inductive function types is restricted
to the case where \grm{k \equiv \Pc}.
\end{defn}

We assume for the remainder of this chapter, that if \grm{\Gamma} specifies an
inductive family, we are provided with $\con{\Gamma} : \grm{\Gamma}^\CC$ and
$\elim_\grm{\Gamma} : \grm{\Gamma}^\EE(\con{\Gamma}, m)$ for each
$m : \grm{\Gamma}^\MM$.

TODO: explain reduction to W-types maybe

\section{Type Erasure}

As seen in the examples, the first step to prove the reducability is to formally
define the operation which we will call \emph{flattening} or -- inspired by
the syntax example -- \emph{type erasure}.
This operation strips away any dependencies between the sorts of a signature
as well as all external indices to sorts.
The operation should take arbitrary inductive-inductive signatures (contexts) and
return signatures for inductive families.
Let us look at what type erasure should do with our running examples:

\begin{example}[Natural Numbers]\label{ex:red-e-nat}
Since the inductive-inductive signature of the \emph{natural numbers}~\ref{ex:ii-syntax-nat} doesn't
contain any indexed sorts, type erasure should ``do nothing'' with it.
That is, returning the sort context and point context of the inductive family
syntax which looks like a obvious correspondence to it (cf. Example~\ref{ex:if-natvec}):
\begin{align*}
  &\tqm{\grm{(\cdot,\, \UU,\, \El(\vz),\, \Pi\left(\vs(\vz),\, \El(\vs(\vs(\vz)))\right))}^\EE_\Sc} \\
= &\tqm{(\cdot_\Sc,\, \UU)} \text{ and} \\
  &\tqm{\grm{(\cdot,\, \UU,\, \El(\vz),\, \Pi\left(\vs(\vz),\, \El(\vs(\vs(\vz)))\right))}^\EE} \\
= &\tqm{(\cdot,\, \El(\var(\vz)),\, \var(\vz) \Rightarrow_\Pc \El(\var(\vz)))} \text{.}
\end{align*}
\end{example}

\begin{example}[Vectors]
In the example of vectors \ref{ex:ii-syntax-vec} we need to erase the natural numbers
index of the only sort under consideration:
\begin{align*}
\tqm{\grm{\Gamma_{vec}}^\EE_\Sc}
 &= \tqm{(\cdot_\Sc,\, \UU)} \text{ and} \\
\tqm{\grm{\Gamma_{vec}}^\EE}
  &= \tqm{(\cdot,\, \El(\var(\vz)),\, 
    \ExtPiP{A}{\blm{\lambda a.\,}\ExtPiP{\N}{\blm{\lambda n.\,}
    \var(\vz) \Rightarrow_\Pc \El(\var(\vz))}})} \text{.}
\end{align*}
Note that the erasure of the vectors does not coincide with the vectors represented
as an inductive family (Example~\ref{ex:if-natvec}), because its sort lacks the
indexing over the natural numbers.
In fact, it's easy to see that the algebras of this signature would no be isomorphic
to the type of lists over the type \blm{A \times \N}.
\end{example}

\begin{example}[Type Theory Syntax]
In our syntax we will now see why the operation is called ``type erasure'':
%TODO
\end{example}

To go from examples to the general case, we will present the different components
of the type erasure operation in roughly the same order in which they appear in
Section~\ref{sec:ii-syntax}, most often needing to distinguish between sort
and point constructors.

\begin{defn}[Type Erasure]
First of all, each context will need to be split into a sort context and a point
context:
\begin{equation*}
\inferrule{\grm{\vdash \Gamma}}
  {\tqm{\SCon \grm{\Gamma}^\EE_\Sc}}
\qquad
\inferrule{\grm{\vdash \Gamma}}
  {\tqm{\vdash_{\grm{\Gamma}^\EE_\Sc} \grm{\Gamma}^\EE }}
\end{equation*}
To descent down the components of the contexts, we will need to define the operation
on types as well.
Since we are erasing all information from the sorts, we will only need this for
point types, though.
Unsurprisingly, we want them to be translated to point types in the appropriate
sort context:
\begin{equation*}
\inferrule{\grm{\Gamma \vdash A :: \Pc}}
  {\tqm{\grm{\Gamma}^\EE_\Sc \SCon \grm{A}^\EE :: \Pc}}
\end{equation*}
Using this we will be able to define the operation creating sort contexts by
\begin{align*}
\tqm{\grm{\cdot}^\EE_\Sc}
  &:\equiv\tqm{\cdot_\Sc} \text{,} \\
\tqm{\grm{(\Gamma,\, B)}^\EE_\Sc}
  &:\equiv \tqm{\left(\grm{\Gamma}^\EE_\Sc,\, \grm{\UU}^\EE_\Sc\right)} \text{ for \grm{B :: \Sc}, and} \\
\tqm{\grm{(\Gamma,\, A)}^\EE_\Sc}
  &:\equiv \tqm{\grm{\Gamma}^\EE_\Sc} \text{ for \grm{A :: \Pc}.}
\end{align*}
The generated point context over this sort context has to be extended in the case
where the input is an extension by a point type.
In the case where it is an extension by a sort type, we want to return the
unextended context, but to make up for the definition above, we need to weaken
to account for the extension of the resulting sort context:
\begin{align*}
\tqm{\grm{\cdot}^\EE}
  &:\equiv\tqm{\cdot} \text{,} \\
\tqm{\grm{(\Gamma,\, B)}^\EE}
  &:\equiv \tqm{\grm{\Gamma}^\EE[\wk_{\id}]} \text{ for \grm{B :: \Sc}, and} \\
\tqm{\grm{(\Gamma,\, A)}^\EE}
  &:\equiv \tqm{\left(\grm{\Gamma}^\EE,\, \grm{A}^\EE\right)} \text{ for \grm{A :: \Pc}.}
\end{align*}
So how do we define \tqm{\grm{A}^\EE} for a point type \grm{A}?
The fact the we have to recurse on \grm{\El(a)} makes it clear that we will have
to extend our operation to terms of sort types at least.
That is, together with \tqm{\grm{A}^\EE} we also need the following:
\begin{equation*}
\inferrule{\grm{\Gamma \vdash t : B :: \Sc}}
  {\tqm{\grm{\Gamma}^\EE_\Sc \SCon \grm{t}^\EE : \UU}}
\end{equation*}
And indeed, with this we can set
\begin{align*}
\tqm{\grm{\El(a)}^\EE}
  &:\equiv \tqm{\El(\grm{a}^\EE)} \text{.}
\end{align*}
For recursive $\Pi$-types, we need only care about the ones yielding point types.
Note that the operation turns a $\Pi$-type into a non-dependent function type!
\begin{align*}
\tqm{\grm{\Pi(a, A)}^\EE}
  &:\equiv \tqm{\grm{a}^\EE \Rightarrow_\Pc \grm{A}^\EE}
\end{align*}
Since we forgot about the indexing of sort types, erasure of sort-kinded application terms
is just erasure of its $\Pi$-type term:
\begin{align*}
\tqm{\grm{\IIapp(f)}^\EE}
  &:\equiv \tqm{\grm{f}^\EE} \text{ for \grm{\Gamma \vdash t : \Pi(a, B) :: \Sc}.}
\end{align*}
External $\Pi$-types and their applications convert directly into their
respective counterparts in the syntax of inductive families:
\begin{align*}
\tqm{\grm{\ExtPi{T}{A}}^\EE}
  &:\equiv \tqm{\ExtPiP{T}{\blm{\lambda \tau.\, }\grm{A(\bltau)}^\EE}} \text{, and} \\
\tqm{\grm{f(\bltau)}^\EE}
  &:\equiv \tqm{\grm{f}^\EE} \text{ for \grm{\Gamma \vdash f : \ExtPi{T}{B} : \Sc}}
\end{align*} %TODO this is a bit confusing since the application is for sorts and the types for points
Defining the erasure on point types and sort terms pulled back along a substitution,
we see that we will also need to erase entire sort substitutions.
This is achieved by extending the operation as follows:
\begin{equation*}
\inferrule{\grm{\IISub{\sigma}{\Gamma}{\Delta}}}
  {\tqm{\IISub{\grm{\sigma}^\EE_\Sc}{\grm{\Gamma}^\EE_\Sc}{\grm{\Delta}^\EE_\Sc}}}
\end{equation*}
We will then be able to use this in a straight forward way to define the pullbacks:
\begin{align*}
\tqm{\grm{A[\sigma]}^\EE}
  &:\equiv \tqm{\grm{A}^\EE[\sigma^\EE_\Sc]} \text{ for \grm{\Gamma \vdash A :: \Pc} and} \\
\tqm{\grm{t[\sigma]}^\EE}
  &:\equiv \tqm{\grm{t}^\EE[\sigma^\EE_\Sc]} \text{ for \grm{\Gamma \vdash t : B :: \Sc}.}
\end{align*}
Erasure of substitutions is built recursively, ignoring point types.
Likewise, the first projection will ignore point types:
\begin{align*}
\tqm{\grm{\epsilon}^\EE_\Sc}
  &:\equiv \tqm{\epsilon} \text{,} \\
\tqm{\grm{(\sigma,\, t)}^\EE_\Sc}
  &:\equiv \tqm{(\grm{\epsilon}^\EE_\Sc,\, \grm{t}^\EE)} \text{ for \grm{\Gamma \vdash t : B :: \Sc}, and} \\
\tqm{\grm{(\sigma,\, t)}^\EE_\Sc}
  &:\equiv \tqm{\grm{\epsilon}^\EE_\Sc} \text{.} \\
\tqm{\grm{\pi_1(\sigma)}^\EE_\Sc}
  &:\equiv \tqm{\pi_1(\grm{\sigma}^\EE_\Sc)} \text{ for \grm{\IISub{\sigma}{\Gamma}{(\Delta,\, B :: \Sc)}},} \\
\tqm{\grm{\pi_1(\sigma)}^\EE_\Sc}
  &:\equiv \tqm{\grm{\sigma}^\EE_\Sc} \text{ for \grm{\IISub{\sigma}{\Gamma}{(\Delta,\, A :: \Pc)}}, and} \\
\tqm{\grm{\pi_2(\sigma)}^\EE}
  &:\equiv \tqm{\pi_2(\grm{\sigma}^\EE_\Sc)} \text{.}
\end{align*} %TODO laws
This concludes the definition of the erasure operation.
\end{defn}

For the steps that follow it will be necessary to equip the \emph{algebras}
of the resulting signatures with a substitution calculus that also considers
point contexts instead of only sort contexts.
To this end, we defined what we called lifted substitution algebras in Definition~\ref{def:if-alg-lsub}

\section{(OLD) -- Type Erasure}

Since we cannot define type families simultaneously together with their index
type, we will first produce a version of a given code, whose dependencies between
sorts have been \emph{erased}.
For this, we will use an operation on contexts, types and terms, which we will
call \emph{flattening}.
The resulting code will specify a type which is now a mutual definition
of a number of plain types instead of families,
but it will, in general, contain too many elements, because it lacks all the
restrictions on which fiber arguments of inductively generated elements should
lie in.

Note that $\flatten{}$ erases all uses of the function type in the
specifications of sorts.

\section{(OLD) -- Wellformedness Predicates}

From this section onwards we will, for the code $\grm{\Gamma \equiv (\cdot, x_1 : A_1, \ldots, x_n : A_n)}$
in consideration, assume that we have the projections $\bar{x_1} : \bar{A_1}, \ldots, \bar{x_n} : \bar{A_1}$ of
$\con{\grm{\flatten{\Gamma}}} : \grm{\flatten{\Gamma}}^\CC$ at our disposal in the target
language.

To motivate the next step of our construction, consider the following code:
\begin{equation*}
\grm{
A : \UU,\, B : A \to \UU,\, a_0, a_1 : \underline{A},\, b : \underline{B(a_0)}
} \text{.}
\end{equation*}
In its flattened form
\begin{equation*}
\grm{
\bar{A} : \UU,\, \bar{B} : \UU,\, \bar{a_0}, \bar{a_1} : \underline{\bar{A}},\, \bar{b} : \underline{\bar{B}}
} \text{,}
\end{equation*}
there is no way to recognize, whether \grm{b} was meant to be in \grm{B(a_0)},
in \grm{B(a_1)}.
To reintroduce this piece of information, we will transform a given code into a
mutually defined predicate over the flattened code, which shall indicate, in which
fiber of a type family a given constructor (or one of its inductive arguments)
should be located.

\begin{defn}[Wellformedness]
Like with \grm{\flatten{}}, we define a \emph{annotation}
transformation \annotate{} on contexts, types, and
terms by structural recursion on the syntax.
On contexts and sort types it is defined like this:
\begin{align*}
\annotate{\cdot}			&:\equiv \grm{\cdot} \\
\annotate{(\Gamma, x : A)}		&:\equiv
	\begin{cases}
	\grm{\annotate{\Gamma}, W_x : \blm{\bar{x}} \to \annotate{A}}	& \text{for \grm{A :: \Sc}} \\
	\grm{\annotate{\Gamma}, w_x : \anntwo{\bar{x}}{A}}		& \text{for \grm{A :: \Pc}}
	\end{cases} \\
\annotate{\UU}				&:\equiv \grm{\UU} \\
\annotate{(x : a) \to B}		&:\equiv \grm{\flatten{a} \to \annotate{B}} \text{ for \grm{B :: \Sc}} \\
\annotate{(\blm{x : A}) \to B}		&:\equiv \grm{\blm{A} \to \annotate{B}}  \text{ for \grm{B :: \Sc}}
\end{align*} %TODO alignment
As we can see, on point types, the operation takes another argument, which is a
term of the target theory.
\begin{align*}
\anntwo{y}{\UU}				&:\equiv \grm{\UU} \\
\anntwo{y}{\underline{a}}		&:\equiv \grm{\underline{\anntwo{y}{a}}} \\
\anntwo{y}{(x : a) \to B}		&:\equiv
	\grm{(\blm{\bar{x} : \flatten{a}})(\annotate{\blm{\bar{x}}, a})}  \\
	& \grm{\hfill{} \to \anntwo{y(\bar{x})}{B}} & \text{for \grm{B :: \Pc}} \\
\anntwo{y}{(\blm{x : A}) \to B}	&:\equiv
	\grm{(\blm{x : A}) \to \anntwo{y(x)}{B}} & \text{for \grm{B :: \Pc}}
\end{align*}
Annotate is also defined where the second argument is a term of a sort type:
\begin{align*}
\anntwo{y}{x}				&:\equiv \grm{W_x(\blm{y})} \text{ for variables \grm{x}} \\
\anntwo{y}{t(u)}			&:\equiv \grm{\anntwo{y}{t}(\flatten{u})} \\
\anntwo{y}{t(\blm{u})}			&:\equiv \grm{\anntwo{y}{t}(\blm{u})} \\
\anntwo{y}{(\blm{x : A}) \to b}		&:\equiv \grm{(\blm{x : A}) \to \anntwo{y(x)}{b}} \\
\anntwo{y}{t \app \blm{u}}		&:\equiv \grm{\anntwo{y}{t} \app \blm{u}}
\end{align*}
\end{defn}

\begin{example}[Type Theory Syntax]
Let us look at one sort and one point constructor of our prime example~\ref{ex:ttintt}.
The sort \grm{\mathop{Ty} : \mathop{Con} \to \UU} gets transformed to the following:
\begin{align*}
\grm{W_{\mathop{Ty}} :} 
&\phantom{\equiv~} \grm{\blm{\overline{Ty}} \to \annotate{\mathop{Con} \to \UU} } \\
&\equiv \grm{\blm{\overline{Ty}} \to \flatten{\mathop{Con}} \to \annotate{\UU}} \\
&\equiv \grm{\blm{\overline{Ty}} \to \blm{\overline{Con}} \to \UU} \text{.}
\end{align*}
The point constructor
\grm{\mathop{pi} : (\Gamma : \mathop{Con})(A : \mathop{Ty}(\Gamma))(\mathop{Ty}(\mathop{ext}(\Gamma, A))) \to \underline{\mathop{Ty}(\Gamma)}}
becomes a term \grm{w_{\mathop{pi}}} of the following type:
\begin{align*}
& \grm{\anntwo{\overline{x}}{(\Gamma : \mathop{Con})(A : \mathop{Ty}(\Gamma))(\mathop{Ty}(\mathop{ext}(\Gamma, A)))
	\to \mathop{Ty}(\Gamma)}} \\
\equiv & \grm{(\blm{\overline{\Gamma} : \overline{Con}})(\anntwo{\overline{\Gamma}}{\mathop{Con}})
	\to \anntwo{\overline{pi}(\overline{\Gamma})}{(A : \mathop{Ty}(\Gamma))(\mathop{Ty}(\mathop{ext}(\Gamma, A))) \to \underline{\mathop{Ty}(\Gamma)}}} \\
\equiv & \grm{
	(\blm{\overline{\Gamma} : \overline{Con}})(W_{\mathop{Con}}(\blm{\overline{\Gamma}}))
	(\blm{\overline{A} : \overline{Ty}})(\anntwo{\overline{A}}{\mathop{Ty}(\Gamma)}) } \\
& \grm{
	\to \anntwo{\overline{pi}(\overline{\Gamma}, \overline{A})}{(\mathop{Ty}(\mathop{ext}(\Gamma, A))) \to \underline{\mathop{Ty}(\Gamma)}}
} \\
\equiv & \grm{
	(\blm{\overline{\Gamma} : \overline{Con}})(W_{\mathop{Con}}(\blm{\overline{\Gamma}}))
	(\blm{\overline{A} : \overline{Ty}})(\anntwo{\overline{A}}{\mathop{Ty}}(\blm{\overline{\Gamma}}))
	(\blm{\overline{B} : \overline{Ty}})(\anntwo{\overline{B}}{\mathop{Ty}(\mathop{ext}(\Gamma, A))}) } \\
& \grm{
	\to \anntwo{\overline{pi}(\overline{\Gamma}, \overline{A}, \overline{B})}{\underline{Ty(\Gamma)}}
} \\
\equiv & \grm{
	(\blm{\overline{\Gamma} : \overline{Con}})(W_{\mathop{Con}}(\blm{\overline{\Gamma}}))
	(\blm{\overline{A} : \overline{Ty}})(W_{\mathop{Ty}}(\blm{\overline{A}}, \blm{\overline{\Gamma}}))
	(\blm{\overline{B} : \overline{Ty}})(\anntwo{\overline{B}}{\mathop{Ty}}(\blm{\overline{ext}(\overline{\Gamma}, \overline{A})}))
} \\
& \grm{
	\to \underline{\anntwo{\overline{pi}(\overline{\Gamma}, \overline{A}, \overline{B})}{\mathop{Ty}(\Gamma)}}
} \\
\equiv & \grm{
	(\blm{\overline{\Gamma} : \overline{Con}})(W_{\mathop{Con}}(\blm{\overline{\Gamma}}))
	(\blm{\overline{A} : \overline{Ty}})(W_{\mathop{Ty}}(\blm{\overline{A}}, \blm{\overline{\Gamma}}))
	(\blm{\overline{B} : \overline{Ty}})(W_{\mathop{Ty}}(\blm{\overline{B}}, \blm{\overline{ext}(\overline{\Gamma}, \overline{A})}))
} \\
& \grm{
	\to \underline{\anntwo{\overline{pi}(\overline{\Gamma}, \overline{A}, \overline{B})}{\mathop{Ty}}(\blm{\overline{\Gamma}})}
} \\
\equiv & \grm{
	(\blm{\overline{\Gamma} : \overline{Con}})(W_{\mathop{Con}}(\blm{\overline{\Gamma}}))
	(\blm{\overline{A} : \overline{Ty}})(W_{\mathop{Ty}}(\blm{\overline{A}}, \blm{\overline{\Gamma}}))
	(\blm{\overline{B} : \overline{Ty}})(W_{\mathop{Ty}}(\blm{\overline{B}}, \blm{\overline{ext}(\overline{\Gamma}, \overline{A})}))
} \\
& \grm{ \hspace*{0pt}\hfill
	\to \underline{W_{\mathop{Ty}}(\blm{\overline{pi}(\overline{\Gamma}, \overline{A}, \overline{B})},\blm{\overline{\Gamma}})}
} 
\end{align*}
\end{example}

\begin{lemma}
We have \grm{\vdash \annotate{\Gamma}} for every \grm{\vdash \Gamma}.
\end{lemma}

\begin{proof}
We simultaneouly prove that all of the following rules are admissible:
\begin{equation*}
\begin{gathered}
\inferrule{\grm{\vdash \Gamma}}{\grm{\vdash \annotate{\Gamma}}}
\qquad
\inferrule{\grm{\Gamma \vdash B :: \Sc}}{\grm{\annotate{\Gamma} \vdash \annotate{B} :: \Sc}}
\\[.7em]
\inferrule{\grm{\Gamma \vdash A :: \Pc} \\ \bar{x} : \flatten{A}}{\grm{\annotate{\Gamma} \vdash \anntwo{\bar{x}}{A} :: \Pc}}
\\[.7em]
\inferrule{\grm{\Gamma \vdash a : B :: \Sc} \\ \bar{x} : \flatten{a}} %TODO underline here??
	{\grm{\annotate{\Gamma} \vdash \anntwo{\bar{x}}{a} : \annotate{B} :: \Sc}}
\end{gathered}
\end{equation*}
For context extension by sort constructors,
assume that \grm{\vdash \annotate{\Gamma}} and \grm{\annotate{\Gamma} \vdash \annotate{B} :: \Sc}.
Since \blm{\overline{x} : \overline{A} \equiv \UU}, we infer \grm{\annotate{\Gamma} \vdash \blm{\overline{x}} \to \annotate{B}},
and thus
\begin{equation*}
\grm{\vdash \annotate{\Gamma}, W_x : \blm{\overline{x}} \to \annotate{B} \equiv \annotate{\Gamma, x : A}} \text{.}
\end{equation*}
For extension by point constructors we similarly reason that given
\grm{\vdash \annotate{\Gamma}} and \grm{\Gamma \vdash A :: \Pc}, from which we
can infer \grm{\annotate{\Gamma} \vdash \anntwo{\overline{x}}{A}} for \blm{\overline{x} : \overline{A}},

Let us next handle the case of types in \grm{\Sc}.
Given \grm{\vdash \annotate{\Gamma}},
we obviously have \grm{\annotate{\Gamma} \vdash \annotate{\UU} \equiv \UU}.
To show that \grm{\annotate{\Gamma} \vdash \annotate{(x : a) \to B}}, we may assume
that \grm{\vdash \annotate{\Gamma}} and that for \blm{\overline{x} : \overline{a}}
we have \grm{\annotate{\Gamma} \vdash \anntwo{\overline{x}}{a} : \UU},
and that \grm{\annotate{\Gamma, x : \underline{a}} \vdash \annotate{B}}.
That latter reduces to
\begin{equation*}
\grm{\annotate{\Gamma}, w_x : \underline{\anntwo{\overline{x}}{a}} \vdash \annotate{B} :: \Sc} \text{,}
\end{equation*}
but since, by definition, \grm{B} cannot contain any reference to \grm{w_x},
we conclude \grm{\annotate{\Gamma} \vdash \annotate{B}}, which, together with
\grm{\annotate{\Gamma}} and \blm{\overline{A} : \UU}, suffices to show
the desired result.
The case of non-inductive functions works likewise.

There are three formers for \grm{\Pc}-types: Small types, inductive functions, and
non-inductive functions:
\begin{itemize}
\item Assume that \grm{\Gamma \vdash a : \UU} and thus that for each
\blm{\overline{x} : \overline{a}} we have
\grm{\annotate{\Gamma} \vdash \anntwo{\overline{x}}{a} : \UU}.
We know that \grm{\annotate{\Gamma} \vdash \underline{\anntwo{\overline{x}}{a}}}
which is enough to infer \grm{\annotate{\Gamma} \vdash \underline{\anntwo{\overline{x}}{a}} \equiv \anntwo{\overline{x}}{\underline{a}}}.
\item Given \grm{\Gamma \vdash a : \UU} and \grm{\Gamma, x : \underline{a} \vdash B :: \Pc}
we may assume that \grm{\annotate{\Gamma} \vdash \anntwo{\overline{x}}{a} : \UU} for \blm{\overline{x} : \overline{a}}
and that for \blm{\overline{y}:\overline{B}} we have
\begin{equation*}
\grm{\annotate{\Gamma}, w_x : \anntwo{\overline{x}}{a} \vdash \anntwo{\overline{y}}{B} :: \Pc} \text{.}
\end{equation*}
We want to show that for \blm{\overline{f} : \overline{(x : a) \to B} \equiv (\overline{x} : \overline{a}) \to \overline{B}},
\begin{align*}
\grm{\annotate{\Gamma}} &\vdash \grm{\anntwo{\overline{f}}{(x : a) \to B} :: \Pc} \\
&\equiv \grm{(\blm{\overline{x} : \overline{a}})(\anntwo{\overline{x}}{a}) \to \anntwo{\overline{f}(\overline{x})}{B}} \text{.}
\end{align*}
To this end, we need to show that for \blm{\overline{x} : \overline{a}} we have
\grm{\annotate{\Gamma} \vdash \anntwo{\overline{x}}{a} \to \anntwo{\overline{f}(\overline{x})}{B}},
but this is true by induction and weakening.
\end{itemize}
As a last part of this proof, we need to show that \grm{\annotate{}} behave well
on the terms of \grm{\Sc}-types: Variables and applications of inductive and
non-inductive functions.
\begin{itemize}
\item Given \grm{\Gamma \vdash B :: \Sc} and assuming \grm{\annotate{\Gamma} \vdash \annotate{B}},
we want to show that for \blm{\overline{x}:\overline{a}} we have
\begin{equation*}
\grm{\annotate{\Gamma, a : B} \vdash \anntwo{\overline{x}}{a} : \annotate{B}} \text{,}
\end{equation*}
but this reduces to
\begin{equation*}
\grm{\annotate{\Gamma}, W_a : \blm{\overline{a}} \to \annotate{B} \vdash W_a(\blm{\overline{x}}) : \annotate{B}} \text{,}
\end{equation*}
which is obviously true.
\item Let \grm{\Gamma \vdash t : (x : a) \to B :: \Sc}, which lets us assume
\begin{equation*}
\grm{\annotate{\Gamma} \vdash \anntwo{\overline{f}}{t} : \blm{\overline{a}} \to \annotate{B}} \text{ for \blm{\overline{f} : \overline{t}}.}
\end{equation*}
For \grm{u : \underline{a}} we than have that
\begin{align*}
\grm{\annotate{\Gamma} \vdash} & \grm{\anntwo{\overline{f}}{t}(\overline{u}) : \annotate{B}} \\
& \grm{\equiv \anntwo{\overline{f}}{t(u)} : \annotate{B[x \mapsto u]}} \text{,}
\end{align*}
since \grm{x} is turned into \grm{w_x} which \grm{\annotate{B}} doesn't depend on. %TODO make this cleaner
\item For \blm{A : \UU}, \grm{\vdash \annotate{\Gamma}}, and
\begin{align*}
\grm{\annotate{\Gamma} \vdash} & \anntwo{\overline{f}}{t} : \annotate{(\blm{x : A}) \to B} \\
& \equiv \anntwo{\overline{f}}{t} : A \to \annotate{B} \text{ for \blm{\overline{f} : \overline{t}},}
\end{align*}
we see that for any \blm{u : A} we have \grm{\anntwo{\overline{f}}{t}(u) \equiv \anntwo{\overline{f}}{t(u)} : \annotate{B[\blm{x} \mapsto \blm{u}]}}
as desired.
\end{itemize}
\end{proof}

Like \grm{\flatten{}}, \grm{\annotate{}} only produces inductive families, so
we can assume to have $\con{\grm{\annotate{\Gamma}}} : \grm{\annotate{\Gamma}}^\CC$:

\begin{lemma}
If \grm{\vdash{\Gamma}}, then \grm{\annotate{\Gamma}} is an inductive family.
\end{lemma}

\begin{proof}
All occurences of inductive functions in sort constructors are replaced
by non-inductive functions.
\end{proof}

\section{Constructing the initial algebra}

With \grm{\flatten{\Gamma}} and \grm{\annotate{\Gamma}} with have two inductive
families which will suffice to construct the initial algebra for any code \grm{\Gamma}:
We use $\Sigma$-types over the wellformedness predicates to select which ones of
the objects in the flattened types and type families we want to include in the
initial algebra.

\begin{defn}(Initial algebra)
Define \blm{\con{}} on contexts by:
\begin{align*}
\con{\cdot} &:\equiv \star : \unit \\
\con{\Gamma, a : B :: \Sc} &:\equiv (\con{\Gamma}, \contwo{W_a}{B}) \\
\con{\Gamma, x : A :: \Pc} &:\equiv (\con{\Gamma}, \conthree{\bar{x}}{w_x}{A}) \\
\end{align*}
On sort constructors we need:
\begin{align*}
\contwo{f}{\UU} &:\equiv
\begin{cases}
(x : A) \to \contwo{b}{\UU} & \text{if \blm{f \equiv (x : A) \to b : \UU},} \\
\Sigma(f) & \text{otherwise} %TODO avoid this distinction
\end{cases}\\
\contwo{f}{(x : a) \to B :: \Sc} &:\equiv \lambda x : \conthree{\flatten{a}}{\annotate{a}}{\UU}. \contwo{\lambda y. f(y, \pr_1(x))}{B} \\ %TODO fix
\contwo{f}{(\blm{x : A}) \to B :: \Sc} &:\equiv \lambda x : A. \contwo{\lambda y. f(y, x)}{B}
\end{align*}
On point constructor we have:
\begin{align*}
\conthree{f}{w}{\underline{a}} &:\equiv (f, w) \text{ for a variable \grm{a}} \\
\conthree{f}{w}{\underline{t(u)}} &:\equiv \conthree{f}{w}{\underline{t}} \\
\conthree{f}{w}{\underline{(\blm{x : A}) \to b}} &:\equiv \lambda x : A. \conthree{f(x)}{w(x)}{\underline{b}} \\
\conthree{f}{w}{(x : a) \to B :: \Pc} &:\equiv
\lambda x : \conthree{\flatten{a}}{\annotate{a}}{\UU}. \conthree{f(\pr_1(x))}{w(\pr_1(x), \pr_2(x))}{B} \\ %TODO fix
\conthree{f}{w}{(\blm{x : A}) \to B :: \Pc} &:\equiv
\lambda x : A. \conthree{f(x)}{w(x)}{B}
\end{align*}
Here, \blm{\pr_1} and \blm{\pr_2} are generalized projections, defined on universe
terms and for $i \in {1, 2}$ as follows:
\begin{align*}
\pr_i(y, \grm{a}) &:\equiv \pr_i(y) \text{ for a variable \grm{a}} \\
\pr_i(y, \grm{t(u)}) &:\equiv \pr_i(y) \\ %TODO this really seems fishy
\pr_i(y, \grm{(\blm{x : A}) \to b}) &:\equiv \lambda x : A. \pr_i(y(x), \grm{b})
\end{align*}
Also we need for universe terms a way to reference them as the domain of function
types:
\begin{align*}
\mathsf{ref}(\grm{a}) &:\equiv \contwo{W_a}{\UU} \text{ for a variable \grm{a}} \\
\mathsf{ref}(\grm{t(u)}) &:\equiv \mathsf{ref}(\grm{t})(\bar{u}) \\
\mathsf{ref}(\grm{(\blm{x : A}) \to b}) &:\equiv (x : A) \to \mathsf{ref}(\grm{b})
\end{align*}
\end{defn}

\section{Constructing the Eliminator}

TODO: Define $r$


\bibliographystyle{unsrtnat}
\bibliography{references}

\end{document}

