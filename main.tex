\documentclass[12pt,headings=optiontohead,openany,oneside,a4paper]{book}
%TODO: page layout
\usepackage[backref=page,
unicode,
pdfauthor={Jakob von Raumer},
pdftitle={Reducing Inductive-Inductive Types to Indexed Inductive Types},
pdfsubject={Mathematics, Computer Science},
pdfkeywords={interactive theorem proving}]{hyperref}

%code listings
\usepackage{mathpartir}
\usepackage{newunicodechar}
%\renewcommand{\MintedPygmentize}{./pygments-main/pygmentize}
\usepackage{fontspec,xunicode}
\usepackage[osf]{mathpazo}
%\usepackage{shellesc}
\usepackage{minted}
\setmainfont[Ligatures=TeX]{TeX Gyre Pagella}
\setmonofont{Iosevka}
%\newfontfamily{\freeserif}{DejaVu Sans Mono}
%\newunicodechar{⦃}{\ensuremath{\texttt{\{}\mkern-7mu\texttt{|}}}
	%{\makebox[.5em]{\freeserif⦃}}
%\newunicodechar{⦄}{\ensuremath{\texttt{|}\mkern-7mu\texttt{\}}}}
	%{\makebox[.5em]{\freeserif⦄}}
%\newunicodechar{→}{\texttt{\freeserif{→}}}
%\newunicodechar{⟶}{\texttt{\freeserif{⟶}}}
%\newunicodechar{⁻}{\texttt{\freeserif{⁻}}}
%\newunicodechar{▹}{\freeserif{▹}}
%\newunicodechar{ℕ}{\texttt{\freeserif{ℕ}}}
%\newunicodechar{⟨}{\texttt{\freeserif{⟨}}}
%\newunicodechar{⟩}{\texttt{\freeserif{⟩}}}
%\newunicodechar{⬝}{\ensuremath{\ct}}
%\newunicodechar{∼}{\freeserif{∼}}
%\newunicodechar{≃}{\freeserif{≃}}
%\newunicodechar{≅}{\freeserif{≅}}
%\newunicodechar{∘}{\freeserif{∘}}
%\newunicodechar{ʰ}{\freeserif{ʰ}}
%\newunicodechar{ᵍ}{\freeserif{ᵍ}}
%\newunicodechar{⇒}{\freeserif{⇒}}
%\newunicodechar{⋆}{\freeserif{⋆}}
%\newunicodechar{∘}{\freeserif{∘}}
%\newunicodechar{←}{\freeserif{←}}
%\newunicodechar{∙}{\freeserif{∙}}
%\newunicodechar{▶}{\freeserif{▶}}
%\newunicodechar{∀}{\freeserif{∀}}

%workaround to remove red boxes
\AtBeginEnvironment{minted}{\renewcommand{\fcolorbox}[4][]{#4}}
\usemintedstyle{default}
\newminted{lean}{
	linenos,
	numbersep=5pt,
	frame=single,
	fontsize=\footnotesize,
	framesep=1mm,
	samepage,
	autogobble}
\newminted[leancodebr]{lean}{
	linenos,
	numbersep=5pt,
	frame=single,
	fontsize=\footnotesize,
	framesep=1mm,
	autogobble}
\newcommand{\leani}[1]{\mintinline{lean}{#1}}
\newminted{agda}{
	linenos,
	numbersep=5pt,
	frame=single,
	fontsize=\footnotesize,
	framesep=1mm,
	samepage,
	autogobble}
\newcommand{\agdai}[1]{\mintinline{agda}{#1}}
\linespread{1.07}
\usepackage[dvipsnames]{xcolor}
\usepackage{graphicx}
\usepackage{array}
\usepackage[all,2cell,cmtip]{xy}
\usepackage{tikz}
\usetikzlibrary{decorations.pathmorphing,arrows}
\usepackage[numbered]{bookmark}
\usepackage{fancyhdr}
\usepackage{amssymb,amsmath,amsthm,mathrsfs,wasysym}
\usepackage{enumitem,mathtools,xspace}
\usepackage[nottoc]{tocbibind}
\usepackage{cleveref}
\usepackage{aliascnt}
\usepackage{natbib}
\usepackage{mathtools}
\usepackage{footnote}
\usepackage{booktabs}
\usepackage{xifthen}
%\usepackage[top=1in, bottom=1.25in, left=1.25in, right=1.25in]{geometry} %TODO decide

\fancyhead[LO]{\leftmark}
\fancyhead[RE]{\rightmark}
\fancyhead[LE,RO]{\thepage}
\cfoot{}
\pagestyle{fancy}

\def\defthm#1#2#3{
	\newaliascnt{#1}{thm}
	\newtheorem{#1}[#1]{#2}
	\aliascntresetthe{#1}
	\crefname{#1}{#2}{#3}
}

\newtheorem{thm}{Theorem}[section]
\crefname{thm}{Theorem}{Theorems}
\defthm{lemma}{Lemma}{Lemmas}
\defthm{axiom}{Axiom}{Axioms}
\defthm{corollary}{Corollary}{Corollaries}
\theoremstyle{definition}
\defthm{defn}{Definition}{Definitions}
\defthm{example}{Example}{Examples}
\defthm{remark}{Remark}{Remarks}

\newcommand{\upperf}{\partial^-_1}
\newcommand{\lowerf}{\partial^+_1}
\newcommand{\leftf}{\partial^-_2}
\newcommand{\rightf}{\partial^+_2}
\newcommand{\inv}{^{-1}}
\newcommand{\DCat}{\mathbf{DCat}}
\newcommand{\DGpd}{\mathbf{DGpd}}
\newcommand{\XMod}{\mathbf{XMod}}
\newcommand{\twotype}{\mathbf{2}}
\newcommand{\unit}{\mathbf{1}}
\newcommand{\emptytype}{\mathbf{0}}
\newcommand{\UU}{\mathcal{U}}
\newcommand{\isProp}{\mathsf{isProp}}
\newcommand{\isSet}{\mathsf{isSet}}
\newcommand{\PropU}{\mathsf{Prop}}
\newcommand{\SetU}{\mathsf{Set}}
\newcommand{\seg}{\mathsf{seg}}
\newcommand{\isContr}{\mathsf{isContr}}
\newcommand{\isIso}{\mathsf{isIso}}
\newcommand{\idtoiso}{\mathsf{idtoiso}}
\newcommand{\idtoeqv}{\mathsf{idtoeqv}}
\newcommand{\ua}{\mathsf{ua}}
\newcommand{\alliso}{\mathsf{alliso}}
\newcommand{\refl}{\mathsf{refl}}
\newcommand{\ap}{\mathsf{ap}}
\newcommand{\coe}{\mathsf{coe}}
\newcommand{\apd}{\mathsf{apd}}
\newcommand{\happly}{\mathsf{happly}}
\newcommand{\ind}{\mathsf{ind}}
\newcommand{\rec}{\mathsf{rec}}
\newcommand{\pr}{\mathsf{pr}}
\newcommand{\inl}{\mathsf{inl}}
\newcommand{\inr}{\mathsf{inr}}
\newcommand{\ishae}{\mathsf{ishae}}
\newcommand{\apiop}{\ap_{\iota'}}
\DeclareMathOperator{\comp}{comp}
\newcommand{\id}{\mathsf{id}}
\DeclareMathOperator{\invv}{inv_1}
\DeclareMathOperator{\invh}{inv_2}
\DeclareMathOperator{\obj}{obj}
\DeclareMathOperator{\EI}{EI}
\DeclareMathOperator{\IE}{IE}
\DeclareMathOperator{\op}{op}
\DeclareMathOperator{\inj}{inj}
\newcommand{\assoc}{\mathsf{assoc}}
\newcommand{\assocv}{\mathsf{assoc}_1}
\newcommand{\assoch}{\mathsf{assoc}_2}
\newcommand{\idLeft}{\mathsf{idLeft}}
\newcommand{\idLeftv}{\mathsf{idLeft}_1}
\newcommand{\idLefth}{\mathsf{idLeft}_2}
\newcommand{\idRight}{\mathsf{idRight}}
\newcommand{\idRightv}{\mathsf{idRight}_1}
\newcommand{\idRighth}{\mathsf{idRight}_2}
\newcommand{\leftInv}{\mathsf{leftInv}}
\newcommand{\leftInvv}{\mathsf{leftInv}_1}
\newcommand{\leftInvh}{\mathsf{leftInv}_2}
\newcommand{\rightInv}{\mathsf{rightInv}}
\newcommand{\rightInvv}{\mathsf{rightInv}_1}
\newcommand{\rightInvh}{\mathsf{rightInv}_2}
\newcommand{\respectId}{\mathsf{respectId}}
\newcommand{\respectIdv}{\mathsf{respectId}_1}
\newcommand{\respectIdh}{\mathsf{respectId}_2}
\newcommand{\respectComp}{\mathsf{respectComp}}
\newcommand{\respectCompv}{\mathsf{respectComp}_1}
\newcommand{\respectComph}{\mathsf{respectComp}_2}
\newcommand{\isequiv}{\mathsf{isequiv}}
\newcommand{\isodd}{\mathsf{isodd}}
\newcommand{\thin}{\mathsf{thin}}
\newcommand{\Sbase}{\mathsf{base}}
\newcommand{\Sloop}{\mathsf{loop}}
\newcommand{\Smerid}{\mathsf{merid}}
\newcommand{\vecty}{\mathsf{vec}}
\newcommand{\nil}{\mathsf{nil}}
\newcommand{\cons}{\mathsf{cons}}
\newcommand{\ct}{
	\mathchoice{\mathbin{\raisebox{0.5ex}{$\displaystyle\centerdot$}}}
		{\mathbin{\raisebox{0.5ex}{$\centerdot$}}}
		{\mathbin{\raisebox{0.25ex}{$\scriptstyle\,\centerdot\,$}}}
		{\mathbin{\raisebox{0.1ex}{$\scriptscriptstyle\,\centerdot\,$}}}
}
\newcommand{\trunc}[2]{\left\Vert #2\right\Vert_{#1}}
\newcommand{\squash}[1]{\trunc{}{#1}}
\newcommand{\isntype}[1]{\mathsf{is}\mbox{-}{#1}\mbox{-}\mathsf{type}}
\newcommand{\N}{\mathbb{N}}
\newcommand{\Sph}{\mathbb{S}}
\newcommand{\Set}{\mathbf{Set}}
\newcommand{\Cat}{\mathbf{Cat}}
\newcommand{\Fam}{\mathbf{Fam}}
\newcommand{\Alg}{\mathbf{Alg}}
\newcommand{\CT}{\mathbf{CT}}
\newcommand{\CTw}{\mathbf{CTw}}
\newcommand{\old}[1]{\overline{#1}}
\newcommand{\oldold}[1]{\overline{\overline{#1}}}
\newcommand{\gr}[1]{{\color{ForestGreen}#1}}
\newcommand{\grm}[1]{\ensuremath{\gr{#1}}}
\newcommand{\blm}[1]{\ensuremath{{\color{Black}#1}}}
\newcommand{\tqm}[1]{\ensuremath{{\color{Blue}#1}}}
%\newcommand{\grinfer}[2]{\grm{\inferrule{{#1}}{{#2}}}}
\newcommand{\Sc}{\mathsf{S}}
\newcommand{\Pc}{\mathsf{P}}
\newcommand{\CC}{\mathsf{A}} %TODO maybe jus replace
\renewcommand{\AA}{\mathsf{A}}
\newcommand{\EE}{\mathsf{E}}
\newcommand{\PP}{\mathsf{P}}
\newcommand{\mm}{\mathsf{m}}
\newcommand{\MM}{\mathsf{M}}
\newcommand{\DD}{\mathsf{D}}
\newcommand{\WW}{\mathsf{W}}
\newcommand{\RR}{\mathsf{R}}
\renewcommand{\SS}{\mathsf{S}}
\newcommand{\Sg}{\mathsf{\Sigma}}
\newcommand{\con}[1]{\mathsf{con}(\grm{#1})}
\newcommand{\contwo}[2]{\mathsf{con}(#1,\grm{#2})}
\newcommand{\conthree}[3]{\mathsf{con}(#1, #2, \grm{#3})}
\newcommand{\IFconS}[1]{\mathsf{con_\Sc}(\tqm{#1})}
\newcommand{\IFcon}[1]{\mathsf{con}(\tqm{#1})}
\newcommand{\IFelimS}[2]{\ifthenelse{\isempty{#2}}{\mathsf{elim_\Sc}(\tqm{#1})}{\mathsf{elim_\Sc}(\tqm{#1}, #2)}}
\newcommand{\IFelim}[2]{\ifthenelse{\isempty{#2}}{\mathsf{elim}(\tqm{#1})}{\mathsf{elim}(\tqm{#1}, #2)}}
\newcommand{\elim}{\mathsf{elim}}
\newcommand{\flatten}[1]{\blm{\mathsf{flatten}(\grm{#1})}}
\newcommand{\annotate}[1]{\blm{\mathsf{an}(\grm{#1})}}
\newcommand{\anntwo}[2]{\blm{\mathsf{an}(#1, \grm{#2})}}
\newcommand{\app}{\mathrel{@}}
\newcommand{\SCon}{\vdash_\Sc}
\newcommand{\var}{\mathsf{var}}
\newcommand{\vz}{\mathsf{vz}}
\newcommand{\vs}{\mathsf{vs}}
\newcommand{\El}{\mathsf{El}} %TODO change to operator?
\newcommand{\LSub}[3]{\mathsf{LSub}_{\tqm{#1}}(\tqm{#2}, \tqm{#3})}
\newcommand{\wk}{\mathsf{wk}}
\newcommand{\IFSub}[3]{#2 \overset{#1}{\longrightarrow} #3}
\newcommand{\IISub}[3]{#2 \overset{#1}{\longrightarrow} #3}
\newcommand{\IIapp}{\mathsf{app}}
\newcommand{\ExtPi}[2]{\hat{\Pi}(\blm{#1},\, #2)}
\newcommand{\ExtPiS}[2]{\hat{\Pi}_\Sc(\blm{#1},\, #2)}
\newcommand{\ExtPiP}[2]{\hat{\Pi}_\Pc(\blm{#1},\, #2)}
\newcommand{\bltau}{\blm{\tau}}
\newcommand{\blalpha}{\blm{\alpha}}
\newcommand{\blgamma}{\blm{\gamma}}
\newcommand{\blpi}{\blm{\pi}}
\newcommand{\blphi}{\blm{\phi}}

\renewcommand{\backrefalt}[4]{
	\ifcase #1
		(No citations.)
	\or
		(Cited on page\ #2.)
	\else
		(Cited on pages\ #2.)
	\fi
}

\begin{document}

\title{Reducing Inductive-Inductive Types to Indexed Inductive Types}
\author{Jakob von Raumer}

\frontmatter
\maketitle

\tableofcontents

\mainmatter

\chapter{Introduction}

\section{Background}

This thesis explores several problems in the field of \emph{type theory}.
By type theory we will always mean various flavors of what is usually
referred to as Martin-Löf type theory or dependent type theory.
Martin-Löf type theory (MLTT) can serve as a foundational framework for
mathematics as well as an organization principle for functional programming languages.

In the field of type theory, many researchers either apply theoretical considerations
achieve cleaner formalizations of mathematical content,
they create implementations of type theory which can be used as
interactive theorem proving systems, and they try to extend type
theory to improve its usability and justify these extensions with models.

In this spirit, this thesis will also explore ways to make certain kinds of formalizations
and certain constructions in type theory smoother and easier to use.
While it is \emph{mainly theoretical work}, it has consequences for the application of
interactive theorem provers, as they are used today.
This thesis is broadly split in two halfs.
Both halfs will explore different classes of language elements both of which are generalizations
of a language feature which is
called \emph{inductive types}.
Inductive types are a common way to define collections of data in mathematics
as well as computer science.

In the first half, we will try to encapsulate a common proof strategy which
is often used in the field of homotopy type theory and especially in
synthetic homotopy thoery.
We will do this by
proving a very general result about \emph{higher inductive types}.
These are inductive types in which statements about \emph{equality} between
elements of these data types carry a \emph{higher-dimensional structure},
making them on the one hand an interesting object of study in terms of their topology,
but on the other hand they are sometimes hard to handle.
Our theorem allows an easier way to prove propositions about these
equaities of elements.

In the second half, we will explore the topic of induction-induction.
\emph{Inductive-inductive types} are a class of inductive types which
allows us to define a data type simultaneously with data types depending on
values of the former type.
We will give an exact definition of what inductive-inductive types are,
and, with a new definition of \emph{inductive families} provide a point of
comparison which allows us to attempt to represent each example of
an inductive-inductive type as a construction based on a series of
inductive families.

\section{Homotopy Type Theory and Higher Inductive Types}

Homotopy type theory is a relatively new field which connects the study of
dependent types with the field of higher category theory and homotopy theory.
This synthesis has since offered a new perspective on how to constructively
have a formal representation of homotopy theory which is \emph{synthetic},
i.\,e.\ builds the spaces which are considered out of just a few fundamental
operations.

The connection to homotopy theory is based on the observation that one might
consider equality types, which by the proposition-as-types interpretation of type
theory, represent the statement that two elements $x$ and $y$ of a type $A$
are equal, as representing the spaces of \emph{paths modulo homotopy}.
Extending on this equivalence, we can view types to represent spaces,
and types depending on the data of other types as fibrations.

In this setting, we want to consider types, which are inductively defined,
like for example the natural numbers, but which, besides the \emph{points}
of the type also allow the (free) generation of new paths between points
and ``higher'' paths between
other paths.
These types are called \emph{higher inductive types}.

To prove facts about higher inductive types, for example in order to get the
type theoretic equivalent of the fact that the fundamental group of the circle
is equivalent to the integers, or the type theoretic Seifert-van Kampen theorem,
which characterizes the fundamental groupoid of a pushout of spaces,
a proof strategy with the name ``encode-decode method'' is employed.

In this thesis we provide a theorem which can be seen a generalization of
encode-decode proofs.
The application of this theorem can help to reduce ``boiler plate'' overhead
in formalizations and in reasoning about the equalities between points
of higher inductive types.

\section{The Concept of Induction-Induction}\label{sec:intro-iit}

While the first half of this thesis is about ways to make inductive types carry
higher-dimensional structure, the second half is about allowing for inductive types
which are more \emph{interdependent}:
There are situations in which we might not only want to define one single type
or type family, but instead we want to define a type $A$ and a type family $B : A \to \UU$,
or even a whole system of new types, indexed over each other,
mutually.
``Mutually'' here means that the point constructors, which specify which elements
we can form can refer to any of the types being defined.
And more than that, also the signature of the type families which we define can
be indexed over other type families the definition of which is not finished.

To illustrate one main application, imagine we wanted to formalize and reason
about the syntax of type theory in a type theoretic setting.
The environment of variables which are at our disposal at a given point in a piece
of type theoretic syntax are captured in what is called a \emph{context}.
We can model contexts as a type $Con : \UU$.
But at the same time we want to model types as existing in a context, so we want
a simultaneous definition of a type family $Ty : Con \to \UU$.
To see that we can not first define $Con$ and then move on to define $Ty$, consider
the following data which $Con$ and $Ty$ should include:

Contexts can be seen as lists of types depending on previous entries in that
list.
In that sense, it is obvious that for a context $\Gamma : Con$ and a type
$A : Ty(\Gamma)$ in that context, we want a context which represents the
extension of $\Gamma$ by $A$.
This means that $Con$ should have a constructor of the form
\begin{equation*}
ext: (\Gamma : Con) \to Ty(\Gamma) \to Con \text{,}
\end{equation*}
and the fact that this constructor mentions $Ty$ is already enough to exclude
a sequential definition of $Con$ and $Ty$.

Given the need for inductive-inductive types we can ask the question of whether
this concept is really stronger than inductive families without dependencies
between the sorts.
At first glance it might seem as if they are more expressive,
but on a closer look we can discover that we can actually reduce every
example of an inductive-inductive type to an inductive family.

The second half of this thesis is about making this reduction, which can easily
be seen to work on specific examples, more general.
To make it a formal statement we will have to give precise definitions of
what inductive-inductive types are and what, as our reference point,
inductive families are.
While we don't succeed at providing a full formal proof for the reduction,
the essential steps of it are complete and formalized in Agda. %TODO re-read and improve

\section{Contributions and Publications}

While parts of this thesis consists of the review and introduction of constructions
and knowledge which is already established,
other parts offer new contributions which stand on the shoulders of these ``giants''.
The contributions of this thesis include the following:
\begin{itemize}
\item The formalization of lots of homotopy theoretic notions in the theorem prover
Lean as described in \Cref{sec:tt-lean}.
\item The formulation of the characterization theorem for path spaces
of homotopy Coequalizers \Cref{thm:paths-main-thm}, as well as its proof
as given in \Cref{sec:paths-main}.
\item The adaptation of this theorem for pushouts as described in \Cref{thm:paths-main-pushout}.
\item The formulation of a possible higher Seifert-van Kampen theorem
as stated in \Cref{thm:paths-higher-SvK}.
\item The formalization of \Cref{sec:paths-main} and \Cref{thm:paths-main-pushout}
in Lean.
\item An adapation of the syntax for higher inductive-inductive types by
\citet{ambrussyntax} to separate sort and point constructors, as described, together with
its semantics, in \Cref{chp:iit}.
\item A syntax of signatures for inductive families as given in \Cref{chp:if}.
\item A formal specification of type erasure, wellformedness relation and eliminator
relation given as syntactic translations as described in \Cref{sec:red-e},
\Cref{sec:red-w}, and \Cref{sec:red-r}.
\item A formal definition of the ``sigma construction'' for an initial algebra
for inductive-inductive types as proposed in \Cref{sec:red-sg}.
\end{itemize}

Parts of this thesis have been peer-reviewed and published already, while
other parts, especially \Cref{chp:if} and \Cref{chp:red} are not yet published
elsewhere.
\begin{itemize}
\item Together with Floris van Doorn and Ulrik Buchholtz, a more
detailed description of our homotopy type theory formalizations was given
in the proceedings of the conference \emph{Interactive Theorem Proving -- 8th International Conference}
in 2017 \citep{leanhott}.
\item In joint work with Nicolai Kraus, the characterization of path spaces
was published in the proceeding of the conference
\emph{Thirty-Fourth Annual ACM/IEEE Symposium on
Logic in Computer Science (LICS)} in 2019 \citep{paths}.
\end{itemize}

\section{Structure of this Thesis}

We will start off this thesis by giving a more detailled exposition of the
background in type theory, which is the basis for the further content.
In this endevour, \Cref{chp:tt} does not only serve to give the necessary background
to readers unfamiliar with dependent types, it also sets notations and terminology
which we will re-use in the later chapters.
In \Cref{sec:tt-provers}, it furthermore gives a short characterization of the interactive
theorem pro\-vers Lean and Agda.

\Cref{chp:hit} will then first introduce some examples of higher inductive types
and will then go on to propose homotopy coequalizers as a fundamental higher
inductive type which can serve as a common generalization of all of these examples.
Sketches for two proofs using the ``encode-decode'' method will be given in
\Cref{sec:hit-encode-decode}.

Following the introduction of homotopy coequalizers we will then (\Cref{chp:paths})
see how we can
characterize their path spaces in order to replace encode-decode proofs.
Apart from proving this characterization (\Cref{sec:paths-main})
we will demonstrate this use in some examples (\Cref{sec:paths-applications}
and \Cref{sec:paths-svk}).

Switching not only the type theoretical setting from homotopy type theory
to set-truncated (or even extensional) type theory
but also the focus of the thesis, the second half will explore types which
instead of higher dimensional structure carry intricate dependencies between
multiple types to be defined.

We will start this second half by first giving a syntax for inductive-inductive
types (\Cref{chp:iit}) including its semantics,
before comparing it to the simpler fragment of inductive families (\Cref{chp:if}).
The latter will be reduced to indexed W-types in \Cref{sec:if-ex}.

After reducing inductive families to indexed W-types we will try to reduce
inductive-inductive types to inductive families:
In \Cref{chp:red} we will give a formal description about how to generate the
inductive families which correspond to
erasing the inductive-inductive typing information,
recovering it with a wellformedness predicate, yielding a candidate for an inital
object in the target type theory.
Then we will present a binary relation which could be used in future to prove
its initiality.












\chapter{Basic Type Theory}

\chapter{Higher Inductive Types}

\chapter{Path Spaces of HITs}

\chapter{Specification of Inductive-Inductive Types}

Inductive-Inductive Types are specified by giving a context in  a small type
theoretic syntax which we will refer to as \emph{source type theory}.

%TODO add waay more explanation
This idea originates from Ambrus Kaposi's work on the syntax of \emph{higher}
inductive-inductive types (~\cite{ambrussyntax}) which we adapt and rid of equality
constructors to only allow for inductive-inductive types.
In contrast to their presentation we will leave the context of the ambient type
theory implicit and, instead of highlighting syntax of the ambient type theory,
mark elements of the source type theory in \gr{green}.

\section{Signatures for Inductive-Inductive Types}\label{sec:ii-syntax}

We assume that the source type theory makes use of the standard syntax of type
theory, using contexts, types, terms, and variables.
Types and terms are uniquely ascribed to one of two \emph{kinds}:
Either their kind is \grm{\Sc} which indicates that the type contains sort
constructors, or their kind is \grm{\Pc} because elements of it describe
point constructors.
We will write \grm{\Gamma \vdash A :: k} to say that \grm{A} is a type of kind
\grm{k} and \grm{\Gamma \vdash t : A :: k} to state that \grm{t} is a term of the
type \grm{A} which in turn has kind \grm{k}.
Often, we will omit the annotation of the sort, meaning that a judgment is to
hold true for both \grm{\Sc} and \grm{\Pc}, or that the kind of a term's type
has already been specified.

It's important that contexts can be extended by sort and point types in any order
(see Example~\ref{ex:tmnil}) to be able to capture sorts which depend on previously
defined point constructors.
So we have the usual two rules for context formation:
\begin{equation*}
\inferrule{}{\grm{\vdash \cdot}}
\qquad
\inferrule{\grm{\Gamma \vdash A :: k}}
  {\grm{\vdash \Gamma, A}}
\end{equation*}

We need one atomic constructor for sort types:
For plain types we and the codomain we need a type \grm{\UU} which serves as a
token for the \emph{universe}.
We will call terms of this universe ``small types''.
Positiviy requires that these are the only (internal) types which are allowed in
the domain of functions.
An operation \grm{\El} reifies these small types to big types, making our version
of universe what is commonly referred to as ``Tarski-style universe'' (cf. ~\cite{luotarski}):
\begin{equation*}
\inferrule{\grm{\vdash \Gamma}}{\grm{\Gamma \vdash \UU :: \Sc}}
\qquad
\inferrule{\grm{\Gamma \vdash a : \UU}}{\grm{\Gamma \vdash \El(a) :: \Pc}}
\end{equation*}

For sorts which are type families over other sorts that we seek to define, and for
constructors which recursively refer to other constructors, we need $\Pi$-types
which have a small type as their codomain.
Note that whether we want to build a sort or a point type only depends on the
kind of the \emph{codomain} of such a $\Pi$-type.
To eliminate from $\Pi$-types we want a rule for its \emph{application} which
turns a term of a $\Pi$-type into a term of its codomain:
\begin{equation*}
\inferrule{\grm{\Gamma \vdash a : \UU} \\
  \grm{\Gamma, \El(a) \vdash B :: k}}
  {\grm{\Gamma \vdash \Pi(a, B) :: k}}
\qquad
\inferrule{\grm{\Gamma \vdash f : \Pi(a, B)}}
  {\grm{\Gamma, \El(a) \vdash \IIapp(f) : B}}
\end{equation*}

Additionally, we want sorts to be able to be parameterized by previously defined
types which are not part of the signature itself.
The same goes for point constructors.
Since this cannot be captured using the previous $\Pi$-type, we will do the obvious
and just introduce another type former for this occasion.
We will usuall call it \emph{external} or \emph{non-recursive} function type.
Note that external functions must have a fixed kind.
This is to prevent a function which, depending on the input returns sometimes
a sort and sometimes a point constructor.
\begin{equation*}
\inferrule{T : \UU \\
  (\bltau : T) \to \grm{\Gamma \vdash B(\bltau) :: k}}
  {\grm{\Gamma \vdash \ExtPi{T}{B} :: k}} \\
\qquad
\inferrule{\grm{\Gamma \vdash f : \ExtPi{T}{B}} \\
  \tau : T}
  {\grm{\Gamma \vdash f(\bltau) : B(\bltau)}}
\end{equation*}

Since we are working with explicit substitutions, we need to postulate a calculus
for substitutions \grm{\IISub{\sigma}{\Gamma}{\Delta}} between any two contexts
\grm{\Gamma} and \grm{\Delta}.
The substitutions should form a category as postulated by the following rules:
\begin{equation*}
\inferrule{\grm{\vdash \Gamma}}
  {\grm{\IISub{\id}{\Gamma}{\Gamma}}}
\qquad
\inferrule{\grm{\IISub{\sigma}{\Delta}{\Sigma}} \\
  \grm{\IISub{\delta}{\Gamma}{\Delta}}}
  {\grm{\IISub{\sigma \circ \delta}{\Gamma}{\Sigma}}}
\end{equation*}
\begin{align*}
\grm{\id \circ \sigma} &= \grm{\sigma} \\
\grm{\sigma \circ \id} &= \grm{\sigma} \\
\grm{(\sigma \circ \delta) \circ \gamma} &= \grm{\sigma \circ (\delta \circ \gamma)}
\end{align*}

We can pull back types and terms along substitutions, and these pullbacks are functorial
in the categorical structure:
\begin{equation*}
\inferrule{\grm{\Delta \vdash A :: k} \\
  \grm{\IISub{\sigma}{\Gamma}{\Delta}}}
  {\grm{\Gamma \vdash A[\sigma] :: k}}
\qquad
\inferrule{\grm{\Delta \vdash t : A} \\
  \grm{\IISub{\sigma}{\Gamma}{\Delta}}}
  {\grm{\Gamma \vdash t[\sigma] : A[\sigma]}}
\end{equation*}
\begin{align*}
\grm{A[\id]} &= \grm{A} \\
\grm{A[\sigma \circ \delta]} &= \grm{A[\sigma][\delta]} \\
\grm{t[\id]} &= \grm{t} \\
\grm{t[\sigma \circ \delta]} &= \grm{t[\sigma][\delta]}
\end{align*}

We have a canonical substitution into the empty context and we can extend substitutions
by giving a term in a type bulled back to their domain.
\begin{equation*}
\inferrule{\grm{\vdash \Gamma}}
  {\grm{\IISub{\epsilon}{\Gamma}{\cdot}}}
\qquad
\inferrule{\grm{\IISub{\sigma}{\Gamma}{\Delta}} \\
  \grm{\Delta \vdash A} \\
  \grm{\Gamma \vdash t : A[\sigma]}}
  {\grm{\IISub{\sigma, t}{\Gamma, A}{\Delta}}}
\end{equation*}
Empty substituition and extension simplify by the following laws:
\begin{align*}
\grm{\sigma} &= \grm{\epsilon}
  &\text{ for all \grm{\IISub{\sigma}{\Gamma}{\cdot}}, and} \\
\grm{(\delta , t) \circ \sigma} &= \grm{(\delta \circ \sigma), t[\sigma]}
  &\text{ for \grm{\IISub{\delta}{\Gamma}{\Delta}}, \grm{\IISub{\sigma}{\Sigma}{\Gamma}}.}
\end{align*}
A substitution into an extended context allows us to project out
a ``shorter'' substitution and a term in the terminal component:
\begin{equation*}
\inferrule{\grm{\IISub{\sigma}{\Gamma}{\Delta,A}}}
  {\grm{\IISub{\pi_1(\sigma)}{\Gamma}{\Delta}}}
\qquad
\inferrule{\grm{\IISub{\sigma}{\Gamma}{\Delta,A}}}
  {\grm{\Gamma \vdash \pi_2(\sigma) : A[\pi_1(\sigma)]}} \text{, with}
\end{equation*}
\begin{align*}
\grm{\pi_1(\sigma, t)} &= \grm{\sigma} \text{,} \\
\grm{\pi_2(\sigma, t)} &= \grm{t} \text{, and} \\
\grm{(\pi_1(\sigma), \pi_2(\sigma))} &= \grm{\sigma} \text{.}
\end{align*}
Finally, we also need rules that tell us, how the constructors of the universe
and the $\Pi$-types behave when pulled back along an arbitrary substitution
\grm{\IISub{\sigma}{\Gamma}{\Delta}}:
\begin{align*}
\grm{\UU[\sigma]} &= \grm{\UU} \text{,} \\
\grm{\El(a)[\sigma]} &= \grm{\El(a[\sigma])} \text{,} \\
\grm{\Pi(a, B)[\sigma]} &= \grm{\Pi(a[\sigma],B[\sigma \wedge \El(a)])} \text{,} \\
\grm{\IIapp(f)[\sigma \wedge El(a)]} &= \grm{\IIapp(f[\sigma])} \text{,} \\ %TODO use a better symbol than wedge
\grm{\ExtPi{T}{B}[\sigma]} &= \grm{\ExtPi{T}{\blm{\lambda \tau. \grm{B(\bltau)[\sigma]}}}} \text{, and} \\
\grm{f(\bltau)[\sigma]} &= \grm{f[\sigma](\bltau)} \text{.}
\end{align*}

\begin{defn}
Above, \grm{\sigma \wedge A} is one of several auxiliary constructions on the syntax
which are helpful when dealing with substitutions and which can be derived from
the other rules.

The first one is the operation known as \emph{weakening} which for any \grm{\Gamma \vdash A}
gives a substition \grm{\IISub{\wk}{\Gamma, A}{\Gamma}} from the extended into
the original context by \grm{\wk := \pi_1(\id)}.

Likewise, we can apply the second projection to the identity substitution such that
whenever \grm{\Gamma \vdash A}, we have the first \emph{variable} of the context
$\grm{\vz} := \grm{\pi_2(\id)}$ with \grm{\Gamma, A \vdash \vz : A[\wk]}.
Transporting a term \grm{\Gamma \vdash t : A} along the weakening substitution
defined above for any \grm{\Gamma \vdash B}, we get \grm{\Gamma, B : t[\wk] : A[\wk]}.
We will write $\grm{\vs(t)} := \grm{t[\wk]}$ for this variable.
Together, \grm{\vz} and \grm{\vs} form \emph{typed de Bruijn indices} to select
variables from a context via numbering them with a zero (\grm{\vz}) and a
successor (\grm{\vs}).

For \grm{\IISub{\sigma}{\Gamma}{\Delta}} and \grm{\Gamma \vdash A} we can
``lift'' \grm{\sigma} along \grm{A} to get a substitution
\grm{\IISub{\sigma \wedge A}{\Gamma, A[\sigma]}{\Delta, A}}.
This operation can be defined by
\begin{equation*}
\grm{\sigma \wedge A} := \grm{\sigma \circ \wk, \vz(A[\sigma])} \text{.}
\end{equation*}
At last, every term \grm{\Gamma \vdash t : A} gives rise to a substitution
\grm{\IISub{\langle t \rangle}{\Gamma}{\Gamma, A}}, representing the extension
of \grm{\Gamma} by \grm{t}, via
\begin{equation*}
\grm{\langle t \rangle} := \grm{id, t} \text{.}
\end{equation*}
\end{defn}

\begin{defn}[Application]
Most often, we want to use the application of the inductive function type in the
form where we give the function term and its input separately, as in the following
rule:
\begin{equation*}
\inferrule{\grm{\Gamma \vdash f : \Pi(a, B)} \\
  \grm{\Gamma \vdash u : \El(a)}}
  {\grm{\Gamma \vdash f(u) : B[\sigma]}}
\text{,}
\end{equation*}
for some substitution \grm{\sigma}.
Now we know that we can define this substitution by
\begin{equation*}
\grm{f(u)} := \grm{\IIapp(f)[\langle u \rangle]} \text{,}
\end{equation*}
and $\grm{\sigma} = \grm{\langle u \rangle}$.
\end{defn}

We will now look at a few example to make it clearer on how to encode inductive-inductive
declarations in the syntax presented above. %TODO more fluff
To see what the different function types are used for, consider the following two
examples:

\begin{example}[Natural numbers]\label{ex:ii-syntax-nat}
The encoding of the natural numbers as would correspond to the
following source type theory context using the first :
\begin{equation*}
\grm{
\cdot,\, \UU,\, \El(\vz),\, \ExtPi{\vs(\vz)}{\El(\vs(\vs(\vz)))}
}\text{.}
\end{equation*}
Often, we will, instead of denoting variables using de~Bruijn indices, use names
as binders in contexts and domains of $\Pi$-types to make example contexts more
legible.
Assuming we always use fresh names, this is not any more imprecise than restricting
ourselves to use \grm{\vz} and \grm{\vz} instead:
\begin{equation*}
\grm{
\cdot,\, \N : \UU,\, 0 : \El(\N),\, \mathop{S} : \ExtPi{n : \N}{\El(\N)}
}\text{.}
\end{equation*}
\end{example}

\begin{example}[Vectors]\label{ex:ii-syntax-vec}
The type of vectors over a type $A : \UU$ can be represented using the ``external''
natural numbers $\N$.
In the following, the constructor \grm{\mathop{cons}} uses both the non-inductive
and the inductive function type:
\begin{equation*}
\begin{gathered}
\grm{ \cdot,\, \mathop{vec} : \ExtPi{\N}{\lambda n.\, \UU},\, \mathop{nil} : \El(\mathop{vec}(\blm{0})),} \\
\grm{ \mathop{cons} : \ExtPi{A}{\blm{\lambda a.\,
  \grm{\ExtPi{\N}{\blm{\lambda n.\, \grm{\Pi(v : \mathop{vec}(\blm{n}),\, \El(\mathop{vec}(\blm{n + 1})))}}}}}}}
\end{gathered}
\end{equation*}
With de-Bruijn indices instead of names the signature \grm{\Gamma_{vec}} would be
\begin{equation*}
\begin{gathered}
\grm{ \cdot,\, \ExtPi{\N}{\lambda n.\, \UU},\, \El(\vz(\blm{0})),} \\
\grm{ \ExtPi{A}{\blm{\lambda a.\,
  \grm{\ExtPi{\N}{\blm{\lambda n.\, \grm{\Pi(\vs(\vz)(\blm{n}),\, \El(\vs(\vs(\vz))(\blm{n + 1})))}}}}}}}
\end{gathered}
\end{equation*}
%An example for a type which uses the third, small function type is the following
%definition of full $\omega$-ary rooted trees:
%\begin{equation*}
%\grm{
%\cdot,\, T : \UU,\, \mathop{leaf} : \underline{T},\,
%	\mathop{node} : (\blm{\N} \to T) \to \underline{T}
%} \text{.}
%\end{equation*}
\end{example}

\begin{example}[Type Theory Syntax]
The example of the syntax of type theory~\ref{ex:ttintt} is represented by the
following signature \grm{\Gamma_{ConTy}}:
\begin{equation*}
\begin{gathered}
\grm{\cdot,\, \mathop{Con} : \UU,\, \mathop{Ty} : \Pi(\Gamma : \mathop{Con},\, \UU),} \\
\grm{\mathop{nil} : \El(\mathop{Con}),} \\
\grm{\mathop{ext} : \Pi(\Gamma : \mathop{Con},\, \Pi(A : \mathop{Ty} ,\, \El(\mathop{Con}))),} \\
\grm{\mathop{unit} : \Pi(\Gamma : \mathop{Con},\, \El(\IIapp(\mathop{Ty}))),} \\
\grm{\mathop{pi} : \Pi(\Gamma : \mathop{Con},\, \Pi(A : \IIapp(\mathop{Ty}),\,
   \Pi(B : ?,\, ?)))} %TODO complete this after doing it in agda
\end{gathered}
\end{equation*}

\end{example}

\section{Algebras of Inductive-Inductive Types}

To give meaning to the codes expressed in the source type theory, we need to
interpret the contexts as a type in the ambient type theory whose elements are
the algebras of of the specified inductive-inductive type.
This means that the interpretation of our contexts must give the types of the
sort and point constructors they specify.

\begin{defn}[Algebra operator] %TODO name?
By structural recursion over the source syntax, we define an operation $-^\CC$
which assigns types to source contexts, fibrations over those to types, sections of
these fibrations to terms, and maps between types to substitutions:
\begin{equation*}
\begin{gathered}
\inferrule{\grm{\vdash \Gamma}}{\grm{\Gamma}^\CC : \UU_1}
\qquad
\inferrule{\grm{\Gamma \vdash A :: \Sc}}{\grm{A}^\CC : \grm{\Gamma}^\CC \to \UU_1}
\qquad
\inferrule{\grm{\Gamma \vdash A :: \Pc}}{\grm{A}^\CC : \grm{\Gamma}^\CC \to \UU_0}
\\[.7em]
\inferrule{\grm{\Gamma \vdash t : A}}
  {\grm{t}^\CC : (\gamma : \grm{\Gamma}^\CC) \to \grm{A}^\CC(\gamma)}
\qquad
\inferrule{\grm{\IISub{\sigma}{\Gamma}{\Delta}}}
  {\grm{\sigma}^\CC : \grm{\Gamma}^\CC \to \grm{\Delta}^\CC}
\end{gathered}
\end{equation*}

We will give the construction on contexts, types, subsitutions and terms in the
same order as there were presented in Section~\ref{sec:ii-syntax}.
On contexts, the operation is defined by iterated $\Sigma$-types:
\begin{align*}
\grm{\cdot}^\CC &:\equiv \unit \text{ and} \\
\grm{(\Gamma,\,A)}^\CC &:\equiv (\gamma : \grm{\Gamma}^\CC) \times \grm{A}^\CC(\gamma)
\end{align*}

The universe in the syntax needs of course be mapped to the metatheoretic universe.
We chose $\UU_1$ as a target for context interpretation to make sure $\UU_0$ fits
in there.
The operation \grm{\El} is just there to make the conversation between small and
big types, which in turn is needed to ensure positivity of the constructors.
Since this distinction doesn't have any semantic meaning, \grm{\El} will just be
ignored by the algebra operator:
\begin{align*}
\grm{\UU}^\CC(\gamma) 			&:\equiv \UU_0 \\
\grm{(\El(a))}^\CC(\gamma)		&:\equiv \grm{a}^\CC(\gamma)
\end{align*}
Recursive $\Pi$-types become metatheoretic dependent function spaces, with \grm{\IIapp} the
usual function application:
\begin{align*}
\grm{\Pi(a, B)}^\CC(\gamma)		&:\equiv (\alpha : \grm{a}^\CC(\gamma)) \to \grm{B}^\CC(\gamma, \alpha) \\
\grm{\IIapp(t)}^\CC(\gamma, \alpha)	&:\equiv \grm{t}^\CC(\gamma, \alpha)
\end{align*}
Non-recursive $\Pi$-types also become functions, but here we have to apply the
external argument to the codomain to be able to evaluate its interpretation:
\begin{align*}
\grm{\ExtPi{T}{B}}^\CC(\gamma)		&:\equiv (\tau : T) \to B(\bltau)^\CC(\gamma) \\
\grm{f(\bltau)}^\CC(\gamma)		&:\equiv \grm{f}^\CC(\gamma, \tau)
\end{align*}

Unsurprisingly the category structure of substitution is achieved by interpreting
it into the one of metatheoretic functions between context interpretations:
\begin{align*}
\grm{\id}^\CC(\gamma)			&:\equiv \gamma \\
\grm{(\sigma \circ \delta)}^\CC(\gamma)	&:\equiv \grm{\sigma}^\CC(\grm{\delta}^\CC(\gamma))
\end{align*}
Pulling back a type or a term along a substitution means interpreting it after
applying the function which we get from interpreting the substitution:
\begin{align*}
\grm{A[\sigma]}^\CC(\gamma)		&:\equiv \grm{A}^\CC(\grm{\sigma}^\CC(\gamma)) \\
\grm{t[\sigma]}^\CC(\gamma)		&:\equiv \grm{t}^\CC(\grm{\sigma}^\CC(\gamma))
\end{align*}
The interpretation of the empty substitution is the unique map into the interpretation
of the empty context.
Extension of and projections from a substitution now justify their name by being
interpreted as the extension of a function and the projections of a $\sigma$-type:
\begin{align*}
\grm{\epsilon}^\CC(\gamma)		&:\equiv \star \\
\grm{(\sigma, t)}^\CC(\gamma)		&:\equiv (\grm{\sigma}^\CC(\gamma), \grm{t}^\CC(\gamma)) \\
\grm{\pi_1(\sigma)}^\CC(\gamma)		&:\equiv \pr_1(\grm{\sigma}^\CC(\gamma)) \\
\grm{\pi_2(\sigma)}^\CC(\gamma)		&:\equiv \pr_2(\grm{\sigma}^\CC(\gamma))
\end{align*}

All the rules of the substitution calculus which were mentioned in Section~\ref{sec:ii-syntax}
are preserved \emph{definitionally}, which, in the end, is due to types and
functions forming a \emph{strict} category. %TODO more explanation here

%\begin{align*}
%\grm{(t \app \blm{u})}^\CC(\gamma)	&:\equiv (\grm{t}^\CC(\gamma))(u) \text{.}
%\end{align*}

\end{defn}

\begin{example}[Natural numbers]
Strictly speaking, the algebra interpretation of the
context from our example of natrual numbers (Example~\ref{ex:ii-syntax-nat})
would compute to the following iterated $\Sigma$-type:
\begin{equation*}
\grm{\left(N' : (N : \top \times \UU) \times \pr_2(N)\right)
  \times \left(\pr_2(\pr_1(N')) \to \pr_2(\pr_1(N')) \right) } \text{.}
\end{equation*}
Obviously, this is unnecessarily complicated and we can easily transform this
type using equivalences to see that we can also express the algebras as being
elements of the type
\begin{equation*}
\grm{(N : Set) \times N \times (N \to N)} \text{.}
\end{equation*}
\end{example}

%TODO more examples

\section{Initiality}

\begin{defn}[Initial algebra]
For a source context \grm{\vdash \Gamma}, an algebra $s : \grm{\Gamma}^\CC$ is
called \textbf{initial}, if there is a \emph{unique} algebra morphism from $s$ to any
other algebra, which is to say that the following two rules hold true:
\begin{equation*}
\begin{gathered}
\inferrule{a : \grm{\Gamma}^\CC}{m_a : \grm{\Gamma}^\mm(s, a)}
\qquad
\inferrule{a : \grm{\Gamma}^\CC \\ m' : \grm{\Gamma}^\mm(s, a)}{m' = m_a}
\end{gathered}
\end{equation*}
We will call the first rule \emph{weak initiality} and the second one \emph{universality}
of the algebra.
\end{defn}

\section{Morphisms of Algebras}

%TODO

\section{Displayed Algebras and Sections}

%TODO

\section{Existence of Inductive-Inductive Types}

%TODO

\section{Initiality implies Elimination}

TODO: use sigma construction for this, cite whoever's done this first









%\section{(OLD) --The Source Type Theory}

Since for some examples of inductive-inductive types, especially ones which are
\emph{infinitely branching}, we need functions with external domain as the domains of
other functions, we will also add another function type which is itself small, and
has external domain and small codomain:
\begin{equation*}
\begin{gathered}
\inferrule{\grm{\vdash \Gamma} \\ A : \UU_i \\ (x : A) \to (\grm{\Gamma \vdash b : \UU})}
	{\grm{\Gamma \vdash ((\blm{x : A}) \to b) : \UU}}
\qquad
\inferrule{\grm{\Gamma \vdash t : \underline{(\blm{x : A}) \to b}} \\ u : A}
	{\grm{\Gamma \vdash (t \app \blm{u}) : \underline{b[\blm{x} \mapsto \blm{u}]}}}
\end{gathered}
\end{equation*}

\begin{example}
To see what the different function types are used for, consider the following three
examples:

The encoding of the natural numbers as would correspond to the
following source type theory context using the first :
\begin{equation*}
\grm{
\cdot,\, \N : \UU,\, 0 : \underline{\N},\, \mathop{S} : (n : \N) \to \underline{\N}
}\text{.}
\end{equation*}

The type of vectors over a type $A : \UU$ can be represented using the ``external''
natural numbers $\N$.
In the following, the constructor \blm{\mathop{cons}} uses both the non-inductive
and the inductive function type:
\begin{equation*}
\grm{
\cdot,\, \vecty : \blm{\N} \to \UU,\, \mathop{nil} : \underline{\vecty(\blm{0})},\,
	\mathop{cons} : (\blm{n : \N})(\blm{a : A})(v : \vecty(\blm{n})) \to \underline{\vecty(\blm{S(n)})}
}
\end{equation*}

An example for a type which uses the third, small function type is the following
definition of full $\omega$-ary rooted trees:
\begin{equation*}
\grm{
\cdot,\, T : \UU,\, \mathop{leaf} : \underline{T},\,
	\mathop{node} : (\blm{\N} \to T) \to \underline{T}
} \text{.}
\end{equation*}
\end{example}

\begin{example}[Type Theory Syntax]
The example of the syntax of type theory~\ref{ex:ttintt} is represented by the
following code:
\begin{multline*}
\grm{\cdot,\, \mathop{Con} : \UU,\, \mathop{Ty} : \mathop{Con} \to \UU,} \\
\grm{\mathop{nil} : \mathop{Con},} \\
\grm{\mathop{ext} : (\Gamma : \mathop{Con})(A : \mathop{Ty}(\Gamma)) \to \underline{\mathop{Con}},} \\
\grm{\mathop{unit} : (\Gamma : \mathop{Con}) \to \underline{\mathop{Ty}(\Gamma)},} \\
\grm{\mathop{pi} : (\Gamma : \mathop{Con})(A : \mathop{Ty}(\Gamma))(B : \mathop{Ty}(\mathop{ext}(\Gamma, A))) \to \underline{\mathop{Ty}(\Gamma)}}
\end{multline*} %TODO find a better way to typeset this!
\end{example}

\section{(OLD) --Algebras for the Codes}

To give meaning to the codes expressed in the source type theory, we need to
interpret the contexts as a type in the ambient type theory whose elements are
the algebras of of the specified inductive-inductive type.
This means that the interpretation of our contexts must give the types of the
sort and point constructors they specify.

\begin{defn}[Constructor operator]
By structural recursion over the source syntax, we define an operation $-^\CC$
which assigns types to source contexts, fibrations over those to types and
sections to terms:
\begin{equation*}
\begin{gathered}
\inferrule{\grm{\vdash \Gamma}}{\grm{\Gamma}^\CC : \UU_1}
\qquad
\inferrule{\grm{\Gamma \vdash A}}{\grm{A}^\CC : \grm{\Gamma}^\CC \to \UU_1}
\qquad
\inferrule{\grm{\Gamma \vdash t : A}}{\grm{t}^\CC : (\gamma : \grm{\Gamma}^\CC) \to \grm{A}^\CC(\gamma)}
\end{gathered}
\end{equation*}

We will give the construction on contexts, types, and terms, while relying on
the freshness of names to make sure that the construction respects the substitution
calculus which we left implicit in this presentation. On contexts, the operation
is defined by iterated $\Sigma$-types:
\begin{align*}
\grm{\cdot}^\CC &:\equiv \unit \text{ and} \\
\grm{(\Gamma, x : A)}^\CC &:\equiv (\gamma : \grm{\Gamma}^\CC) \times \grm{A}^\CC(\gamma)
\end{align*}
Alternatively, context interpretation $\grm{(\cdot, x_1 : A_1, \ldots, x_n : A_n)}^\CC$
can be considered a \emph{record type} with fields $x_1 : \grm{A_1}^\CC, \ldots, x_n : \grm{A_n}^\CC$.

Sorts will be interpreted as functions into the universe, constructors as inhabiting
these functions:
\begin{align*}
\grm{\UU}^\CC(\gamma) 			&:\equiv \UU_0 \text{,} \\
\grm{(\underline{a})}^\CC(\gamma)	&:\equiv \grm{a}^\CC(\gamma) \text{,} \\
\grm{((x : a) \to B)}^\CC(\gamma)	&:\equiv (x : \grm{a}^\CC(\gamma)) \to \grm{B}^\CC(\gamma, x) \text{, and} \\ %TODO notation?
\grm{((\blm{x : A}) \to B)}^\CC(\gamma) &:\equiv (x : A) \to \grm{B}^\CC(\gamma) \text{.}
\end{align*}
The interpretation of terms is the straightforward translation into the type interpretation
above:
\begin{align*}
\grm{x}^\CC(\gamma) 			&:\equiv \gamma.\grm{x} \qquad \text{for variables \grm{x},} \\
\grm{(t(u))}^\CC(\gamma)		&:\equiv (\grm{t}^\CC(\gamma))(\grm{u}^\CC(\gamma)) \text{,} \\
\grm{((\blm{x : A}) \to b)}^\CC(\gamma)	&:\equiv (x : A) \to \grm{b}^\CC(\gamma) \text{,} \\
\grm{(t(\blm{u}))}^\CC(\gamma)		&:\equiv (\grm{t}^\CC(\gamma))(u) \text{, and} \\
\grm{(t \app \blm{u})}^\CC(\gamma)	&:\equiv (\grm{t}^\CC(\gamma))(u) \text{.}
\end{align*}

\end{defn}

TODO: Add some examples here.

\section{(OLD) --Motives and Methods}

TODO: Add definition of $\grm{-}^\MM$.

TODO: Add examples.

\section{(OLD) --Recursion and Computation}

TODO: Add definition of $\grm{-}^\EE$.

TODO: Add examples.

\section{(OLD) --Existence of HIITs}

\begin{defn}[Dependent Eliminator]
An algebra $c : \grm{\Gamma}^\CC$ is said to admit \emph{dependent elimination}
if for each motive $m : \grm{\Gamma}^\MM(c)$ there is a dependent eliminator
$\elim_{\grm{\Gamma}}(m) : \grm{\Gamma}^\EE(c, m)$.
\end{defn}

\begin{thm}[Admissibility of Inductive-Inductive Types]
Our type theory admits inductive-inductive types if for each wellformed context
$\grm{\Gamma}$ we can find a constructor $\con{\grm{\Gamma}}$ which admits dependent
elimination $\elim_{\grm{\Gamma}}$.
\end{thm}

\section{(OLD) --Morphisms of Algebras}

In the following, it will be useful to regard the algebras of a given context
\grm{\Gamma} as a category.
To this end, we need to make clear what a morphism between \grm{\Gamma}-algebras
is.
Intutitively a morphism is given by maps between the interpretations of the sorts,
together with evidence that those maps preserve the interpretation of point
constructors.

\begin{defn}[Morphisms of Algebras]
We want to define the following by mutual recursion on contexts, types and terms:
\begin{equation*}
\begin{gathered}
\inferrule{\grm{\vdash \Gamma} \\ \gamma_0, \gamma_1 : \grm{\Gamma}^\AA}
	{\grm{\Gamma}^\mm(\gamma_0,\gamma_1) : \UU} \\[.7em]
\inferrule{\grm{\Gamma \vdash A} %\\ \gamma_0, \gamma_1 : \grm{\Gamma}^\AA TODO decide if show
		\\ g : \grm{\Gamma}^\mm(\gamma_0, \gamma_1) \\
		\\ \alpha_0 : \grm{A}^\AA(\gamma_0) \\ \alpha_1 : \grm{A}^\AA(\gamma_1)}
	{\grm{A}^\mm(g, \alpha_0, \alpha_1) : \UU} \\[.7em]
\inferrule{\grm{\Gamma \vdash t : A} \\ \gamma_0, \gamma_1 : \grm{\Gamma}^\AA 
		\\ g : \grm{\Gamma}^\mm(\gamma_0, \gamma_1)}
	{\grm{t}^\mm(g) : \grm{A}^\mm(g, \grm{t}^\AA(\gamma_0), \grm{t}^\AA(\gamma_1))}
\end{gathered}
\end{equation*}

Like we did for the definition of algebras, we want morphisms of contexts to be
just iterated $\Sigma$-types of the respective interpretation of types:
\begin{align*}
\grm{\cdot}^\mm(\gamma_0, \gamma_1) &:\equiv \unit \text{ and} \\
\grm{(\Gamma, x : A)}^\mm(\gamma_0, \gamma_1) &:\equiv
	(g : \grm{\Gamma}^\mm(\pr_1(\gamma_0), \pr_1(\gamma_1))) \times \grm{A}^\mm(g, \pr_2(\gamma_0), \pr_2(\gamma_1)) \text{.}
\end{align*}
The core of the definition on sort types is that the universe is interpreted as
a function space:
\begin{align*}
\grm{\UU}^\mm(g, \alpha_0, \alpha_1)  			&:\equiv \alpha_0 \to \alpha_1 \text{,} \\
\grm{(\underline{a})}^\mm(g, \alpha_0, \alpha_1)	&:\equiv (\grm{a}^\mm(g, \alpha_0) = \alpha_1) \text{,} \\
\grm{((x : a) \to B)}^\mm(g, \alpha_0, \alpha_1)	&:\equiv (x : \grm{a}^\CC(\gamma_0))
							\to \grm{B}^\mm((g, \refl), \alpha_0(x), \alpha_1(\grm{a}^\mm(g, x))) \text{,} \\
\grm{((\blm{x : A}) \to B)}^\mm(g, \alpha_0, \alpha_1)  &:\equiv (x : A)
							\to \grm{(B(\blm{x}))}^\mm(g, \alpha_0(x), \alpha_1(x)) \text{.}
\end{align*}
On terms, consider the foo
\begin{align*}
\grm{x}^\mm(g) 				&:\equiv g.\grm{x} \qquad \text{for variables \grm{x},} \\ %???
\grm{(t(u))}^\mm(g)			&:\equiv (\grm{u}^\mm(g))_* (\grm{t}^\mm(g)(\grm{u}^\AA(\gamma_0))) \text{,} \\ %TODO check this
\grm{((\blm{x : A}) \to b)}^\mm(g)	&:\equiv (x : A) \to \grm{b}^\mm(g) \text{,} \\
\grm{(t(\blm{u}))}^\mm(g)		&:\equiv \grm{t}^\mm(g)(u) \text{, and} \\
\grm{(t \app \blm{u})}^\mm(g)		&:\equiv (\grm{t}^\mm(g))(u) \text{.} %TODO use happly here sometimes!!! uargh
\end{align*}

\end{defn}

TODO: Add examples.



\chapter{Specification of Inductive Families}

%TODO's as per thorsten july 9
%* code snippet besser teasern 

As we have already seen in the last chapter, sometimes it is helpful to allow
point constructors of a collection of inductive types to be \emph{mutually dependent}.
This means that to define several sorts simultaneously whose point constructors
may refer to the other types being defined.
We will refer to this class as \emph{inductive families}, though others might
call them, for example, mutual inductive types.
Inductive families are a class of inductive types which at first glance seems more
powerful than indexed W-types but less than inductive-inductive types --
sorts are \emph{not} allowed to depend on other sorts but only point constructors.

Previous specifications of mutual inductive families have taken different approaches:
Some are based on the notion of a polynomial functor
\citep{indexedcontainers, dybjer1999finite}
while others, like
the original \citet{dybjer94} description are based on a schematic description.


\section{Signatures for Inductive Families}

Applying the same principle as in the case of inductive-inductive types we want
to create a specification based on the contexts of type theory syntax.
We could imagine that we can obtain such a specification by just restricting the
syntax for inductive-inductive types to not use the recursive $\Pi$-type for sorts,
but this approach doesn't capture the full %TODO
extent of inductive families being a much simpler concept than inductive-inductive
types.
Given the strategy of our reduction we want the specification to capture at least
the following features of inductive families:
\begin{itemize}
\item Sorts are either types or functions over existing types.
\item Point constructors can also be indexed over existing (``external'') types.
\item Point constructors can refer to any sort being defined.
\end{itemize}

We will use {\color{Blue}blue font} to distinguish the new syntax from the ambient
type theory.
The first point above says that we want the \emph{sort types} \blm{\tqm{\Sc}}
to be inductively generated by a \emph{universe} token and a constructor
of external functions for sorts which are meant to be \emph{type families}:
\begin{equation*}
\begin{gathered}
\inferrule{ }{\tqm{\UU :: \Sc}}
\qquad
\inferrule{T : \UU \\ \tqm{B : \blm{T \to \tqm{\Sc}}}}{\tqm{\ExtPiS{T}{B :: \Sc}}}
\end{gathered}
\end{equation*}
For example, the sort of vectors over a type $A : \UU$ would be described by
\tqm{\ExtPiS{A}{\ExtPiS{\N}{\UU}}}.
Note that in contrast to the sort types of inductive-inductive definitions these
do not depend on a context.

Instead, we say that a \emph{sort context} is just a list of sort types without
any interdependencies:
\begin{equation*}
\begin{gathered}
\inferrule{}{\tqm{\SCon \cdot_\Sc}}
\qquad
\inferrule{\tqm{\SCon \Gamma_\Sc} \\ \tqm{B :: \Sc}}{\tqm{\SCon \Gamma_\Sc, B}}
\end{gathered}
\end{equation*}

In order to refer to sorts we introduce a simplified term calculus based on typed
de Bruijn indices for bound variables and an application operation for type families:
\begin{equation*}
\begin{gathered}
\inferrule{\tqm{\SCon \Gamma_\Sc} \\ \tqm{B :: \Sc}}{\tqm{\Gamma_\Sc, B \SCon \var(\vz) : B}}
\qquad
\inferrule{\tqm{\Gamma_\Sc \SCon \var(v) : B}}{\tqm{\Gamma_\Sc, B' \SCon \var(\vs(v)) : B}}
\\[.7em]
\inferrule{\tqm{\Gamma_\Sc \SCon t : \ExtPiS{T}{B}} \\ \blm{\tau : T}}
  {\tqm{\Gamma_\Sc \SCon t(\blm{\tau}) : B(\blm{\tau})}}
\end{gathered}
\end{equation*}

Point constructors will be represented by \emph{point types} over a given sort
context.
This means that in contrast to inductive-inductive types, they cannot depend on
other point types.
The type formers we need are the element type for the universe \tqm{\UU}, an
external, non-recursive function type like the one we have for sorts, and an
internal function type used for recursive point constructors -- which are
\emph{non-dependent} since point constructors only depend on the \emph{sort context}:
\begin{equation*}
\begin{gathered}
\inferrule{\tqm{\Gamma_\Sc \SCon a : \UU}}{\tqm{\Gamma_\Sc \SCon \El(a)}}
\qquad
\inferrule{\blm{T : \UU} \\ \blm{(\tau : T) \to \tqm{\Gamma_\Sc \SCon B(\blm{\tau})}}}
  {\tqm{\Gamma_\Sc \SCon \ExtPiP{T}{B}}}
\\[.7em]
\inferrule{\tqm{\Gamma_\Sc \SCon a : \UU} \\ \tqm{\Gamma_\Sc \SCon A}}
  {\tqm{\Gamma_\Sc \SCon a \Rightarrow_\Pc A}}
\end{gathered}
\end{equation*}

As a last building block of the syntax, we can now form contexts consisting
of point constructors over a given sort context.
Such a context \tqm{\Gamma} can be formed over a given sort context \tqm{\Gamma_\Sc}
which we will denote as a subscript to the turnstile or omit when inferrable.
%The empty context can be formed over the empty sort context, an extension of
%a context by a sort constructor happens in parallel to an extension of its sort
%context, and an extension by a point constructor leaves the sort context fixed:
The empty context can be formed over any sort context, and an extension by a point
constructor leaves the sort context fixed:
\begin{equation*}
\begin{gathered}
\inferrule{\tqm{\SCon \Gamma_\Sc}}{\tqm{\vdash_{\Gamma_\Sc} \Gamma}}
%\inferrule{}{\tqm{\vdash_{\cdot_\Sc} \cdot}}
\qquad
%\inferrule{\tqm{\vdash_{\Gamma_\Sc} \Gamma} \\ \tqm{B : \Sc}}
%  {\tqm{\vdash_{\Gamma_\Sc, B} \Gamma, B}}
%\qquad
\inferrule{\tqm{\vdash_{\Gamma_\Sc} \Gamma} \\ \tqm{\Gamma_\Sc \SCon A}}
  {\tqm{\vdash_{\Gamma_\Sc} \Gamma, A}}
\end{gathered}
\end{equation*}

\begin{example}[Natural numbers, Vectors]\label{ex:if-natvec}
A common example for inductive types, the natural numbers, with one constructor for
zero and one for the successor function, are represented by the sort context
\tqm{\cdot_\Sc,\, \UU} and the points
\begin{equation*}
\tqm{\El(\var(\vz)), \var(\vz) \Rightarrow_\Pc \El(\var(\vz))} \text{.}
\end{equation*}

An example of a real indexed type would be the type family of vectors over a fixed
type \blm{A : \UU} which is defined over the sort context
\tqm{\cdot_\Sc,\, \ExtPiS{n : \N}{\UU}} by
\begin{equation*}
\begin{gathered}
\tqm{\cdot, \El(\var(\vz)(\blm{0})),}\\
\tqm{\ExtPiP{a : A}{\ExtPiP{n : \N}{\var(\vz)(\blm{n}) \Rightarrow_\Pc \El(\var(\vz)(\blm{n + 1}))}}} \text{.}
\end{gathered}
\end{equation*}

An easy example with non-trivial mutual dependency between the point constructors
is the predicate of evenness and oddness on natural numbers: The sorts are
represented by \tqm{\cdot_\Sc, \Pi_\Sc(\blm{\N}, \blm{\lambda n. \tqm{\UU}}), \Pi_\Sc(\blm{\N}, \blm{\lambda n. \tqm{\UU}})}
and the point constructors by
\begin{equation*}
\begin{gathered}
\tqm{\cdot , \El(\var(\vs(\vz))(\blm{0})),} \\
\tqm{\ExtPiP{n : \N}{\var(\vz)(\blm{n}) \Rightarrow_\Pc \El(\var(\vs(\vz))(\blm{n + 1}))},} \\
\tqm{\ExtPiP{n:\N}{\var(\vs(\vz))(\blm{n}) \Rightarrow_\Pc \El(\var(\vz)(\blm{n + 1}))}} \text{.}
\end{gathered}
\end{equation*}
Here, the first sort constructor represents evenness, the second one oddness,
the first point constructor the proof that \blm{0} is even and the other two the
proof that evenness implies oddness of the successor and vice versa.
\end{example}

\begin{defn}[Sort Substitutions]\label{def:if-sort-subs}
One component of the syntax which has completely gone missing are substitutions.
Since we cant refer to previous point constructors, we certainly don't need them
for the point contexts.
But since we also got rid of sort interdependencies, we could reduce the recursive
function types on points to a non-depenent one and thus don't need to use
substitutions in the definition of application.
It will still be helpful for syntax transformations to use substitutions
of the sort contexts which we define as generated by
\begin{equation*}
\begin{gathered}
\inferrule{\tqm{\SCon \Gamma_\Sc}}
  {\tqm{\IFSub{\epsilon}{\Gamma_\Sc}{\cdot_\Sc}}}
\quad\text{and}\quad
\inferrule{\tqm{\IFSub{\sigma}{\Gamma_\Sc}{\Delta_\Sc}} \\ 
  \tqm{\Gamma_\Sc \SCon t : B}}
  {\tqm{\IFSub{\sigma, t}{\Gamma_\Sc}{(\Delta_\Sc, B)}}}
\text{.}
\end{gathered}
\end{equation*}
These then allow us to substitute point types, point contexts, and sort terms via
the following ``pullback'' operations:
\begin{equation*}
\begin{gathered}
\inferrule{\tqm{\Delta_\Sc \SCon A} \\
  \tqm{\IFSub{\sigma}{\Gamma_\Sc}{\Delta_\Sc}}}
  {\tqm{\Gamma_\Sc \SCon A[\sigma]}}
\qquad
\inferrule{\tqm{\Delta_\Sc \SCon t : B} \\
  \tqm{\IFSub{\sigma}{\Gamma_\Sc}{\Delta_\Sc}}}
  {\tqm{\Gamma_\Sc \SCon t[\sigma] : B}}
\\[.7em]
\inferrule{\tqm{\vdash_{\Delta_\Sc} \Delta} \\
  \tqm{\IFSub{\sigma}{\Gamma_\Sc}{\Delta_\Sc}}}
  {\tqm{\vdash_{\Gamma_\Sc} \Delta[\sigma]}}
\end{gathered}
\end{equation*}
given by the defining rules for substitution
\begin{align*}
\tqm{\ExtPiP{T}{A}[\sigma]}			&= \tqm{\ExtPiP{T}{\blm{\lambda \tau.\, \tqm{A(\blm{\tau})[\sigma]}}}} \text{,} \\
\tqm{\El(a)[\sigma]}				&= \tqm{\El(a[\sigma])} \text{,} \\
\tqm{(a \Rightarrow_\Pc A)[\sigma]}		&= \tqm{a[\sigma] \Rightarrow_\Pc A[\sigma]} \text{,} \\
\tqm{\var(\vz)[\sigma, t]}			&= \tqm{t} \text{,} \\
\tqm{\var(\vs(t))[\sigma, s]}			&= \tqm{\var(t)[\sigma]} & \text{ for \tqm{\Delta_\Sc \SCon \var(t) : B},} \\
\tqm{f(\bltau)[\sigma]}				&= \tqm{f[\sigma](\bltau)} & \text{ for \tqm{\Delta_\Sc \SCon f : \ExtPiS{T}{B}},}\\
\tqm{\cdot[\sigma]}				&= \tqm{\cdot} \text{, and} \\
\tqm{(\Gamma,\, A)[\sigma]}			&= \tqm{(\Gamma[\sigma],\, A[\sigma])} \text{.}
\end{align*}

We can derive from this the gadgets of the substitutional calculus which we
are already acquainted with from the syntax of inductive-inductive Types:
We can define the \emph{weakening} of a subsitution
\tqm{\IFSub{\sigma}{\Gamma_\Sc}{\Delta_\Sc}} to 
\tqm{\IFSub{\wk_\sigma}{\Gamma_\Sc, B}{\Delta_\Sc}} via recursion on \tqm{\sigma} by
\begin{align*}
\tqm{\wk_\epsilon} 				&:\equiv \tqm{\epsilon} \text{ and} \\
\tqm{\wk_{\sigma, t}}				&:\equiv \tqm{(\wk_\sigma, \vs(t))} \text{.}
\end{align*}
Using \tqm{\wk}, we can then recover the categorical structure of the substitutions
by defining the identity \tqm{\IFSub{\id_{\Gamma_\Sc}}{\Gamma_\Sc}{\Gamma_\Sc}}
by recursion of the context \tqm{\Gamma_\Sc}:
\begin{align*}
\tqm{\id_{(\cdot_\Sc)}}				&:\equiv \tqm{\epsilon} \text{ and} \\
\tqm{\id_{\Gamma_\Sc,\, B}}			&:\equiv \tqm{(\wk_{\id_{\Gamma_\Sc}},\, \vz)} \text{.}
\end{align*}
Composition \tqm{\IFSub{\sigma \circ \delta}{\Gamma_\Sc}{\Sigma_\Sc}} of substitutions
\tqm{\IFSub{\sigma}{\Delta_\Sc}{\Sigma_\Sc}} and \tqm{\IFSub{\delta}{\Gamma_\Sc}{\Delta_\Sc}}
is defined by recursion on the first context:
\begin{align*}
\tqm{\epsilon \circ \delta}			&:\equiv \tqm{\epsilon} \text{,} \\
\tqm{(\sigma, t) \circ \delta}			&:\equiv \tqm{(\sigma \circ \delta, t[\delta])} \text{.}
\end{align*}
Projections \tqm{\IFSub{\pi_1(\sigma)}{\Gamma_\Sc}{\Delta_\Sc}} and
\tqm{\Gamma_\Sc \SCon \pi_2(\sigma) : B} of a substitution
\tqm{\IFSub{\sigma}{\Gamma_\Sc}{\Delta_\Sc, B}} can be defined as just that --
projections.
Any substitution between \tqm{\Gamma_\Sc} and \tqm{\Delta_\Sc,\, B} is of the form
\tqm{\sigma,\, t} and we can just set
\begin{align*}
\tqm{\pi_1(\sigma,\, t)}				&:\equiv \tqm{\sigma} \text{ and} \\
\tqm{\pi_2(\sigma,\, t)}				&:\equiv \tqm{t} \text{.}
\end{align*}
\end{defn}

%TODO address equalities

\section{Algebras of Inductive Families}\label{sec:if-alg}

Like for inductive-inductive types, we have to give a way to semantify the signatures
by stating what kind of data they should represent.

\begin{defn}[Algebra operator]
Again, sort contexts will be mapped to types, sort constructors to families over these
types, their terms to sections of these families.
Point contexts will give the same data, but depending on an interpretation of the sort
contexts:
\begin{equation*}
\begin{gathered}
\inferrule{\tqm{B :: \Sc}}
  {\tqm{B}^\CC : \UU}
\qquad
\inferrule{\tqm{\SCon \Gamma_\Sc}}
  {\tqm{\Gamma_\Sc}^\CC : \UU}
\qquad
\inferrule{\tqm{\Gamma_\Sc \SCon t : B :: \Sc}}
  {\tqm{t}^\CC : \tqm{\Gamma_\Sc}^\CC \to \tqm{B}^\CC}
\\[.7em]
\inferrule{\tqm{\Gamma_\Sc \SCon A :: \Pc}}
  {\tqm{A}^\CC : \tqm{\Gamma_\Sc}^\CC \to \UU}
\qquad
\inferrule{\tqm{\vdash_{\Gamma_\Sc} \Gamma}}
  {\tqm{\Gamma}^\CC : \UU}
\end{gathered}
\end{equation*}

Going through all of these translation in order, we first define the algebras
of sorts to be interpreted into functions over the universe:
\begin{align*}
\tqm{\UU}^\CC				&:\equiv \UU \\
\tqm{\ExtPiS{T}{B}}^\CC			&:\equiv (\tau : T) \to \tqm{B(\bltau)}^\CC
\end{align*}
Sort contexts become iterated product types -- note that we don't even need to
use $\Sigma$-types since there are no dependencies between sorts:
\begin{align*}
\tqm{\cdot_\Sc}^\CC			&:\equiv \unit \\
\tqm{(\Gamma_\Sc,\, B)}^\CC		&:\equiv \tqm{\Gamma_\Sc}^\CC \times \tqm{B}^\CC
\end{align*}
We use terms to navigate these iterated product via iterated projects, and to
apply function sorts:
\begin{align*}
\tqm{\var(\vz)}^\CC(\gamma, \alpha)	&:\equiv \alpha \\
\tqm{\var(\vs(t))}^\CC(\gamma, \alpha)	&:\equiv \tqm{\var(t)}^\CC(\gamma) \\
\tqm{t(\bltau)}^\CC(\gamma)		&:\equiv \tqm{t}^\CC(\gamma)(\bltau)
\end{align*}
For point constructors, we need to interpret both types of functions into functions
while erasing the element operator, since it does not have any semantic meaning:
\begin{align*}
\tqm{\El(a)}^\CC(\gamma)		&:\equiv \tqm{a}^\CC(\gamma) \\
\tqm{\ExtPiP{T}{A}}^\CC(\gamma)		&:\equiv (\tau : T) \to \tqm{A(\bltau)}^\CC(\gamma) \\
\tqm{(a \Rightarrow_\Pc A)}^\CC(\gamma)	&:\equiv \tqm{a}^\CC(\gamma) \to \tqm{A}^\CC(\gamma)
\end{align*}
Just like for the sort contexts, point contexts are interdependency-free lists
of point constructors and as such can be interpreted as simple products instead
of $\Sigma$-types:
\begin{align*}
\tqm{\cdot}^\CC(\gamma)			&:\equiv \unit \\
\tqm{(\Gamma,\, A)}^\CC(\gamma)		&:\equiv \tqm{\Gamma}^\CC(\gamma) \times \tqm{A}^\CC(\gamma)
\end{align*}
\end{defn}

\begin{example}[Natural numbers]
Looking at the signature of the natural numbers from Example~\ref{ex:if-natvec},
we see that the algebra interpretation of its sort context evaluates to
\begin{equation*}
\unit \times \UU
\end{equation*}
and given an element \blm{(\star, N) : \unit \times \UU}, the algebras of its
point contexts, evaluated at this point result in
\begin{equation*}
N \times (N \to N) \text{.}
\end{equation*}
%TODO vector example?
\end{example}

In the previous section, we introduced a substition calculus for the sort contexts.
Obviously, we might also want to consider algebras over these substitutions.

\begin{defn}[Algebras of substitutions]\label{def:if-alg-sub}
We can extend the algebra operator by defining it on substitutions by functions
between the interpretations of sort contexts:
\begin{equation*}
\inferrule{\tqm{\IFSub{\sigma}{\Gamma_\Sc}{\Delta_\Sc}}}
  {\tqm{\sigma}^\CC : \tqm{\Gamma_\Sc}^\CC \to \tqm{\Gamma_\Sc}^\CC}
\end{equation*}
This is done by setting
\begin{align*}
\tqm{\epsilon}^\CC		&:\equiv \star \text{ and}\\
\tqm{(\sigma,\, t)}^\CC		&:\equiv (\tqm{\sigma}^\CC , \tqm{t}^\CC) \text{.}
\end{align*}
\end{defn}

\begin{lemma}
It's easy to check that this definition of algebras of a subtitution respects
the substitution calculus given in Definition~\ref{def:if-sort-subs} in the following
sense:
\begin{align*}
\tqm{A[\sigma]}^\CC(\gamma)		&= \tqm{A}^\CC(\tqm{\sigma}^\CC(\gamma)) \text{,} \\
\tqm{t[\sigma]}^\CC(\gamma)		&= \tqm{t}^\CC(\tqm{\sigma}^\CC(\gamma)) \text{,} \\
\tqm{\id}^\CC(\gamma)			&= \gamma \text{,} \\
\tqm{(\sigma \circ \delta)}^\CC(\gamma)	&= \tqm{\sigma}^\CC(\tqm{\delta}^\CC(\gamma)) \text{,} \\
\tqm{\wk_\sigma}^\CC(\gamma, \alpha)	&= \tqm{\sigma}^\CC(\gamma) \text{,} \\
\tqm{\pi_1(\sigma)}^\CC(\gamma)		&= \pr_1(\tqm{\sigma}^\CC(\gamma)) \text{, and } \\
\tqm{\pi_2(\sigma)}^\CC(\gamma)		&= \pr_2(\tqm{\sigma}^\CC(\gamma)) \text{.}
\end{align*}
\end{lemma}
\begin{proof}
We can prove the first rule by recursion on the point type \tqm{\Gamma_\Sc \SCon A :: \Pc},
the second rule by recursing on the term \tqm{\Gamma_\Sc \SCon t : B :: \Sc},
the third by induction on the context,
and all other by induction by the substitution.
\end{proof}

%Note that we never introduced substitutions between point constructors.
%While we don't need them to specify inductive families, they are still
%a useful tool, especially on the level of algebras.
%
%\begin{defn}[Point Substitution Algebras]\label{def:if-alg-lsub}
%For a sort substitution \tqm{\IFSub{\sigma}{\Gamma_\Sc}{\Delta_\Sc}} we can
%define a type of \emph{lifted point substitution algebras} $\LSub{\sigma}{\Gamma}{\Delta}$ between
%point contexts \tqm{\vdash_{\Gamma_\Sc} \Gamma} and
%\tqm{\vdash_{\Delta_\Sc} \Delta} inductively as generated by
%\begin{align*}
%\epsilon		&: \LSub{\sigma}{\Gamma}{\cdot} \text{ and} \\
%(\sigma_\Pc,\, \phi)	&: \LSub{\sigma}{\Gamma}{(\Delta,\, A)} \text{,}
%\end{align*}
%for $\sigma_\Pc : \LSub{\sigma}{\Gamma}{\Delta}$ and
%$\phi : \left\{\gamma_\Sc : \tqm{\Gamma_\Sc}^\CC\right\} \left(\gamma : \tqm{\Gamma}^\CC(\gamma_\Sc)\right) \to \tqm{A}^\CC(\sigma^\CC(\gamma_\Sc))$.
%\end{defn}

%TODO define calculus for lifted subs

\section{Displayed Algebras and their Sections}\label{sec:if-ds}

To represent the dependent eliminator, we need algebras which vary over other
algebras.
To get a feeling about what these should look like, let us first look at our
usual simplest example:

\begin{example}\label{ex:if-ds-nat}
Take algebras \blm{(\star, N) : \unit \times \UU} and
\blm{(\star, z, s) : \unit \times N \times (N \to N)} of
the the signature of natural numbers (Example~\ref{ex:if-natvec}).
A \emph{displayed algebra} over this should contain the data which the dependent
eliminator of the natural numbers takes as input:
A type family \blm{P : N \to \UU} together with a point \blm{p_z : P(z)}
and a family of functions \blm{p_s : (n : N) \to P(n) \to P(s(n))}.

A \emph{section} of this algebra would be a section \blm{f : (n : N) \to P(n)} of
\blm{P} respecting the other data by ensuring that \blm{f(z) = p_z}
and that for all \blm{n : N}, we have \blm{f(s(n)) = p_s(f(n))}.
%TODO add another example
\end{example}

Let's first concentrate on the first piece of data:
\begin{defn}[Displayed Algebra Operator]\label{def:if-ds}
As seen above, we want to map sorts to type families over the given algebra.
Sort context will likewise be type families over an algebra:
\begin{equation*}
\inferrule{\tqm{B :: \Sc}}
  {\tqm{B}^\DD : \tqm{B}^\CC \to \UU}
\qquad
\inferrule{\tqm{\SCon \Gamma_\Sc}}
  {\tqm{\Gamma_\Sc}^\DD : \tqm{\Gamma_\Sc}^\CC \to \UU}
\end{equation*}
Since sorts can themselves be interpreted as functions, we have to apply them
whenever we encounter a sort function.
Sort contexts will again be interpreted as iterated products.
\begin{align*}
\tqm{\UU}^\DD(\alpha)				&:\equiv \alpha \to \UU \\
\tqm{\ExtPiS{T}{B}}^\DD(\alpha)			&:\equiv (\tau : T) \to \tqm{B(\bltau)}^\DD(\alpha(\bltau)) \\
\tqm{\cdot_\Sc}^\DD(\star)			&:\equiv \unit \\
\tqm{(\Gamma_\Sc,\, B)}^\DD(\gamma, \alpha)	&:\equiv \tqm{\Gamma}^\DD(\gamma) \times \tqm{B}^\DD(\alpha)
\end{align*}

The interpretation of point constructors and of point contexts now not only depends
on the algebra, but also on the interpretation of the underlying sorts:
\begin{equation*}
\begin{gathered}
\inferrule{\tqm{\Gamma_\Sc \SCon A :: \Pc}}
  {\tqm{A}^\DD : \{\gamma : \tqm{\Gamma_\Sc}^\CC\} \to \tqm{\Gamma_\Sc}^\DD(\gamma) 
    \to \tqm{A}^\CC(\gamma) \to \UU }
\\[.7em]
\inferrule{\tqm{\vdash_{\Gamma_\Sc} \Gamma}}
  {\tqm{\Gamma}^\DD : \{\gamma : \tqm{\Gamma_\Sc}^\CC\} \to \tqm{\Gamma_\Sc}^\DD(\gamma)
    \to \tqm{\Gamma}^\CC(\gamma) \to \UU }
\\[.7em]
\inferrule{\tqm{\Gamma_\Sc \SCon t : B :: \Sc}}
  {\tqm{t}^\DD : \{\gamma : \tqm{\Gamma_\Sc}^\CC\} \to \tqm{\Gamma_\Sc}^\DD(\gamma)
    \to \tqm{B}^\DD(\tqm{t}^\CC(\gamma)) }
\end{gathered}
\end{equation*}
The definition on terms is almost the same as for fixed algebras:
\begin{align*}
\tqm{\var(\vz)}^\DD(\gamma^\DD, \alpha^\DD)
  &:\equiv \alpha^\DD \text{,} \\
\tqm{\var(\vs(t))}^\DD(\gamma^\DD, \alpha^\DD)
  &:\equiv \tqm{\var(t)}^\DD(\gamma^\DD) \text{, and} \\
\tqm{f(\bltau)}^\DD(\gamma^\DD)
  &:\equiv \tqm{f}^\DD(\gamma^\DD)(\tau) \text{.}
\end{align*}
Displayed algebras on point constructors are defined fiberwise, like the ones for
sorts:
\begin{align*}
\tqm{\El(a)}^\DD(\gamma_\Sc^\DD, \alpha)
  &:\equiv \tqm{a}^\DD(\gamma_\Sc^\DD, \alpha) \\
\tqm{\ExtPiP{T}{A}}^\DD(\gamma_\Sc^\DD, \pi)
  &:\equiv (\tau : T) \to \tqm{A(\bltau)}^\DD(\gamma_\Sc^\DD, \pi(\tau)) \\
\tqm{(a \Rightarrow_\Pc A)}^\DD(\gamma_\Sc^\DD, \pi)
  &:\equiv \{\alpha : \tqm{a}^\CC(\gamma_\Sc) \} \to \tqm{a}^\DD(\gamma_\Sc^\DD, \alpha)
    \to \tqm{A}^\DD(\gamma_\Sc^\DD, \pi(\alpha))
\end{align*}
Finally, point contexts are interpreted as iterated products again:
\begin{align*}
\tqm{\cdot}^\DD(\gamma_\Sc^\DD, \gamma)
  &:\equiv \unit \\
\tqm{(\Gamma,\, A)}^\DD(\gamma_\Sc^\DD, (\gamma, \alpha))
  &:\equiv \tqm{\Gamma}^\DD(\gamma_\Sc^\DD, \gamma) \times \tqm{A}^\DD(\gamma_\Sc^\DD, \alpha)
\end{align*}
\end{defn}

%TODO section fluff
\begin{defn}[Section Operator]
For sorts and sort contexts, we want the sections of a displayed algebra to be
the sections of the type family they represent:
\begin{equation*}
\begin{gathered}
\inferrule{\tqm{B :: \Sc}}
  {\tqm{B}^\SS : \{\alpha : \tqm{B}^\CC\} \to \tqm{B}^\DD(\alpha) \to \UU}
\\[.7em]
\inferrule{\tqm{\SCon \Gamma_\Sc}}
  {\tqm{\Gamma_\Sc}^\SS : \{\gamma_\Sc : \tqm{\Gamma_\Sc}^\CC\} \to 
    \tqm{\Gamma_\Sc}^\DD(\gamma_\Sc) \to \UU }
\end{gathered}
\end{equation*}
Both follow the structure of the underlying displayed algebra -- fibrewise for
sort functions and by iterated products on sort contexts:
\begin{align*}
\tqm{\UU}^\SS(\alpha^\DD)
  &:\equiv (x : \alpha) \to \alpha^\DD(x) \\
\tqm{\ExtPiS{T}{B}}^\SS(\pi^\DD)
  &:\equiv (\tau : T) \to \tqm{B(\bltau)}^\SS(\pi^\DD(\tau)) \\
\tqm{\cdot_\Sc}(\gamma_\Sc^\DD)
  &:\equiv \unit \\
\tqm{(\Gamma_\Sc, B)}^\SS(\gamma_\Sc^\DD, \alpha^\DD)
  &:\equiv \tqm{\Gamma_\Sc}^\SS(\gamma_\Sc^\DD) \times \tqm{B}^\SS(\alpha^\DD)
\end{align*}

Sections of point constructors, point contexts, and sort terms will clearly
have to depend on a section of the underlying sort interpretation:
\begin{equation*}
\begin{gathered}
\inferrule{\tqm{\Gamma_\Sc \SCon A :: k} \\
  \gamma_\Sc : \tqm{\Gamma_\Sc}^\CC \\
    \gamma_\Sc^\DD : \tqm{\Gamma_\Sc}^\DD(\gamma_\Sc) \\
    \gamma_\Sc^\SS : \tqm{\Gamma_\Sc}^\SS(\gamma_\Sc^\DD)  }
  {\tqm{A}^\SS(\gamma_\Sc^\SS) : 
    \left\{\alpha : \tqm{A}^\CC(\gamma_\Sc)\right\}
    \left(\alpha^\DD : \tqm{A}^\DD(\gamma_\Sc^\DD, \alpha)\right)
    \to \UU }
\\[.7em]
\inferrule{\tqm{\vdash_{\Gamma_\Sc} \Gamma} \\
  \gamma_\Sc : \tqm{\Gamma_\Sc}^\CC \\
    \gamma_\Sc^\DD : \tqm{\Gamma_\Sc}^\DD(\gamma_\Sc) \\
    \gamma_\Sc^\SS : \tqm{\Gamma_\Sc}^\SS(\gamma_\Sc^\DD) } 
  {\tqm{\Gamma}^\SS(\gamma_\Sc^\SS) : 
    \left\{\gamma : \tqm{\Gamma}^\CC(\gamma_\Sc)\right\}
    \left(\gamma^\DD : \tqm{\Gamma}^\DD(\gamma_\Sc^\DD, \gamma)
    \to \UU \right)}
\\[.7em]
\inferrule{\tqm{\Gamma_\Sc \SCon t : B :: \Sc} \\
    \gamma_\Sc : \tqm{\Gamma_\Sc}^\CC \\
    \gamma_\Sc^\DD : \tqm{\Gamma_\Sc}^\DD(\gamma_\Sc) \\
    \gamma_\Sc^\SS : \tqm{\Gamma_\Sc}^\SS(\gamma_\Sc^\DD) }
  {\tqm{t}^\SS(\gamma_\Sc^\SS) : \tqm{B}^\SS\left(\tqm{t}^\DD(\gamma_\Sc^\DD)\right)}
\end{gathered}
\end{equation*}
For point types we again descent fibrewise, but what to do about the element
operator?
This is where the equations which we have seen in Example~\ref{ex:if-ds-nat}
come into play:
The element which we get out of the interpretation of the section must coincide
with the one we provided by giving the displayed algebra:
\begin{align*}
\tqm{\El(a)}^\SS(\gamma_\Sc^\SS, \alpha^\DD)
  &:\equiv \left(\tqm{a}^\SS(\gamma_\Sc^\SS, \alpha) = \alpha^\DD  \right) \\
\tqm{\ExtPiP{T}{A}}^\SS(\gamma_\Sc^\SS, \pi^\DD)
  &:\equiv (\tau : T) \to \tqm{A(\bltau)}^\SS(\gamma_\Sc^\SS, \pi^\DD(\tau)) \\
\tqm{(a \Rightarrow_\Pc A)}^\SS(\gamma_\Sc^\SS, \pi^\DD)
  &:\equiv (\alpha : \tqm{a}^\CC(\gamma_\Sc))
    \to \tqm{A}^\SS\left(\gamma_\Sc^\SS, \pi^\DD(\tqm{a}^\SS(a, \gamma_\Sc^\SS)(\alpha))\right)
\end{align*}
The definition of sections of point contexts is easier as it is, again,
just an iteration of products:
\begin{align*}
\tqm{\cdot}^\SS(\gamma_\Sc^\SS, \gamma^\DD)
  &:\equiv \unit \\
\tqm{(\Gamma,\, A)}^\SS(\gamma_\Sc^\SS, (\gamma^\DD, \alpha^\DD))
  &:\equiv \tqm{\Gamma}^\SS(\gamma_\Sc^\SS, \gamma^\DD)
    \times \tqm{A}^\SS(\gamma_\Sc^\SS, \alpha^\DD)
\end{align*}
At last, also terms follow the usual pattern of variables selecting sort interpretations
via projections of products and interpreting the application by metatheoretic application:
\begin{align*}
\tqm{\var(\vz)}^\SS(\gamma_\Sc^\SS, \alpha^\SS)
  &:\equiv \alpha^\SS \\
\tqm{\var(\vs(t))}^\SS(\gamma_\Sc^\SS, \alpha^\SS)
  &:\equiv \tqm{\var(t)}^\SS(\gamma_\Sc^\SS) \\
\tqm{f(\bltau)}^\SS(\gamma_\Sc^\SS)
  &:\equiv \tqm{f}^\SS(\gamma_\Sc^\SS)(\tau)
\end{align*}
\end{defn}

Later on, we will need that, following Definition~\ref{def:if-alg-sub}, we can
interpret sort substitutions with the means of displayed algebras, for which
we also need a definition of a section:
\begin{defn}[Displayed Algebras of Substitutions]
Given a sort substitution, its type of displayed algebras should be the type
of function between the displayed algebras of its domain and codomain, where in the
latter we have to apply the function which we get from the \emph{algebra} over
the substitution:
\begin{equation*}
\inferrule{\tqm{\IFSub{\sigma}{\Gamma_\Sc}{\Delta_\Sc}}}
  {\tqm{\sigma}^\DD :
    \left\{\gamma_\Sc : \tqm{\Gamma_\Sc}^\CC\right\}
    \to \tqm{\Gamma_\Sc}^\DD(\gamma_\Sc)
    \to \tqm{\Delta_\Sc}^\DD(\tqm{\sigma}^\CC(\gamma_\Sc))}
\end{equation*}
These are defined, like in the non-displayed case, by
\begin{align*}
\tqm{\epsilon}^\DD(\gamma_\Sc^\DD)
  &:\equiv \star \text{ and} \\
\tqm{(\sigma,\, t)}^\DD(\gamma_\Sc^\DD)
  &:\equiv \left(\tqm{\sigma}^\DD(\gamma_\Sc^\DD), \tqm{t}^\DD(\gamma_\Sc^\DD)\right) \text{.}
\end{align*}
\end{defn}

\begin{defn}[Sections of Substitutions]\label{def:if-ds-sub}
A section of a displayed algebra of a sort substitution is supposed to map
sections of its domain to sections of its codomain:
\begin{equation*}
\inferrule{\tqm{\IFSub{\sigma}{\Gamma_\Sc}{\Delta_\Sc}}}
  {\tqm{\sigma}^\SS :
    \left\{\gamma_\Sc : \tqm{\Gamma_\Sc}^\CC\right\}
    \left\{\gamma_\Sc^\DD : \tqm{\Gamma_\Sc}^\DD(\gamma_\Sc)\right\}
    \to \tqm{\Gamma_\Sc}^\SS(\gamma_\Sc^\DD)
    \to \tqm{\Delta_\Sc}^\SS(\tqm{\sigma}^\DD(\gamma_\Sc^\DD)) }
\end{equation*}
Again, this is happening componentwise:
\begin{align*}
\tqm{\epsilon}^\SS(\gamma_\Sc^\SS)
  &:\equiv \star \text{ and} \\
\tqm{(\sigma,\, t)}^\SS(\gamma_\Sc^\SS)
  &:\equiv \left(\tqm{\sigma}^\SS(\gamma_\Sc^\SS), \tqm{t}^\SS(\gamma_\Sc^\SS) \right) \text{.}
\end{align*}
\end{defn}

\section{Existence of Inductive Families}\label{sec:if-ex}

Having a specification for Inductive Families is not worth much if there is no
way to know what it means for a type theory to actually ``support'' types of this
specification.
The intended meaning of the signatures is clear from the definition of their algebras
as seen in \Cref{sec:if-alg} and as discussed in \Cref{sec:if-ds},
candidates for their eliminators and computation rules are specified in the definition
of sections displayed algebras.
This means that we can formally say what it means for inductive
families to exist in a type theory.
In this section, we will prove that any metatheory as premised in \Cref{sec:if-ex}
actually supports inductive families as specified here.
Since we make heavy use of indexed W-types, we can also see this endeavour as
\emph{reducing} inductive families to indexed W-types.

\begin{thm}[Existence of Inductive Families]\label{thm:if-ex}
For every signature of inductive families given by a sort context \tqm{\SCon \Omega_\Sc}
and a point context \tqm{\vdash_{\Omega_\Sc} \Omega}, there are are sort and point
\emph{constructors} in the form of
\begin{align*}
\IFconS{\Omega}		&: \tqm{\Omega_\Sc}^\CC \text{ and} \\
\IFcon{\Omega}		&: \tqm{\Omega}^\CC(\IFconS{\Omega})
\end{align*}
such that for each displayed algebra given by motives $\omega_\Sc^\DD : \tqm{\Omega_\Sc}^\DD(\IFconS{\Omega})$
and methods $\omega^\DD : \tqm{\Omega}^\DD(\omega_\Sc^\DD, \IFcon{\Omega})$
we can prove an \emph{eliminator} by the means of giving sections
\begin{align*}
\IFelimS{\Omega}{\omega^\DD}	&: \tqm{\Omega_\Sc}^\SS(\omega_\Sc^\DD) \text{ with} \\
\IFelim{\Omega}{\omega^\DD}	&: \tqm{\Omega}^\SS(\IFelimS{\Omega}{\omega^\DD}, \omega^\DD) \text{.}
\end{align*}
\end{thm}

Our strategy to prove this theorem is to first extend
our syntax with elements that have been missing: terms and substitutions for
point types.
For the extended syntax, we will than show that indexed W-types allow us
to find an \emph{internal representation} of the syntax (\Cref{sec:if-internal})
and then
construct a \emph{term model} using the internalization, which we can then
show to be the initial algebra (\Cref{sec:if-termmodel}).

\subsection{Internalization of the Syntax}\label{sec:if-internal}

At first, we will need to make up for some of the
short cuts and simplifications in our definition of signatures.
In the theory of semantics of type theory, which studies various models of different
type theories, the model which is initial in the category of all models
is usually called the \emph{term model}.
This is because in this model, a type get interpreted as the set of all of its
terms.
Since our signatures form -- or are at least strongly
inspired by -- a type theoretic syntax as well, we might hope to deploy the same
strategy for inductive families.
In the core of this interpretation is the issue of how to find an interpretation
for a given sort term \tqm{a} of the universe token \tqm{\UU}.
The interpretation of this ought to be the terms of the \emph{point type}
\tqm{\El(a)} associated with this sort term.
But our syntax does not mention terms of point types at all, since point constructor
are not interdependent!
So our solution is to retrofit the theory with terms, as well as substitutions
for the point contexts:

\begin{defn}[Point Substitution Calculus]\label{def:if-ex-subp}
Let us fix a sort context \tqm{\SCon {\Gamma_\Sc}}.
In total, there are four ways to construct reasonable terms of point types
in \tqm{\Gamma_\Sc}:
Via two constructors for de-Bruijn indices to navigate point contexts and
by an application constructor for each of the two kinds of $\Pi$-type present in
the syntax.
\begin{equation*}
\begin{gathered}
\inferrule{\tqm{\vdash_{\Gamma_\Sc} \Gamma} \\ \tqm{\Gamma_\Sc \SCon A}}
  {\tqm{\Gamma,\, A \vdash \var(\vz) : A :: \Pc}}
\qquad
\inferrule{\tqm{\Gamma_\Sc \SCon A} \\ \tqm{\Gamma_\Sc \SCon A'} \\
  \tqm{\Gamma \vdash \var(t) : A :: \Pc}}
  {\tqm{\Gamma,\, A' \vdash \var(\vs(t)) : A :: \Pc}}
\\[.7em]
\inferrule{\tqm{\Gamma \vdash f : (a \Rightarrow_\Pc A)} \\
  \tqm{\Gamma  \vdash t : \El(a)}}
  {\tqm{\Gamma \vdash f(t) : A :: \Pc}}
\\[.7em]
\inferrule{\tqm{\Gamma \vdash f : \ExtPiP{T}{A}} \\ \tau : T}
  {\tqm{\Gamma \vdash f(\bltau) : A(\bltau) :: \Pc}}
\end{gathered}
\end{equation*}
Like with the sort substitutions defined in \Cref{def:if-sort-subs}, we
define substitutions between point contexts over a fixed sort context
\tqm{\SCon \Gamma_\Sc} to be lists of point terms:
\begin{equation*}
\begin{gathered}
\inferrule{\tqm{\vdash_{\Gamma_\Sc} \Gamma}}
  {\tqm{\IFSub{\epsilon_\Pc}{\Gamma}{\cdot}}}
\qquad
\inferrule{\tqm{\IFSub{\sigma_\Pc}{\Gamma}{\Delta}} \\
  \tqm{\Gamma \vdash t : A :: \Pc}}
  {\tqm{\IFSub{\sigma_\Pc,\, t}{\Gamma}{\Delta,\, A}}}
\end{gathered}
\end{equation*}
We can again define a pullback operation for terms -- this time for point terms --
along substitutions in the form of
\begin{equation*}
\begin{gathered}
\inferrule{\tqm{\Delta \vdash_{\Gamma_\Sc} t : A :: \Pc} \\
  \tqm{\IFSub{\sigma_\Pc}{\Gamma}{\Delta}}}
  {\tqm{\Gamma \vdash_{\Gamma_\Sc} t[\sigma] : A :: \Pc}}
\end{gathered}
\end{equation*}
which is recursively defined by
\begin{align*}
\tqm{\var(\vz)[\sigma_\Pc,\, t_\Pc]}
  &:\equiv \tqm{t_\Pc} \text{,} \\
\tqm{\var(\vs(v_\Pc))[\sigma_\Pc,\, t_\Pc]}
  &:\equiv \tqm{\var(v_\Pc)[\sigma_\Pc]} \text{,} \\
\tqm{f(t)[\sigma_\Pc]}
  &:\equiv \tqm{f[\sigma_\Pc](t[\sigma_\Pc]) } \text{, and} \\
\tqm{f(\bltau)[\sigma_\Pc]}
  &:\equiv \tqm{f[\sigma_\Pc](\bltau)} \text{.}
\end{align*}
Analogously to \ref{def:if-sort-subs} we can define the weakening
\tqm{\IFSub{\wk_{\sigma_\Pc}}{\Gamma,\,A}{\Delta}} of a point substitution
\tqm{\IFSub{\sigma_\Pc}{\Gamma}{\Delta}} along a point type \tqm{\Gamma_\Sc \SCon A :: \Pc},
the identity substitution \tqm{\IFSub{\id_\Pc}{\Gamma}{\Gamma}}, and
the Composition \tqm{\IFSub{\sigma_\Pc \circ \delta_\Pc}{\Gamma}{\Sigma}} of substitutions
\tqm{\IFSub{\sigma_\Pc}{\Delta}{\Sigma}} and \tqm{\IFSub{\delta_\Pc}{\Gamma}{\Delta}}
causing the analogous effect when being used to pullback point terms:
\begin{align*}
\tqm{t_\Pc[\wk_{\sigma_\Pc}]}
  &= \tqm{\vs(t_\Pc[\sigma_\Pc])} \\
\tqm{t_\Pc[\id]}
  &= \tqm{t_\Pc} \\
\tqm{t_\Pc[\sigma_\Pc \circ \delta_\Pc]}
  &= \tqm{t_\Pc[\sigma_\Pc][\delta_\Pc]}
\end{align*}
\end{defn}

As an auxiliary construction for our existence proof we will furthermore need
notions of algebra, displayed algebras, and sections for the point terms and
point substitutions:

\begin{defn}[Algebras of Point Substitutions \& Terms]\label{def:if-ex-psub-a}
We can give semantic meaning to point types and point substitution by extending
the algebra operator with the following components, all over a fixed sort
context \tqm{\SCon \Gamma_\Sc}:
\begin{equation*}
\begin{gathered}
\inferrule{\tqm{\Gamma \vdash_{\Gamma_\Sc} t_\Pc : A :: \Pc}}
  {\tqm{t_\Pc}^\CC :
   \left\{\gamma_\Sc : \tqm{\Gamma_\Sc}^\CC\right\} \to \tqm{\Gamma}^\CC(\gamma_\Sc) \to
   \tqm{A}^\CC(\gamma_\Sc)}
\\[.7em]
\inferrule{\tqm{\IFSub{\sigma_\Pc}{\Gamma}{\Delta}}}
  {\tqm{\sigma_\Pc}^\CC :
    \left\{\gamma_\Sc : \tqm{\Gamma_\Sc}^\CC\right\}
    \to \tqm{\Gamma}^\CC(\gamma_\Sc)
    \to \tqm{\Delta}^\CC(\gamma_\Sc)}
\end{gathered}
\end{equation*}
These components are, in essence, defined as iterated tuples and projections.
For point terms, these defining equations are
\begin{align*}
\tqm{\var(\vz)}^\CC(\gamma, \alpha)
  &:\equiv \alpha \text{,} \\
\tqm{\var(\vs(t))}^\CC(\gamma, \alpha)
  &:\equiv \tqm{\var(t)}^\CC(\gamma) \text{,} \\
\tqm{f(t)}^\CC(\gamma)
  &:\equiv \tqm{f}^\CC(\gamma)(\tqm{t}^\CC(\gamma)) \text{, and} \\
\tqm{f(\bltau)}^\CC(\gamma)
  &:\equiv \tqm{f}^\CC(\gamma)(\tau) \text{,}
\end{align*}
while for point contexts we have the usual
\begin{align*}
\tqm{\epsilon_\Pc}^\CC(\gamma)
  &:\equiv \star \text{ and} \\
\tqm{(\Gamma,\, A)}^\CC(\gamma)
  &:\equiv \left(\tqm{\Gamma}^\CC(\gamma), \tqm{A}^\CC(\gamma)\right) \text{.}
\end{align*}
Of course, apart from these defining equations, this definition of algebras is also
well-behaved under the other components of substitutional calculus:
\begin{align*}
\tqm{t_\Pc[\sigma_\Pc]}^\CC(\gamma)
  &= \tqm{t_\Pc}^\CC(\tqm{\sigma_\Pc}^\CC(\gamma)) \text{,} \\
\tqm{\vs(t_\Pc)}^\CC(\gamma, \alpha)
  &= \tqm{t_\Pc}^\CC(\gamma) \text{,} \\
\tqm{\wk_{\sigma_\Pc}}^\CC(\gamma, \alpha)
  &= \tqm{\sigma_\Pc}^\CC(\gamma) \text{,} \\
\tqm{\id_\Pc}^\CC(\gamma)
  &= \gamma \text{, and} \\
\tqm{(\sigma_\Pc \circ \delta_\Pc)}^\CC(\gamma)
  &= \tqm{\sigma_\Pc}^\CC(\tqm{\delta_\Pc}^\CC(\gamma)) \text{.}
\end{align*}
\end{defn}

\begin{defn}[Displayed Algebras of Point Terms \& Subsitutions]\label{def:if-ex-psub-ds}
Let us for this definition fix a sort context \tqm{\SCon \Gamma_\Sc} with an
algebra \tqm{\gamma_\Sc : \tqm{\Gamma_\Sc}^\CC} as well as a displayed
algebra \tqm{\gamma_\Sc^\DD : \tqm{\Gamma_\Sc}^\DD(\gamma_\Sc)} over \tqm{\gamma_\Sc}.
For the displayed version of these algebras, the interpretation of point terms
and of point substitutions needs to depend on these and, additionally, on
an algebra and displayed algebra of the underlying point context.
This leads to the addition of the following rules:
\begin{equation*}
\begin{gathered}
\inferrule{\tqm{\Gamma \vdash_{\Gamma_\Sc} t_\Pc : A :: \Pc}}
  {\tqm{t_\Pc}^\DD :
    \left\{\gamma : \tqm{\Gamma}^\CC(\gamma_\Sc) \right\}
    \to \tqm{\Gamma}^\DD(\gamma_\Sc^\DD, \gamma)
    \to \tqm{A}^\DD(\gamma_\Sc^\DD, \tqm{t_\Pc}^\CC(\gamma))}
\\[.7em]
\inferrule{\tqm{\IFSub{\sigma_\Pc}{\Gamma}{\Delta}}}
  {\tqm{\sigma_\Pc}^\DD :
    \left\{\gamma : \tqm{\Gamma}^\CC(\gamma_\Sc) \right\}
    \to \tqm{\Gamma}^\DD(\gamma_\Sc^\DD, \gamma) 
    \to \tqm{\Delta}^\DD(\gamma_\Sc^\DD, \tqm{\sigma_\Pc}^\CC(\gamma))}
\end{gathered}
\end{equation*}
We define them by setting
\begin{align*}
\tqm{\var(\vz)}^\DD(\gamma^\DD, \alpha^\DD)
  &:\equiv \alpha^\DD \text{,} \\
\tqm{\var(\vs(t_\Pc))}^\DD(\gamma^\DD, \alpha^\DD)
  &:\equiv \tqm{\var(t_\Pc)}^\DD(\gamma^\DD) \text{,} \\
\tqm{f_\Pc(t_\Pc)}^\DD(\gamma^\DD)
  &:\equiv \tqm{f_\Pc}^\DD(\gamma^\DD)\left(\tqm{t_\Pc}^\CC(\gamma),\, \tqm{t_\Pc}^\DD(\gamma^\DD)\right) \text{, and} \\
\tqm{f_\Pc(\bltau)}^\DD(\gamma^\DD)
  &:\equiv \tqm{f_\Pc}^\DD(\gamma^\DD)(\tau) \text{ for terms, and} \\
\tqm{\epsilon_\Pc}^\DD(\gamma^\DD)
  &:\equiv \star \text{ and} \\
\tqm{(\sigma_\Pc,\, t_\Pc)}^\DD(\gamma^\DD)
  &:\equiv \left(\tqm{\sigma_\Pc}^\DD(\gamma^\DD), \tqm{t_\Pc}^\DD(\gamma^\DD) \right) \text{ for point substitutions.}
\end{align*}
Again, substitution rules analogous to the ones in \ref{def:if-ex-psub-a} hold:
\begin{align*}
\tqm{t_\Pc[\sigma_\Pc]}^\DD(\gamma^\DD)
  &= \tqm{t_\Pc}^\DD(\tqm{\sigma_\Pc}^\DD(\gamma^\DD)) \text{,} \\
\tqm{\vs(t_\Pc)}^\DD(\gamma^\DD, \alpha^\DD)
  &= \tqm{t_\Pc}^\DD(\gamma^\DD) \text{,} \\
\tqm{\wk_{\sigma_\Pc}}^\DD(\gamma^\DD, \alpha^\DD)
  &= \tqm{\sigma_\Pc}^\DD(\gamma^\DD) \text{,} \\
\tqm{\id_\Pc}^\DD(\gamma^\DD)
  &= \gamma^\DD \text{, and} \\
\tqm{(\sigma_\Pc \circ \delta_\Pc)}^\DD(\gamma^\DD)
  &= \tqm{\sigma_\Pc}^\DD(\tqm{\delta_\Pc}^\DD(\gamma^\DD)) \text{.}
\end{align*}
\end{defn}

%\begin{defn}[Sections of Point Terms \& Substitutions]
%I'm not sure if we need these...
%\end{defn}

As a next step after having extended our syntax and defined the semantics of this
extension, we will show that any type theory with indexed W-types is able
to represent the whole syntax for inductive families internally.

\begin{remark}\label{rmk:if-impl}
While for the signatures of inductive-inductive types, contexts, types, and terms
depend on each other, we can here define sort types, sort contexts, terms, point
types, and contexts in the presented order without referring to later constructions.
This means that unlike mentioned in \Cref{rmk:iit-qit}, we  can
internalize this syntax just using inductive families, as shown in the
following agda implementation:
\begin{agdacodebr}
data TyS : Set₁ where
  U  : TyS
  Π̂S : (T : Set) → (T → TyS) → TyS

data ConS : Set₁ where
  ∙c   : ConS
  _▶c_ : ConS → TyS → ConS

data VarS : ConS → TyS → Set₁ where
  vvz : ∀{Γc B} → Var (Γc ▶c B) B
  vvs : ∀{Γc B B'} → Var Γc B → Var (Γc ▶c B') B

data TmS (Γc : ConS) : TyS → Set₁ where
  var  : ∀{A} → Var Γc A → TmS Γc A
  _@S_ : ∀{T B} → TmS Γc (Π̂S T B) → (τ : T) → TmS Γc (B τ)

data TyP (Γc : ConS) : Set₁ where
  El   : TmS Γc U → TyP Γc
  Π̂P   : (T : Set) → (T → TyP Γc) → TyP Γc
  _⇒P_ : TmS Γc U → TyP Γc → TyP Γc

data Con (Γc : ConS) : Set₁ where
  ∙    : Con Γc
  _▶P_ : Con Γc → TyP Γc → Con Γc
\end{agdacodebr}
Note, that in the implementaion, variables and terms are defined in separate
types to allow for \tqm{\var(v)} to appear as a premise for the introduction rule
for \tqm{\vs(v)}.
The extension of the syntax by sort substitutions of Definition~\ref{def:if-sort-subs}
as well as the subsequent extension by point terms and point substitutions as presented in
Definition~\ref{def:if-ex-subp} is implementable as well:
\begin{agdacodebr}
data SubS : ConS → ConS → Set₁ where
  ε   : ∀{Γc} → SubS Γc ∙c
  _,_ : ∀{Γc Δc B} → SubS Γc Δc → TmS Γc B → SubS Γc (Δc ▶c B)

data VarP {Γc} : Con Γc → TyP Γc → Set₁ where
  vvzP : ∀{Γ A} → VarP (Γ ▶P A) A
  vvsP : ∀{Γ A B} → VarP Γ A → VarP (Γ ▶P B) A

data TmP {Γc}(Γ : Con Γc) : TyP Γc → Set₁ where
  varP : ∀{A} → VarP Γ A → TmP Γ A
  _@P_ : ∀{a A} → TmP Γ (a ⇒P A) → TmP Γ (El a) → TmP Γ A
  _^@P_ : ∀{T A} → TmP Γ (Π̂P T A) → (τ : T) → TmP Γ (A τ)

data SubP {Γc} : Con Γc → Con Γc → Set₁ where
  εP   : ∀{Γ} → SubP Γ ∙
  _,P_ : ∀{Γ Δ A} → SubP Γ Δ → TmP Γ A → SubP Γ (Δ ▶P A)
\end{agdacodebr}
\end{remark}

To make this proof of internalizability formal, we will present the exact definition
of all the parts of the extended syntax as indexed W-types.
This means giving a type ${Con_\Sc} : \UU$ of sort contexts,
a type ${Ty_\Sc} : \UU$ of sort types,
a family ${Var_\Sc} : {Con_\Sc} \to {Ty_\Sc} \to \UU$ of variables of a given
sort context and sort type,
extending the latter, a family $W_{Tm_\Sc} : {Con_\Sc} \to {Ty_\Sc} \to \UU$
of sort terms,
a family ${Ty_\Pc} : {Con_\Sc} \to \UU$ of point types in a given sort context,
and finally a type ${Con_\Pc} : {Con_\Sc} \to \UU$ of point contexts over a
given sort context.
Afterwards, we will give the same treatment to the extensions with
sort substitutions between two sort contexts, with variables
of a point type, terms of a point type and point substitution between two point
contexts over the same sort context in the form of
\begin{align*}
{Sub_\Sc}
  &: {Con_\Sc} \to {Con_\Sc} \to \UU \text{,} \\
{Var_\Pc}
  &: \{\Gamma_\Sc : {Con_\Sc}\} \to {Con_\Pc}(\Gamma_\Sc) \to {Ty_\Pc}(\Gamma_\Sc) \to \UU \text{,}\\
{Tm_\Pc}
  &: \{\Gamma_\Sc : {Con_\Sc}\} \to {Con_\Pc}(\Gamma_\Sc) \to {Ty_\Pc}(\Gamma_\Sc) \to \UU \text{, and}\\
{Sub_\Pc}
  &: \{\Gamma_\Sc : {Con_\Sc}\} \to {Con_\Pc}(\Gamma_\Sc) \to {Con_\Pc}(\Gamma_\Sc) \to \UU \text{.}
\end{align*}
In the following definition we will give all of these ten types and type families
in general by giving the respective input data for indexed W-types as described
in Chapter~\ref{sec:tt-w}.

\begin{defn}[IF-Syntax as W-Types]
We define the types mentioned above as follows:
\begin{align*}
\begin{split}
{Ty_\Sc}
  &:\equiv \IW{A_{Ty_\Sc}}{B_{Ty_\Sc}}{o_{Ty_\Sc}}{r_{Ty_\Sc}}(\star) \text{,} \\
{Con_\Sc}
  &:\equiv \IW{A_{Con_\Sc}}{B_{Con_\Sc}}{o_{Con_\Sc}}{r_{Con_\Sc}}(\star) \text{,} \\
{Var_\Sc}(\_, B)
  &:\equiv \IW{A_{Var_\Sc}(B)}{B_{Var_\Sc}(B)}{o_{Var_\Sc}(B)}{r_{Var_\Sc}(B)} \text{,} \\
W_{Tm_\Sc}(\Gamma_\Sc)
  &:\equiv \IW{A_{Tm_\Sc}(\Gamma_\Sc)}{B_{Tm_\Sc}(\Gamma_\Sc)}{o_{Tm_\Sc}(\Gamma_\Sc)}{r_{Tm_\Sc}(\Gamma_\Sc)} \text{,} \\
{Sub_\Sc}(\Gamma_\Sc)
  &:\equiv \IW{A_{Sub_\Sc}(\Gamma_\Sc)}{B_{Sub_\Sc}(\Gamma_\Sc)}{o_{Sub_\Sc}(\Gamma_\Sc)}{r_{Sub_\Sc}(\Gamma_\Sc)} \text{,}
\end{split}
\begin{split}
{Ty_\Pc}(\Gamma_\Sc)
  &:\equiv \IW{A_{Ty_\Pc}}{B_{Ty_\Pc}}{o_{Ty_\Pc}}{r_{Ty_\Pc}}(\star) \text{,} \\
{Con_\Pc}(\Gamma_\Sc)
  &:\equiv \IW{A_{Con_\Pc}}{B_{Con_\Pc}}{o_{Con_\Pc}}{r_{Con_\Pc}}(\star) \text{,} \\
{Var_\Pc}(\_, A)
  &:\equiv \IW{A_{Var_\Pc}(A)}{B_{Var_\Pc}(A)}{o_{Var_\Pc}(A)}{r_{Var_\Pc}(A)} \text{,} \\
{Tm_\Pc}(\Gamma)
  &:\equiv \IW{A_{Tm_\Pc}(\Gamma)}{B_{Tm_\Pc}(\Gamma)}{o_{Tm_\Pc}(\Gamma)}{r_{Tm_\Pc}(\Gamma)} \text{,} \\
{Sub_\Pc}(\Gamma)
  &:\equiv \IW{A_{Sub_\Pc}(\Gamma)}{B_{Sub_\Pc}(\Gamma)}{o_{Sub_\Pc}(\Gamma)}{r_{Sub_\Pc}(\Gamma)} \text{,}
\end{split}
\end{align*}
where the respective indices for the indexed W-types are given in Table~\ref{tbl:if-iws}.
\end{defn}

\begin{sidewaystable}\label{tbl:if-iws}
\centering
{\footnotesize$\begin{array}{r|lllll}
i & I_i : \UU & A_i : \UU & B_i : A_i \to \UU & o_i : A_i \to I_i & r_i : (a : A_i) \to B_i(a) \to I_i \\
\hline
Ty_\Sc
  & \unit
  & \makecell[cl]{\unit \\ + \UU}
  & \makecell[cl]{\inl(\star) \mapsto \emptytype \\ \inr(T) \mapsto T}
  & \_ \mapsto \star
  & \_ \mapsto \star \\
Con_\Sc
  & \unit
  & \makecell[cl]{\unit \\ + {Ty_\Sc}}
  & \makecell[cl]{\inl(\star) \mapsto \emptytype \\ \inr(B) \mapsto \unit}
  & \_ \mapsto \star
  & \_ \mapsto \star \\
Var_\Sc
  & {Con_\Sc}
  & \makecell[cl]{{Con_\Sc} \\ + {Con_\Sc} \times W_{Ty_{\Sc}} }
  & \makecell[cl]{\inl(\Gamma_\Sc) \mapsto \emptytype \\ \inr(\Gamma_\Sc, B') \mapsto \unit}
  & \makecell[cl]{\inl(\Gamma_\Sc) \mapsto \tqm{(\Gamma_\Sc,\,B)} \\ \inr(\Gamma_\Sc, B') \mapsto \tqm{(\Gamma_\Sc,\,B')} }
  & \makecell[cl]{ - \\ \inr(\Gamma_\Sc, B')(\star) \mapsto \Gamma_\Sc } \\
Tm_\Sc(\Gamma_\Sc)
  & {Ty_\Sc}
  & \makecell[cl]{{Ty_\Sc} \\ + (T : \UU) \times (T \to {Ty_\Sc}) \times T }
  & \makecell[cl]{\inl(B) \mapsto \emptytype \\ \inr(\_) \mapsto \unit}
  & \makecell[cl]{\inl(B) \mapsto \emptytype \\ \inr(T, B, \tau) \mapsto B(\tau) }
  & \makecell[cl]{ - \\ \inr(T, B, \tau)(\star) \mapsto \tqm{\ExtPiS{T}{B}} } \\
Sub_\Sc(\Gamma_\Sc)
  & \unit
  & \makecell[cl]{\unit \\ + (B : {Ty_\Sc}) \times W_{Tm_\Sc}(\Gamma_\Sc, B) }
  & \makecell[cl]{\inl(\star) \mapsto \emptytype \\ \inr(B, t) \mapsto \unit }
  & \_ \mapsto \star
  & \_ \mapsto \star \\
Ty_\Pc(\Gamma_\Sc)
  & \unit
  & \makecell[cl]{W_{Tm_\Sc}(\Gamma, \tqm{\UU}) \\ + \UU \\ + W_{Tm_\Sc}(\Gamma, \tqm{\UU})}
  & \makecell[cl]{\inl(a) \mapsto \emptytype \\ \inr(\inl(\tau)) \mapsto T \\ \inr(\inr(a)) \mapsto \unit }
  & \_ \mapsto \star
  & \_ \mapsto \star \\
Con_\Pc(\Gamma_\Sc)
  & \unit
  & \makecell[cl]{\unit \\ + W_{Ty_{\Pc}(\Gamma_\Sc)} }
  & \makecell[cl]{\inl(\star) \mapsto \emptytype \\ \inr(A) \mapsto \unit}
  & \_ \mapsto \star
  & \_ \mapsto \star \\
Var_\Pc(A)
  & {Con_\Pc}(\Gamma_\Sc)
  & \makecell[cl]{{Con_\Pc}(\Gamma_\Sc) \\ + {Con_\Pc}(\Gamma_\Sc) \times {Ty_\Pc}(\Gamma_\Sc)}
  & \makecell[cl]{\inl(\Gamma) \mapsto \emptytype \\ \inr(\Gamma, A') \mapsto \unit}
  & \makecell[cl]{\inl(\Gamma) \mapsto \tqm{(\Gamma,\,A)} \\ \inr(\Gamma,A') \mapsto \tqm{(\Gamma,\,A')}}
  & \makecell[cl]{ - \\ \inr(\Gamma, A')(\star) \mapsto \Gamma} \\
Tm_\Pc(\Gamma)
  & {Ty_\Pc}(\Gamma_\Sc)
  & \makecell[cl]{(A : {Ty_\Pc}(\Gamma_\Sc)) \times {Var_\Pc}(\Gamma, A)}
  & \makecell[cl]{\inl(A, v) \mapsto \emptytype \\ \inr(\inl(\_)) \mapsto \twotype \\ \\ \inr(\inr(\_)) \mapsto \unit }
  & \makecell[cl]{\inl(A, v) \mapsto A          \\ \inr(\inl(A, a)) \mapsto A      \\ \\ \inr(\inr(T, A, \tau)) \mapsto A(\tau)}
  & \makecell[cl]{ -                            \\ \inr(\inl(A,a))(0) \mapsto \tqm{(a \Rightarrow_\Pc A)} \\ \inr(\inl(A,a))(1) \mapsto \tqm{\El(a)}  \\ \inr(\inr(T, A, \tau))(\star) \mapsto \tqm{\ExtPiP{T}{A}} } \\
Sub_\Pc(\Gamma)
  & \unit
  & \makecell[cl]{\unit \\ + (A : {Ty_\Pc}(\Gamma_\Sc)) \times {Tm_\Pc}(\Gamma, A) }
  & \makecell[cl]{\inl(\star) \mapsto \emptytype \\ \inr(A, t) \mapsto \unit }
  & \_ \mapsto \star
  & \_ \mapsto \star
\end{array}$}
\caption{The input data for the indexed W-types representing the internalized
syntax for inductive families.}\label{tbl:if-iws}
\end{sidewaystable}

\subsection{Constructing the Term Model}\label{sec:if-termmodel}

For the remainder of this section, let us fix the sort context \tqm{\SCon \Omega_\Sc}
and the point context \tqm{\vdash_{\Omega_\Sc} \Omega} which we want to construct
by giving
\begin{align*}
\IFconS{\Omega}		&: \tqm{\Omega_\Sc}^\CC \text{ and} \\
\IFcon{\Omega}		&: \tqm{\Omega}^\CC(\IFconS{\Omega}) \text{.}
\end{align*}
Our definition of the constructor uses the trick %TODO cite whoever invented this trick
to index several of the constructions by a second sort or point context together
with a sort or point substitution from \tqm{\Omega_\Sc} or \tqm{\Omega}.
We can think of this second context as some sort of a ``sub-context'' of a fixed
context.

\begin{defn}[The Sort Constructor]\label{def:if-ex-cons}
The generalized sort constructor consists of the following data:
\begin{equation*}
\inferrule{\tqm{\SCon \Gamma_\Sc} \\
  \tqm{\IFSub{\sigma}{\Omega_\Sc}{\Gamma_\Sc}}}
  {\IFconS{\sigma} : \tqm{\Gamma_\Sc}^\CC }
\end{equation*}
We can define this recursively via
\begin{align*}
\IFconS{\epsilon}
  &:\equiv \star \text{ and} \\
\IFconS{\sigma,\, t}
  &:\equiv \left(\IFconS{\sigma},\, \IFconS{t}\right) \text{,}
\end{align*}
where on sort terms we will define a constructor operation yielding an algebra
of the respective sort type:
\begin{equation*}
\inferrule{\tqm{\Omega_\Sc \SCon t : B :: \Sc}}
  {\IFconS{t} : \tqm{B}^\CC }
\end{equation*}
This operation will on universe terms consist of the type of point terms, while
on external sort functions, it will return a function with constructor of the applied
term: %TODO formulation
\begin{align*}
\IFconS{a}
  &:\equiv {Tm_\Pc}(\tqm{\Omega}, \tqm{\El(a)})
  & \text{ for \tqm{\Omega_\Sc \SCon a : \UU :: \Sc} and} \\
\IFconS{f}
  &:\equiv \lambda \tau.\, \IFconS{f(\bltau)}
  & \text{ for \tqm{\Omega_\Sc \SCon f : \ExtPiS{T}{B} :: \Sc}.}
\end{align*}
\end{defn}

This construction is already enough to give the sort constructor required in
Theorem~\ref{thm:if-ex} by pinning the substitution to be the identity:
\begin{equation}\label{eq:if-ex-sorts}
\IFconS{\Omega} :\equiv \IFconS{\id_{\Omega_\Sc}} : \tqm{\Omega_\Sc}^\CC
\end{equation}
It is not immediately clear that the operation on substitutions and the operation
on sort terms is well-behaved under the pullback along substitutions.
We can, however, show that this is indeed the case.

\begin{lemma}[Coherence of the Sort Constructor]\label{lem:if-ex-cont}
For all subsitutions \tqm{\IFSub{\sigma}{\Gamma_\Sc}{\Delta_\Sc}} and sort terms
\tqm{\Gamma_\Sc \SCon t : B :: \Sc}, taking a constructor of \tqm{t} pulled back
along \tqm{\sigma} has the same effect as taking the term algebra over the context
algebra generated by the constructor on \tqm{\sigma}, i.\,e.
\begin{equation*}
\tqm{t}^\CC(\IFconS{\sigma}) = \IFconS{t[\sigma]} \text{.}
\end{equation*}
\end{lemma}

\begin{proof}
Let us first do a case distinction on the substitution.
If it is \tqm{\epsilon}, then $\tqm{\Gamma_\Sc} = \tqm{\cdot_\Sc} $,
and it is easy to see that there are no terms in the empty sort context.
Thus, we can assume the substitution to be of the form $\tqm{(\sigma,\, s)}$.
In this case, lets recurse on the term and see that
\begin{align*}
\tqm{\var(\vz)}^\CC(\IFconS{\sigma,\, s})
  &= \tqm{\var(\vz)}^\CC(\IFconS{\sigma}, \IFconS{s}) \\
  &= \IFconS{s} \\
  &= \IFconS{\var(\vz)[\sigma,\, s]} \text{,} \\[.7em]
\tqm{\var(\vs(t))}^\CC(\IFconS{\sigma,\, s})
  &= \tqm{\var(\vs(t))}^\CC(\IFconS{\sigma}, \IFconS{s}) \\
  &= \tqm{\var(t)}^\CC(\IFconS{\sigma}) \\
  &= \IFconS{\var(t)[\sigma]} & \text{ by induction} \\
  &= \IFconS{\var(\vs(t))[\sigma,\, s]} \text, & \text{ and lastly} \\[.7em]
\tqm{f(\bltau)}^\CC(\IFconS{\sigma,\, s})
  &= \tqm{f}^\CC(\IFconS{\sigma,\, s})(\bltau) \\
  &= \IFconS{f[\sigma,\, s]}(\bltau) & \text{ by induction} \\
  &= \IFconS{f(\bltau)[\sigma,\, s]} & \text{ for \tqm{f : \ExtPiS{T}{B}}.}
\end{align*}
\end{proof}

We can now use this lemma to be able to do a trick with $\IFcon{\Omega}$ similar
to the trick we did for $\IFconS{\Omega}$:
Replace the fixed point context with a variable one, together with a substitution
from \tqm{\Omega}, and define the constructor recursively on point types.

\begin{defn}[The Point Constructor]\label{def:if-ex-con}
We define operations on point contexts and point terms, resulting in algebras, in
the form of the following:
\begin{equation*}
\begin{gathered}
\inferrule{\tqm{\vdash_{\Gamma_\Sc} \Gamma} \\ \tqm{\IFSub{\sigma_\Pc}{\Omega}{\Gamma}}}
  {\IFcon{\sigma_\Pc} : \tqm{\Gamma}^\CC(\IFconS{\Omega})}
\qquad
\inferrule{\tqm{\Omega \vdash_{\Omega\Sc} t_\Pc : A :: \Pc}}
  {\IFcon{t_\Pc} : \tqm{A}^\CC(\IFconS{\Omega}) }
\end{gathered}
\end{equation*}
The operation on point substitutions is defined recursively by
\begin{align*}
\IFcon{\epsilon_\Pc}
  &:\equiv \star \text{ and} \\
\IFcon{\sigma_\Pc,\, t_\Pc}
  &:\equiv \left(\IFcon{\sigma_\Pc}, \IFcon{t_\Pc} \right) \text{,}
\end{align*}
wheres for point terms, note that if \tqm{\Omega \vdash t_\Pc : \El(a) :: \Pc},
then by Lemma~\ref{lem:if-ex-cont}
\begin{equation*}
\tqm{t_\Pc} : \IFconS{a} \equiv \IFconS{a[\id]} = \tqm{a}^\CC(\IFconS{\id_{\Omega_\Sc}}) \equiv \tqm{\El(a)}^\CC(\IFconS{\Omega}) \text{,}
\end{equation*}
which allows us to define the constructor operator by
\begin{align*}
\IFcon{t_\Pc}
  &:\equiv \tqm{t_\Pc} & \text{ for \tqm{\Omega \vdash t_\Pc : \El(a)},} \\
\IFcon{f_\Pc}
  &:\equiv \lambda \tqm{t_\Pc}.\, \IFcon{f_\Pc(t_\Pc)} & \text{ for \tqm{\Omega \vdash f_\Pc : a \Rightarrow_\Pc A}, and} \\
\IFcon{f_\Pc}
  &:\equiv \lambda \tau.\, \IFcon{f_\Pc(\bltau)} & \text{ for \tqm{\Omega \vdash f_\Pc : \ExtPiP{T}{A}}.}
\end{align*}
\end{defn}

This concludes the definition of the constructors, since we can set, like for the
sort constructor
\begin{equation}
\IFcon{\Omega} :\equiv \IFcon{\id_\Omega} : \tqm{\Omega}^\CC(\IFconS{\Omega}) \text{.}
\end{equation}
Again, the construction comes with a property that makes it coherent under
pulled back point terms.
Analogously to Lemma~\ref{lem:if-ex-cont}, this coherence looks as follows:

\begin{lemma}[Coherence of the Point Constructor]\label{lem:if-ex-contp}
For all point subsitutions \tqm{\IFSub{\sigma_\Pc}{\Omega}{\Delta}} and
point terms \tqm{\Gamma \vdash_{\Omega_\Sc} t_\Pc : A :: \Pc}, pulling back has the 
same effect as the point constructor as in
\begin{equation}
\tqm{t_\Pc}^\CC(\IFcon{\sigma_\Pc}) = \IFcon{t_\Pc[\sigma_\Pc]} \text{.}
\end{equation}
\end{lemma}

\begin{proof}
Repeating the strategy of the proof of Lemma~\ref{lem:if-ex-cont}, we again see
that we can assume the substitution to be of an extended form \tqm{(\sigma_\Pc,\, s_\Pc)},
since there are no point terms in the empty point context.
Now, by recursion on the term we see that
\begin{align*}
\tqm{\var(\vz)}^\CC(\IFcon{\sigma_\Pc,\, s_\Pc})
  &= \tqm{\var(\vz)}^\CC(\IFcon{\sigma_\Pc}, \IFcon{s_\Pc}) \\
  &= \IFcon{s_\Pc} \\
  &= \IFcon{\var(\vz)[\sigma_\Pc,\, s_\Pc]} \text{,} \\[.7em]
\tqm{\var(\vs(t_\Pc))}^\CC(\IFcon{\sigma_\Pc,\, s_\Pc})
  &= \tqm{\var(\vs(t_\Pc))}^\CC(\IFcon{\sigma_\Pc}, \IFcon{s_\Pc}) \\
  &= \tqm{\var(t_\Pc)}^\CC(\IFcon{\sigma_\Pc}) \\
  &= \IFcon{\var(t_\Pc)[\sigma_\Pc]} & \text{ by induction} \\
  &= \IFcon{\var(\vs(t_\Pc))[\sigma_\Pc,\, s_\Pc]} \text{,} \\[.7em]
\tqm{f_\Pc(t_\Pc)}^\CC(\IFcon{\sigma_\Pc})
  &= \tqm{f_\Pc}^\CC(\IFcon{\sigma_\Pc})\left(\tqm{t_\Pc}^\CC(\IFcon{\sigma_\Pc})\right) \\
  &= \IFcon{f_\Pc[\sigma_\Pc]}(\IFcon{t_\Pc[\sigma_\Pc]}) & \text{ by induction} \\
  &= \IFcon{f_\Pc(t_\Pc)[\sigma_\Pc]} \text{, and} \\[.7em]
\tqm{f_\Pc(\bltau)}^\CC(\IFcon{\sigma_\Pc})
  &= \tqm{f_\Pc}^\CC(\IFcon{\sigma_\Pc})(\tau) \\
  &= \IFcon{f_\Pc[\sigma_\Pc]}(\tau) & \text{ by induction} \\
  &= \IFcon{f_\Pc(\tau)[\sigma_\Pc]} \text{.}
\end{align*}
\end{proof}

With the constructors defined let us move on the construction of the eliminator.
Let us from now on fix displayed algebras $\omega_\Sc^\DD : \tqm{\Omega_\Sc}^\DD(\IFconS{\Omega})$
and $\omega^\DD : \tqm{\Omega}^\DD(\omega_\Sc^\DD, \IFcon{\Omega})$.
We will proceed in the same order as for the constructors and start by generalizing
$\IFelimS{\Omega}{\omega^\DD}$ to arbitrary subcontexts of \tqm{\Omega} by giving constructions
on sort substitutions and sort terms.

\begin{defn}[The Eliminator]
The generalized eliminator will take substitutions or sort terms to give sections
of sort types or sort contexts, respectively:
\begin{equation*}
\begin{gathered}
\inferrule{\tqm{\SCon \Gamma_\Sc} \\ \tqm{\IFSub{\sigma}{\Omega_\Sc}{\Gamma_\Sc}}}
  {\IFelimS{\sigma}{} : \tqm{\Gamma_\Sc}^\SS(\tqm{\sigma}^\DD(\omega_\Sc^\DD))}
\qquad
\inferrule{\tqm{\Omega_\Sc \SCon t : B :: \Sc}}
  {\IFelimS{t}{} : \tqm{B}^\SS(\tqm{t}^\DD(\omega_\Sc^\DD))}
\end{gathered}
\end{equation*}
The first rule is defined by recursion using the second construction as usual:
\begin{align*}
\IFelimS{\epsilon}{}
  &:\equiv \star \text{ and} \\
\IFelimS{\sigma,\,t}{}
  &:\equiv \left(\IFelimS{\sigma}{}, \IFelimS{t}{}\right) \text{.}
\end{align*}
For the sort terms, we observe that, by Lemmas~\ref{lem:if-ex-cont} and \ref{lem:if-ex-contp}, for
\tqm{\Omega_\Sc \SCon a : \UU} and \blm{\tqm{t_\Pc} : \tqm{a}^\CC(\IFconS{\Omega})} we have
\begin{equation*}
\begin{gathered}
%\tqm{a}^\CC(\IFconS{\Omega})
%  \equiv foo \text{, implying} \\
\tqm{\UU}^\SS(\tqm{a}^\DD(\omega_\Sc^\DD), \tqm{t_\Pc}^\CC(\IFcon{\Omega}))
  = \tqm{a}^\DD(\omega_\Sc^\DD, \tqm{t_\Pc})
\end{gathered}
\end{equation*}
and thus we can set, disregarding transports,
\begin{align*}
\IFelimS{a}{}
  &:\equiv \lambda \tqm{t_\Pc}.\, \tqm{t_\Pc}^\DD(\omega^\DD) & \text{ for \tqm{\Omega_\Sc \SCon a : \UU} and} \\
\IFelimS{f}{}
  &:\equiv \lambda \tau.\, \IFelimS{f(\bltau)}{} & \text{ for \tqm{\Omega_\Sc \SCon f : \ExtPiS{T}{B}}.}
\end{align*}
\end{defn}

Similar to Lemma~\ref{lem:if-ex-cont}, these definitions are coherent in the following
form:
\begin{lemma}\label{lem:if-ex-elimt}
Given a sort substitution \tqm{\IFSub{\sigma}{\Omega_\Sc}{\Gamma_\Sc}} and a sort
term \tqm{\Gamma_\Sc \SCon t : B :: \Sc}, the eliminator of a pulled back term
is the section of the term, evaluated at the eliminator on a substitution:
\begin{equation*}
\IFelimS{t[\sigma]}{} = \tqm{t}^\SS(\IFelimS{\sigma}{}) \text{.}
\end{equation*}
\end{lemma}

\begin{proof}
The proof strategy is exactly the same as for Lemma~\ref{lem:if-ex-cont}.
\end{proof}

As a last step, we still need to prove the computation rules for the eliminator,
consisting of section of given point contexts.
Consistent with~\ref{def:if-ex-cons}, we generalize them to arbitrary point
substitutions and point terms.

\begin{lemma}[Computation Rules]\label{lem:if-ex-elim}
We prove the computation rule for our eliminator $\IFelimS{\Omega}{}$ to be a
section of subcontexts of \tqm{\Omega} and on point terms of \tqm{\Omega}:
\begin{equation*}
\begin{gathered}
\inferrule{\tqm{\vdash_{\Omega_\Sc} \Gamma} \\ \tqm{\IFSub{\sigma_\Pc}{\Omega}{\Gamma}}}
  {\IFelim{\sigma_\Pc}{} : \tqm{\Gamma}^\SS(\IFelimS{\Omega}{}, \tqm{\sigma_\Pc}^\DD(\omega^\DD))}
\\[.7em]
\inferrule{\tqm{\Omega \vdash_{\Omega_\Sc} t_\Pc : A :: \Pc}}
  {\IFelim{t_\Pc}{} : \tqm{A}^\SS(\IFelimS{\Omega}{}, \tqm{t_\Pc}^\DD(\omega^\DD))}
\end{gathered}
\end{equation*}
\end{lemma}

\begin{proof}
Using the second rule, the first one can be proved in a straightforward way by
recursion on the point substitution:
\begin{align*}
\IFelim{\epsilon_\Pc}{}
  &:\equiv \star \text{ and} \\
\IFelim{\sigma_\Pc,\, t_\Pc}{}
  &:\equiv \left( \IFelim{\sigma_\Pc}{}, \IFelim{t_\Pc}{} \right) \text{.}
\end{align*}
For the second rule we again need to consider the types needed for the element
case.
The previous lemmas tell us that for \tqm{\Omega \vdash t_\Pc : \El(a) :: \Pc} we
can prove the required rule by
\begin{align*}
  & \tqm{a}^\SS\left(\IFelimS{\id_{\Omega_\Sc}}{}, \tqm{t_\Pc}^\CC(\IFcon{\Omega})\right) \\
= & \tqm{a}^\SS(\IFelimS{\id_{\Omega_\Sc}}{}, \tqm{t_\Pc})
  & \text { by Lemma~\ref{lem:if-ex-contp}} \\
= & \IFelimS{a}{}(\tqm{t_\Pc})
  & \text { by Lemma~\ref{lem:if-ex-elimt}} \\
= & \tqm{t_\Pc}^\DD(\omega^\DD) \text{.}
%  & \text{ for \tqm{\Omega \vdash t_\Pc : \El(a)},} \\[.7em]
\end{align*}
For the case of \tqm{\Omega \vdash f_\Pc : \ExtPiP{T}{A}}, we see that we
can recursively define $\IFelim{f_\Pc}{}$ by proving $\IFelim{f_\Pc(\bltau)}{}$
for all $\tau : T$.
Likewise in the case of a recursive function term
\tqm{\Omega \vdash f_\Pc : a \Rightarrow_\Pc A}, we prove
$\IFelim{f_\Pc}{}$ recursively by $\IFelim{f_\Pc(t_\Pc)}{}$. %TODO more detail?
\end{proof}

\begin{proof}[Proof of Theorem~\ref{thm:if-ex}]
Lemma~\ref{lem:if-ex-elim} completes the construction of the eliminator and
setting
\begin{align*}
\IFelimS{\Omega}{\omega^\DD}
  &:\equiv \IFelimS{\id_{\Omega_\Sc}}{} \text{ and} \\
\IFelim{\Omega}{\omega^\DD}
  &:\equiv \IFelim{\id_\Omega}{}
\end{align*}
completes the existence proofs for our specification of inductive families.
\end{proof}






\chapter{Reducing Inductive-Inductive Types to Inductive Families}
\chaptermark{Reducing Inductive-Inductive Types}

\section{Example: Type Theory Syntax}

Before we explain the reduction in the general case, it is useful to first
look at how it works in a special case.
As a prime example we chose Example~\ref{ex:ttintt} which describes
the contexts and types of a type theoretic syntax with a base type and $\Pi$-types.
To recall the specifics of the example: We want to define a type
$Con : \UU$ of contexts and a type family $Ty : Con \to \UU$ which gives the
type of types over a context.
These are populated by constructors, providing the empty context,
context extension, the base type former and the type former for the $\Pi$-types:
\begin{align*}
nil &: Con \text{,} \\
ext &: (\Gamma : Con) \to Ty(\Gamma) \to Con \text{,} \\
unit &: (\Gamma : Con) \to Ty(\Gamma) \text{, and} \\
pi &: (\Gamma : Con) (A : Ty(\Gamma)) \to Ty(ext(\Gamma, A)) \to Ty(\gamma) \text{.}
\end{align*}

Since the dependency between the two sorts $Con$ and $Ty$ can not be represented
directly with inductive families, we might, as a first approximation, simply forget
about all the indices of the sort -- that is, the $Con$-index in $Ty$ --
and adapt the point constructors accordingly:
Let $Con' : \UU$ and $Ty' : \UU$ be plain types generated by the following four
mutually dependent constructors:
\begin{align*}
nil' &: Con' \text{,} \\
ext' &: Con' \to Ty' \to Con' \text{,} \\
unit' &: Con' \to Ty' \text{, and} \\
pi' &: Con' \to Ty' \to Ty' \to Ty' \text{.}
\end{align*}

But this transformation, which we will call \emph{type erasure} loses important
information about the constructed types:
In the syntax generated by $Con'$ and $Ty'$, all types exist in the same context.
There is no way to tell that the codomain of the $\Pi$-types may depend on its domain,
and that the $\Pi$-type itself exists in the same context as its domain.
This justifies that we might call the above types the \emph{presyntax} associated
to the syntax given by $Con$ and $Ty$, consisting of \emph{precontexts} and
\emph{pretypes}.

To counteract this shortcoming, we reintroduce the \emph{typing relation} as a pair of
predicates over the presyntax.
These inductively defined predicates capture whether an instance of $Con'$ or $Ty'$ is \emph{wellformed} according to the original typing.
For the contexts, this is a simple property $w_{Con} : Con' \to \UU$, while for
types, it needs to state what precontext a pretype is wellformed in:
$w_{Ty} : Con' \to Ty' \to \UU$.
Note that these are inductive families since the definition of all indexing types
is concluded at the point of the definition of $W_{Con}$ and $W_{Ty}$.
The point constructors for the wellformedness predicates simply state that
is preserved by all constructors of $Con'$ and $Ty'$, in the case of $Ty'$ given the
correct index:
\begin{align*}
w_{nil} &: W_{Con}(nil') \text{,} \\
w_{ext} &: \{\Gamma : Con'\} \{A : Ty'\}
    \to W_{Con}(\Gamma) \to W_{Ty}(\Gamma, A) \\
  & \qquad \to W_{Con}(ext'(\Gamma, A)) \text{,}\\
w_{unit} &: \{\Gamma : Con'\} \to W_{Con}(\Gamma) \to W_{Ty}(unit'(\Gamma)) \text{, and}\\
w_{pi} &: \{\Gamma : Con'\} \{A, B : Ty'\} \to
    W_{Con}(\Gamma) \\
  & \qquad \to W_{Ty}(\Gamma, A) \to W_{Ty}(ext'(\Gamma, A), B)
    \to W_{Ty}(\Gamma, pi'(\Gamma, A, B)) \text{.}
\end{align*}

Now we can use the predicates to cut out the correct subset of $Con'$ and
$Ty'$ which we deem correct:
A context is a precontext together with a proof of its wellformedness just
as a type is a pretype together with a welltypedness witness:
\begin{align*}
Con &:\equiv (\Gamma : Con') \times W_{Con}(\Gamma) \text{ and}\\
Ty(\Gamma) &:\equiv (A : Ty') \times W_{Ty}(\pr_1(\Gamma), A) \text{.}
\end{align*}
The four point constructors are then easy to define as pairs:
\begin{align*}
nil &:\equiv (nil', w_{nil}) \text{,}\\
ext(\Gamma, A) &:\equiv \left(ext'(\pr_1(\Gamma),
    \pr_1(A)), w_{ext}(\pr_2(\Gamma), \pr_2(A)) \right) \text{,}\\
unit(\Gamma) &:\equiv \left(unit'(\pr_1(\Gamma)),
    w_{unit}(\pr_2(\Gamma)) \right) \text{, and}\\
pi(\Gamma, A, B) &:\equiv \left(pi'(\pr_1(\Gamma), \pr_1(A), \pr_1(B)),
   w_{pi}(\pr_2(\Gamma), \pr_2(A), \pr_2(B)) \right) \text{.}
\end{align*}

This definition clearly has the correct type signature but for it to be the correct
replacement for the intended inductive-inductive type,
we also need to construct its eliminator:
For any given $C : \UU$ and $T : C \to \UU$ with
\begin{align*}
n &: C \text{,} \\
e &: (\gamma : C) \to T(\gamma) \to C \text{,} \\
u &: (\gamma : C) \to T(\gamma) \text{, and} \\
p &: (\gamma : C) (a : T(\gamma)) \to T(e(\gamma, a)) \to T(\gamma) \text{,}
\end{align*}
we need to construct functions $\rec^{Con} : Con \to C$ and 
$\rec^{Ty} : \{\Gamma : Con\} \to Ty(\Gamma) \to T(\rec^{Con}(\Gamma))$ such
that the preservation of the point constructors is manifested in the
following $\beta$-rules:
\begin{align*}
\rec^{Con}(nil)
  &= n \text{,} \\
\rec^{Con}(ext(\Gamma, A))
  &= e(\rec^{Con}(\Gamma), \rec^{Ty}(A)) \text{,} \\
\rec^{Ty}(unit(\Gamma))
  &= u(\rec^{Con}(\Gamma)) \text{, and} \\
\rec^{Ty}(pi(\Gamma, A, B))
  &= p(\rec^{Con}(\Gamma), \rec^{Ty}(A), \rec^{Ty}(B)) \text{.}
\end{align*}

It is difficult to define these functions straight away, but it turns out that
we will be able to define an \emph{eliminator relation} between the presyntax
$(Con', Ty')$ and the motive $(C, T)$ which we can show restricts to the graph
of a function on the wellformed parts of the syntax.
Just like the wellformedness predicate, this relation is defined inductively
as a type family over the presyntax.
The signature of this relation is $R_{Con} : Con' \to C \to \UU$ for contexts
and $R_{Ty} : (\Gamma : Ty') \{\gamma : C\} \to T(\gamma) \to \UU$ and the
constructors for the relation state that relatedness is preserved by each
constructor of the presyntax:
\begin{align*}
r_{nil} 
  &: R_{Con}(nil', n) \text{,} \\
r_{ext}(\Gamma, A, \gamma, a) 
  &: R_{Con}(\Gamma, \gamma) \to R_{Ty}(A, a)
    \to R_{Con}(ext'(\Gamma, A), e(\gamma, a)) \text{,}\\
r_{unit}(\Gamma, \gamma)
  &: R_{Con}(\Gamma, \gamma) \to R_{Ty}(unit'(\Gamma), u(\gamma)) \text{, and} \\
r_{pi}(\Gamma, A, B, \gamma, a, b)
  &: R_{Con}(\Gamma, \gamma) \to R_{Ty}(A, a) \to R_{Ty}(B, b) \\
  & \qquad \to R_{Ty}(pi'(\Gamma, A, B), p(\gamma, a, b)) \text{.}
\end{align*}

Since we want a morphism to the model $(C, T, n, e, u, p)$ instead of a relation,
we now need to prove that the relation is in fact the graph of a function --
i.\,e. it is right-unique and left-total.

\begin{lemma}\label{lem:red-ex-right-unique}
The relation is right-unique on contexts and types. That is, for
$\gamma, \gamma' : C$ with $R_{Con}(\Gamma, \gamma)$ and $R_{Con}(\Gamma, \gamma')$,
we have $\gamma = \gamma'$, and, regarding types, for $\gamma : C$ and
$a, a' : T(\gamma)$, with $R_{Ty}(A, a)$ and $R_{Ty}(A, a')$,
we have $a = a'$.
\end{lemma}

\begin{proof}
Let us first apply induction on the presyntactic variables $\Gamma$ and $A$,
respectively.
This leaves us to consider the cases of the four constructors of $Con'$ and $Ty'$.
For the case of $nil'$, we observe that the only constructor resulting in
$R_{Con}(nil', \gamma)$ for some $\gamma$ is $r_{nil} : R_{Con}(nil, n)$ and we
can conclude that both $\gamma$ and $\gamma'$ must be equal to $n$.
The reasoning analogously applies to the other cases as well:
There is only one relatedness constructor for each of the constructors of the
presyntax, so we can always obtain the right-uniqueness for all arguments via
the induction hypothesis and, by congruence, infer that the uniqueness carries
over to the constructor in consideration.
As an example, in the case of $\delta$ and $\delta'$ with $R_{Con}(ext'(\Gamma, A), \delta)$
and $R_{Con}(ext'(\Gamma, A), \delta')$, we first conclude that
$\delta = e(\gamma, a)$ and $\delta' = e(\gamma', a')$ for some $\gamma$, $\gamma'$,
$a$, and $a'$, we see that for these $R_{Con}(\Gamma, \gamma)$, $R_{Con}(\Gamma, \gamma')$,
$R_{Ty}(A, a)$, and $R_{Ty}(A, a')$ have to hold and from this we infer
that $\gamma = \gamma'$ as well as $a = a'$ and thus $\delta = \delta'$.
\end{proof}

\begin{lemma}\label{lem:red-ex-total}
The eliminator relation is left-total on wellformed presyntax:
For $\Gamma : Con'$ with $W_{Con}(\Gamma)$ there is $\gamma : C$
such that $R_{Con}(\Gamma, \gamma)$.
Analogously, for $A : Ty'(\Gamma)$ with $W_{Ty}(\Gamma, A)$
and $\gamma : C$ with $R_{Con}(\Gamma, \gamma)$
there is $a : T(\gamma)$
such that $R_{Ty}(A, a)$.
\end{lemma}

\begin{proof}
Again, we first perform induction on the presyntactic argument to the statement
-- that is, $\Gamma$ or $A$.
The case of $nil'$ is trivial by providing $n$ and $r_{nil}$.
So let us look at the case of $ext'(\Gamma, A)$.
From the induction hypothesis, we get witnesses for the wellformedness of the
arguments in the form of $W_{Con}(\Gamma)$ and $W_{Ty}(\Gamma, A)$,
as well as related data from the algebra:
$\gamma : C$ with $R_{Con}(\Gamma, \gamma)$, and
$a : T(\gamma)$ with $R_{Ty}(A, a)$.
But this is all the input to use $r_{ext}$ to obtain $R_{Con}(ext'(\Gamma, A), e(\gamma, a))$.
The other two cases can be proved analogously.
\end{proof}

The left-totality will suffice to define the recursor functions by simply setting
$\rec^{Con}(\Gamma)$ and $\rec^{Ty}(A)$ to be the respective witnesses gained from
Lemma~\ref{lem:red-ex-total}.
This means that the $\beta$-rule for the non-recursive constructor $nil$ will be
definitional, while to prove $\beta$-rules for the recursive constructors will
require the use of Lemma~\ref{lem:red-ex-right-unique}:

To prove, for example, that $\rec^{Con}(ext(\Gamma, A))
= e(\rec^{Con}(\Gamma), \rec^{Ty}(A))$ holds, we observe that both the left-hand side
and the right-hand side provide elements in $C$ which by $R_{Con}$ are related
to $ext'(\Gamma, A)$, so the lemma give us the desired equality.


\section{(OLD) -- Fragments of Inductive-Inductive Types}

As we have seen in the previous sections, inductive-inductive types as specified
allow for a very broad variety of definitions.
We will now see that it is easy to carve out different subsets of specifications
to obtain more restrictive fragments of inductive types.
Starting from the largest of these subsets, we will first see that there is a
straightforward way to restrict inductive-inductive types to those whose constructors
are finitary in the sense that no point constructor depends on an infinite
amount of data: %TODO improve that last sentence

\begin{defn}[Finitary IITs]
Given a specification \grm{\Gamma} for an inductive-inductive types we say that
it is \textbf{finitary} if the infinitary $\Pi$-type is not used.
\end{defn}

One example of a specification which does not meet this requirement are the
infinitely branching trees. %TODO cite example

Instead of only preventing the use of external data in ``small functions''

We want to reduce inductive-inductive types to inductive families.
This means, we postulate that inductive families be admissible in our target
type theory and show that this implies the existence of all inductive-inductive
types.
Since we want to reuse the way of specifying inductive-inductive types to specify
instances which are as well inductive families, we want to rediscover the specifications
of inductive families as a subset of all inductive-inductive specifications:

\begin{defn}[Inductive families]
A context \grm{\Gamma} is said to specify a \textbf{inductive family} if it is
generated without using the inductive function type in the specification of sorts
and thus, no sorts depend on other sorts but are only iterated function depending
on external types.
This means that the formation rule for inductive function types is restricted
to the case where \grm{k \equiv \Pc}.
\end{defn}

We assume for the remainder of this chapter, that if \grm{\Gamma} specifies an
inductive family, we are provided with $\con{\Gamma} : \grm{\Gamma}^\CC$ and
$\elim_\grm{\Gamma} : \grm{\Gamma}^\EE(\con{\Gamma}, m)$ for each
$m : \grm{\Gamma}^\MM$.

TODO: explain reduction to W-types maybe

\section{Type Erasure}

As seen in the examples, the first step to prove the reducability is to formally
define the operation which we will call \emph{flattening} or -- inspired by
the syntax example -- \emph{type erasure}.
This operation strips away any dependencies between the sorts of a signature
as well as all external indices to sorts.
The operation should take arbitrary inductive-inductive signatures (contexts) and
return signatures for inductive families.
Let us look at what type erasure should do with our running examples:

\begin{example}[Natural Numbers]\label{ex:red-e-nat}
Since the inductive-inductive signature of the \emph{natural numbers}~\ref{ex:ii-syntax-nat} doesn't
contain any indexed sorts, type erasure should ``do nothing'' with it.
That is, returning the sort context and point context of the inductive family
syntax which looks like a obvious correspondence to it (cf. Example~\ref{ex:if-natvec})
while ignoring all entries of the other kind:
Let
\begin{equation*}
\grm{\Gamma_{nat}} 
  :\equiv \grm{(\cdot,\, \UU,\, \El(\vz),\, \Pi\left(\vs(\vz),\, \El(\vs(\vs(\vz)))\right))}
  \text{.}
\end{equation*}
We want to have the following split into sort types and point types:
\begin{align*}
\tqm{\grm{\Gamma_{nat}}^\EE_\Sc}
 &= \tqm{(\cdot_\Sc,\, \UU)} \text{ and} \\
\tqm{\grm{\Gamma_{nat}}^\EE}
 &= \tqm{(\cdot,\, \El(\var(\vz)),\, \var(\vz) \Rightarrow_\Pc \El(\var(\vz)))} \text{.}
\end{align*}
\end{example}

\begin{example}[Vectors]\label{ex:red-e-vec}
In the example of vectors \ref{ex:ii-syntax-vec} we need to erase the natural numbers
index of the only sort under consideration:
\begin{align*}
\tqm{\grm{\Gamma_{vec}}^\EE_\Sc}
 &= \tqm{(\cdot_\Sc,\, \UU)} \text{ and} \\
\tqm{\grm{\Gamma_{vec}}^\EE}
  &= \tqm{(\cdot,\, \El(\var(\vz)),\, 
    \ExtPiP{A}{\blm{\lambda a.\,}\ExtPiP{\N}{\blm{\lambda n.\,}
    \var(\vz) \Rightarrow_\Pc \El(\var(\vz))}})} \text{.}
\end{align*}
Note that the erasure of the vectors does not coincide with the vectors represented
as an inductive family (Example~\ref{ex:if-natvec}), because its sort lacks the
indexing over the natural numbers.
In fact, it's easy to see that the algebras of this signature would not be isomorphic
to the type of lists over the type \blm{A \times \N}.
\end{example}

\begin{example}[Type Theory Syntax]
In our syntax we will now see why the operation is called ``type erasure'':
%TODO
\end{example}

To go from examples to the general case, we will present the different components
of the type erasure operation in roughly the same order in which they appear in
Section~\ref{sec:ii-syntax}, most often needing to distinguish between sort
and point constructors.

\begin{defn}[Type Erasure]
First of all, each context will need to be split into a sort context and a point
context:
\begin{equation*}
\inferrule{\grm{\vdash \Gamma}}
  {\tqm{\SCon \grm{\Gamma}^\EE_\Sc}}
\qquad
\inferrule{\grm{\vdash \Gamma}}
  {\tqm{\vdash_{\grm{\Gamma}^\EE_\Sc} \grm{\Gamma}^\EE }}
\end{equation*}
To descent down the components of the contexts, we will need to define the operation
on types as well.
Since we are erasing all information from the sorts, we will only need this for
point types, though.
Unsurprisingly, we want them to be translated to point types in the appropriate
sort context:
\begin{equation*}
\inferrule{\grm{\Gamma \vdash A :: \Pc}}
  {\tqm{\grm{\Gamma}^\EE_\Sc \SCon \grm{A}^\EE :: \Pc}}
\end{equation*}
Using this we will be able to define the operation creating sort contexts by
\begin{align*}
\tqm{\grm{\cdot}^\EE_\Sc}
  &:\equiv\tqm{\cdot_\Sc} \text{,} \\
\tqm{\grm{(\Gamma,\, B)}^\EE_\Sc}
  &:\equiv \tqm{\left(\grm{\Gamma}^\EE_\Sc,\, \grm{\UU}^\EE_\Sc\right)} \text{ for \grm{B :: \Sc}, and} \\
\tqm{\grm{(\Gamma,\, A)}^\EE_\Sc}
  &:\equiv \tqm{\grm{\Gamma}^\EE_\Sc} \text{ for \grm{A :: \Pc}.}
\end{align*}
The generated point context over this sort context has to be extended in the case
where the input is an extension by a point type.
In the case where it is an extension by a sort type, we want to return the
unextended context, but to make up for the definition above, we need to weaken
to account for the extension of the resulting sort context:
\begin{align*}
\tqm{\grm{\cdot}^\EE}
  &:\equiv\tqm{\cdot} \text{,} \\
\tqm{\grm{(\Gamma,\, B)}^\EE}
  &:\equiv \tqm{\grm{\Gamma}^\EE[\wk_{\id}]} \text{ for \grm{B :: \Sc}, and} \\
\tqm{\grm{(\Gamma,\, A)}^\EE}
  &:\equiv \tqm{\left(\grm{\Gamma}^\EE,\, \grm{A}^\EE\right)} \text{ for \grm{A :: \Pc}.}
\end{align*}
So how do we define \tqm{\grm{A}^\EE} for a point type \grm{A}?
The fact the we have to recurse on \grm{\El(a)} makes it clear that we will have
to extend our operation to terms of sort types at least.
That is, together with \tqm{\grm{A}^\EE} we also need the following:
\begin{equation*}
\inferrule{\grm{\Gamma \vdash t : B :: \Sc}}
  {\tqm{\grm{\Gamma}^\EE_\Sc \SCon \grm{t}^\EE : \UU}}
\end{equation*}
And indeed, with this we can set
\begin{align*}
\tqm{\grm{\El(a)}^\EE}
  &:\equiv \tqm{\El(\grm{a}^\EE)} \text{.}
\end{align*}
For recursive $\Pi$-types, we need only care about the ones yielding point types.
Note that the operation turns a $\Pi$-type into a non-dependent function type!
\begin{align*}
\tqm{\grm{\Pi(a, A)}^\EE}
  &:\equiv \tqm{\grm{a}^\EE \Rightarrow_\Pc \grm{A}^\EE}
\end{align*}
Since we forgot about the indexing of sort types, erasure of sort-kinded application terms
is just erasure of its $\Pi$-type term:
\begin{align*}
\tqm{\grm{\IIapp(f)}^\EE}
  &:\equiv \tqm{\grm{f}^\EE} \text{ for \grm{\Gamma \vdash t : \Pi(a, B) :: \Sc}.}
\end{align*}
External $\Pi$-types convert directly into their
respective counterparts in the syntax of inductive families.
For application of terms of sort-kinded $\Pi$-types we need to erase the argument
since we erased the $\Pi$-type itself.
\begin{align*}
\tqm{\grm{\ExtPi{T}{A}}^\EE}
  &:\equiv \tqm{\ExtPiP{T}{\blm{\lambda \tau.\, }\grm{A(\bltau)}^\EE}} \text{, and} \\
\tqm{\grm{f(\bltau)}^\EE}
  &:\equiv \tqm{\grm{f}^\EE} \text{ for \grm{\Gamma \vdash f : \ExtPi{T}{B} : \Sc}}
\end{align*} %TODO this is a bit confusing since the application is for sorts and the types for points
Defining the erasure on point types and sort terms pulled back along a substitution,
we see that we will also need to erase entire sort substitutions.
This is achieved by extending the operation as follows:
\begin{equation*}
\inferrule{\grm{\IISub{\sigma}{\Gamma}{\Delta}}}
  {\tqm{\IISub{\grm{\sigma}^\EE_\Sc}{\grm{\Gamma}^\EE_\Sc}{\grm{\Delta}^\EE_\Sc}}}
\end{equation*}
We will then be able to use this in a straight forward way to define the pullbacks:
\begin{align*}
\tqm{\grm{A[\sigma]}^\EE}
  &:\equiv \tqm{\grm{A}^\EE[\grm{\sigma}^\EE_\Sc]}
  & \text{ for \grm{\Gamma \vdash A :: \Pc} and} \\
\tqm{\grm{t[\sigma]}^\EE}
  &:\equiv \tqm{\grm{t}^\EE[\grm{\sigma}^\EE_\Sc]}
  & \text{ for \grm{\Gamma \vdash t : B :: \Sc}.}
\end{align*}
Erasure of substitutions is built recursively, ignoring point types.
Likewise, the first projection will ignore point types:
\begin{align*}\label{eq:red-e-sub}
\tqm{\grm{\id}^\EE_\Sc}
  &:\equiv \tqm{\id} \text{,}
  & \\
\tqm{\grm{(\sigma \circ \delta)}^\EE_\Sc}
  &:\equiv \tqm{\grm{\sigma}^\EE_\Sc \circ \grm{\delta}^\EE_\Sc} \text{,}
  & \\
\tqm{\grm{\epsilon}^\EE_\Sc}
  &:\equiv \tqm{\epsilon} \text{,}
  & \\
\tqm{\grm{(\sigma,\, t)}^\EE_\Sc}
  &:\equiv \tqm{(\grm{\sigma}^\EE_\Sc,\, \grm{t}^\EE)}
  & \text{ for \grm{\Gamma \vdash t : B[\sigma] :: \Sc},} \\
\tqm{\grm{(\sigma,\, t)}^\EE_\Sc}
  &:\equiv \tqm{\grm{\sigma}^\EE_\Sc}
  & \text{ for \grm{\Gamma \vdash t : A[\sigma] :: \Pc},} \\
\tqm{\grm{\pi_1(\sigma)}^\EE_\Sc}
  &:\equiv \tqm{\pi_1(\grm{\sigma}^\EE_\Sc)}
  & \text{ for \grm{\IISub{\sigma}{\Gamma}{(\Delta,\, B :: \Sc)}},} \\
\tqm{\grm{\pi_1(\sigma)}^\EE_\Sc}
  &:\equiv \tqm{\grm{\sigma}^\EE_\Sc}
  & \text{ for \grm{\IISub{\sigma}{\Gamma}{(\Delta,\, A :: \Pc)}}, and} \\
\tqm{\grm{\pi_2(\sigma)}^\EE}
  &:\equiv \tqm{\pi_2(\grm{\sigma}^\EE_\Sc)} \text{.}
  &
\end{align*} %TODO laws
This concludes the definition of the erasure operation.
\end{defn} %TODO example derivations

For the steps that follow it will be necessary to equip the \emph{algebras}
of the resulting signatures with a substitution calculus that also considers
point contexts instead of only sort contexts.
To this end, we extend the operation of type erasure by assigning a map between
the types of algebras of the erasure to each substitution:

\begin{defn}[Replacement for Point Substitutions]\label{def:red-e-points}
We define the following operation on substitutions:
\begin{equation*}
\inferrule{\grm{\IISub{\sigma}{\Gamma}{\Delta}}}
  {\grm{\sigma}^\EE : \left\{\gamma_\Sc : \tqm{\grm{\Gamma}^\EE_\Sc}^\AA \right\}
    \to \tqm{\grm{\Gamma}^\EE}^\AA(\gamma_\Sc)
    \to \tqm{\grm{\Delta}^\EE}^\AA\left(\grm{\sigma}^\EE_\Sc(\gamma_\Sc)\right)}
\end{equation*}
While in for \tqm{\grm{\sigma}^\EE_\Sc} we ignored point constructors
(see \ref{eq:red-e-sub} above)
this time we will to the opposite and ignore all sort constructors:
\begin{align*}
\grm{\id}^\EE(\gamma)
  &:\equiv \gamma \text{,} \\
\grm{\sigma \circ \delta}^\EE(\gamma)
  &:\equiv \grm{\sigma}^\EE\left(\grm{\delta}^\EE(\gamma)\right) \text{,} \\
\grm{\epsilon}^\EE(\gamma)
  &:\equiv \star \text{,} \\
\grm{(\sigma,\, t)}^\EE(\gamma)
  &:\equiv \grm{\sigma}^\EE(\gamma)
  & \text{ for \grm{\Gamma \vdash t : B[\sigma] :: \Sc },} \\
\grm{(\sigma,\, t)}^\EE(\gamma)
  &:\equiv \left(\grm{\sigma}^\EE, \grm{t}^\EE\right)
  & \text{ for \grm{\Gamma \vdash t : A[\sigma] :: \Pc},} \\
\grm{\pi_1(\sigma)}^\EE(\gamma)
  &:\equiv \grm{\sigma}^\EE(\gamma)
  & \text{ for \grm{\IISub{\sigma}{\Gamma}{(\Delta,\, B :: \Sc)}},} \\
\grm{\pi_1(\sigma)}^\EE(\gamma, \alpha)
  &:\equiv \grm{\sigma}^\EE(\gamma)
  & \text{ for \grm{\IISub{\sigma}{\Gamma}{(\Delta,\, A :: \Pc)}}.}
\end{align*}
\end{defn}

\section{The Wellformedness Predicate}

To remove the ambiguity created by the type erasure we will now have to find
a way to select those instances of the types which are ``wellformed'' in the
sense that the lie in the correct fibers of dependent sorts.
This predicate will be a proposition dependent on a realization of the erased
signature, i.\,e. on contexts, it will be a function on the type of algebras
of the erasure.
It is important keep this dependencies and not only to use the initial such
algebra, since when we will recursively define this wellformedness predicate,
the corresponding piece of signature will not always be initial
-- in the same way in which a projection of an initial algebra is not necessarily
initial anymore.

\begin{example}[Natural Numbers]\label{ex:red-w-nat}
Taking up the example of \grm{\Gamma_{nat}} from \ref{ex:red-e-nat},
we observe that algebras of \tqm{\grm{\Gamma_{nat}}^\EE_\Sc} take the form of
$(\star, N)$ with $N : \UU$ and, given $N$, those of
\tqm{\grm{\Gamma_{nat}}^\EE} are of the form $(\star, z, s)$ with
$z : N$ and $s : N \to N$.
Our wellformedness predicate in this case will encode a type family on $N$, inductively
populated by elements ``over'' $z$ and $n$.
The code for its sort and point constructors looks as follows:
\begin{align*}
\tqm{\grm{\Gamma_{nat}}^\WW_\Sc}(\star, z, s)
  &= \tqm{\left( \cdot_\Sc,\, \ExtPiS{N}{\UU} \right)} \text{ and} \\
\tqm{\grm{\Gamma_{nat}}^\WW}(\star, z, s)
  &= \tqm{\left( \cdot,\,  \El(\var(\vz)(\blm{z})),\,
    \ExtPiP{n : N}{\var(\vz)(\blm{n}) \Rightarrow_\Pc \El(\var(\vz)(\blm{s(n)}))} \right)}
\end{align*} %TODO
It's easy to see that the initial algebra of this signature is nothing more than
the trivial (final) type family on $N$.
\end{example}

\begin{example}[Vectors]\label{ex:red-w-vec}
For vectors on a type $A : \UU$, the duties of the wellformednes predicate are less trivial:
We have to add back the length information which we erased, as described in
\ref{ex:red-e-vec}:
Empty vectors should have length zero and appending an element should increase its
length by one.
This can be achieved by, given the data from an erasure algebra in the form of
$V : \UU$, $n: V$, and $c : A \to \N \to V \to V$,
having a predicate encoded by
\begin{equation*}
\tqm{\grm{\Gamma_{vec}}^\WW_\Sc}(\star, n, c)
  = \tqm{\left(\cdot_\Sc,\, \ExtPiS{n : \N}{\ExtPiS{v : V}{\UU}} \right)} \text{,}
\end{equation*}
with point constructors that ensure the correct lengh by setting $\tqm{\grm{\Gamma_{vec}}^\WW}(\star, n, c)$ to be the point context
\begin{equation*}
\begin{gathered}
\tqm{\cdot,\, \El(\var(\vz)(\blm{0}, \blm{n})),\,} \\
\tqm{\ExtPiP{a: A}{\ExtPiP{n : \N}{\ExtPiP{v : V}{\var(\vz)(\blm{n}, \blm{v}) \Rightarrow_\Pc 
  \El(\var(\vz)(\blm{n + 1}, \blm{c(a, n, v)}) ) } } } } \text{.}
\end{gathered}
\end{equation*}
\end{example}

Like for the type erasure, we will no proceed to generalize this to arbitrary
inductive-inductive types.

\begin{defn}[Wellformedness Predicates]
Again, we start by considering what the resulting type on context.
Clearly, we want the operation to result in the sort context and the point context
of another signature of an inductive family.
As we have alredy seen in the previous exapmles,
there needs to be a dependency on an erasure algebra which leads to the following
rules:
\begin{equation*}
\begin{gathered}
\inferrule{\grm{\vdash \Gamma} \\
  \gamma_\Sc : \tqm{\grm{\Gamma}^\EE_\Sc }^\CC \\
  \gamma : \tqm{\grm{\Gamma}^\EE }^\CC(\gamma_\Sc) }
  {\tqm{\SCon \grm{\Gamma}^\WW_\Sc(\blm{\gamma})} }
\\[.7em]
\inferrule{\grm{\vdash \Gamma} \\
  \gamma_\Sc : \tqm{\grm{\Gamma}^\EE_\Sc }^\CC \\
  \gamma : \tqm{\grm{\Gamma}^\EE }^\CC(\gamma_\Sc) }
  {\tqm{\vdash_{\grm{\Gamma}^\WW_\Sc(\blm{\gamma})} \grm{\Gamma}^\WW(\blm{\gamma}) }}
\end{gathered}
\end{equation*}

To be able to do recursion we will again need to provide a suitable operation
on types.
We need to distinguish between sort and point types.
For sort types, note that we don't have an erasure operation of which we could take
an algebra, but since, implicitly, every input sort turns into the inductive-family
universe token \tqm{\UU}, we know that we can act as if its universe is a plain type.
Also, we need to know the interpretation of the erasure of the context the type is based on.
\begin{equation*}
\begin{gathered}
\inferrule{\grm{\Gamma \vdash B :: \Sc} \\
  \gamma_\Sc : \tqm{\grm{\Gamma}^\EE_\Sc }^\CC \\
  \gamma : \tqm{\grm{\Gamma}^\EE }^\CC(\gamma_\Sc) \\
  \alpha : \UU }
  {\tqm{\SCon \grm{B}^\WW(\blgamma, \blalpha) :: \Sc }}
\\[.7em]
\inferrule{\grm{\Gamma \vdash A :: \Pc} \\
  \gamma_\Sc : \tqm{\grm{\Gamma}^\EE_\Sc }^\CC \\
  \gamma : \tqm{\grm{\Gamma}^\EE }^\CC(\gamma_\Sc) \\
  \alpha : \tqm{\grm{A}^\EE}^\CC(\gamma_\Sc) }
  {\tqm{\grm{\Gamma}^\WW_\Sc(\blm{\gamma}) \SCon \grm{A}^\WW(\blgamma, \blalpha) :: \Pc}}
\end{gathered}
\end{equation*}
The recursion of the context then looks very much like the on in the definition
of type erasure:
Extending the sort context whenever we encounter a sort type in the 
inductive-inductive signature and extending the point case for each point type.
Again, we can not leave the point context fixed ``on the nose'' when encountering
a sort type since we need to weaken it to account for the new sort:
\begin{align*}
\tqm{\grm{\cdot}^\WW_\Sc(\blgamma)}
  &:\equiv \tqm{\cdot_\Sc} \\
\tqm{\grm{(\Gamma,\, B :: \Sc)}^\WW_\Sc\{\gamma_\Sc, \alpha\}(\blgamma)}
  &:\equiv \tqm{\left( \grm{\Gamma}^\WW_\Sc(\blgamma)
    ,\, \grm{B}^\WW(\blgamma, \blalpha)\right)} \\
\tqm{\grm{(\Gamma,\, A :: \Pc)}^\WW_\Sc(\blgamma, \blalpha)}
  &:\equiv \tqm{\grm{\Gamma}^\WW_\Sc(\blgamma) } \\[.7em]
\tqm{\grm{\cdot}^\WW(\blgamma)}
  &:\equiv \tqm{\cdot} \\
\tqm{\grm{(\Gamma,\, B :: \Sc)}^\WW(\blgamma)}
  &:\equiv \tqm{\grm{\Gamma}^\WW(\blgamma)[\wk_\id]} \\
\tqm{\grm{(\Gamma,\, A :: \Pc)}^\WW(\blgamma, \blalpha)}
  &:\equiv \tqm{\left(\grm{\Gamma}^\WW(\blgamma),\, \grm{A}^\WW(\blgamma, \blalpha)\right) }
\end{align*}

Like in the definition of type erasure, recursing on \grm{\El(a)} makes it
necessary to extend the definition at least to sort types.
So we will also give an operation producing the following data:
\begin{equation*}
\inferrule{\grm{\Gamma \vdash t : B :: \Sc} \\
  \gamma_\Sc : \tqm{\grm{\Gamma}^\EE_\Sc }^\CC \\
  \gamma : \tqm{\grm{\Gamma}^\EE }^\CC(\gamma_\Sc)}
  {\tqm{\grm{\Gamma}^\WW_\Sc(\blgamma) \SCon \grm{t}^\WW(\blgamma)
    : \grm{B}^\WW(\blgamma, \blm{\tqm{\grm{t}^\EE}^\CC(\gamma_\Sc)})}}
\end{equation*}

Let us now proceed to give the definition on all type formers.
The each sort of the input signature should become a predicate.
Since a predicate is the same as a type family with propositional values,
we set the wellformedness on the universe to be a type family, the domain of which
is given by the set we obtain from the algebra of the erased context.
Not that this type family is a non-dependent, \emph{non-recursive} $\Pi$-type.
The interpretation of \grm{\El(a)} has to make up for this shift by applying
to the wellformedness predicate corresponding the sort term \grm{a} the
element we get from the erasure of \grm{\El(a)}: %TODO this still reads horrible
\begin{align*}
\tqm{\grm{\UU}^\WW(\blgamma, \blalpha)}
  &:\equiv \tqm{\ExtPiS{x : \blalpha}{\UU}} \text{ and} \\
\tqm{\grm{\El(a)}^\WW(\blgamma, \blalpha)}
  &:\equiv \tqm{\El\left(\grm{a}^\WW(\blgamma)(\blalpha)\right)} \text{.}
\end{align*}

For sort-kinded, recursive $\Pi$-types, we again need to remember that in the
definition of type erasure, we turned them into instances of \tqm{\UU}, so to
add the information back which we erased, the wellformedness has to turn them into
non-recursive $\Pi$-types over the erasure of sort term which is the domain of the
$\Pi$-type we started with.
The iterpretation of application terms has to follow this step accordingly:
\begin{align*}
\tqm{\grm{\Pi(a, B :: \Sc)}^\WW\{\blm{\gamma_\Sc}\}(\blgamma, \blphi)}
  &:\equiv \tqm{\ExtPiS{\blalpha : \tqm{\grm{a}^\EE}^\CC(\gamma_\Sc)}
    {\grm{B}^\WW((\blgamma, \blalpha), \blphi)}} \text { and} \\
\tqm{\grm{\IIapp(f)}^\WW(\blgamma, \blalpha)}
  &:\equiv \tqm{\grm{f}^\WW(\blgamma)(\blalpha)}
  \text{ for \grm{\Gamma \vdash f : \Pi(a, B :: \Sc)}.}
\end{align*}

The treatment of $\Pi$-types in point constructors is arguably the trickiest part
of the definition.
A non-technical description of the effect of the wellformedness operation on
these $\Pi$-types is the following:
For each bit of input data from an algebra of the erasure, wellformedness of this
input data should imply wellformedness of the result.
\begin{equation*}
\tqm{\grm{\Pi(a, A :: \Pc)}^\WW\{\blm{\gamma_\Sc}\}(\blgamma, \blphi)}
  :\equiv \tqm{\ExtPi{\blalpha : \tqm{\grm{a}^\EE}^\CC(\gamma_\Sc)}
    { \grm{a}^\WW(\blgamma)(\blalpha)
      \Rightarrow_\Pc \grm{A}^\WW((\gamma, \alpha), \phi(\alpha))}}
%\tqm{\grm{\IIapp(f)}^\WW\{\blm{\gamma_\Sc}\}
%  \{\blgamma, \blphi\}(\blm{\gamma^\WW}, \blm{\phi^\WW}) }
%  &:\equiv \tqm{\grm{f}^\WW(\blm{\gamma^\WW})(\blphi)(\blm{\phi^\WW}) }
%  \text{ for \grm{\Gamma \vdash f : \Pi(a, A :: \Pc)}.} TODO this belongs to extra constr
\end{equation*}

Let us next look at the non-recursive function types.
Since we erased them just like the recursive ones, they are processed similar to
the definitions above, with the difference that for point constructors, there is
no wellformedness of the domain that we have to presuppose to infer wellformedness
of the codomain:
\begin{align*}
\tqm{\grm{\ExtPi{T}{B :: \Sc}}^\WW(\blgamma, \blphi)}
  &:\equiv \tqm{\ExtPiS{\tau : T}{\grm{B(\bltau)}^\WW(\blgamma, \blphi)}} \text{,} \\
\tqm{\grm{\ExtPi{T}{A :: \Pc}}^\WW(\blgamma, \blphi)}
  &:\equiv \tqm{\ExtPiP{\tau : T}{\grm{A(\bltau)}^\WW(\blgamma, \blm{\phi(\tau)})}}
  \text{, and} \\
\tqm{\grm{f(\bltau)}^\WW(\blgamma)}
  &:\equiv \tqm{\grm{f}^\WW(\blgamma)(\bltau) } \text{.}
\end{align*}

Again, we need to extend the definition to substitutions to be able to specify
it on pulled back types and terms:
\begin{equation*}
\inferrule{\grm{\IISub{\sigma}{\Gamma}{\Delta}}\\
  \gamma_\Sc : \tqm{\grm{\Gamma}^\EE_\Sc }^\CC \\
  \gamma : \tqm{\grm{\Gamma}^\EE }^\CC(\gamma_\Sc)}
  {\tqm{\IFSub{\grm{\sigma}^\WW_\Sc(\blgamma)}{\grm{\Gamma}^\WW_\Sc(\blgamma)}
    {\grm{\Delta}^\WW_\Sc(\blgamma)}}}
\end{equation*}
Their category structure is a direct translation to the sort substitutions of
the inductive family syntax.
Note that here, we need to refer to \ref{def:red-e-points} to carry the
algebra of the erase point context along the substitution:
\begin{align*}
\tqm{\grm{\id}^\WW_\Sc(\blgamma)}
  &:\equiv \tqm{\id} \text{ and} \\
\tqm{\grm{(\sigma \circ \delta)}^\WW_\Sc(\blgamma)}
  &:\equiv \tqm{\grm{\sigma}^\WW_\Sc(\blm{\grm{\delta}^\EE(\gamma)})
    \circ \grm{\delta}^\WW_\Sc(\blgamma)} \text{.}
\end{align*}
The pullback operations can afterwards defined by
\begin{align*}
\tqm{\grm{B[\sigma]}^\WW(\blgamma, \blalpha)}
  &:\equiv \tqm{\grm{B}^\WW(\blm{\grm{\sigma}^\EE(\gamma)}, \blalpha)} \text{,} \\
\tqm{\grm{A[\sigma]}^\WW(\blgamma)}
  &:\equiv \tqm{\grm{A}^\WW(\blgamma, \blalpha)
    [\grm{\sigma}^\WW_\Sc(\blgamma)]} \text{, and} \\
\tqm{\grm{t[\sigma]}^\WW(\blgamma)}
  &:\equiv \tqm{\grm{t}^\WW_\Sc(\blm{\grm{\sigma}^\EE(\gamma)})
    [\grm{\sigma}^\WW_\Sc(\blgamma)] } \text{ for \grm{\Gamma \vdash t : B :: \Sc}.}
\end{align*}
The remaining pieces of substitutional calculus are straightforward and look the
same as for the type erasure:
\begin{align*}
\tqm{\grm{\epsilon}^\WW_\Sc(\blgamma)}
  &:\equiv \tqm{\epsilon} \text{,} \\
\tqm{\grm{(\sigma,\, t)}^\WW_\Sc(\blgamma)}
  &:\equiv \tqm{\left( \grm{\sigma}^\WW_\Sc(\blgamma),\,\grm{t}^\WW(\blgamma)\right)}
  \text{ for \grm{t : B :: \Sc},} \\
\tqm{\grm{(\sigma,\, t)}^\WW_\Sc(\blgamma)}
  &:\equiv \tqm{\grm{\sigma}^\WW_\Sc(\blgamma)} \\
\tqm{\grm{\pi_1(\sigma)}^\WW_\Sc(\blgamma)}
  &:\equiv \tqm{\pi_1(\grm{\sigma}^\WW_\Sc(\blgamma))}
  \text{ for \grm{\IISub{\sigma}{\Gamma}{\Delta,\, B :: \Sc}},} \\
\tqm{\grm{\pi_1(\sigma)}^\WW_\Sc(\blgamma)}
  &:\equiv \tqm{\grm{\sigma}^\WW_\Sc(\blgamma)}
  \text{ for \grm{\IISub{\sigma}{\Gamma}{\Delta,\, A :: \Pc}},} \\
\tqm{\grm{\pi_2(\sigma)}^\WW_\Sc(\blgamma)}
  &:\equiv \tqm{\pi_2(\grm{\sigma}^\WW_\Sc(\blgamma))}
  \text{ for \grm{\IISub{\sigma}{\Gamma}{\Delta,\, B :: \Sc}},} \\
\tqm{\grm{\pi_2(\sigma)}^\WW_\Sc(\blgamma)}
  &:\equiv \tqm{\grm{\sigma}^\WW_\Sc(\blgamma)}
  \text{ for \grm{\IISub{\sigma}{\Gamma}{\Delta,\, A :: \Pc}},}
\end{align*}

\section{The Initial Object}

Since we now have a way to ``carve out'' the wellformed elements from the types
we created via type erasure, can now define our desired inductive-inductive types
itself.
In this section, this will amount to defining just one specific algebra over the
given inductive-inductive signature.
This corresponds to giving sorts with the correct \emph{point constructors}.
What distinguishes this algebra from others is that we will in the later sections
demonstrate how to show that, besides constructors, it also admits a dependent
\emph{eliminator}, or, equivalently, that it is initial among all algebras.

The construction of the initial object obviously presupposes the existence of
initial algebras of inductive families.
Nevertheless, we need to apply the same strategy as in the definition of the
wellformedness predicate:
The construction will depend on arbitrary algebras of type erasure and wellformedness
instead of just relying on the initial on.
This allows us to descend recursively and still refer to the correct algebra
of the respective inductive families.

Like in the last two steps of the construction, let us again start off by taking
a look at our set of running examples:

\begin{example}[Natural Numbers]\label{ex:red-init-nat}
Continuing from Example \ref{ex:red-w-nat}, we again assume %TODO fix red-w-nat
sort and point algebras $(\star, N')$ and $(\star, z', s')$ of the erasure of
natural numbers \tqm{\grm{\Gamma_{nat}}^\EE_\Sc} and \tqm{\grm{\Gamma_{nat}}^\EE}.
Given this data,
the algebras of the wellformedness predicate take the form of
$(\star, W_N)$ and $(\star, w_z, w_s)$ with types
\begin{align*}
W_N &: N' \to \UU \text{,} \\
w_z &: W(z') \text{, and} \\
w_s &: (n' : N') \to W_N(n') \to W_N(s'(n')) \text{.}
\end{align*}
Then, we want the inductive-inductive algebra
$\con{\Gamma_{nat}} : \grm{\Gamma_{nat}}^\CC$ to consist of the subsets of
erased types which (in this case trivially) fulfil the wellformedness condition,
with the point constructors lifted to these subsets:
$\con{\Gamma_{nat}} = (\star, N, z, s)$ with
\begin{align*}
N &= (n' : N') \times W_N(n') \text{,} \\
z &= (z', w_z) \text{, and} \\
s &= \lambda ((n', w_n) : N).\, (s'(n'), w_s(n', w_n)) \text{.}
\end{align*}
\end{example}

\begin{example}[Vectors]
Let us next consider the type of vectors on a type $A : \UU$.
The assumed algebras of the type erasure give us $V'$, $n'$, and $c'$ as in
\ref{ex:red-w-vec}.
With those as input, algebras of the sort and point part of the 
wellformedness predicate \tqm{\grm{\Gamma_{vec}}^\WW_\Sc} and
\tqm{\grm{\Gamma_{vec}}^\WW} look like
$(\star, W_V)$ and $(\star, w_n, w_c)$ with
\begin{align*}
W_V &: \N \to V' \to \UU \text{,} \\
w_n &: W_V(0, n') \text{, and} \\
w_c &: (a : A)(m : \N)(v' : V) \to W_V(m, v') \to W_V(m + 1, c'(a, m, v')) \text{.}
\end{align*}
This suggests that we will have an algebra $\con{\Gamma_{vec}} : \Gamma_{vec}^\CC$
defined by $\con{\Gamma_{vec}} = (\star, V, n, c)$ with
\begin{align*}
V &= \lambda (n : \N).\, (v' : V') \times W_V(n, v') \text{,} \\
n &= (n', w_n) \text{, and} \\
c &= \lambda (a : A)(m : \N)((v', w_{v'}) : V).\, (c'(a, m, v'), w_c(a, m, v', w_{v'})) \text{.}
\end{align*}
\end{example}

\begin{example}[Type Theory Syntax]
baz
\end{example}

Let us now consider the case of an arbitrary signature \grm{\vdash \Gamma}.
The form which our operation will take is clear -- for each signature we need
to return an algebra of that signature:
\begin{equation*}
\inferrule{\grm{\vdash \Gamma}}
  {\con{\Gamma} : \grm{\Gamma}^\CC}
\end{equation*}
But as we saw before, recursion is easier when we make the dependent on arbitrary
algebras of the previous steps -- that is, arbitrary algebras over type erasure
and the wellformedness predicate.
After we succeed in defining this more general construction $\grm{\Gamma}^\Sg$,
we will eliminate this dependency by fixing these algebras to be the initial ones
which we assume to exist in this chapter.

\begin{defn}[Sigma Construction]
As mentioned, the more general construction will depend on both the type erasure
and the wellformedness, so that the operation will take the following form:
\begin{equation*}
\inferrule{\grm{\vdash \Gamma} \\
  \gamma_\Sc : \tqm{\grm{\Gamma}^\EE_\Sc }^\CC \\
  \gamma : \tqm{\grm{\Gamma}^\EE }^\CC(\gamma_\Sc) \\
  \delta_\Sc : \tqm{\grm{\Gamma}^\WW_\Sc(\blgamma)}^\CC \\
  \delta : \tqm{\grm{\Gamma}^\WW(\blgamma)}^\CC }
  {\grm{\Gamma}^\Sg(\gamma, \delta) : \grm{\Gamma}^\CC }
\end{equation*}

To recurse on the contexts, we again need to extend the operations to types,
distinguishing between sort and point constructors, resulting in the following
two rules:
\begin{equation*}
\begin{gathered}
\inferrule{\grm{\Gamma \vdash B :: \Sc} \\
  \gamma_\Sc : \tqm{\grm{\Gamma}^\EE_\Sc }^\CC \\
  \gamma : \tqm{\grm{\Gamma}^\EE }^\CC(\gamma_\Sc) \\
  \delta_\Sc : \tqm{\grm{\Gamma}^\WW_\Sc(\blgamma)}^\CC \\
  \delta : \tqm{\grm{\Gamma}^\WW(\blgamma)}^\CC \\
  \alpha : \UU \\
  \omega : \tqm{\grm{B}^\WW(\blgamma, \blalpha)}^\CC }
  {\grm{B}^\Sg(\gamma, \delta, \omega) 
   : \grm{B}^\CC\left(\grm{\Gamma}^\Sg(\gamma, \delta) \right) }
\\[.7em]
\inferrule{\grm{\Gamma \vdash A :: \Pc} \\
  \gamma_\Sc : \tqm{\grm{\Gamma}^\EE_\Sc }^\CC \\
  \gamma : \tqm{\grm{\Gamma}^\EE }^\CC(\gamma_\Sc) \\
  \delta_\Sc : \tqm{\grm{\Gamma}^\WW_\Sc(\blgamma)}^\CC \\
  \delta : \tqm{\grm{\Gamma}^\WW(\blgamma)}^\CC \\
  \alpha : \tqm{\grm{A}^\EE}^\CC(\gamma_\Sc) \\
  \omega : \tqm{\grm{A}^\WW(\blgamma, \blalpha) }^\CC(\delta_\Sc) }
  {\grm{A}^\Sg(\gamma, \delta, \omega)
     : \grm{A}^\CC\left(\grm{\Gamma}^\Sg(\gamma, \delta) \right) }
\end{gathered}
\end{equation*}
These operations allow us two define the sigma construction as straightforward
as we have seen in the previous constructions:
\begin{align*}
\grm{\cdot}^\Sg(\gamma, \delta)
  &:\equiv \star \\
\grm{(\Gamma,\,B :: \Sc)}^\Sg(\gamma)\{\delta_\Sc, \omega\}( \delta)
  &:\equiv \left( \grm{\Gamma}^\Sg(\gamma,\delta) ,
    \grm{B}^\Sg(\gamma, \delta, \omega)\right) \\
\grm{(\Gamma,\,A :: \Pc)}^\Sg((\gamma, \alpha), (\delta, \omega))
  &:\equiv \left( \grm{\Gamma}^\Sg(\gamma, \delta) ,
    \grm{A}^\Sg(\gamma, \delta, \omega)\right)
\end{align*}
\end{defn}

Again, the necessity to define $\grm{\El(a)}^\Sg$ forces us to extend the definition
on terms as well.
The treatment for sort and point terms differs because type erasure and
wellformedness predicate are defined as maps between algebras of
inductive family syntax instead of syntax itself:
\begin{equation*}
\begin{gathered}
\inferrule{\grm{\Gamma \vdash t : B :: \Sc} \\
  \gamma_\Sc : \tqm{\grm{\Gamma}^\EE_\Sc }^\CC \\
  \gamma : \tqm{\grm{\Gamma}^\EE }^\CC(\gamma_\Sc) \\
  \delta_\Sc : \tqm{\grm{\Gamma}^\WW_\Sc(\blgamma)}^\CC \\
  \delta : \tqm{\grm{\Gamma}^\WW(\blgamma)}^\CC }
  {\grm{t}^\CC\left(\grm{\Gamma}^\Sg(\gamma, \delta)\right)
    = \grm{B}^\Sg\left(\gamma, \delta,
      \tqm{\grm{t}^\WW(\blgamma)}^\CC(\delta_\Sc) \right)}
\\[.7em]
\inferrule{\grm{\Gamma \vdash t : A :: \Pc} \\
  \gamma_\Sc : \tqm{\grm{\Gamma}^\EE_\Sc }^\CC \\
  \gamma : \tqm{\grm{\Gamma}^\EE }^\CC(\gamma_\Sc) \\
  \delta_\Sc : \tqm{\grm{\Gamma}^\WW_\Sc(\blgamma)}^\CC \\
  \delta : \tqm{\grm{\Gamma}^\WW(\blgamma)}^\CC }
  {\grm{t}^\CC\left(\grm{\Gamma}^\Sg(\gamma, \delta)\right)
    = \grm{A}^\Sg\left(\gamma, \delta, \grm{t}^\WW(\delta)\right)}
\end{gathered}
\end{equation*}

We will now go all the type formers in order, starting with the universe.
It justifies the name of the construction, producing a sigma type of the erasure and
its wellformedness.
For the element operator, we need the above equation for terms to be able to
give to populate these sigma types accordingly:
\begin{align*}
\grm{\UU}^\Sg(\gamma, \delta, \omega)
  &:\equiv (x : \alpha) \times \omega(x) \text{ and} \\
\grm{\El(a)}^\Sg(\gamma, \delta, \omega)
  &:\equiv \left(\grm{a}^\Sg(\gamma, \delta)\inv\right)^*(\alpha, \omega)
\end{align*}

Let us next look at the recursive $\Pi$-types and their application:
Let $(\alpha, \omega)$ be the result of
$\left(\grm{a}^\Sg(\gamma, \delta)\right)^*(\xi)$, then
we can set
\begin{align*}
\grm{\Pi(a, B :: \Sc)}^\Sg(\gamma, \delta, \phi, \xi)
  &:\equiv p^*\left(\grm{B}^\Sg((\gamma, \alpha), (\delta, \omega), \phi(\alpha))\right)
    \text{ and} \\
\grm{\Pi(a, A :: \Pc)}^\Sg(\gamma, \delta, \phi, \xi)
  &:\equiv p^*\left(\grm{B}^\Sg((\gamma, \alpha), (\delta, \omega), \phi(\alpha, \omega))\right)
    \text{,}
\end{align*}
where $p$ is an equation providing proof for
\begin{equation*}
\grm{B}^\CC\left(\grm{\Gamma}^\Sg(\gamma, \delta), \left(\grm{a}^\Sg(\gamma, \delta)\inv\right)^*(\alpha, \omega) \right)
  = \grm{B}^\CC\left(\grm{\Gamma}^\Sg(\gamma, \delta), \xi \right) \text{.}
\end{equation*}
For the application we provide 
$\grm{\IIapp(f)}^\Sg((\gamma, \alpha), (\delta, \omega))$ for
\grm{\Gamma \vdash f : \Pi(a, B)} by the following
identity proofs:
\begin{align*}
  & \grm{\IIapp(f)}^\CC\left(\grm{(\Gamma,\,\El(a))}^\Sg((\gamma, \alpha), (\delta, \omega))\right) \\
  &\equiv \grm{f}^\CC\left(\grm{\Gamma}^\Sg(\gamma), \left(\grm{a}^\Sg(\gamma, \delta)\inv\right)^*(\alpha, \omega)\right) \\
  &= \grm{\Pi(a, B)}^\Sg\left(\gamma, \delta, \left(\grm{a}^\Sg(\gamma, \delta)\inv\right)^*(\alpha, \omega)\right) \text{ by $\grm{f}^\Sg$}\\
  &\equiv 
    \begin{cases}
    \grm{B}^\Sg\left((\gamma, \alpha), (\delta, \omega),
    \tqm{\grm{f}^\EE }^\CC(\gamma_\Sc)
      \left(\tqm{\grm{f}^\WW(\blgamma)}^\CC(\delta_\Sc)\right)\right)
     & \text{for \grm{B :: \Sc} and} \\
    \grm{B}^\Sg\left((\gamma, \alpha), (\delta, \omega),
     \grm{f}^\EE(\gamma)\left(\grm{f}^\WW(\delta)\right)\right)
     & \text{for \grm{B :: \Pc}.}
    \end{cases}
\end{align*}

The case for non-recursive $\Pi$-types is a rather straightforward descent,
compared to the recursive ones:
\begin{align*}
\grm{\Pi(a, B)}^\Sg(\gamma, \delta, \phi, \tau)
  &:\equiv \grm{B}^\Sg(\gamma, \delta, \phi(\tau)) \text{ and}\\
\grm{f(\bltau)}^\Sg(\gamma, \delta)
  &:\equiv \happly\left(\grm{f}^\Sg(\gamma, \delta), \tau\right)
\end{align*}

This concludes all type formers, though we still need to have a definition on
types which are the result of pullback along a substitution, and thus need
to extend the operation to substitution with the following rule, which introduces
equalities similar to the ones that we already saw for terms:
\begin{equation*}
\inferrule{\grm{\IISub{\sigma}{\Gamma}{\Delta}} \\
  \gamma_\Sc : \tqm{\grm{\Gamma}^\EE_\Sc }^\CC \\
  \gamma : \tqm{\grm{\Gamma}^\EE }^\CC(\gamma_\Sc) \\
  \delta_\Sc : \tqm{\grm{\Gamma}^\WW_\Sc(\blgamma)}^\CC \\
  \delta : \tqm{\grm{\Gamma}^\WW(\blgamma)}^\CC  }
  {\grm{\sigma}^\Sg(\gamma, \delta) :
    \grm{\sigma}^\CC\left(\grm{\Gamma}^\Sg(\gamma, \delta)\right)
    = \grm{\Delta}^\Sg\left(\grm{\sigma}^\EE(\gamma), \grm{\sigma}^\WW(\delta) \right) }
\end{equation*}
With this rule we can set the provide the correct operations on pulled back types
and terms:
\begin{align*}
\grm{B[\sigma]}^\Sg(\gamma, \delta, \omega)
  &:\equiv \left(\grm{\sigma}^\Sg(\gamma, \delta)\inv\right)^*
    \left(\grm{B}^\Sg(\grm{\sigma}^\EE(\gamma), \grm{\sigma}^\WW(\delta), \omega ) \right)
\end{align*}
for \grm{\Gamma \vdash B :: \Sc}, and for a term \grm{\Gamma \vdash t[\sigma] : B[\sigma]}, we
use $\grm{\sigma}^\Sg$ in the proof of the equality $\grm{t[\sigma]}^\Sg$:
\begin{align*}
  & \grm{t[\sigma]}^\CC(\grm{\Gamma}^\Sg(\gamma, \delta)) \\
  &\equiv \grm{t}^\CC(\grm{\sigma}^\CC(\grm{\Gamma}^\Sg(\gamma, \delta))) \\
  &= \left(\grm{\sigma}^\Sg(\gamma, \delta)\inv\right)^*
    \left(\grm{t}^\CC(\grm{\Delta}^\Sg(\grm{\sigma}^\EE(\gamma), \grm{\sigma}^\WW(\delta)))\right) \\
  &= \left(\grm{\sigma}^\Sg(\gamma, \delta)\inv\right)^*
     \left(\grm{B}^\Sg(\grm{\sigma}^\EE(\gamma), \grm{\sigma}^\WW(\gamma, \delta), 
       \tqm{\grm{t}^\WW(\gamma)}^\CC(\tqm{\grm{\sigma}^\WW_\Sc(\gamma)}^\CC(\delta_\Sc)  )
       )\right) \\
  &\equiv \grm{B[\sigma]}^\Sg(\gamma, \delta, \tqm{\grm{t}^\WW[\grm{\sigma}^\WW_\Sc(\gamma) ]  }^\CC(\delta_\Sc) ) %TODO re-check all this
\end{align*}
and similarly for \grm{\Gamma \vdash A :: \Pc}.


The substitutional calculus for this rule is easily defined since
$\grm{\id}^\Sg(\gamma, \delta)$ and $\grm{\epsilon}^\Sg(\gamma, \delta)$ hold
definitionally,
the composition rule $\grm{\sigma \circ \delta}^\Sg(\gamma, \delta')$
for \grm{\IISub{\sigma}{\Delta}{\Sigma}} and
\grm{\IISub{\delta}{\Gamma}{\Delta}} is justified
by
\begin{align*}
\grm{\sigma}^\CC\left(\grm{\delta}^\CC\left(\grm{\Gamma}^\Sg(\gamma, \delta')\right)\right)
  &= \grm{\sigma}^\CC\left(\grm{\Delta}^\Sg(\grm{\delta}^\EE(\gamma), \grm{\delta}^\WW(\gamma, \delta') ) \right)
  & \text{by $\grm{\delta}^\Sg(\gamma, \delta') $} \\
  & = \grm{\Sigma}^\Sg(\grm{\sigma}^\EE(\grm{\delta}^\EE(\gamma)),
     \grm{\sigma}^\WW(\grm{\delta}^\EE(\gamma),\grm{\delta}^\WW(\gamma, \delta') ) )
  & \text{by $\grm{\sigma}^\Sg(\ldots) $.}
\end{align*}
To prove the coherence of a substitution extended by a term
$\grm{(\sigma,\,t)}$ with \grm{\IISub{\sigma}{\Gamma}{\Delta}} and
\grm{\Gamma \vdash B[\sigma] :: \Sc}, we use
$\grm{\sigma}^\Sg(\gamma, \delta)$ and $\grm{t}^\Sg(\gamma, \delta)$
simultaneously to obtain the required equation (the variant for point terms follows
in similar fashion):
\begin{align*}
  &\left( \grm{\sigma}^\CC(\grm{\Delta}^\Sg(\gamma, \delta)) ,\,
    \grm{t}^\CC(\grm{\Delta}^\Sg(\gamma, \delta))   \right) \\
  = &\left( \grm{\Delta}^\Sg(\grm{\sigma}^\EE(\gamma), \grm{\sigma}^\WW(\gamma, \delta) ) ,
    \grm{B}^\Sg(\grm{\sigma}^\EE(\gamma), \grm{\sigma}^\WW(\gamma, \delta),
      \tqm{\grm{t}^\WW(\blgamma)}^\CC(\delta_\Sc) ) \right) \text{.}
\end{align*}
The remaining equations follow in similarly obvious way.


\end{defn}

\begin{defn}[Initial Object]
Using the generalized sigma construction we are now able to define the initial
object by plugging in the respective initial objects of the inductive families:
\begin{equation*}
\con{\Gamma} :\equiv \grm{\Gamma}^\Sg(\IFcon{\grm{\Gamma}^\EE}, \IFcon{\grm{\Gamma}^\WW(\blm{\IFcon{\grm{\Gamma}^\EE}})})
  : \grm{\Gamma}^\CC
\end{equation*}
\end{defn}

\section{The Eliminator Relation}













\bibliographystyle{unsrtnat}
\bibliography{references}

\end{document}

