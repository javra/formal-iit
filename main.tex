\documentclass[12pt,headings=optiontohead,openany,oneside,a4paper]{book}
%TODO: page layout
\usepackage[backref=page,
unicode,
pdfauthor={Jakob von Raumer},
pdftitle={Reducing Inductive-Inductive Types to Indexed Inductive Types},
pdfsubject={Mathematics, Computer Science},
pdfkeywords={interactive theorem proving}]{hyperref}

%code listings
\usepackage{mathpartir}
\usepackage{newunicodechar}
%\renewcommand{\MintedPygmentize}{./pygments-main/pygmentize}
\usepackage{fontspec,xunicode}
\usepackage[osf]{mathpazo}
%\usepackage{shellesc}
\usepackage{minted}
\setmainfont[Ligatures=TeX]{TeX Gyre Pagella}
\setmonofont{Iosevka}
%\newfontfamily{\freeserif}{DejaVu Sans Mono}
%\newunicodechar{⦃}{\ensuremath{\texttt{\{}\mkern-7mu\texttt{|}}}
	%{\makebox[.5em]{\freeserif⦃}}
%\newunicodechar{⦄}{\ensuremath{\texttt{|}\mkern-7mu\texttt{\}}}}
	%{\makebox[.5em]{\freeserif⦄}}
%\newunicodechar{→}{\texttt{\freeserif{→}}}
%\newunicodechar{⟶}{\texttt{\freeserif{⟶}}}
%\newunicodechar{⁻}{\texttt{\freeserif{⁻}}}
%\newunicodechar{▹}{\freeserif{▹}}
%\newunicodechar{ℕ}{\texttt{\freeserif{ℕ}}}
%\newunicodechar{⟨}{\texttt{\freeserif{⟨}}}
%\newunicodechar{⟩}{\texttt{\freeserif{⟩}}}
%\newunicodechar{⬝}{\ensuremath{\ct}}
%\newunicodechar{∼}{\freeserif{∼}}
%\newunicodechar{≃}{\freeserif{≃}}
%\newunicodechar{≅}{\freeserif{≅}}
%\newunicodechar{∘}{\freeserif{∘}}
%\newunicodechar{ʰ}{\freeserif{ʰ}}
%\newunicodechar{ᵍ}{\freeserif{ᵍ}}
%\newunicodechar{⇒}{\freeserif{⇒}}
%\newunicodechar{⋆}{\freeserif{⋆}}
%\newunicodechar{∘}{\freeserif{∘}}
%\newunicodechar{←}{\freeserif{←}}
%\newunicodechar{∙}{\freeserif{∙}}
%\newunicodechar{▶}{\freeserif{▶}}
%\newunicodechar{∀}{\freeserif{∀}}

%workaround to remove red boxes
\AtBeginEnvironment{minted}{\renewcommand{\fcolorbox}[4][]{#4}}
\usemintedstyle{default}
\newminted{lean}{
	linenos,
	numbersep=5pt,
	frame=single,
	fontsize=\footnotesize,
	framesep=1mm,
	samepage,
	autogobble}
\newminted[leancodebr]{lean}{
	linenos,
	numbersep=5pt,
	frame=single,
	fontsize=\footnotesize,
	framesep=1mm,
	autogobble}
\newcommand{\leani}[1]{\mintinline{lean}{#1}}
\newminted{agda}{
	linenos,
	numbersep=5pt,
	frame=single,
	fontsize=\footnotesize,
	framesep=1mm,
	samepage,
	autogobble}
\newcommand{\agdai}[1]{\mintinline{agda}{#1}}
\linespread{1.07}
\usepackage[dvipsnames]{xcolor}
\usepackage{graphicx}
\usepackage{array}
\usepackage[all,2cell,cmtip]{xy}
\usepackage{tikz}
\usetikzlibrary{decorations.pathmorphing,arrows}
\usepackage[numbered]{bookmark}
\usepackage{fancyhdr}
\usepackage{amssymb,amsmath,amsthm,mathrsfs,wasysym}
\usepackage{enumitem,mathtools,xspace}
\usepackage[nottoc]{tocbibind}
\usepackage{cleveref}
\usepackage{aliascnt}
\usepackage{natbib}
\usepackage{mathtools}
\usepackage{footnote}
\usepackage{booktabs}
\usepackage{xifthen}
%\usepackage[top=1in, bottom=1.25in, left=1.25in, right=1.25in]{geometry} %TODO decide

\fancyhead[LO]{\leftmark}
\fancyhead[RE]{\rightmark}
\fancyhead[LE,RO]{\thepage}
\cfoot{}
\pagestyle{fancy}

\def\defthm#1#2#3{
	\newaliascnt{#1}{thm}
	\newtheorem{#1}[#1]{#2}
	\aliascntresetthe{#1}
	\crefname{#1}{#2}{#3}
}

\newtheorem{thm}{Theorem}[section]
\crefname{thm}{Theorem}{Theorems}
\defthm{lemma}{Lemma}{Lemmas}
\defthm{axiom}{Axiom}{Axioms}
\defthm{corollary}{Corollary}{Corollaries}
\theoremstyle{definition}
\defthm{defn}{Definition}{Definitions}
\defthm{example}{Example}{Examples}
\defthm{remark}{Remark}{Remarks}

\newcommand{\upperf}{\partial^-_1}
\newcommand{\lowerf}{\partial^+_1}
\newcommand{\leftf}{\partial^-_2}
\newcommand{\rightf}{\partial^+_2}
\newcommand{\inv}{^{-1}}
\newcommand{\DCat}{\mathbf{DCat}}
\newcommand{\DGpd}{\mathbf{DGpd}}
\newcommand{\XMod}{\mathbf{XMod}}
\newcommand{\twotype}{\mathbf{2}}
\newcommand{\unit}{\mathbf{1}}
\newcommand{\emptytype}{\mathbf{0}}
\newcommand{\UU}{\mathcal{U}}
\newcommand{\isProp}{\mathsf{isProp}}
\newcommand{\isSet}{\mathsf{isSet}}
\newcommand{\PropU}{\mathsf{Prop}}
\newcommand{\SetU}{\mathsf{Set}}
\newcommand{\seg}{\mathsf{seg}}
\newcommand{\isContr}{\mathsf{isContr}}
\newcommand{\isIso}{\mathsf{isIso}}
\newcommand{\idtoiso}{\mathsf{idtoiso}}
\newcommand{\idtoeqv}{\mathsf{idtoeqv}}
\newcommand{\ua}{\mathsf{ua}}
\newcommand{\alliso}{\mathsf{alliso}}
\newcommand{\refl}{\mathsf{refl}}
\newcommand{\ap}{\mathsf{ap}}
\newcommand{\apd}{\mathsf{apd}}
\newcommand{\happly}{\mathsf{happly}}
\newcommand{\ind}{\mathsf{ind}}
\newcommand{\rec}{\mathsf{rec}}
\newcommand{\pr}{\mathsf{pr}}
\newcommand{\inl}{\mathsf{inl}}
\newcommand{\inr}{\mathsf{inr}}
\newcommand{\ishae}{\mathsf{ishae}}
\newcommand{\apiop}{\ap_{\iota'}}
\DeclareMathOperator{\comp}{comp}
\newcommand{\id}{\mathsf{id}}
\DeclareMathOperator{\invv}{inv_1}
\DeclareMathOperator{\invh}{inv_2}
\DeclareMathOperator{\obj}{obj}
\DeclareMathOperator{\EI}{EI}
\DeclareMathOperator{\IE}{IE}
\DeclareMathOperator{\op}{op}
\DeclareMathOperator{\inj}{inj}
\newcommand{\assoc}{\mathsf{assoc}}
\newcommand{\assocv}{\mathsf{assoc}_1}
\newcommand{\assoch}{\mathsf{assoc}_2}
\newcommand{\idLeft}{\mathsf{idLeft}}
\newcommand{\idLeftv}{\mathsf{idLeft}_1}
\newcommand{\idLefth}{\mathsf{idLeft}_2}
\newcommand{\idRight}{\mathsf{idRight}}
\newcommand{\idRightv}{\mathsf{idRight}_1}
\newcommand{\idRighth}{\mathsf{idRight}_2}
\newcommand{\leftInv}{\mathsf{leftInv}}
\newcommand{\leftInvv}{\mathsf{leftInv}_1}
\newcommand{\leftInvh}{\mathsf{leftInv}_2}
\newcommand{\rightInv}{\mathsf{rightInv}}
\newcommand{\rightInvv}{\mathsf{rightInv}_1}
\newcommand{\rightInvh}{\mathsf{rightInv}_2}
\newcommand{\respectId}{\mathsf{respectId}}
\newcommand{\respectIdv}{\mathsf{respectId}_1}
\newcommand{\respectIdh}{\mathsf{respectId}_2}
\newcommand{\respectComp}{\mathsf{respectComp}}
\newcommand{\respectCompv}{\mathsf{respectComp}_1}
\newcommand{\respectComph}{\mathsf{respectComp}_2}
\newcommand{\isequiv}{\mathsf{isequiv}}
\newcommand{\isodd}{\mathsf{isodd}}
\newcommand{\thin}{\mathsf{thin}}
\newcommand{\Sbase}{\mathsf{base}}
\newcommand{\Sloop}{\mathsf{loop}}
\newcommand{\Smerid}{\mathsf{merid}}
\newcommand{\vecty}{\mathsf{vec}}
\newcommand{\nil}{\mathsf{nil}}
\newcommand{\cons}{\mathsf{cons}}
\newcommand{\ct}{
	\mathchoice{\mathbin{\raisebox{0.5ex}{$\displaystyle\centerdot$}}}
		{\mathbin{\raisebox{0.5ex}{$\centerdot$}}}
		{\mathbin{\raisebox{0.25ex}{$\scriptstyle\,\centerdot\,$}}}
		{\mathbin{\raisebox{0.1ex}{$\scriptscriptstyle\,\centerdot\,$}}}
}
\newcommand{\trunc}[2]{\left\Vert #2\right\Vert_{#1}}
\newcommand{\squash}[1]{\trunc{}{#1}}
\newcommand{\isntype}[1]{\mathsf{is}\mbox{-}{#1}\mbox{-}\mathsf{type}}
\newcommand{\N}{\mathbb{N}}
\newcommand{\Sph}{\mathbb{S}}
\newcommand{\Set}{\mathbf{Set}}
\newcommand{\Cat}{\mathbf{Cat}}
\newcommand{\Fam}{\mathbf{Fam}}
\newcommand{\Alg}{\mathbf{Alg}}
\newcommand{\CT}{\mathbf{CT}}
\newcommand{\CTw}{\mathbf{CTw}}
\newcommand{\old}[1]{\overline{#1}}
\newcommand{\oldold}[1]{\overline{\overline{#1}}}
\newcommand{\gr}[1]{{\color{ForestGreen}#1}}
\newcommand{\grm}[1]{\ensuremath{\gr{#1}}}
\newcommand{\blm}[1]{\ensuremath{{\color{Black}#1}}}
\newcommand{\tqm}[1]{\ensuremath{{\color{Blue}#1}}}
%\newcommand{\grinfer}[2]{\grm{\inferrule{{#1}}{{#2}}}}
\newcommand{\Sc}{\mathsf{S}}
\newcommand{\Pc}{\mathsf{P}}
\newcommand{\CC}{\mathsf{A}} %TODO maybe jus replace
\renewcommand{\AA}{\mathsf{A}}
\newcommand{\EE}{\mathsf{E}}
\newcommand{\PP}{\mathsf{P}}
\newcommand{\mm}{\mathsf{m}}
\newcommand{\MM}{\mathsf{M}}
\newcommand{\DD}{\mathsf{D}}
\newcommand{\WW}{\mathsf{W}}
\renewcommand{\SS}{\mathsf{S}}
\newcommand{\con}[1]{\mathsf{con}(\grm{#1})}
\newcommand{\contwo}[2]{\mathsf{con}(#1,\grm{#2})}
\newcommand{\conthree}[3]{\mathsf{con}(#1, #2, \grm{#3})}
\newcommand{\IFconS}[1]{\mathsf{con_\Sc}(\tqm{#1})}
\newcommand{\IFcon}[1]{\mathsf{con}(\tqm{#1})}
\newcommand{\IFelimS}[2]{\ifthenelse{\isempty{#2}}{\mathsf{elim_\Sc}(\tqm{#1})}{\mathsf{elim_\Sc}(\tqm{#1}, #2)}}
\newcommand{\IFelim}[2]{\ifthenelse{\isempty{#2}}{\mathsf{elim}(\tqm{#1})}{\mathsf{elim}(\tqm{#1}, #2)}}
\newcommand{\elim}{\mathsf{elim}}
\newcommand{\flatten}[1]{\blm{\mathsf{flatten}(\grm{#1})}}
\newcommand{\annotate}[1]{\blm{\mathsf{an}(\grm{#1})}}
\newcommand{\anntwo}[2]{\blm{\mathsf{an}(#1, \grm{#2})}}
\newcommand{\app}{\mathrel{@}}
\newcommand{\SCon}{\vdash_\Sc}
\newcommand{\var}{\mathsf{var}}
\newcommand{\vz}{\mathsf{vz}}
\newcommand{\vs}{\mathsf{vs}}
\newcommand{\El}{\mathsf{El}} %TODO change to operator?
\newcommand{\LSub}[3]{\mathsf{LSub}_{\tqm{#1}}(\tqm{#2}, \tqm{#3})}
\newcommand{\wk}{\mathsf{wk}}
\newcommand{\IFSub}[3]{#2 \overset{#1}{\longrightarrow} #3}
\newcommand{\IISub}[3]{#2 \overset{#1}{\longrightarrow} #3}
\newcommand{\IIapp}{\mathsf{app}}
\newcommand{\ExtPi}[2]{\hat{\Pi}(\blm{#1},\, #2)}
\newcommand{\ExtPiS}[2]{\hat{\Pi}_\Sc(\blm{#1},\, #2)}
\newcommand{\ExtPiP}[2]{\hat{\Pi}_\Pc(\blm{#1},\, #2)}
\newcommand{\bltau}{\blm{\tau}}
\newcommand{\blalpha}{\blm{\alpha}}
\newcommand{\blgamma}{\blm{\gamma}}

\renewcommand{\backrefalt}[4]{
	\ifcase #1
		(No citations.)
	\or
		(Cited on page\ #2.)
	\else
		(Cited on pages\ #2.)
	\fi
}

\begin{document}

\title{Reducing Inductive-Inductive Types to Indexed Inductive Types}
\author{Jakob von Raumer}

\frontmatter
\maketitle

\tableofcontents

\mainmatter

\chapter{Introduction}

\section{Examples of Inductive-Inductive Types}

In the following, we will take a look at a few examples which we are going to
revisit at various steps throughout this presentation:

\begin{example}[Type Theory Syntax]\label{ex:ttintt}
\cite{ttintt} showed how to internalise the syntax of type theory inside type
theory itself, using a quotient inductive-inductive type.
Leaving out terms and substitutions, we arrive at a fragment of type theoretical
syntax specifying a type of contexts and a type of types over a certain contexts,
together with type formers for a unit type and a $\Pi$-type:
We want $\mathop{Con} : \UU$ to be inductively defined by
\begin{align*}
\mathop{nil}	&: \mathop{Con} \text{and } \\
\mathop{ext}	&: (\Gamma : \mathop{Con})(A : \mathop{Ty}(\Gamma)) \to \mathop{Con} \text{,}
\end{align*}
while simultaneously defining a family $\mathop{Ty} : \mathop{Con} \to \UU$ with constructors
\begin{align*}
\mathop{unit}	&: (\Gamma : \mathop{Con}) \to \mathop{Ty}(\Gamma) \text{ and} \\
\mathop{pi}	&: (\Gamma : \mathop{Con})(A : \mathop{Ty}(\Gamma))(B : \mathop{Ty}(\mathop{ext}(\Gamma, A))) \to \mathop{Ty}(\Gamma) \text{.}
\end{align*}
\end{example}

\begin{example}[Free Dense Completion]
\cite{nordvallinductive} proposed the example of a ``free dense completion'' of
an order (or, more general, any relation) which for any type $A : \UU$ and
any type valued relation $\_<\_ : A \to A \to \UU$ on $A$ freely adds midpoints
to all pairs of related elements by $\_<\_$.
It does so by introducing a new type $A' : \UU$ inductively generated by the
original points and their midpoints:
\begin{align*}
\iota_A		&: A \to A' \text{ and} \\
\mathop{mid}	&: \{x, y : A'\}(p : x <' y) \to A' \text{.}
\end{align*}
But since our relation was only defined on $A$, we have to extend it to $A'$ by
postulating
\begin{align*}
\iota_<		&: \{a, b : A\}(p : a < b) \to \iota_A(a) <' \iota_A(b) \text{,} \\
\mathop{mid}_l	&: \{x, y : A'\}(p : x <' y) \to x < \mathop{mid}(p) \text{, and} \\
\mathop{mid}_r	&: \{x, y : A'\}(p : x <' y) \to \mathop{mid}(p) < y \text{.}
\end{align*}
\end{example}


\chapter{Basic Type Theory}

\chapter{Higher Inductive Types}

\chapter{Path Spaces of HITs}

\chapter{Specification of Inductive-Inductive Types}

Inductive-Inductive Types are specified by giving a context in  a small type
theoretic syntax which we will refer to as \emph{source type theory}.

%TODO add waay more explanation
This idea originates from Ambrus Kaposi's work on the syntax of \emph{higher}
inductive-inductive types (~\cite{ambrussyntax}) which we adapt and rid of equality
constructors to only allow for inductive-inductive types.
In contrast to their presentation we will leave the context of the ambient type
theory implicit and, instead of highlighting syntax of the ambient type theory,
mark elements of the source type theory in \gr{green}.

\section{Signatures for Inductive-Inductive Types}

We assume that the source type theory makes use of the standard syntax of type
theory, using contexts, types, terms, and variables.
Types and terms are uniquely ascribed to one of two \emph{kinds}:
Either their kind is \grm{\Sc} which indicates that the type contains sort
constructors, or their kind is \grm{\Pc} because elements of it describe
point constructors.
We will write \grm{\Gamma \vdash A :: k} to say that \grm{A} is a type of kind
\grm{k} and \grm{\Gamma \vdash t : A :: k} to state that \grm{t} is a term of the
type \grm{A} which in turn has kind \grm{k}.
Often, we will omit the annotation of the sort, meaning that a judgment is to
hold true for both \grm{\Sc} and \grm{\Pc}, or that the kind of a term's type
has already been specified.

It's important that contexts can be extended by sort and point types in any order
(see Example~\ref{ex:tmnil}) to be able to capture sorts which depend on previously
defined point constructors.
So we have the usual two rules for context formation:
\begin{equation*}
\inferrule{}{\grm{\vdash \cdot}}
\qquad
\inferrule{\grm{\Gamma \vdash A :: k}}
  {\grm{\vdash \Gamma, A}}
\end{equation*}

We need one atomic constructor for sort types:
For plain types we and the codomain we need a type \grm{\UU} which serves as a
token for the \emph{universe}.
We will call terms of this universe ``small types''.
Positiviy requires that these are the only (internal) types which are allowed in
the domain of functions.
An operation \grm{\El} reifies these small types to big types, making our version
of universe what is commonly referred to as ``Tarski-style universe'' (cf. ~\cite{luotarski}):
\begin{equation*}
\inferrule{\grm{\vdash \Gamma}}{\grm{\Gamma \vdash \UU :: \Sc}}
\qquad
\inferrule{\grm{\Gamma \vdash a : \UU}}{\grm{\Gamma \vdash \El(a) :: \Pc}}
\end{equation*}

For sorts which are type families over other sorts that we seek to define, and for
constructors which recursively refer to other constructors, we need $\Pi$-types
which have a small type as their codomain.
Note that whether we want to build a sort or a point type only depends on the
kind of the \emph{codomain} of such a $\Pi$-type.
To eliminate from $\Pi$-types we want a rule for its \emph{application} which
turns a term of a $\Pi$-type into a term of its codomain:
\begin{equation*}
\inferrule{\grm{\Gamma \vdash a : \UU} \\ 
  \grm{\Gamma, \El(a) \vdash B :: k}}
  {\grm{\Gamma \vdash \Pi(a, B) :: k}}
\qquad
\inferrule{\grm{\Gamma \vdash t : \Pi(a, B)}}
  {\grm{\Gamma, \El(a) \vdash \IIapp(t) : B}}
\end{equation*}

Since we are working with explicit substitutions, we need to postulate a calculus
for substitutions \grm{\IISub{\sigma}{\Gamma}{\Delta}} between any two contexts
\grm{\Gamma} and \grm{\Delta}.
The substitutions should form a category as postulated by the following rules:
\begin{equation*}
\inferrule{\grm{\vdash \Gamma}}
  {\grm{\IISub{\id}{\Gamma}{\Gamma}}}
\qquad
\inferrule{\grm{\IISub{\sigma}{\Delta}{\Sigma}} \\
  \grm{\IISub{\delta}{\Gamma}{\Delta}}}
  {\grm{\IISub{\sigma \circ \delta}{\Gamma}{\Sigma}}}
\end{equation*}
\begin{align*}
\grm{\id \circ \sigma} &= \grm{\sigma} \\
\grm{\sigma \circ \id} &= \grm{\sigma} \\
\grm{(\sigma \circ \delta) \circ \gamma} &= \grm{\sigma \circ (\delta \circ \gamma)}
\end{align*}










%\section{(OLD) --The Source Type Theory}

Since for some examples of inductive-inductive types, especially ones which are
\emph{infinitely branching}, we need functions with external domain as the domains of
other functions, we will also add another function type which is itself small, and
has external domain and small codomain:
\begin{equation*}
\begin{gathered}
\inferrule{\grm{\vdash \Gamma} \\ A : \UU_i \\ (x : A) \to (\grm{\Gamma \vdash b : \UU})}
	{\grm{\Gamma \vdash ((\blm{x : A}) \to b) : \UU}}
\qquad
\inferrule{\grm{\Gamma \vdash t : \underline{(\blm{x : A}) \to b}} \\ u : A}
	{\grm{\Gamma \vdash (t \app \blm{u}) : \underline{b[\blm{x} \mapsto \blm{u}]}}}
\end{gathered}
\end{equation*}

\section{(OLD) --Motives and Methods}

TODO: Add definition of $\grm{-}^\MM$.

TODO: Add examples.

\section{(OLD) --Recursion and Computation}

TODO: Add definition of $\grm{-}^\EE$.

TODO: Add examples.

\section{(OLD) --Existence of HIITs}

\begin{defn}[Dependent Eliminator]
An algebra $c : \grm{\Gamma}^\CC$ is said to admit \emph{dependent elimination}
if for each motive $m : \grm{\Gamma}^\MM(c)$ there is a dependent eliminator
$\elim_{\grm{\Gamma}}(m) : \grm{\Gamma}^\EE(c, m)$.
\end{defn}

\begin{thm}[Admissibility of Inductive-Inductive Types]
Our type theory admits inductive-inductive types if for each wellformed context
$\grm{\Gamma}$ we can find a constructor $\con{\grm{\Gamma}}$ which admits dependent
elimination $\elim_{\grm{\Gamma}}$.
\end{thm}

\section{(OLD) --Morphisms of Algebras}

In the following, it will be useful to regard the algebras of a given context
\grm{\Gamma} as a category.
To this end, we need to make clear what a morphism between \grm{\Gamma}-algebras
is.
Intutitively a morphism is given by maps between the interpretations of the sorts,
together with evidence that those maps preserve the interpretation of point
constructors.

\begin{defn}[Morphisms of Algebras]
We want to define the following by mutual recursion on contexts, types and terms:
\begin{equation*}
\begin{gathered}
\inferrule{\grm{\vdash \Gamma} \\ \gamma_0, \gamma_1 : \grm{\Gamma}^\AA}
	{\grm{\Gamma}^\mm(\gamma_0,\gamma_1) : \UU} \\[.7em]
\inferrule{\grm{\Gamma \vdash A} %\\ \gamma_0, \gamma_1 : \grm{\Gamma}^\AA TODO decide if show
		\\ g : \grm{\Gamma}^\mm(\gamma_0, \gamma_1) \\
		\\ \alpha_0 : \grm{A}^\AA(\gamma_0) \\ \alpha_1 : \grm{A}^\AA(\gamma_1)}
	{\grm{A}^\mm(g, \alpha_0, \alpha_1) : \UU} \\[.7em]
\inferrule{\grm{\Gamma \vdash t : A} \\ \gamma_0, \gamma_1 : \grm{\Gamma}^\AA 
		\\ g : \grm{\Gamma}^\mm(\gamma_0, \gamma_1)}
	{\grm{t}^\mm(g) : \grm{A}^\mm(g, \grm{t}^\AA(\gamma_0), \grm{t}^\AA(\gamma_1))}
\end{gathered}
\end{equation*}

Like we did for the definition of algebras, we want morphisms of contexts to be
just iterated $\Sigma$-types of the respective interpretation of types:
\begin{align*}
\grm{\cdot}^\mm(\gamma_0, \gamma_1) &:\equiv \unit \text{ and} \\
\grm{(\Gamma, x : A)}^\mm(\gamma_0, \gamma_1) &:\equiv
	(g : \grm{\Gamma}^\mm(\pr_1(\gamma_0), \pr_1(\gamma_1))) \times \grm{A}^\mm(g, \pr_2(\gamma_0), \pr_2(\gamma_1)) \text{.}
\end{align*}
The core of the definition on sort types is that the universe is interpreted as
a function space:
\begin{align*}
\grm{\UU}^\mm(g, \alpha_0, \alpha_1)  			&:\equiv \alpha_0 \to \alpha_1 \text{,} \\
\grm{(\underline{a})}^\mm(g, \alpha_0, \alpha_1)	&:\equiv (\grm{a}^\mm(g, \alpha_0) = \alpha_1) \text{,} \\
\grm{((x : a) \to B)}^\mm(g, \alpha_0, \alpha_1)	&:\equiv (x : \grm{a}^\CC(\gamma_0))
							\to \grm{B}^\mm((g, \refl), \alpha_0(x), \alpha_1(\grm{a}^\mm(g, x))) \text{,} \\
\grm{((\blm{x : A}) \to B)}^\mm(g, \alpha_0, \alpha_1)  &:\equiv (x : A)
							\to \grm{(B(\blm{x}))}^\mm(g, \alpha_0(x), \alpha_1(x)) \text{.}
\end{align*}
On terms, consider the foo
\begin{align*}
\grm{x}^\mm(g) 				&:\equiv g.\grm{x} \qquad \text{for variables \grm{x},} \\ %???
\grm{(t(u))}^\mm(g)			&:\equiv (\grm{u}^\mm(g))_* (\grm{t}^\mm(g)(\grm{u}^\AA(\gamma_0))) \text{,} \\ %TODO check this
\grm{((\blm{x : A}) \to b)}^\mm(g)	&:\equiv (x : A) \to \grm{b}^\mm(g) \text{,} \\
\grm{(t(\blm{u}))}^\mm(g)		&:\equiv \grm{t}^\mm(g)(u) \text{, and} \\
\grm{(t \app \blm{u})}^\mm(g)		&:\equiv (\grm{t}^\mm(g))(u) \text{.} %TODO use happly here sometimes!!! uargh
\end{align*}

\end{defn}

TODO: Add examples.



\chapter{Specification of Inductive Families}

\section{Signatures for Inductive Families}

Previous specifications of mutual inductive families have taken different approaches:
Some are based on the notion of a polynomial functor while others...

Applying the same principle as in the case of inductive-inductive types we want
to create a specification based on the contexts of type theory syntax.
We already saw that we can obtain such a specification by just restricting the
syntax for inductive-inductive types to not use the recursive $\Pi$-type for sorts,
but this approach doesn't capture the full %TODO
extent of inductive families being a much simpler concept than inductive-inductive
types.
Given the strategy of our recursion we want the specification to capture at least
the following features of inductive families:
\begin{itemize}
\item Sorts are either types of funcions over existing types.
\item Point constructor can also be indexed over existing (``external'') types.
\item Point constructors can refer to any sort being defined.
\end{itemize}

The first point above says that we want the type of \emph{sort types} \blm{\tqm{\Sc} : \UU}
to be inductively generated by a \emph{universe} token \tqm{\UU : \Sc} and a constructor
of external functions for sorts which are meant to be \emph{type families}:
\tqm{\Pi_\Sc(\blm{T}, B) : \Sc} for a type \blm{T : \UU} and a function
\tqm{B : \blm{T \to \tqm{\Sc}}}.
Note that in contrast to the sort types of inductive-inductive definitions these
do not depend on a context.

Instead, we say that a \emph{sort context} is just a list of sort types without
any interdependencies:
\begin{equation*}
\begin{gathered}
\inferrule{}{\tqm{\SCon \cdot_\Sc}}
\qquad
\inferrule{\tqm{\SCon \Gamma_\Sc} \\ \tqm{B : \Sc}}{\tqm{\SCon \Gamma_\Sc, B}}
\end{gathered}
\end{equation*}

In order to refer to sorts we introduce a simplified term calculus based on typed
de Bruijn indices for bound variables and an application operation for type families:
\begin{equation*}
\begin{gathered}
\inferrule{\tqm{\SCon \Gamma_\Sc} \\ \tqm{B : \Sc}}{\tqm{\Gamma_\Sc, B \SCon \var(\vz) : B}}
\qquad
\inferrule{\tqm{\Gamma_\Sc \SCon \var(v) : B}}{\tqm{\Gamma_\Sc, B' \SCon \var(\vs(v)) : B}}
\\[.7em]
\inferrule{\tqm{\Gamma_\Sc \SCon t : \Pi_\Sc(\blm{T}, B)} \\ \blm{\tau : T}}
  {\tqm{\Gamma_\Sc \SCon t(\blm{\tau}) : B(\blm{\tau})}}
\end{gathered}
\end{equation*}

Point constructors will be represented by \emph{point types} over a given sort
context.
This means that opposite to inductive-inductive types, they cannot depend on
other point types.
The type formers we need are the element type for the universe \tqm{\UU}, an
external, non-recursive function type like the one we have for sorts, and an
internal function type used for recursive point constructors:
\begin{equation*}
\begin{gathered}
\inferrule{\tqm{\Gamma_\Sc \SCon a : \UU}}{\tqm{\Gamma_\Sc \SCon \El(a)}}
\qquad
\inferrule{\blm{T : \UU} \\ \blm{(\tau : T) \to \tqm{\Gamma_\Sc \SCon B(\blm{\tau})}}}
  {\tqm{\Gamma_\Sc \SCon \Pi_\Pc(\blm{T}, B)}}
\\[.7em]
\inferrule{\tqm{\Gamma_\Sc \SCon a : \UU} \\ \tqm{\Gamma_\Sc \SCon A}}
  {\tqm{\Gamma_\Sc \SCon a \Rightarrow_\Pc A}}
\end{gathered}
\end{equation*}

As a last building block of the syntax, we can now form full contexts consisting
of sort and point constructors.
Such a context \tqm{\Gamma} can be formed over a given sort context \tqm{\Gamma_\Sc}
which we will denote as a subscript to the turnstile or omit when inferrable.
The empty context can be formed over the empty sort context, an extension of
a context by a sort constructor happens in parallel to an extension of its sort
context, and an extension by a point constructor leaves the sort context fixed:
\begin{equation*}
\begin{gathered}
\inferrule{}{\tqm{\vdash_{\cdot_\Sc} \cdot}}
\qquad
\inferrule{\tqm{\vdash_{\Gamma_\Sc} \Gamma} \\ \tqm{B : \Sc}}
  {\tqm{\vdash_{\Gamma_\Sc, B} \Gamma, B}}
\qquad
\inferrule{\tqm{\vdash_{\Gamma_\Sc} \Gamma} \\ \tqm{\Gamma_\Sc \SCon A}}
  {\tqm{\vdash_{\Gamma_\Sc} \Gamma, A}}
\end{gathered}
\end{equation*}

\begin{remark}
While for the signatures of inductive-inductive types, contexts, types, and terms
depend on each other we can here define sort types, sort contexts, terms, point
types, and contexts in the presented order without referring to later constructions.
This means that unlike mentioned in Remark~\ref{rmk:iit-syntax}, we  can %TODO cite remark
internalize this syntax just using inductive families.

An Agda formalization of the syntax looks as follows, with variables and terms
separated:
\begin{agdacode}
data TyS : Set₁ where
  U  : TyS
  Π̂S : (T : Set) → (T → TyS) → TyS

data SCon : Set₁ where
  ∙c   : SCon
  _▶c_ : SCon → TyS → SCon

data Var : SCon → TyS → Set₁ where
  vvz : ∀{Γc}{B} → Var (Γc ▶c B) B
  vvs : ∀{Γc}{B}{B'} → Var Γc B → Var (Γc ▶c B') B

data Tm : SCon → TyS → Set₁ where
  var  : ∀{Γc}{A} → Var Γc A → Tm Γc A
  _\$S_ : ∀{Γc}{T}{B} → Tm Γc (Π̂S T B) → (α : T) → Tm Γc (B α)

data TyP : SCon → Set₁ where
  El   : ∀{Γc} → Tm Γc U → TyP Γc
  Π̂P   : ∀{Γc}(T : Set) → (T → TyP Γc) → TyP Γc
  _⇒P_ : ∀{Γc} → Tm Γc U → TyP Γc → TyP Γc

data Con : SCon → Set₁ where
  ∙    : Con ∙c
  _▶S_ : ∀{Γc} → Con Γc → (A : TyS) → Con (Γc ▶c A)
  _▶P_ : ∀{Γc} → Con Γc → (B : TyP Γc) → Con Γc
\end{agdacode}
%TODO remove backslash
\end{remark}




\chapter{Reducing Inductive-Inductive Types to Inductive Families}
\chaptermark{Reducing Inductive-Inductive Types}

\section{(OLD) -- Fragments of Inductive-Inductive Types}

As we have seen in the previous sections, inductive-inductive types as specified
allow for a very broad variety of definitions.
We will now see that it is easy to carve out different subsets of specifications
to obtain more restrictive fragments of inductive types.
Starting from the largest of these subsets, we will first see that there is a
straightforward way to restrict inductive-inductive types to those whose constructors
are finitary in the sense that no point constructor depends on an infinite
amount of data: %TODO improve that last sentence

\begin{defn}[Finitary IITs]
Given a specification \grm{\Gamma} for an inductive-inductive types we say that
it is \textbf{finitary} if the infinitary $\Pi$-type is not used.
\end{defn}

One example of a specification which does not meet this requirement are the
infinitely branching trees. %TODO cite example

Instead of only preventing the use of external data in ``small functions''

We want to reduce inductive-inductive types to inductive families.
This means, we postulate that inductive families be admissible in our target
type theory and show that this implies the existence of all inductive-inductive
types.
Since we want to reuse the way of specifying inductive-inductive types to specify
instances which are as well inductive families, we want to rediscover the specifications
of inductive families as a subset of all inductive-inductive specifications:

\begin{defn}[Inductive families]
A context \grm{\Gamma} is said to specify a \textbf{inductive family} if it is
generated without using the inductive function type in the specification of sorts
and thus, no sorts depend on other sorts but are only iterated function depending
on external types.
This means that the formation rule for inductive function types is restricted
to the case where \grm{k \equiv \Pc}.
\end{defn}

We assume for the remainder of this chapter, that if \grm{\Gamma} specifies an
inductive family, we are provided with $\con{\Gamma} : \grm{\Gamma}^\CC$ and
$\elim_\grm{\Gamma} : \grm{\Gamma}^\EE(\con{\Gamma}, m)$ for each
$m : \grm{\Gamma}^\MM$.

TODO: explain reduction to W-types maybe

\section{Type Erasure}

As seen in the examples, the first step to prove the reducability is to formally
define the operation which we will call \emph{flattening} or -- inspired by
the syntax example -- \emph{type erasure}.
This operation strips away any dependencies between the sorts of a signature
as well as all external indices to sorts.
The operation should take arbitrary inductive-inductive signatures (contexts) and
return signatures for inductive families.
Let us look at what type erasure should do with our running examples:

\begin{example}[Natural Numbers]\label{ex:red-e-nat}
Since the inductive-inductive signature of the \emph{natural numbers}~\ref{ex:ii-syntax-nat} doesn't
contain any indexed sorts, type erasure should ``do nothing'' with it.
That is, returning the sort context and point context of the inductive family
syntax which looks like a obvious correspondence to it (cf. Example~\ref{ex:if-natvec}):
\begin{align*}
  &\tqm{\grm{(\cdot,\, \UU,\, \El(\vz),\, \Pi\left(\vs(\vz),\, \El(\vs(\vs(\vz)))\right))}^\EE_\Sc} \\
= &\tqm{(\cdot_\Sc,\, \UU)} \text{ and} \\
  &\tqm{\grm{(\cdot,\, \UU,\, \El(\vz),\, \Pi\left(\vs(\vz),\, \El(\vs(\vs(\vz)))\right))}^\EE} \\
= &\tqm{(\cdot,\, \El(\var(\vz)),\, \var(\vz) \Rightarrow_\Pc \El(\var(\vz)))} \text{.}
\end{align*}
\end{example}

\begin{example}[Vectors]
In the example of vectors \ref{ex:ii-syntax-vec} we need to erase the natural numbers
index of the only sort under consideration:
\begin{align*}
\tqm{\grm{\Gamma_{vec}}^\EE_\Sc}
 &= \tqm{(\cdot_\Sc,\, \UU)} \text{ and} \\
\tqm{\grm{\Gamma_{vec}}^\EE}
  &= \tqm{(\cdot,\, \El(\var(\vz)),\, 
    \ExtPiP{A}{\blm{\lambda a.\,}\ExtPiP{\N}{\blm{\lambda n.\,}
    \var(\vz) \Rightarrow_\Pc \El(\var(\vz))}})} \text{.}
\end{align*}
Note that the erasure of the vectors does not coincide with the vectors represented
as an inductive family (Example~\ref{ex:if-natvec}), because its sort lacks the
indexing over the natural numbers.
In fact, it's easy to see that the algebras of this signature would no be isomorphic
to the type of lists over the type \blm{A \times \N}.
\end{example}

\begin{example}[Type Theory Syntax]
In our syntax we will now see why the operation is called ``type erasure'':
%TODO
\end{example}

To go from examples to the general case, we will present the different components
of the type erasure operation in roughly the same order in which they appear in
Section~\ref{sec:ii-syntax}, most often needing to distinguish between sort
and point constructors.

\begin{defn}[Type Erasure]
First of all, each context will need to be split into a sort context and a point
context:
\begin{equation*}
\inferrule{\grm{\vdash \Gamma}}
  {\tqm{\SCon \grm{\Gamma}^\EE_\Sc}}
\qquad
\inferrule{\grm{\vdash \Gamma}}
  {\tqm{\vdash_{\grm{\Gamma}^\EE_\Sc} \grm{\Gamma}^\EE }}
\end{equation*}
To descent down the components of the contexts, we will need to define the operation
on types as well.
Since we are erasing all information from the sorts, we will only need this for
point types, though.
Unsurprisingly, we want them to be translated to point types in the appropriate
sort context:
\begin{equation*}
\inferrule{\grm{\Gamma \vdash A :: \Pc}}
  {\tqm{\grm{\Gamma}^\EE_\Sc \SCon \grm{A}^\EE :: \Pc}}
\end{equation*}
Using this we will be able to define the operation creating sort contexts by
\begin{align*}
\tqm{\grm{\cdot}^\EE_\Sc}
  &:\equiv\tqm{\cdot_\Sc} \text{,} \\
\tqm{\grm{(\Gamma,\, B)}^\EE_\Sc}
  &:\equiv \tqm{\left(\grm{\Gamma}^\EE_\Sc,\, \grm{\UU}^\EE_\Sc\right)} \text{ for \grm{B :: \Sc}, and} \\
\tqm{\grm{(\Gamma,\, A)}^\EE_\Sc}
  &:\equiv \tqm{\grm{\Gamma}^\EE_\Sc} \text{ for \grm{A :: \Pc}.}
\end{align*}
The generated point context over this sort context has to be extended in the case
where the input is an extension by a point type.
In the case where it is an extension by a sort type, we want to return the
unextended context, but to make up for the definition above, we need to weaken
to account for the extension of the resulting sort context:
\begin{align*}
\tqm{\grm{\cdot}^\EE}
  &:\equiv\tqm{\cdot} \text{,} \\
\tqm{\grm{(\Gamma,\, B)}^\EE}
  &:\equiv \tqm{\grm{\Gamma}^\EE[\wk_{\id}]} \text{ for \grm{B :: \Sc}, and} \\
\tqm{\grm{(\Gamma,\, A)}^\EE}
  &:\equiv \tqm{\left(\grm{\Gamma}^\EE,\, \grm{A}^\EE\right)} \text{ for \grm{A :: \Pc}.}
\end{align*}
So how do we define \tqm{\grm{A}^\EE} for a point type \grm{A}?
The fact the we have to recurse on \grm{\El(a)} makes it clear that we will have
to extend our operation to terms of sort types at least.
That is, together with \tqm{\grm{A}^\EE} we also need the following:
\begin{equation*}
\inferrule{\grm{\Gamma \vdash t : B :: \Sc}}
  {\tqm{\grm{\Gamma}^\EE_\Sc \SCon \grm{t}^\EE : \UU}}
\end{equation*}
And indeed, with this we can set
\begin{align*}
\tqm{\grm{\El(a)}^\EE}
  &:\equiv \tqm{\El(\grm{a}^\EE)} \text{.}
\end{align*}
For recursive $\Pi$-types, we need only care about the ones yielding point types.
Note that the operation turns a $\Pi$-type into a non-dependent function type!
\begin{align*}
\tqm{\grm{\Pi(a, A)}^\EE}
  &:\equiv \tqm{\grm{a}^\EE \Rightarrow_\Pc \grm{A}^\EE}
\end{align*}
Since we forgot about the indexing of sort types, erasure of sort-kinded application terms
is just erasure of its $\Pi$-type term:
\begin{align*}
\tqm{\grm{\IIapp(f)}^\EE}
  &:\equiv \tqm{\grm{f}^\EE} \text{ for \grm{\Gamma \vdash t : \Pi(a, B) :: \Sc}.}
\end{align*}
External $\Pi$-types and their applications convert directly into their
respective counterparts in the syntax of inductive families:
\begin{align*}
\tqm{\grm{\ExtPi{T}{A}}^\EE}
  &:\equiv \tqm{\ExtPiP{T}{\blm{\lambda \tau.\, }\grm{A(\bltau)}^\EE}} \text{, and} \\
\tqm{\grm{f(\bltau)}^\EE}
  &:\equiv \tqm{\grm{f}^\EE} \text{ for \grm{\Gamma \vdash f : \ExtPi{T}{B} : \Sc}}
\end{align*} %TODO this is a bit confusing since the application is for sorts and the types for points
Defining the erasure on point types and sort terms pulled back along a substitution,
we see that we will also need to erase entire sort substitutions.
This is achieved by extending the operation as follows:
\begin{equation*}
\inferrule{\grm{\IISub{\sigma}{\Gamma}{\Delta}}}
  {\tqm{\IISub{\grm{\sigma}^\EE_\Sc}{\grm{\Gamma}^\EE_\Sc}{\grm{\Delta}^\EE_\Sc}}}
\end{equation*}
We will then be able to use this in a straight forward way to define the pullbacks:
\begin{align*}
\tqm{\grm{A[\sigma]}^\EE}
  &:\equiv \tqm{\grm{A}^\EE[\sigma^\EE_\Sc]} \text{ for \grm{\Gamma \vdash A :: \Pc} and} \\
\tqm{\grm{t[\sigma]}^\EE}
  &:\equiv \tqm{\grm{t}^\EE[\sigma^\EE_\Sc]} \text{ for \grm{\Gamma \vdash t : B :: \Sc}.}
\end{align*}
Erasure of substitutions is built recursively, ignoring point types.
Likewise, the first projection will ignore point types:
\begin{align*}
\tqm{\grm{\epsilon}^\EE_\Sc}
  &:\equiv \tqm{\epsilon} \text{,} \\
\tqm{\grm{(\sigma,\, t)}^\EE_\Sc}
  &:\equiv \tqm{(\grm{\epsilon}^\EE_\Sc,\, \grm{t}^\EE)} \text{ for \grm{\Gamma \vdash t : B :: \Sc}, and} \\
\tqm{\grm{(\sigma,\, t)}^\EE_\Sc}
  &:\equiv \tqm{\grm{\epsilon}^\EE_\Sc} \text{.} \\
\tqm{\grm{\pi_1(\sigma)}^\EE_\Sc}
  &:\equiv \tqm{\pi_1(\grm{\sigma}^\EE_\Sc)} \text{ for \grm{\IISub{\sigma}{\Gamma}{(\Delta,\, B :: \Sc)}},} \\
\tqm{\grm{\pi_1(\sigma)}^\EE_\Sc}
  &:\equiv \tqm{\grm{\sigma}^\EE_\Sc} \text{ for \grm{\IISub{\sigma}{\Gamma}{(\Delta,\, A :: \Pc)}}, and} \\
\tqm{\grm{\pi_2(\sigma)}^\EE}
  &:\equiv \tqm{\pi_2(\grm{\sigma}^\EE_\Sc)} \text{.}
\end{align*} %TODO laws
This concludes the definition of the erasure operation.
\end{defn}

For the steps that follow it will be necessary to equip the \emph{algebras}
of the resulting signatures with a substitution calculus that also considers
point contexts instead of only sort contexts.
To this end, we defined what we called lifted substitution algebras in Definition~\ref{def:if-alg-lsub}

\section{(OLD) -- Type Erasure}

Since we cannot define type families simultaneously together with their index
type, we will first produce a version of a given code, whose dependencies between
sorts have been \emph{erased}.
For this, we will use an operation on contexts, types and terms, which we will
call \emph{flattening}.
The resulting code will specify a type which is now a mutual definition
of a number of plain types instead of families,
but it will, in general, contain too many elements, because it lacks all the
restrictions on which fiber arguments of inductively generated elements should
lie in.

Note that $\flatten{}$ erases all uses of the function type in the
specifications of sorts.

\section{(OLD) -- Wellformedness Predicates}

From this section onwards we will, for the code $\grm{\Gamma \equiv (\cdot, x_1 : A_1, \ldots, x_n : A_n)}$
in consideration, assume that we have the projections $\bar{x_1} : \bar{A_1}, \ldots, \bar{x_n} : \bar{A_1}$ of
$\con{\grm{\flatten{\Gamma}}} : \grm{\flatten{\Gamma}}^\CC$ at our disposal in the target
language.

To motivate the next step of our construction, consider the following code:
\begin{equation*}
\grm{
A : \UU,\, B : A \to \UU,\, a_0, a_1 : \underline{A},\, b : \underline{B(a_0)}
} \text{.}
\end{equation*}
In its flattened form
\begin{equation*}
\grm{
\bar{A} : \UU,\, \bar{B} : \UU,\, \bar{a_0}, \bar{a_1} : \underline{\bar{A}},\, \bar{b} : \underline{\bar{B}}
} \text{,}
\end{equation*}
there is no way to recognize, whether \grm{b} was meant to be in \grm{B(a_0)},
in \grm{B(a_1)}.
To reintroduce this piece of information, we will transform a given code into a
mutually defined predicate over the flattened code, which shall indicate, in which
fiber of a type family a given constructor (or one of its inductive arguments)
should be located.

\begin{defn}[Wellformedness]
Like with \grm{\flatten{}}, we define a \emph{annotation}
transformation \annotate{} on contexts, types, and
terms by structural recursion on the syntax.
On contexts and sort types it is defined like this:
\begin{align*}
\annotate{\cdot}			&:\equiv \grm{\cdot} \\
\annotate{(\Gamma, x : A)}		&:\equiv
	\begin{cases}
	\grm{\annotate{\Gamma}, W_x : \blm{\bar{x}} \to \annotate{A}}	& \text{for \grm{A :: \Sc}} \\
	\grm{\annotate{\Gamma}, w_x : \anntwo{\bar{x}}{A}}		& \text{for \grm{A :: \Pc}}
	\end{cases} \\
\annotate{\UU}				&:\equiv \grm{\UU} \\
\annotate{(x : a) \to B}		&:\equiv \grm{\flatten{a} \to \annotate{B}} \text{ for \grm{B :: \Sc}} \\
\annotate{(\blm{x : A}) \to B}		&:\equiv \grm{\blm{A} \to \annotate{B}}  \text{ for \grm{B :: \Sc}}
\end{align*} %TODO alignment
As we can see, on point types, the operation takes another argument, which is a
term of the target theory.
\begin{align*}
\anntwo{y}{\UU}				&:\equiv \grm{\UU} \\
\anntwo{y}{\underline{a}}		&:\equiv \grm{\underline{\anntwo{y}{a}}} \\
\anntwo{y}{(x : a) \to B}		&:\equiv
	\grm{(\blm{\bar{x} : \flatten{a}})(\annotate{\blm{\bar{x}}, a})}  \\
	& \grm{\hfill{} \to \anntwo{y(\bar{x})}{B}} & \text{for \grm{B :: \Pc}} \\
\anntwo{y}{(\blm{x : A}) \to B}	&:\equiv
	\grm{(\blm{x : A}) \to \anntwo{y(x)}{B}} & \text{for \grm{B :: \Pc}}
\end{align*}
Annotate is also defined where the second argument is a term of a sort type:
\begin{align*}
\anntwo{y}{x}				&:\equiv \grm{W_x(\blm{y})} \text{ for variables \grm{x}} \\
\anntwo{y}{t(u)}			&:\equiv \grm{\anntwo{y}{t}(\flatten{u})} \\
\anntwo{y}{t(\blm{u})}			&:\equiv \grm{\anntwo{y}{t}(\blm{u})} \\
\anntwo{y}{(\blm{x : A}) \to b}		&:\equiv \grm{(\blm{x : A}) \to \anntwo{y(x)}{b}} \\
\anntwo{y}{t \app \blm{u}}		&:\equiv \grm{\anntwo{y}{t} \app \blm{u}}
\end{align*}
\end{defn}

\begin{example}[Type Theory Syntax]
Let us look at one sort and one point constructor of our prime example~\ref{ex:ttintt}.
The sort \grm{\mathop{Ty} : \mathop{Con} \to \UU} gets transformed to the following:
\begin{align*}
\grm{W_{\mathop{Ty}} :} 
&\phantom{\equiv~} \grm{\blm{\overline{Ty}} \to \annotate{\mathop{Con} \to \UU} } \\
&\equiv \grm{\blm{\overline{Ty}} \to \flatten{\mathop{Con}} \to \annotate{\UU}} \\
&\equiv \grm{\blm{\overline{Ty}} \to \blm{\overline{Con}} \to \UU} \text{.}
\end{align*}
The point constructor
\grm{\mathop{pi} : (\Gamma : \mathop{Con})(A : \mathop{Ty}(\Gamma))(\mathop{Ty}(\mathop{ext}(\Gamma, A))) \to \underline{\mathop{Ty}(\Gamma)}}
becomes a term \grm{w_{\mathop{pi}}} of the following type:
\begin{align*}
& \grm{\anntwo{\overline{x}}{(\Gamma : \mathop{Con})(A : \mathop{Ty}(\Gamma))(\mathop{Ty}(\mathop{ext}(\Gamma, A)))
	\to \mathop{Ty}(\Gamma)}} \\
\equiv & \grm{(\blm{\overline{\Gamma} : \overline{Con}})(\anntwo{\overline{\Gamma}}{\mathop{Con}})
	\to \anntwo{\overline{pi}(\overline{\Gamma})}{(A : \mathop{Ty}(\Gamma))(\mathop{Ty}(\mathop{ext}(\Gamma, A))) \to \underline{\mathop{Ty}(\Gamma)}}} \\
\equiv & \grm{
	(\blm{\overline{\Gamma} : \overline{Con}})(W_{\mathop{Con}}(\blm{\overline{\Gamma}}))
	(\blm{\overline{A} : \overline{Ty}})(\anntwo{\overline{A}}{\mathop{Ty}(\Gamma)}) } \\
& \grm{
	\to \anntwo{\overline{pi}(\overline{\Gamma}, \overline{A})}{(\mathop{Ty}(\mathop{ext}(\Gamma, A))) \to \underline{\mathop{Ty}(\Gamma)}}
} \\
\equiv & \grm{
	(\blm{\overline{\Gamma} : \overline{Con}})(W_{\mathop{Con}}(\blm{\overline{\Gamma}}))
	(\blm{\overline{A} : \overline{Ty}})(\anntwo{\overline{A}}{\mathop{Ty}}(\blm{\overline{\Gamma}}))
	(\blm{\overline{B} : \overline{Ty}})(\anntwo{\overline{B}}{\mathop{Ty}(\mathop{ext}(\Gamma, A))}) } \\
& \grm{
	\to \anntwo{\overline{pi}(\overline{\Gamma}, \overline{A}, \overline{B})}{\underline{Ty(\Gamma)}}
} \\
\equiv & \grm{
	(\blm{\overline{\Gamma} : \overline{Con}})(W_{\mathop{Con}}(\blm{\overline{\Gamma}}))
	(\blm{\overline{A} : \overline{Ty}})(W_{\mathop{Ty}}(\blm{\overline{A}}, \blm{\overline{\Gamma}}))
	(\blm{\overline{B} : \overline{Ty}})(\anntwo{\overline{B}}{\mathop{Ty}}(\blm{\overline{ext}(\overline{\Gamma}, \overline{A})}))
} \\
& \grm{
	\to \underline{\anntwo{\overline{pi}(\overline{\Gamma}, \overline{A}, \overline{B})}{\mathop{Ty}(\Gamma)}}
} \\
\equiv & \grm{
	(\blm{\overline{\Gamma} : \overline{Con}})(W_{\mathop{Con}}(\blm{\overline{\Gamma}}))
	(\blm{\overline{A} : \overline{Ty}})(W_{\mathop{Ty}}(\blm{\overline{A}}, \blm{\overline{\Gamma}}))
	(\blm{\overline{B} : \overline{Ty}})(W_{\mathop{Ty}}(\blm{\overline{B}}, \blm{\overline{ext}(\overline{\Gamma}, \overline{A})}))
} \\
& \grm{
	\to \underline{\anntwo{\overline{pi}(\overline{\Gamma}, \overline{A}, \overline{B})}{\mathop{Ty}}(\blm{\overline{\Gamma}})}
} \\
\equiv & \grm{
	(\blm{\overline{\Gamma} : \overline{Con}})(W_{\mathop{Con}}(\blm{\overline{\Gamma}}))
	(\blm{\overline{A} : \overline{Ty}})(W_{\mathop{Ty}}(\blm{\overline{A}}, \blm{\overline{\Gamma}}))
	(\blm{\overline{B} : \overline{Ty}})(W_{\mathop{Ty}}(\blm{\overline{B}}, \blm{\overline{ext}(\overline{\Gamma}, \overline{A})}))
} \\
& \grm{ \hspace*{0pt}\hfill
	\to \underline{W_{\mathop{Ty}}(\blm{\overline{pi}(\overline{\Gamma}, \overline{A}, \overline{B})},\blm{\overline{\Gamma}})}
} 
\end{align*}
\end{example}

\begin{lemma}
We have \grm{\vdash \annotate{\Gamma}} for every \grm{\vdash \Gamma}.
\end{lemma}

\begin{proof}
We simultaneouly prove that all of the following rules are admissible:
\begin{equation*}
\begin{gathered}
\inferrule{\grm{\vdash \Gamma}}{\grm{\vdash \annotate{\Gamma}}}
\qquad
\inferrule{\grm{\Gamma \vdash B :: \Sc}}{\grm{\annotate{\Gamma} \vdash \annotate{B} :: \Sc}}
\\[.7em]
\inferrule{\grm{\Gamma \vdash A :: \Pc} \\ \bar{x} : \flatten{A}}{\grm{\annotate{\Gamma} \vdash \anntwo{\bar{x}}{A} :: \Pc}}
\\[.7em]
\inferrule{\grm{\Gamma \vdash a : B :: \Sc} \\ \bar{x} : \flatten{a}} %TODO underline here??
	{\grm{\annotate{\Gamma} \vdash \anntwo{\bar{x}}{a} : \annotate{B} :: \Sc}}
\end{gathered}
\end{equation*}
For context extension by sort constructors,
assume that \grm{\vdash \annotate{\Gamma}} and \grm{\annotate{\Gamma} \vdash \annotate{B} :: \Sc}.
Since \blm{\overline{x} : \overline{A} \equiv \UU}, we infer \grm{\annotate{\Gamma} \vdash \blm{\overline{x}} \to \annotate{B}},
and thus
\begin{equation*}
\grm{\vdash \annotate{\Gamma}, W_x : \blm{\overline{x}} \to \annotate{B} \equiv \annotate{\Gamma, x : A}} \text{.}
\end{equation*}
For extension by point constructors we similarly reason that given
\grm{\vdash \annotate{\Gamma}} and \grm{\Gamma \vdash A :: \Pc}, from which we
can infer \grm{\annotate{\Gamma} \vdash \anntwo{\overline{x}}{A}} for \blm{\overline{x} : \overline{A}},

Let us next handle the case of types in \grm{\Sc}.
Given \grm{\vdash \annotate{\Gamma}},
we obviously have \grm{\annotate{\Gamma} \vdash \annotate{\UU} \equiv \UU}.
To show that \grm{\annotate{\Gamma} \vdash \annotate{(x : a) \to B}}, we may assume
that \grm{\vdash \annotate{\Gamma}} and that for \blm{\overline{x} : \overline{a}}
we have \grm{\annotate{\Gamma} \vdash \anntwo{\overline{x}}{a} : \UU},
and that \grm{\annotate{\Gamma, x : \underline{a}} \vdash \annotate{B}}.
That latter reduces to
\begin{equation*}
\grm{\annotate{\Gamma}, w_x : \underline{\anntwo{\overline{x}}{a}} \vdash \annotate{B} :: \Sc} \text{,}
\end{equation*}
but since, by definition, \grm{B} cannot contain any reference to \grm{w_x},
we conclude \grm{\annotate{\Gamma} \vdash \annotate{B}}, which, together with
\grm{\annotate{\Gamma}} and \blm{\overline{A} : \UU}, suffices to show
the desired result.
The case of non-inductive functions works likewise.

There are three formers for \grm{\Pc}-types: Small types, inductive functions, and
non-inductive functions:
\begin{itemize}
\item Assume that \grm{\Gamma \vdash a : \UU} and thus that for each
\blm{\overline{x} : \overline{a}} we have
\grm{\annotate{\Gamma} \vdash \anntwo{\overline{x}}{a} : \UU}.
We know that \grm{\annotate{\Gamma} \vdash \underline{\anntwo{\overline{x}}{a}}}
which is enough to infer \grm{\annotate{\Gamma} \vdash \underline{\anntwo{\overline{x}}{a}} \equiv \anntwo{\overline{x}}{\underline{a}}}.
\item Given \grm{\Gamma \vdash a : \UU} and \grm{\Gamma, x : \underline{a} \vdash B :: \Pc}
we may assume that \grm{\annotate{\Gamma} \vdash \anntwo{\overline{x}}{a} : \UU} for \blm{\overline{x} : \overline{a}}
and that for \blm{\overline{y}:\overline{B}} we have
\begin{equation*}
\grm{\annotate{\Gamma}, w_x : \anntwo{\overline{x}}{a} \vdash \anntwo{\overline{y}}{B} :: \Pc} \text{.}
\end{equation*}
We want to show that for \blm{\overline{f} : \overline{(x : a) \to B} \equiv (\overline{x} : \overline{a}) \to \overline{B}},
\begin{align*}
\grm{\annotate{\Gamma}} &\vdash \grm{\anntwo{\overline{f}}{(x : a) \to B} :: \Pc} \\
&\equiv \grm{(\blm{\overline{x} : \overline{a}})(\anntwo{\overline{x}}{a}) \to \anntwo{\overline{f}(\overline{x})}{B}} \text{.}
\end{align*}
To this end, we need to show that for \blm{\overline{x} : \overline{a}} we have
\grm{\annotate{\Gamma} \vdash \anntwo{\overline{x}}{a} \to \anntwo{\overline{f}(\overline{x})}{B}},
but this is true by induction and weakening.
\end{itemize}
As a last part of this proof, we need to show that \grm{\annotate{}} behave well
on the terms of \grm{\Sc}-types: Variables and applications of inductive and
non-inductive functions.
\begin{itemize}
\item Given \grm{\Gamma \vdash B :: \Sc} and assuming \grm{\annotate{\Gamma} \vdash \annotate{B}},
we want to show that for \blm{\overline{x}:\overline{a}} we have
\begin{equation*}
\grm{\annotate{\Gamma, a : B} \vdash \anntwo{\overline{x}}{a} : \annotate{B}} \text{,}
\end{equation*}
but this reduces to
\begin{equation*}
\grm{\annotate{\Gamma}, W_a : \blm{\overline{a}} \to \annotate{B} \vdash W_a(\blm{\overline{x}}) : \annotate{B}} \text{,}
\end{equation*}
which is obviously true.
\item Let \grm{\Gamma \vdash t : (x : a) \to B :: \Sc}, which lets us assume
\begin{equation*}
\grm{\annotate{\Gamma} \vdash \anntwo{\overline{f}}{t} : \blm{\overline{a}} \to \annotate{B}} \text{ for \blm{\overline{f} : \overline{t}}.}
\end{equation*}
For \grm{u : \underline{a}} we than have that
\begin{align*}
\grm{\annotate{\Gamma} \vdash} & \grm{\anntwo{\overline{f}}{t}(\overline{u}) : \annotate{B}} \\
& \grm{\equiv \anntwo{\overline{f}}{t(u)} : \annotate{B[x \mapsto u]}} \text{,}
\end{align*}
since \grm{x} is turned into \grm{w_x} which \grm{\annotate{B}} doesn't depend on. %TODO make this cleaner
\item For \blm{A : \UU}, \grm{\vdash \annotate{\Gamma}}, and
\begin{align*}
\grm{\annotate{\Gamma} \vdash} & \anntwo{\overline{f}}{t} : \annotate{(\blm{x : A}) \to B} \\
& \equiv \anntwo{\overline{f}}{t} : A \to \annotate{B} \text{ for \blm{\overline{f} : \overline{t}},}
\end{align*}
we see that for any \blm{u : A} we have \grm{\anntwo{\overline{f}}{t}(u) \equiv \anntwo{\overline{f}}{t(u)} : \annotate{B[\blm{x} \mapsto \blm{u}]}}
as desired.
\end{itemize}
\end{proof}

Like \grm{\flatten{}}, \grm{\annotate{}} only produces inductive families, so
we can assume to have $\con{\grm{\annotate{\Gamma}}} : \grm{\annotate{\Gamma}}^\CC$:

\begin{lemma}
If \grm{\vdash{\Gamma}}, then \grm{\annotate{\Gamma}} is an inductive family.
\end{lemma}

\begin{proof}
All occurences of inductive functions in sort constructors are replaced
by non-inductive functions.
\end{proof}

\section{Constructing the initial algebra}

With \grm{\flatten{\Gamma}} and \grm{\annotate{\Gamma}} with have two inductive
families which will suffice to construct the initial algebra for any code \grm{\Gamma}:
We use $\Sigma$-types over the wellformedness predicates to select which ones of
the objects in the flattened types and type families we want to include in the
initial algebra.

\begin{defn}(Initial algebra)
Define \blm{\con{}} on contexts by:
\begin{align*}
\con{\cdot} &:\equiv \star : \unit \\
\con{\Gamma, a : B :: \Sc} &:\equiv (\con{\Gamma}, \contwo{W_a}{B}) \\
\con{\Gamma, x : A :: \Pc} &:\equiv (\con{\Gamma}, \conthree{\bar{x}}{w_x}{A}) \\
\end{align*}
On sort constructors we need:
\begin{align*}
\contwo{f}{\UU} &:\equiv
\begin{cases}
(x : A) \to \contwo{b}{\UU} & \text{if \blm{f \equiv (x : A) \to b : \UU},} \\
\Sigma(f) & \text{otherwise} %TODO avoid this distinction
\end{cases}\\
\contwo{f}{(x : a) \to B :: \Sc} &:\equiv \lambda x : \conthree{\flatten{a}}{\annotate{a}}{\UU}. \contwo{\lambda y. f(y, \pr_1(x))}{B} \\ %TODO fix
\contwo{f}{(\blm{x : A}) \to B :: \Sc} &:\equiv \lambda x : A. \contwo{\lambda y. f(y, x)}{B}
\end{align*}
On point constructor we have:
\begin{align*}
\conthree{f}{w}{\underline{a}} &:\equiv (f, w) \text{ for a variable \grm{a}} \\
\conthree{f}{w}{\underline{t(u)}} &:\equiv \conthree{f}{w}{\underline{t}} \\
\conthree{f}{w}{\underline{(\blm{x : A}) \to b}} &:\equiv \lambda x : A. \conthree{f(x)}{w(x)}{\underline{b}} \\
\conthree{f}{w}{(x : a) \to B :: \Pc} &:\equiv
\lambda x : \conthree{\flatten{a}}{\annotate{a}}{\UU}. \conthree{f(\pr_1(x))}{w(\pr_1(x), \pr_2(x))}{B} \\ %TODO fix
\conthree{f}{w}{(\blm{x : A}) \to B :: \Pc} &:\equiv
\lambda x : A. \conthree{f(x)}{w(x)}{B}
\end{align*}
Here, \blm{\pr_1} and \blm{\pr_2} are generalized projections, defined on universe
terms and for $i \in {1, 2}$ as follows:
\begin{align*}
\pr_i(y, \grm{a}) &:\equiv \pr_i(y) \text{ for a variable \grm{a}} \\
\pr_i(y, \grm{t(u)}) &:\equiv \pr_i(y) \\ %TODO this really seems fishy
\pr_i(y, \grm{(\blm{x : A}) \to b}) &:\equiv \lambda x : A. \pr_i(y(x), \grm{b})
\end{align*}
Also we need for universe terms a way to reference them as the domain of function
types:
\begin{align*}
\mathsf{ref}(\grm{a}) &:\equiv \contwo{W_a}{\UU} \text{ for a variable \grm{a}} \\
\mathsf{ref}(\grm{t(u)}) &:\equiv \mathsf{ref}(\grm{t})(\bar{u}) \\
\mathsf{ref}(\grm{(\blm{x : A}) \to b}) &:\equiv (x : A) \to \mathsf{ref}(\grm{b})
\end{align*}
\end{defn}

\section{Constructing the Eliminator}

TODO: Define $r$


\bibliographystyle{unsrtnat}
\bibliography{references}

\end{document}

