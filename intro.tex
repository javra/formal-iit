\section{Examples of Inductive-Inductive Types}

In the following, we will take a look at a few examples which we are going to
revisit at various steps throughout this presentation:

\begin{example}[Type Theory Syntax]\label{ex:ttintt}
\cite{ttintt} showed how to internalise the syntax of type theory inside type
theory itself, using a quotient inductive-inductive type.
Leaving out terms and substitutions, we arrive at a fragment of type theoretical
syntax specifying a type of contexts and a type of types over a certain contexts,
together with type formers for a unit type and a $\Pi$-type:
We want $\mathop{Con} : \UU$ to be inductively defined by
\begin{align*}
\mathop{nil}	&: \mathop{Con} \text{and } \\
\mathop{ext}	&: (\Gamma : \mathop{Con})(A : \mathop{Ty}(\Gamma)) \to \mathop{Con} \text{,}
\end{align*}
while simultaneously defining a family $\mathop{Ty} : \mathop{Con} \to \UU$ with constructors
\begin{align*}
\mathop{unit}	&: (\Gamma : \mathop{Con}) \to \mathop{Ty}(\Gamma) \text{ and} \\
\mathop{pi}	&: (\Gamma : \mathop{Con})(A : \mathop{Ty}(\Gamma))(B : \mathop{Ty}(\mathop{ext}(\Gamma, A))) \to \mathop{Ty}(\Gamma) \text{.}
\end{align*}
\end{example}

\begin{example}[Free Dense Completion]
\cite{nordvallinductive} proposed the example of a ``free dense completion'' of
an order (or, more general, any relation) which for any type $A : \UU$ and
any type valued relation $\_<\_ : A \to A \to \UU$ on $A$ freely adds midpoints
to all pairs of related elements by $\_<\_$.
It does so by introducing a new type $A' : \UU$ inductively generated by the
original points and their midpoints:
\begin{align*}
\iota_A		&: A \to A' \text{ and} \\
\mathop{mid}	&: \{x, y : A'\}(p : x <' y) \to A' \text{.}
\end{align*}
But since our relation was only defined on $A$, we have to extend it to $A'$ by
postulating
\begin{align*}
\iota_<		&: \{a, b : A\}(p : a < b) \to \iota_A(a) <' \iota_A(b) \text{,} \\
\mathop{mid}_l	&: \{x, y : A'\}(p : x <' y) \to x < \mathop{mid}(p) \text{, and} \\
\mathop{mid}_r	&: \{x, y : A'\}(p : x <' y) \to \mathop{mid}(p) < y \text{.}
\end{align*}
\end{example}
