\section{Background}

\section{Homotopy Type Theory and Higher Inductive Types}

\section{The Concept of Induction-Induction}

In the following, we will take a look at a few examples which we are going to
revisit at various steps throughout this presentation:

\begin{example}[Type Theory Syntax]\label{ex:ttintt}
\cite{ttintt} showed how to internalise the syntax of type theory inside type
theory itself, using a quotient inductive-inductive type.
Leaving out terms and substitutions, we arrive at a fragment of type theoretical
syntax specifying a type of contexts and a type of types over a certain contexts,
together with type formers for a unit type and a $\Pi$-type:
We want $\mathop{Con} : \UU$ to be inductively defined by
\begin{align*}
\mathop{nil}	&: \mathop{Con} \text{and } \\
\mathop{ext}	&: (\Gamma : \mathop{Con})(A : \mathop{Ty}(\Gamma)) \to \mathop{Con} \text{,}
\end{align*}
while simultaneously defining a family $\mathop{Ty} : \mathop{Con} \to \UU$ with constructors
\begin{align*}
\mathop{unit}	&: (\Gamma : \mathop{Con}) \to \mathop{Ty}(\Gamma) \text{ and} \\
\mathop{pi}	&: (\Gamma : \mathop{Con})(A : \mathop{Ty}(\Gamma))(B : \mathop{Ty}(\mathop{ext}(\Gamma, A))) \to \mathop{Ty}(\Gamma) \text{.}
\end{align*}
\end{example}

\begin{example}[Free Dense Completion]
\citet{nordvallinductive} proposed the example of a ``free dense completion'' of
an order (or, more general, any relation) which for any type $A : \UU$ and
any type valued relation $\_<\_ : A \to A \to \UU$ on $A$ freely adds midpoints
to all pairs of related elements by $\_<\_$.
It does so by introducing a new type $A' : \UU$ inductively generated by the
original points and their midpoints:
\begin{align*}
\iota_A		&: A \to A' \text{ and} \\
\mathop{mid}	&: \{x, y : A'\}(p : x <' y) \to A' \text{.}
\end{align*}
But since our relation was only defined on $A$, we have to extend it to $A'$ by
postulating
\begin{align*}
\iota_<		&: \{a, b : A\}(p : a < b) \to \iota_A(a) <' \iota_A(b) \text{,} \\
\mathop{mid}_l	&: \{x, y : A'\}(p : x <' y) \to x < \mathop{mid}(p) \text{, and} \\
\mathop{mid}_r	&: \{x, y : A'\}(p : x <' y) \to \mathop{mid}(p) < y \text{.}
\end{align*}
\end{example}

\section{Contributions and Publications}

While parts of this thesis consists of the review and introduction of constructions
and knowledge which is already established,
other parts offer new contributions which stands on the shoulders of these ``giants''.
The major contributions of the thesis are the following:
\begin{itemize}
\item The formalization of lots of homotopy theoretic notions in the theorem prover
Lean as described in \Cref{sec:tt-lean}.
\item The formulation of the characterization theorem for path spaces
of homotopy Coequalizers \Cref{thm:paths-main-thm}, as well as its proof
as given in \Cref{sec:paths-main}.
\item The adaptation of this theorem for pushouts as described in \Cref{thm:paths-main-pushout}.
\item The formulation of a possible higher Seifert-van Kampen theorem
as stated in \Cref{thm:paths-higher-SvK}.
\item The formalization of \Cref{sec:paths-main} and \Cref{thm:paths-main-pushout}
in Lean.
\item An adapation of the syntax for higher inductive-inductive types by
\citet{ambrussyntax} to separate sort and point constructors, as described, including
its semantics in \Cref{chp:iit}.
\item A syntax of signatures for inductive families as given in \Cref{chp:if}.
\item A formal specification of type erasure, wellformedness relation and eliminator
relation given as syntactic translations as described in \Cref{sec:red-e},
\Cref{sec:red-w}, and \Cref{sec:red-r}.
\item A formal definition of the ``sigma construction'' for an initial algebra
for inductive-inductive types as proposed in \Cref{sec:red-sg}.
\end{itemize}

Parts of this thesis have been peer-reviewed and published already:
\begin{itemize}
\item \bibentry{leanhott}
\item \bibentry{paths}
\end{itemize}

\section{Structure of this Thesis}

We will start off this thesis by giving a more detailled exposition of the
background in type theory, which is the basis for the further content.
In this endevour, \Cref{chp:tt} does not only serve to give the necessary background
to readers unfamiliar with dependent types, it also sets notations and terminology
which we will re-use in the later chapters.
In \Cref{sec:tt-provers}, it furthermore gives a short characterization of the interactive
theorem provers Lean and Agda.

\Cref{chp:hit} will then first introduce some examples of higher inductive types
and will then go on to propose homotopy coequalizers as a fundamental higher
inductive type which can serve to encode all of these examples.
Sketches for two proofs using the ``encode-decode'' method will be given in
\Cref{sec:hit-encode-decode}.

Following the introduction of homotopy coequalizers we will then (\Cref{chp:paths})
see how we can
characterize the path spaces of all instances to replace encode-decode proofs.
Apart from proving this characterization (\Cref{sec:paths-main})
we will demonstrate this use in some examples (\Cref{sec:paths-applications}
and \Cref{sec:paths-svk}).

Switching not only the type theoretical setting from homotopy type theory
but also the focus of the thesis, the second half will explore types which
instead of higher dimensional structure carry intricate dependencies between
multiple types to be defined.

We will start this second half by first giving a syntax for inductive-inductive
types (\Cref{chp:iit}) including its semantics,
before comparing it to the simpler fragment of inductive families (\Cref{chp:if}).
The latter will be reduced to indexed W-types in \Cref{sec:if-ex}.

After reducing inductive families to indexed W-types we will try to reduce
inductive-inductive types to inductive families:
In \Cref{chp:red} we will a formal description about how to generate the
inductive families which correspond to
erasing the inductive-inductive typing information,
recovering it with a wellformedness predicate, yielding a candidate for an inital
object in the target type theory.
Then we will present a binary relation which could be used in future to prove
its initiality.










