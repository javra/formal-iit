\section{Background}

This thesis explores several problems in the field of \emph{type theory}.
By type theory we will always mean various flavors of what is usually
referred to as Martin-Löf type theory or dependent type theory.
Martin-Löf type theory (MLTT) can serve as a foundational framework for
mathematics as well as an organization principle for functional programming languages.

In the field of type theory, many researchers either apply theoretical considerations
achieve cleaner formalizations of mathematical content,
they create implementations of type theory which can be used as
interactive theorem proving systems, and they try to extend type
theory to improve its usability and justify these extensions with models.

In this spirit, this thesis will also explore ways to make certain kinds of formalizations
and certain constructions in type theory smoother and easier to use.
While it is \emph{mainly theoretical work}, it has consequences for the application of
interactive theorem provers, as they are used today.
This thesis is broadly split in two halfs.
Both halfs will explore different classes of language elements both of which are generalizations
of a language feature which is
called \emph{inductive types}.
Inductive types are a common way to define collections of data in mathematics
as well as computer science.

In the first half, we will try to encapsulate a common proof strategy which
is often used in the field of homotopy type theory and especially in
synthetic homotopy thoery.
We will do this by
proving a very general result about \emph{higher inductive types}.
These are inductive types in which statements about \emph{equality} between
elements of these data types carry a \emph{higher-dimensional structure},
making them on the one hand an interesting object of study in terms of their topology,
but on the other hand they are sometimes hard to handle.
Our theorem allows an easier way to prove propositions about these
equaities of elements.

In the second half, we will explore the topic of induction-induction.
\emph{Inductive-inductive types} are a class of inductive types which
allows us to define a data type simultaneously with data types depending on
values of the former type.
We will give an exact definition of what inductive-inductive types are,
and, with a new definition of \emph{inductive families} provide a point of
comparison which allows us to attempt to represent each example of
an inductive-inductive type as a construction based on a series of
inductive families.

\section{Homotopy Type Theory and Higher Inductive Types}

Homotopy type theory is a relatively new field which connects the study of
dependent types with the field of higher category theory and homotopy theory.
This synthesis has since offered a new perspective on how to constructively
have a formal representation of homotopy theory which is \emph{synthetic},
i.\,e.\ builds the spaces which are considered out of just a few fundamental
operations.

The connection to homotopy theory is based on the observation that one might
consider equality types, which by the proposition-as-types interpretation of type
theory, represent the statement that two elements $x$ and $y$ of a type $A$
are equal, as representing the spaces of \emph{paths modulo homotopy}.
Extending on this equivalence, we can view types to represent spaces,
and types depending on the data of other types as fibrations.

In this setting, we want to consider types, which are inductively defined,
like for example the natural numbers, but which, besides the \emph{points}
of the type also allow the (free) generation of new paths between points
and ``higher'' paths between
other paths.
These types are called \emph{higher inductive types}.

To prove facts about higher inductive types, for example in order to get the
type theoretic equivalent of the fact that the fundamental group of the circle
is equivalent to the integers, or the type theoretic Seifert-van Kampen theorem,
which characterizes the fundamental groupoid of a pushout of spaces,
a proof strategy with the name ``encode-decode method'' is employed.

In this thesis we provide a theorem which can be seen a generalization of
encode-decode proofs.
The application of this theorem can help to reduce ``boiler plate'' overhead
in formalizations and in reasoning about the equalities between points
of higher inductive types.

\section{The Concept of Induction-Induction}\label{sec:intro-iit}

While the first half of this thesis is about ways to make inductive types carry
higher-dimensional structure, the second half is about allowing for inductive types
which are more \emph{interdependent}:
There are situations in which we might not only want to define one single type
or type family, but instead we want to define a type $A$ and a type family $B : A \to \UU$,
or even a whole system of new types, indexed over each other,
mutually.
``Mutually'' here means that the point constructors, which specify which elements
we can form can refer to any of the types being defined.
And more than that, also the signature of the type families which we define can
be indexed over other type families the definition of which is not finished.

To illustrate one main application, imagine we wanted to formalize and reason
about the syntax of type theory in a type theoretic setting.
The environment of variables which are at our disposal at a given point in a piece
of type theoretic syntax are captured in what is called a \emph{context}.
We can model contexts as a type $Con : \UU$.
But at the same time we want to model types as existing in a context, so we want
a simultaneous definition of a type family $Ty : Con \to \UU$.
To see that we can not first define $Con$ and then move on to define $Ty$, consider
the following data which $Con$ and $Ty$ should include:

Contexts can be seen as lists of types depending on previous entries in that
list.
In that sense, it is obvious that for a context $\Gamma : Con$ and a type
$A : Ty(\Gamma)$ in that context, we want a context which represents the
extension of $\Gamma$ by $A$.
This means that $Con$ should have a constructor of the form
\begin{equation*}
ext: (\Gamma : Con) \to Ty(\Gamma) \to Con \text{,}
\end{equation*}
and the fact that this constructor mentions $Ty$ is already enough to exclude
a sequential definition of $Con$ and $Ty$.

Given the need for inductive-inductive types we can ask the question of whether
this concept is really stronger than inductive families without dependencies
between the sorts.
At first glance it might seem as if they are more expressive,
but on a closer look we can discover that we can actually reduce every
example of an inductive-inductive type to an inductive family.

The second half of this thesis is about making this reduction, which can easily
be seen to work on specific examples, more general.
To make it a formal statement we will have to give precise definitions of
what inductive-inductive types are and what, as our reference point,
inductive families are.
While we don't succeed at providing a full formal proof for the reduction,
the essential steps of it are complete and formalized in Agda. %TODO re-read and improve

\section{Contributions and Publications}

While parts of this thesis consists of the review and introduction of constructions
and knowledge which is already established,
other parts offer new contributions which stand on the shoulders of these ``giants''.
The contributions of this thesis include the following:
\begin{itemize}
\item The formalization of lots of homotopy theoretic notions in the theorem prover
Lean as described in \Cref{sec:tt-lean}.
\item The formulation of the characterization theorem for path spaces
of homotopy Coequalizers \Cref{thm:paths-main-thm}, as well as its proof
as given in \Cref{sec:paths-main}.
\item The adaptation of this theorem for pushouts as described in \Cref{thm:paths-main-pushout}.
\item The formulation of a possible higher Seifert-van Kampen theorem
as stated in \Cref{thm:paths-higher-SvK}.
\item The formalization of \Cref{sec:paths-main} and \Cref{thm:paths-main-pushout}
in Lean.
\item An adapation of the syntax for higher inductive-inductive types by
\citet{ambrussyntax} to separate sort and point constructors, as described, together with
its semantics, in \Cref{chp:iit}.
\item A syntax of signatures for inductive families as given in \Cref{chp:if}.
\item A formal specification of type erasure, wellformedness relation and eliminator
relation given as syntactic translations as described in \Cref{sec:red-e},
\Cref{sec:red-w}, and \Cref{sec:red-r}.
\item A formal definition of the ``sigma construction'' for an initial algebra
for inductive-inductive types as proposed in \Cref{sec:red-sg}.
\end{itemize}

Parts of this thesis have been peer-reviewed and published already, while
other parts, especially \Cref{chp:if} and \Cref{chp:red} are not yet published
elsewhere.
\begin{itemize}
\item Together with Floris van Doorn and Ulrik Buchholtz, a more
detailed description of our homotopy type theory formalizations was given
in the proceedings of the conference \emph{Interactive Theorem Proving -- 8th International Conference}
in 2017 \citep{leanhott}.
\item In joint work with Nicolai Kraus, the characterization of path spaces
was published in the proceeding of the conference
\emph{Thirty-Fourth Annual ACM/IEEE Symposium on
Logic in Computer Science (LICS)} in 2019 \citep{paths}.
\end{itemize}

\section{Structure of this Thesis}

We will start off this thesis by giving a more detailled exposition of the
background in type theory, which is the basis for the further content.
In this endevour, \Cref{chp:tt} does not only serve to give the necessary background
to readers unfamiliar with dependent types, it also sets notations and terminology
which we will re-use in the later chapters.
In \Cref{sec:tt-provers}, it furthermore gives a short characterization of the interactive
theorem pro\-vers Lean and Agda.

\Cref{chp:hit} will then first introduce some examples of higher inductive types
and will then go on to propose homotopy coequalizers as a fundamental higher
inductive type which can serve as a common generalization of all of these examples.
Sketches for two proofs using the ``encode-decode'' method will be given in
\Cref{sec:hit-encode-decode}.

Following the introduction of homotopy coequalizers we will then (\Cref{chp:paths})
see how we can
characterize their path spaces in order to replace encode-decode proofs.
Apart from proving this characterization (\Cref{sec:paths-main})
we will demonstrate this use in some examples (\Cref{sec:paths-applications}
and \Cref{sec:paths-svk}).

Switching not only the type theoretical setting from homotopy type theory
to set-truncated (or even extensional) type theory
but also the focus of the thesis, the second half will explore types which
instead of higher dimensional structure carry intricate dependencies between
multiple types to be defined.

We will start this second half by first giving a syntax for inductive-inductive
types (\Cref{chp:iit}) including its semantics,
before comparing it to the simpler fragment of inductive families (\Cref{chp:if}).
The latter will be reduced to indexed W-types in \Cref{sec:if-ex}.

After reducing inductive families to indexed W-types we will try to reduce
inductive-inductive types to inductive families:
In \Cref{chp:red} we will give a formal description about how to generate the
inductive families which correspond to
erasing the inductive-inductive typing information,
recovering it with a wellformedness predicate, yielding a candidate for an inital
object in the target type theory.
Then we will present a binary relation which could be used in future to prove
its initiality.










