\section{Examples of Higher Inductive Types}

The typal equality discussed in the last chapter works just fine if we want
to \emph{prove} things to be equal.
But in mathematics as well as computer science, we also often want to
\emph{make things equal}, in the sense that we might want to consider a type
$A$ and two of its elements $a, b : A$ and want to obtain
another type which only differs from $A$ in that $a$ and $b$ are equal.
In short, we want to take \emph{a quotient}.
In this chapter we will present different ways to achieve quotients in the setting
of homotopy type theory.
Afterwards we will show how we can derive all of these from the basic notion
of a \emph{homotopy coequalizer}.

The general way to achieve quotients is to go beyond the inductive types
which we encountered in the last chapters -- and which were all examples of
indexed W-types --
and also allow for constructors which,
in contrast to the \emph{point constructors} which we have seen so far,
are \emph{path constructors}.
Instead of adding new elements to the inductive type,
these constructors are there to make instances of other constructors equal.
In a setting where the K-rule is present, and every type is a set (see Remark~\ref{rmk:tt-uip}),
this bigger class of inductive type is called \textbf{quotient inductive types},
while, as we will see, in homotopy type theory it is more
fitting to call them \textbf{higher inductive types} (HITs) given the fact
that equality types carry not only proofs but data.
Higher inductive types
allow the development of a synthetic version of homotopy theory inside HoTT (cf.~\cite{Buchholtz2018,Buchholtz2018CellularCI,buchholtz2016cayley,BuchRijke_projectiveSpaces,favonia:SvK,licataFinster_Eilenberg,licataBrunerie_s1again,Brunerie2017,rijke:join}).
A main objective of this line of research is to describe, classify, and compare path spaces (i.e.\ equality types) or homotopy groups (i.e.\ truncated path spaces) of higher inductive types such as circles and spheres.

A minimal example for a higher inductive type could be the following definition of would be a
type theoretic representation of the topological space of the interval $I$.
Its constructors are two points, but furthermore also a path which
connects these two points:
\begin{equation*}
\begin{gathered}
%\inferrule*[left=$I$-Form]{ }{I : \UU} \qquad
\inferrule*[left=$I$-Intro1]{ }{0_I : I} \qquad
\inferrule*[left=$I$-Intro2]{ }{1_I : I} \qquad
\inferrule*[left=$I$-Intro3]{ }{\seg : 0_I = 1_I} \\[.7em]
\inferrule*[left=$I$-Elim]
	{C : I \to \UU \\ c_0 : C(0_I) \\ c_1 : C(1_I) \\ p : \seg_*(c_0) = c_1 \\ x : I}
	{\elim_I(C, c_0, c_1, p, x) : C(x)}
\end{gathered}
\end{equation*}
It is easy to check that the unique map $I \to \unit$ is an equivalence,
and so $I$ is contractible and thus a set.
But what if instead of two point constructors we only had one,
as in the higher inductive types goverened by the following rules?
\begin{equation*}
\begin{gathered}
%\inferrule*[left=$\Sph^1$-Form]{ }{\Sph^1 : \UU} \qquad
\inferrule*[left=$\Sph^1$-Intro1]{ }{\Sbase : \Sph^1} \qquad
\inferrule*[left=$\Sph^1$-Intro2]{ }{\Sloop : \Sbase =_{\Sph^1} \Sbase} \\[.7em]
\inferrule*[left=$\Sph^1$-Elim]
	{C : \Sph^1 \to \UU \\ c : C(\Sbase) \\ p : \Sloop_*(c) = c \\ x : \Sph^1}
	{\elim_{\Sph^1}(C,c,p,x) : C(x)}
\end{gathered}
\end{equation*}
We can see that $\Sloop$ introduces a new path from $\Sbase$ to itself which
cannot be reduced to $\refl_\Sbase$.
This means that we can interpret it as a loop which is not homotopic
to the identity, and so the type represents the topological space of the circle.
The fact that $\Sloop$ is not equal to $\refl_\Sbase$ also makes it clear that
this type is not a set.

In the same way in which we can add arbitrary paths between constructors,
we can also use iterated equality types to express the addition
of arbitrary higher dimensional cells (surfaces, volumes).
An example for this is the definition of a twodimensional sphere, where we have
one basepoint and one surface (see Figure~\ref{fig:hit-circle}):
\begin{equation*}
\begin{gathered}
%\inferrule*[left=$\Sph^1$-Form]{ }{\Sph^1 : \UU} \qquad
\inferrule*[left=$\Sph^2$-Intro1]{ }{\Sbase : \Sph^2} \qquad
\inferrule*[left=$\Sph^2$-Intro2]{ }{\Ssurf : \refl_{\Sbase} =_{\Sbase = \Sbase} \refl_{\Sbase}} \\[.7em]
%\inferrule*[left=$\Sph^2$-Elim]{ TODO?
%	{C : \Sph^2 \to \UU \\ c : C(\Sbase) \\ p : \mathsf{}(c) = c \\ x : \Sph^1}
%	{\elim_{\Sph^1}(C,c,p,x) : C(x)}
\end{gathered}
\end{equation*}

\begin{figure}
\centering
\begin{tikzpicture}[x=\diagx, y=\diagx]
 \draw[black] (.5,0) circle (.5);
 \draw[black,fill=black] (0,0) circle (.6ex);
 \node at (-.3,.0) {$\mathsf{base}$};
 \node at (1.2,.15) {$\mathsf{loop}$};
\end{tikzpicture}
\caption{A circle with a base point.}\label{fig:hit-circle}
\end{figure}

Besides these closed examples for higher inductive types, we can also
have important constructions which are parametrized over an arbitrary type.
One important operation in topology, especially in the field of homologies,
is the one of the \textbf{suspension} $\Susp(A)$ of spaces $A$ which turn an $n$-dimensional
type into a $(n+1)$-dimensional type.
Suspensions will also provide a way to conveniently define all higher-dimensional
spheres by setting $\Sph^{n + 1} :\equiv \Susp(\Sph^n)$.
The suspension has two point constructors (sometimes called the north and
south pole) and for each point in $A$ a path between those points:
\begin{equation*}
\begin{gathered}
\inferrule{ }
  {\SuspN : \Susp(A)} \qquad
\inferrule{ }
  {\SuspS : \Susp(A)} \qquad
\inferrule{a : A}
  {\Suspmerid(a) : \SuspN = \SuspS} \\[.7em]
\inferrule{C : \Susp(A) \to \UU \\
  c_\SuspN : C(\SuspN) \\
  c_\SuspS : C(\SuspS) \\
  c_\Suspmerid : (a : A) \to \Suspmerid(a)^*(\SuspN) = \SuspS}
  {\elim_{\Susp(A)}(C, c_\SuspN, c_\SuspS, c_\Suspmerid) : (x : \Susp(A)) \to C(x) }
\end{gathered}
\end{equation*}

Another very general construction, which does not only depend on one a single
type but instead has as input a whole \emph{span} of types, meaning
three types $L$, $M$, and $N$, and functions $f : L \to M$ and $g : L \to N$.
It consists of a type $P \equiv \pout{L}{M}{N}$ the point constructors of which are to functions
$\inl$ and $\inr$ as in the following diagram:
\begin{center}
   \begin{tikzpicture}[x=\diagx,y=-\diagy]
   \node (A) at (0,0) {$L$};
   \node (C) at (1,0) {$N$};
   \node (B) at (0,1) {$M$};
   \node (D) at (1,1) {$P$}; %{$\pout L M N$};
  
   \draw[->] (A) to node [left] {$f$} (B);
   \draw[->] (A) to node [above] {$ g$} (C);
   \draw[->, dashed] (B) to node [below] {$\inl$} (D);
   \draw[->, dashed] (C) to node [right] {$\inr$} (D);
  \end{tikzpicture}
\end{center}
The introduction rules are the same as for the sum type, but each instance  of $L$
contributes a new path in the resulting type:
\begin{equation*}
\begin{gathered}
\inferrule{m : M}
  {\inl(m) : \pout{L}{M}{N}} \qquad
\inferrule{n : N}
  {\inr(n) : \pout{L}{M}{N}} \\[.7em]
\inferrule{l : L}
  {\glue(l) : \inl(f(l)) = \inr(g(l)) }
\end{gathered}
\end{equation*}
One can try to visualize the resulting type as the sum of $M$ and $N$ which
was glued along an $L$-shaped overlapping.
The pushout is a very general construction.
In fact, it is easy to check that all the previous higher inductive types we
presented were just special cases of a pushout,
for example the suspension $\Susp(A)$ can also be defined
as the pushout where both $f$ and $g$ are the unique map $A \to \unit$.

An example for a higher inductive types, the importance of which is easily recognizable
even we only care for propositions and sets is the one of the \textbf{truncation operator}.
It is a tool to bring types to the desired level of trunction,
as we can see in the following example:
Remember that we represented the disjunctive logical connective with
the sum type.
But the sum of two propositional types is not necessarily a proposition itself:
For example we have $\unit + \unit \simeq \twotype$.
So how can we make it a proposition?
The solution is to add equations between \emph{all} the points in the sum type
and additionaly make sure that no trivial higher equality proofs exist.
Similar situations can obviously happen for every truncation level.
The elimination principle for this operation has to make sure that only
functions on the truncated type, which represent continuous maps, can
be created, and so it has to require that the fibers of the type family which
we want to inhabit are truncated:
\begin{equation*}
\begin{gathered}
\inferrule*[left=Trunc-Intro]{a : A}
  {\tproj{n}{a}{} : \trunc{n}{A}} \\[.7em]
\inferrule*[left=Trunc-Elim]{C : \trunc{n}{A} \to \UU \\
  p : (x : \trunc{n}{A}) \to \isntype{n}{(A)} \\
  q : (a : A) \to C(\tproj{n}{a}{}) }
  {\elim_{\trunc{n}{A}}(C,p,q) : (x : \trunc{n}{A}) \to C(x) }
\end{gathered}
\end{equation*}

Next, we will discuss how -- parallel to how we unified the examples for
indexed inductive types using indexed W-types --
find a general way to express all higher inductive types which we need.

\section{Coequalizers as a Fundamental HIT}

For a long time after homotopy type theory became its own field,
the question about what could be a suitable syntax and sematics for
higher inductive types remained open.
To make up for the lack of core support for higher inductive types
in interactive theorem provers, users of Coq, Agda, and Lean decided to come
up with pragmatic definitions and implementations.
In Agda and Coq, the common solution was to apply a trick first documented by
\cite{licatatrick} which uses \emph{private inductive types} to
hide the elimination principle generated by the prover and replace it by
a manually defined one.
Another strategy which the homotopy type theory formalizations in Lean 2 are
based on, and which are described by \cite{leanhott},
is to find a single higher inductive types which can serve as a \emph{universal}
example which can be used to derive a big range of other instances.
This is similar to the generalized type of trees which make up indexed W-types.

The type we use as such a fundamental higher inductive type is much simpler
than the definition of indexed W-types, and it is a straightforward
generalization of \textbf{quotients} in set-based type theory.
Suppose that we have a type $A : \UU$ and and equivalence relation
$\sim : A \to A \to \UU$.
For a type to be justifiably called the quotient of $A$ by $\sim$,
we want a projection map from $A$ into the quotient and we need to make
sure that any $a, b : A$ with $a \sim b$ are projected to equal elements
of the quotient.
An elimination principle should say that maps out of the quotient are
the same as makes out of $A$ respecting the relation $\sim$.
And these are exactly the features of the type of \textbf{homotopy coequalizers}
which we will now introduce.
The reason why we refrain from calling them quotients
-- even though \cite{leanhott} call them \emph{typal} quotients --
will become apparent in the examples we will study after giving the inference rules
and is due to the fact that we differ from several common assumptions about
quotients:
\begin{itemize}
\item The base type $A : \UU$ does not need to be a set,
\item the relation $\sim : A \to A \to \UU$ does not need to be an equivalence relation, and
\item it does not have to be a relation at all, because for $a, b : A$, the type of relatedness witnesses $a \sim b : \UU$
does not need to be a proposition but can contain multiple elements and even
structure in higher equalities.
(It might be reasonable to speak of a quotient by a higher relation,
which is the approach taken by
\cite{boulierRijkeTab_higherRels}.)
\end{itemize}
The formation, introduction, and elimination rules for the coequalizer
are the following:
\begin{equation*}
\begin{gathered}
\inferrule*[left=Coeq-Form]{A : \UU \\ \_\sim\_ : A \to A \to \UU}
  {\quot : \UU}\\[.7em]
\inferrule*[left=Coeq-Intro1]{a : A}
  {[a] : \quot} \qquad
\inferrule*[left=Coeq-Intro2]{a, b : A \\ s : a \sim b}
  {\glue(s) : [a] = [b] }\\[.7em]
\inferrule*[left=Coeq-Elim]{P : \quot \to \UU \\
  f : (a : A) \to P([a]) \\
  e : \{a, b : A\} (s : a \sim b) \to \glue(s)^*(f(a)) = f(b)  }
  {\elim_\quot(P,f,e) : (a : \quot) \to P(x)}
\end{gathered}
\end{equation*}

\begin{remark}[Pathovers]
Since we will often encounter equality types with a transported term on one
side, it merits its own name and notation:
Whenever we have a type family $B : A \to \UU$, two points $a, a' : A$,
and a path $p : a = a'$, we will for $b : B(a)$ and $b' : B(a')$ define the
type of
\textbf{paths over} or \textbf{dependent paths over} $p$ to be
\begin{equation*}
(b =_p b') :\equiv (p*(b) = b') \text{.}
\end{equation*}
These path-overs have proven to be useful for many formalizations and their
attributes, as well as their generalizations to square-overs and
cube-overs have been described by \cite{licatacubical}.
\end{remark}

\begin{remark}[Coequalizer of Functions]
In category theory, coequalizers are not defined on an object and a relation
but instead are a universal construction on two functions $f, g : X \to A$
which can be thought as an object in which for each $x : X$, $f(x)$ and $g(x)$ are identified.
This can be expressed in our variant of a coequalizers by setting the relation
on $A$ to be
\begin{equation*}
a \sim b :\equiv (x : X) \times (f(x) = a) \times (g(x) = b) \text{,}
\end{equation*}
or, alternatively by defining $\_\sim\_$ inductively by $f(x) \sim g(x)$ for each
$x : X$.
It is then easy to see that the coequalizer of $f$ and $g$ is $\quot$.

Vice versa, we can view $\quot$ for an arbitrary $A : \UU$ and
$\sim : A \to A \to \UU$ as the coequalizer on the maps
\begin{equation*}
 \pr_1,\pr_2 : \left(\Sigma(a,b : A).a \sim b\right) \rightrightarrows A \text{.}
\end{equation*}
\end{remark}

Another reason why we might avoid to conflate homotopy coequalizers with
ordinary quotients is, that $\quot$ is not always a set, even if $A$ is:
We can express the circle which we have defined earlier as a coequalizer
in the following simple way:
Set $A :\equiv \unit$ and for $a , b : A$, set $a \sim b :\equiv \unit$.
While this, at first glance, might seem to yield a trivial coequalizer,
it actually add an \emph{additional} path to the unit type, and
it is easy to check that $\quot \simeq \Sph^1$.

To see the importance of not requiring the relation to be propositional,
consider the case of the pushout.
It can be written as a coequalizer on $A :\equiv M + N$, but it is easy that the
following definition for $\_\sim\_$ is not necessarily a proposition,
and in general it is neither reflexive, nor symmetric, nor transitive:
\begin{align*}
(\inl(m) \sim \inl(m')) &:\equiv \emptytype \text{,} \\
(\inl(m) \sim \inr(n)) &:\equiv (l : L) \times (f(l) = m) \times (g(l) = n) \text{, and} \\
(\inr(m) \sim x) &:\equiv  \emptytype \text{.}
\end{align*}

With the pushouts (and sequential colimits which are defined similarly),
and given that there are more intricate constructions giving
propositional truncations \citep{floris_proptrunc,kraustrunc} and higher truncations~\citep{rijke:join},
we have presented ways to encode all the higher inductive types
which we have seen so far as coequalizers.

\begin{remark}[Coequalizers as Pushouts]
Basing all necessary higher inductive type is a somewhat arbitrary decision
based on aesthetical considerations, since we can also derive coequalizers
from pushouts in the following way:
\begin{center}
   \begin{tikzpicture}[x=\diagx,y=-\diagy]
   \node (A) at (0,0) {$A + (a, b : A) \times a \sim b$};
   \node (C) at (2,0) {$A$};
   \node (B) at (0,1) {$A$}; %TODO
   \node (D) at (2,1) {$P$}; %{$\pout L M N$};
  
   \draw[->] (A) to node [left] {$f$} (B);
   \draw[->] (A) to node [above] {$ g$} (C);
   \draw[->, dashed] (B) to node [below] {$\inl$} (D);
   \draw[->, dashed] (C) to node [right] {$\inr$} (D);
  \end{tikzpicture}
\end{center}
with
\begin{align*}
f(\inl(a)) &:\equiv a \text{,} \\
f(\inr(a, b, s)) &:\equiv a \text{,} \\
g(\inl(a)) &:\equiv a \text{, and} \\
g(\inr(a, b, s)) &:\equiv b \text{.}
\end{align*}
It is then easy to check that the pushout $P$ is equivalent to $\quot$.
As a result, we can conclude that pushouts and homotopy coequalizers
are \emph{interderivable} in homotopy type theory.
\end{remark}

\section{Encode-Decode Proofs}

To motivate our results, which we will present in the next chapter,
will will now look at some important problems which occur when
we want to prove facts about higher inductive types.
Often, we want to find out what specific equality types of
higher inductive types look like.
For a very concrete example, one of the most basic results in homotopy theory
is the calculation that \emph{the fundamental group of the circle is
isomorphic to the group $\Z$ of integers}.
It is an example of a proplem which in a classical, set theoretic setting,
is solved in a very way which is very different from the approach which is needed
in its type theoretic counterpart.
\cite{licataShulman_circle} were the first to translate this fact into the
setting of homotopy type theory.
Another fundamental theorem in homotopy theory is the \emph{theorem of
Seifert-van Kampen}.
It provides proof that the fundamental groupoid of a pushout of two spaces is
isomorphic to the pushout of the fundamental groupoid of these spaces.
This is an important help when trying to calculate homotopy groups of spaces
which were ``glued'' together from more primitive spaces using pushouts.
\cite{favonia:SvK} documented and formalized in Agda a type theoretic
equivalent of the Seifert-van Kampen thoerem.
In this chapter, we will revisit the proof ideas for both of these results
and discover, what they have in common.

The intuitive way to compare the integers with the loop space
$\Omega(\Sph^1)$ of the sphere is the following:
Each loop can be reached by following the constructor $\Sloop$ a number of
times either in its actual direction or by reversing the direction, i.\,e.\
concatenating with $\Sloop \inv$.
Following an inverted loop after a non-inverted one or vice versa cancels out since
we have
\begin{equation*}
\Sloop \ct \Sloop \inv = \refl = \Sloop \inv \ct \Sloop \text{.}
\end{equation*}
The integer corresponding to a loop would then be the ``exponent'' of the number
of times we concatenate $\Sloop$ (with $\Sloop^0 = \refl$).
But translating this intuition into a formal proof, especially in order
to define a function from the loop space to the integers, requires some tricks.
\begin{thm}[Loop Space of the Circle, \cite{licataShulman_circle}]\label{thm:hit-s1}
The loop space of the circle is equivalent to the
type of the integers: $\Omega(\Sph^1, \Sbase) \simeq \Z$
\end{thm}

\begin{proof}
Instead of directly defining a map $\Omega(\Sph^1, \Sbase) \to \Z$,
we perform the type theoretic equivalent of constructing the
\emph{universal cover} for the circle.
Over each point of the circle, we will have a whole type, i.\,e. a type family
$\code : \Sph^1 \to \UU$
and by the end we will be able to prove that this type family satisfies the
following equivalences:
\begin{equation}\label{eq:hit1}
(\Sbase = x) \simeq \code(x) \simeq \Z \text{,}
\end{equation}
after which we can get the desired result by plugging in $x \equiv \Sbase$.
The definition of this type family is the following:
\begin{equation*}
\code(x) :\equiv \elim_{\Sph^1}(\lambda \_.\UU, \Z, \ua(S)) \text{,}
\end{equation*}
with $S : \Z \to \Z$ being the successor function on the integers, which
is obviously an equivalence.
The definition means that we will, by the reduction rules,
have $\code(\Sbase) \equiv \Z$ and
$\ap_\code(\Sloop) = \ua(S)$.
We can then calculate that
\begin{equation}\label{eq:hit2}
\Sloop^*(z) \equiv (\ap_\code(\Sloop))^*(z) \equiv (\ua(S))^*(z) \equiv S(z) \text{,}
\end{equation}
and that, by the same reasoning, $(\Sloop\inv)^*(z) = P(z)$ where
$P = S\inv$ is the predecessor function on $\Z$.
After we gave the definition of $\code$, the second equivalence of \eqref{eq:hit1}
is immediate by induction on $x$, while the first one is more involved.
We call the map into $\code(x)$, which we have to define,
the \emph{encoding} map and the inverse the \emph{decoding}.

The encoding works by, for each path, transporting along this type in the type
family $\code$, starting from zero:
\begin{align*}
\mathsf{encode} &: \{x : \Sph^1\}(p : \Sbase = x) \to \code(x) \\
\mathsf{encode}(p) &:\equiv p^*(0)
\end{align*}
This is the same as defining $\mathsf{encode}$ as $0$ induction on $p$.
It is easy to check via \eqref{eq:hit2} that $\mathsf{encode}(\Sloop)$
calculates to $S(z)$, $\mathsf{encode}(\Sloop \ct \Sloop)$ to
$S(S(z))$ etc.

Defining the decoding function goes by induction on the circle element
$x$ involved.
For the induction we first have to give the base case
$\mathsf{decode}(\Sbase) : \code(\Sbase) \to (\Sbase = \Sbase)$ which we set to the function which is
constantly $\Sloop$.
We then have to prove that
\begin{equation*}
\Sloop^*(\lambda \_.\, \Sloop) = (\lambda\_.\, \Sloop) \text{,}
\end{equation*}
where the transport is over the type family $\lambda x'.\, code(x') \to (\Sbase = x')$,
a calculation we don't want detail here. This defines $\mathsf{decode}$.
It is then left to prove by induction that
\begin{align*}
\mathsf{encode}(\mathsf{decode})(c) &= c \text{ for all $c : \code(x)$ and} \\
\mathsf{decode}(\mathsf{encode})(p) &= p \text{ for all $x : \Sph^1$ and $p : \Sbase = x$.}
\end{align*}
completing the proof of the theorem.
\end{proof}

The proof for the Seifert-van Kampen theorem employs the same method of
providing a type family acting as a universal cover.
It can be seen as a generalization of the above theorem.
Recall that in a classical topological setting the statement of the
theorem is the following:
\begin{quote}
The fundamental groupoid of the pushout of two spaces is equivalent to the pushout of
its fundamental groupoids.
\end{quote}
So while the for the theorem about the circle the ``right-hand side'' of the
equivalence was rather clearly defined, here the
question is whether we find a good definition of the pushout of groupoids.
Since we attempt another example of an encode-decode proof, we
might as well first define the morphisms of the pushout of groupoids as the
code itself.
Intuitively, the morphisms the pushout $P :\equiv \pout{L}{M}{N}$
are ``zig-zags'' consisting of morphisms
alternatingly in ${M}$ and ${N}$, separated by morphisms in $L$.
\begin{defn}[Codes for Groupoid Pushouts, \cite{favonia:SvK}]
We define a type family $\code : P \to P \to \UU$, by recursion on both arguments.
That means that it suffices to give families $M \to M \to \UU$,
$M \to N \to \UU$, $N \to M \to \UU$, and $N \to N \to \UU$ that agree on
$f(l)$ and $g(l)$ for all $l : L$ in either argument.
We will omit proof of this agreement.
For the first case, i.\,e.\ if we have $m, m' : N$, we want $\code(\inr(m), \inr(m'))$ to be
the inductive type where elements are characterized as lists
\begin{equation*}
 (p_0, x_1, q_1, y_1, p_1, x_2, q_2, y_2, p_2, \ldots, y_k, p_k) \text{,}
\end{equation*}
the entries of which have the following type for each $0 < i \leq k$: (cf. \Cref{fig:hit-svk-zigzag}):
\begin{align*}
x_i, y_i &: L \\
p_0 &: \trunc{0}{m = f(x_1)} \\
p_i &: \begin{cases} \trunc{0}{f(y_i) = f(x_{k + 1})} & \text{for $i < k$} \\
  \trunc{0}{f(y_k) = m'} & \text{for $i = k$}  \end{cases} \\
q_i &: \trunc{0}{g(x_k) = g(y_k)} \text{.}
\end{align*}
These lists have to be quotiented by the path constructors
\begin{align*}
(\ldots, q_i, y_i, \refl_{f(y_i)}, y_i, q_{i+1}, \ldots)
  &= (\ldots, q_i \ct q_{i+1}, \ldots) \text{ and} \\
(\ldots, p_i, x_i, \refl_{g(x_i)}, x_i, p_{i+1}, \ldots)
  &= (\ldots, p_i \ct p_{i+1}, \ldots) \text{,}
\end{align*}
and the resulting type is made a set by the means of truncation.
The type families for the three remaining cases are likewise defined
``zig-zags'' in the form of quotiented lists, where the parity of the length
is changed and/or the position of $p_i$ and $q_i$ are swapped.
\end{defn}

\begin{figure}
 \centering
  \begin{tikzpicture}[x=\diagx,y=-\diagy]
   \node (A) at (0,0) {$n_0$};
   \node (B) at (1,0) {$g(k_1)$};
   \node (C) at (1,1) {$f(k_1)$};
   \node (H) at (4,1) {$f(l_2)$};
   \node (I) at (4,0) {$g(l_2)$};
   \node (J) at (5,0) {$n$};
  
   \draw[->] (A) to node [above] {$p_0$} (B);
   \draw[->] (B) to node {} (C);
   \draw[->] (C) to node [above] {$q_1 \ct q_2$} (H);
   \draw[->] (H) to node {} (I);
   \draw[->] (I) to node [above] {$p_2$} (J);
  \end{tikzpicture}
\caption{``Zig-zag paths'' in the pushout codes.}\label{fig:hit-svk-zigzag}
\end{figure}

With this definition, the statement of the theorem can be stated as the following:
\begin{thm}[Seifert-van Kampen, \cite{favonia:SvK}]\label{thm:paths-svk}
For each $x, y : P$ we have
\begin{equation*}
\trunc{0}{x = y} \equiv \code(x, y) \text{.}
\end{equation*}
\end{thm}












