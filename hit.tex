\section{Examples of Higher Inductive Types}

The typal equality discussed in the last chapter works just fine if we want
to \emph{prove} things to be equal.
But in mathematics as well as computer science, we also often want to
\emph{make things equal}, in the sense that we might want to consider a type
$A$ and two of its elements $a, b : A$ and want to obtain
another type which only differs from $A$ in that $a$ and $b$ are equal.
In short, we want to take \emph{a quotient}.
In this chapter we will present different ways to achieve quotients in the setting
of homotopy type theory.
Afterwards we will show how we can derive all of these from the basic notion
of a \emph{homotopy coequalizer}.

The general way to achieve quotients is to go beyond the inductive types
which we encountered in the last chapters -- and which were all examples of
indexed W-types --
and also allow for constructors which,
in contrast to the \emph{point constructors} which we have seen so far,
are \emph{path constructors}.
Instead of adding new elements to the inductive type,
these constructors are there to make instances of other constructors equal.
In a setting where the K-rule is present, and every type is a set (see Remark~\ref{rmk:tt-uip}),
this bigger class of inductive type is called \textbf{quotient inductive types},
while, as we will see, in homotopy type theory it is more
fitting to call them \textbf{higher inductive types} given the fact
that equality types carry not only proofs but data.

A minimal example for this could be the following definition of would be a
type theoretic representation of the topological space of the interval $I$.
Its constructors are two points, but furthermore also a path which
connects these two points:
\begin{equation*}
\begin{gathered}
%\inferrule*[left=$I$-Form]{ }{I : \UU} \qquad
\inferrule*[left=$I$-Intro1]{ }{0_I : I} \qquad
\inferrule*[left=$I$-Intro2]{ }{1_I : I} \qquad
\inferrule*[left=$I$-Intro3]{ }{\seg : 0_I = 1_I} \\[.7em]
\inferrule*[left=$I$-Elim]
	{C : I \to \UU \\ c_0 : C(0_I) \\ c_1 : C(1_I) \\ p : \seg_*(c_0) = c_1 \\ x : I}
	{\elim_I(C, c_0, c_1, p, x) : C(x)}
\end{gathered}
\end{equation*}
It is easy to check that the unique map $I \to \unit$ is an equivalence,
and so $I$ is contractible and thus a set.
But what if instead of two point constructors we only had one,
as in the higher inductive types goverened by the following rules?
\begin{equation*}
\begin{gathered}
%\inferrule*[left=$\Sph^1$-Form]{ }{\Sph^1 : \UU} \qquad
\inferrule*[left=$\Sph^1$-Intro1]{ }{\Sbase : \Sph^1} \qquad
\inferrule*[left=$\Sph^1$-Intro2]{ }{\Sloop : \Sbase =_{\Sph^1} \Sbase} \\[.7em]
\inferrule*[left=$\Sph^1$-Elim]
	{C : \Sph^1 \to \UU \\ c : C(\Sbase) \\ p : \Sloop_*(c) = c \\ x : \Sph^1}
	{\elim_{\Sph^1}(C,c,p,x) : C(x)}
\end{gathered}
\end{equation*}
We can see that $\Sloop$ introduces a new path from $\Sbase$ to itself which
cannot be reduced to $\refl_\Sbase$.
This means that we can interpret it as a loop which is not homotopic
to the identity, and so the type represents the topological space of the circle.
The fact that $\Sloop$ is not equal to $\refl_\Sbase$ also makes it clear that
this type is not a set.

In the same way in which we can add arbitrary paths between constructors,
we can also use iterated equality types to express the addition
of arbitrary higher dimensional cells (surfaces, volumes).
An example for this is the definition of a twodimensional sphere, where we have
one basepoint and one surface:
\begin{equation*}
\begin{gathered}
%\inferrule*[left=$\Sph^1$-Form]{ }{\Sph^1 : \UU} \qquad
\inferrule*[left=$\Sph^2$-Intro1]{ }{\Sbase : \Sph^2} \qquad
\inferrule*[left=$\Sph^2$-Intro2]{ }{\Ssurf : \refl_{\Sbase} =_{\Sbase = \Sbase} \refl_{\Sbase}} \\[.7em]
%\inferrule*[left=$\Sph^2$-Elim]{ TODO?
%	{C : \Sph^2 \to \UU \\ c : C(\Sbase) \\ p : \mathsf{}(c) = c \\ x : \Sph^1}
%	{\elim_{\Sph^1}(C,c,p,x) : C(x)}
\end{gathered}
\end{equation*}

Besides these closed examples for higher inductive types, we can also
have important constructions which are parametrized over an arbitrary type.
One important operation in topology, especially in the field of homologies,
is the one of the \textbf{suspension} $\Susp(A)$ of spaces $A$ which turn an $n$-dimensional
type into a $(n+1)$-dimensional type.
Suspensions will also provide a way to conveniently define all higher-dimensional
spheres by setting $\Sph^{n + 1} :\equiv \Susp(\Sph^n)$.
The suspension has two point constructors (sometimes called the north and
south pole) and for each point in $A$ a path between those points:
\begin{equation*}
\begin{gathered}
\inferrule{ }
  {\SuspN : \Susp(A)} \qquad
\inferrule{ }
  {\SuspS : \Susp(A)} \qquad
\inferrule{a : A}
  {\Suspmerid(a) : \SuspN = \SuspS} \\[.7em]
\inferrule{C : \Susp(A) \to \UU \\
  c_\SuspN : C(\SuspN) \\
  c_\SuspS : C(\SuspS) \\
  c_\Suspmerid : (a : A) \to \Suspmerid(a)^*(\SuspN) = \SuspS}
  {\elim_{\Susp(A)}(C, c_\SuspN, c_\SuspS, c_\Suspmerid) : (x : \Susp(A)) \to C(x) }
\end{gathered}
\end{equation*}

Another very general construction, which does not only depend on one a single
type but instead has as input a whole \emph{span} of types, meaning
three types $L$, $M$, and $N$, and functions $f : L \to M$ and $g : L \to N$.
It consists of a type $P \equiv \pout{L}{M}{N}$ the point constructors of which are to functions
$\inl$ and $\inr$ as in the following diagram:
\begin{center}
   \begin{tikzpicture}[x=\diagx,y=-\diagy]
   \node (A) at (0,0) {$L$};
   \node (C) at (1,0) {$N$};
   \node (B) at (0,1) {$M$};
   \node (D) at (1,1) {$P$}; %{$\pout L M N$};
  
   \draw[->] (A) to node [left] {$f$} (B);
   \draw[->] (A) to node [above] {$ g$} (C);
   \draw[->, dashed] (B) to node [below] {$\inl$} (D);
   \draw[->, dashed] (C) to node [right] {$\inr$} (D);
  \end{tikzpicture}
\end{center}
The introduction rules are the same as for the sum type, but each instance  of $L$
contributes a new path in the resulting type:
\begin{equation*}
\begin{gathered}
\inferrule{m : M}
  {\inl(m) : \pout{L}{M}{N}} \qquad
\inferrule{n : N}
  {\inr(n) : \pout{L}{M}{N}} \\[.7em]
\inferrule{l : L}
  {\glue(l) : \inl(f(l)) = \inr(g(l)) }
\end{gathered}
\end{equation*}
One can try to visualize the resulting type as the sum of $M$ and $N$ which
was glued along an $L$-shaped overlapping.
The pushout is a very general construction.
In fact, it is easy to check that all the previous higher inductive types we
presented were just special cases of a pushout,
for example the suspension $\Susp(A)$ can also be defined
as the pushout where both $f$ and $g$ are the unique map $A \to \unit$.

\section{Coequalizers as a Fundamental HIT}










