This chapter shall serve to introduce the basic notions of type theory which
we will need for the subsequent content of this thesis.
At first (Chapter~\ref{sec:tt-dtt}), we will have a general look on dependent type theory, its use and
how it differs from a set theoretic foundation,
then (Chapter~\ref{sec:tt-hott}) we will explain how homotopy type theory was created to have a suitable
language to reason about higher equalities and provides a synthetic way to
formalize topological insights.
Finally, in Chapter~\ref{sec:tt-provers}, we will give examples of two theorem
provers based on dependent types, Agda and Lean, and point out some of their differences.

\section{Dependent Type Theory}\label{sec:tt-dtt}

While the term ``type theory'' stems from the early nineteenth centry, when
Bertrand Russell sought to lay out an alternative to set theory which did
suffer from the paradox which Russell discovered,
todays versions of type theory have little in common with Russell's attempts
but rather rely on the considerations of Per Martin-Löf (\cite{martin-lof2, martin-lof1})
who, starting in the 1970's, built a new mathematical foundation based on the
$\lambda$-calculus, himself drawing inspiration from previous logicians and
mathematicians like Alonzo Church and Haskell Curry.
Often, dependent type theory is also referred to as Martin-Löf type theory.

Based on a phenomenon knows as the ``Curry-Howard correspondence'', type theory
can serve both as a theoretical foundation for a formal representation of mathematics,
as well as the principle for  the specification for strongly typed functional
programming languages.
It was implemented in computer languages for programming and theorem proving which
are used massively in the field of formal verification and in the formalization of
mathematics.
Among the most commonly used implementations are the theorem prover Coq (\cite{coq})
which notably has a lot of users in the field of hardware verification,
the prover Agda (\cite{agda}), which is popular amongst type theorists themselves,
and the Microsoft Research based project Lean(\cite{mouracade}), which has
drawn considerable attention from researching mathematicians as a tool to
formally verify their proofs.

Type theory differs from a set theoretic mathematical foundation (let us, as a
point of reference, consider set theories based on (higher-order) propositional logic
like Zermelo-Fraenkel set theory) in several
important aspects:
\begin{itemize}
\item Type theory follows a paradigm called \emph{``propositions-as-types''}.
This means that statements like theorems and conjectures are represented
using the same class of objects as other data like sets or (algebraic) structures.
In contrast to this, most set theoretic foundations are built on a \emph{dichotomy between
the propositions and the objects} they describe:
They first start out with a logical frame work on which axiomatically a theory
of sets is introduced.
The coherence between these two levels must then be created using an axiom like
the excision axiom in Zermelo-Fraenkel set theory.
\item Type theory is \emph{typed} while set theory is \emph{untyped}.
While in set theory, objects can be an element of different sets --
consider the number two which is an element both of the set of even integers
as well as the set of all integers --
type theory is based on the principle that every piece of data (every term)
is assigned a unique type which is known at the point of the creation of the data.
Since this assignment, called typing, is decidable, we consider it a judgment
rather than a provable proposition that a given term $t$ has type $A$.
\end{itemize}


\section{Homotopy Type Theory}\label{sec:tt-hott}

\section{Theorem Provers Based on Type Theory}\label{sec:tt-provers}








