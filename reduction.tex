\section{(OLD) -- Fragments of Inductive-Inductive Types}

As we have seen in the previous sections, inductive-inductive types as specified
allow for a very broad variety of definitions.
We will now see that it is easy to carve out different subsets of specifications
to obtain more restrictive fragments of inductive types.
Starting from the largest of these subsets, we will first see that there is a
straightforward way to restrict inductive-inductive types to those whose constructors
are finitary in the sense that no point constructor depends on an infinite
amount of data: %TODO improve that last sentence

\begin{defn}[Finitary IITs]
Given a specification \grm{\Gamma} for an inductive-inductive types we say that
it is \textbf{finitary} if the infinitary $\Pi$-type is not used.
\end{defn}

One example of a specification which does not meet this requirement are the
infinitely branching trees. %TODO cite example

Instead of only preventing the use of external data in ``small functions''

We want to reduce inductive-inductive types to inductive families.
This means, we postulate that inductive families be admissible in our target
type theory and show that this implies the existence of all inductive-inductive
types.
Since we want to reuse the way of specifying inductive-inductive types to specify
instances which are as well inductive families, we want to rediscover the specifications
of inductive families as a subset of all inductive-inductive specifications:

\begin{defn}[Inductive families]
A context \grm{\Gamma} is said to specify a \textbf{inductive family} if it is
generated without using the inductive function type in the specification of sorts
and thus, no sorts depend on other sorts but are only iterated function depending
on external types.
This means that the formation rule for inductive function types is restricted
to the case where \grm{k \equiv \Pc}.
\end{defn}

We assume for the remainder of this chapter, that if \grm{\Gamma} specifies an
inductive family, we are provided with $\con{\Gamma} : \grm{\Gamma}^\CC$ and
$\elim_\grm{\Gamma} : \grm{\Gamma}^\EE(\con{\Gamma}, m)$ for each
$m : \grm{\Gamma}^\MM$.

TODO: explain reduction to W-types maybe

\section{Type Erasure}

As seen in the examples, the first step to prove the reducability is to formally
define the operation which we will call \emph{flattening} or -- inspired by
the syntax example -- \emph{type erasure}.
This operation strips away any dependencies between the sorts of a signature
as well as all external indices to sorts.
The operation should take arbitrary inductive-inductive signatures (contexts) and
return signatures for inductive families.
Let us look at what type erasure should do with our running examples:

\begin{example}[Natural Numbers]\label{ex:red-e-nat}
Since the inductive-inductive signature of the \emph{natural numbers}~\ref{ex:ii-syntax-nat} doesn't
contain any indexed sorts, type erasure should ``do nothing'' with it.
That is, returning the sort context and point context of the inductive family
syntax which looks like a obvious correspondence to it (cf. Example~\ref{ex:if-natvec})
while ignoring all entries of the other kind:
Let
\begin{equation*}
\grm{\Gamma_{nat}} 
  :\equiv \grm{(\cdot,\, \UU,\, \El(\vz),\, \Pi\left(\vs(\vz),\, \El(\vs(\vs(\vz)))\right))}
  \text{.}
\end{equation*}
We want to have the following split into sort types and point types:
\begin{align*}
\tqm{\grm{\Gamma_{nat}}^\EE_\Sc}
 &= \tqm{(\cdot_\Sc,\, \UU)} \text{ and} \\
\tqm{\grm{\Gamma_{nat}}^\EE}
 &= \tqm{(\cdot,\, \El(\var(\vz)),\, \var(\vz) \Rightarrow_\Pc \El(\var(\vz)))} \text{.}
\end{align*}
\end{example}

\begin{example}[Vectors]\label{ex:red-e-vec}
In the example of vectors \ref{ex:ii-syntax-vec} we need to erase the natural numbers
index of the only sort under consideration:
\begin{align*}
\tqm{\grm{\Gamma_{vec}}^\EE_\Sc}
 &= \tqm{(\cdot_\Sc,\, \UU)} \text{ and} \\
\tqm{\grm{\Gamma_{vec}}^\EE}
  &= \tqm{(\cdot,\, \El(\var(\vz)),\, 
    \ExtPiP{A}{\blm{\lambda a.\,}\ExtPiP{\N}{\blm{\lambda n.\,}
    \var(\vz) \Rightarrow_\Pc \El(\var(\vz))}})} \text{.}
\end{align*}
Note that the erasure of the vectors does not coincide with the vectors represented
as an inductive family (Example~\ref{ex:if-natvec}), because its sort lacks the
indexing over the natural numbers.
In fact, it's easy to see that the algebras of this signature would no be isomorphic
to the type of lists over the type \blm{A \times \N}.
\end{example}

\begin{example}[Type Theory Syntax]
In our syntax we will now see why the operation is called ``type erasure'':
%TODO
\end{example}

To go from examples to the general case, we will present the different components
of the type erasure operation in roughly the same order in which they appear in
Section~\ref{sec:ii-syntax}, most often needing to distinguish between sort
and point constructors.

\begin{defn}[Type Erasure]
First of all, each context will need to be split into a sort context and a point
context:
\begin{equation*}
\inferrule{\grm{\vdash \Gamma}}
  {\tqm{\SCon \grm{\Gamma}^\EE_\Sc}}
\qquad
\inferrule{\grm{\vdash \Gamma}}
  {\tqm{\vdash_{\grm{\Gamma}^\EE_\Sc} \grm{\Gamma}^\EE }}
\end{equation*}
To descent down the components of the contexts, we will need to define the operation
on types as well.
Since we are erasing all information from the sorts, we will only need this for
point types, though.
Unsurprisingly, we want them to be translated to point types in the appropriate
sort context:
\begin{equation*}
\inferrule{\grm{\Gamma \vdash A :: \Pc}}
  {\tqm{\grm{\Gamma}^\EE_\Sc \SCon \grm{A}^\EE :: \Pc}}
\end{equation*}
Using this we will be able to define the operation creating sort contexts by
\begin{align*}
\tqm{\grm{\cdot}^\EE_\Sc}
  &:\equiv\tqm{\cdot_\Sc} \text{,} \\
\tqm{\grm{(\Gamma,\, B)}^\EE_\Sc}
  &:\equiv \tqm{\left(\grm{\Gamma}^\EE_\Sc,\, \grm{\UU}^\EE_\Sc\right)} \text{ for \grm{B :: \Sc}, and} \\
\tqm{\grm{(\Gamma,\, A)}^\EE_\Sc}
  &:\equiv \tqm{\grm{\Gamma}^\EE_\Sc} \text{ for \grm{A :: \Pc}.}
\end{align*}
The generated point context over this sort context has to be extended in the case
where the input is an extension by a point type.
In the case where it is an extension by a sort type, we want to return the
unextended context, but to make up for the definition above, we need to weaken
to account for the extension of the resulting sort context:
\begin{align*}
\tqm{\grm{\cdot}^\EE}
  &:\equiv\tqm{\cdot} \text{,} \\
\tqm{\grm{(\Gamma,\, B)}^\EE}
  &:\equiv \tqm{\grm{\Gamma}^\EE[\wk_{\id}]} \text{ for \grm{B :: \Sc}, and} \\
\tqm{\grm{(\Gamma,\, A)}^\EE}
  &:\equiv \tqm{\left(\grm{\Gamma}^\EE,\, \grm{A}^\EE\right)} \text{ for \grm{A :: \Pc}.}
\end{align*}
So how do we define \tqm{\grm{A}^\EE} for a point type \grm{A}?
The fact the we have to recurse on \grm{\El(a)} makes it clear that we will have
to extend our operation to terms of sort types at least.
That is, together with \tqm{\grm{A}^\EE} we also need the following:
\begin{equation*}
\inferrule{\grm{\Gamma \vdash t : B :: \Sc}}
  {\tqm{\grm{\Gamma}^\EE_\Sc \SCon \grm{t}^\EE : \UU}}
\end{equation*}
And indeed, with this we can set
\begin{align*}
\tqm{\grm{\El(a)}^\EE}
  &:\equiv \tqm{\El(\grm{a}^\EE)} \text{.}
\end{align*}
For recursive $\Pi$-types, we need only care about the ones yielding point types.
Note that the operation turns a $\Pi$-type into a non-dependent function type!
\begin{align*}
\tqm{\grm{\Pi(a, A)}^\EE}
  &:\equiv \tqm{\grm{a}^\EE \Rightarrow_\Pc \grm{A}^\EE}
\end{align*}
Since we forgot about the indexing of sort types, erasure of sort-kinded application terms
is just erasure of its $\Pi$-type term:
\begin{align*}
\tqm{\grm{\IIapp(f)}^\EE}
  &:\equiv \tqm{\grm{f}^\EE} \text{ for \grm{\Gamma \vdash t : \Pi(a, B) :: \Sc}.}
\end{align*}
External $\Pi$-types and their applications convert directly into their
respective counterparts in the syntax of inductive families:
\begin{align*}
\tqm{\grm{\ExtPi{T}{A}}^\EE}
  &:\equiv \tqm{\ExtPiP{T}{\blm{\lambda \tau.\, }\grm{A(\bltau)}^\EE}} \text{, and} \\
\tqm{\grm{f(\bltau)}^\EE}
  &:\equiv \tqm{\grm{f}^\EE} \text{ for \grm{\Gamma \vdash f : \ExtPi{T}{B} : \Sc}}
\end{align*} %TODO this is a bit confusing since the application is for sorts and the types for points
Defining the erasure on point types and sort terms pulled back along a substitution,
we see that we will also need to erase entire sort substitutions.
This is achieved by extending the operation as follows:
\begin{equation*}
\inferrule{\grm{\IISub{\sigma}{\Gamma}{\Delta}}}
  {\tqm{\IISub{\grm{\sigma}^\EE_\Sc}{\grm{\Gamma}^\EE_\Sc}{\grm{\Delta}^\EE_\Sc}}}
\end{equation*}
We will then be able to use this in a straight forward way to define the pullbacks:
\begin{align*}
\tqm{\grm{A[\sigma]}^\EE}
  &:\equiv \tqm{\grm{A}^\EE[\grm{\sigma}^\EE_\Sc]}
  & \text{ for \grm{\Gamma \vdash A :: \Pc} and} \\
\tqm{\grm{t[\sigma]}^\EE}
  &:\equiv \tqm{\grm{t}^\EE[\grm{\sigma}^\EE_\Sc]}
  & \text{ for \grm{\Gamma \vdash t : B :: \Sc}.}
\end{align*}
Erasure of substitutions is built recursively, ignoring point types.
Likewise, the first projection will ignore point types:
\begin{align*}\label{eq:red-e-sub}
\tqm{\grm{\id}^\EE_\Sc}
  &:\equiv \tqm{\id} \text{,}
  & \\
\tqm{\grm{(\sigma \circ \delta)}^\EE_\Sc}
  &:\equiv \tqm{\grm{\sigma}^\EE_\Sc \circ \grm{\delta}^\EE_\Sc} \text{,}
  & \\
\tqm{\grm{\epsilon}^\EE_\Sc}
  &:\equiv \tqm{\epsilon} \text{,}
  & \\
\tqm{\grm{(\sigma,\, t)}^\EE_\Sc}
  &:\equiv \tqm{(\grm{\sigma}^\EE_\Sc,\, \grm{t}^\EE)}
  & \text{ for \grm{\Gamma \vdash t : B[\sigma] :: \Sc},} \\
\tqm{\grm{(\sigma,\, t)}^\EE_\Sc}
  &:\equiv \tqm{\grm{\sigma}^\EE_\Sc}
  & \text{ for \grm{\Gamma \vdash t : A[\sigma] :: \Pc},} \\
\tqm{\grm{\pi_1(\sigma)}^\EE_\Sc}
  &:\equiv \tqm{\pi_1(\grm{\sigma}^\EE_\Sc)}
  & \text{ for \grm{\IISub{\sigma}{\Gamma}{(\Delta,\, B :: \Sc)}},} \\
\tqm{\grm{\pi_1(\sigma)}^\EE_\Sc}
  &:\equiv \tqm{\grm{\sigma}^\EE_\Sc}
  & \text{ for \grm{\IISub{\sigma}{\Gamma}{(\Delta,\, A :: \Pc)}}, and} \\
\tqm{\grm{\pi_2(\sigma)}^\EE}
  &:\equiv \tqm{\pi_2(\grm{\sigma}^\EE_\Sc)} \text{.}
  &
\end{align*} %TODO laws
This concludes the definition of the erasure operation.
\end{defn} %TODO example derivations

For the steps that follow it will be necessary to equip the \emph{algebras}
of the resulting signatures with a substitution calculus that also considers
point contexts instead of only sort contexts.
To this end, we extend the operation of type erasure by assigning a map between
the types of algebras of the erasure to each substitution:

\begin{defn}[Replacement for Point Substitutions]
We define the following operation on substitutions:
\begin{equation*}
\inferrule{\grm{\IISub{\sigma}{\Gamma}{\Delta}}}
  {\grm{\sigma}^\EE : \left\{\gamma_\Sc : \tqm{\grm{\Gamma}^\EE_\Sc}^\AA \right\}
    \to \tqm{\grm{\Gamma}^\EE}^\AA(\gamma_\Sc)
    \to \tqm{\grm{\Delta}^\EE}^\AA\left(\grm{\sigma}^\EE_\Sc(\gamma_\Sc)\right)}
\end{equation*}
While in for \tqm{\grm{\sigma}^\EE_\Sc} we ignored point constructors
(see \ref{eq:red-e-sub} above)
this time we will to the opposite and ignore all sort constructors:
\begin{align*}
\grm{\id}^\EE(\gamma)
  &:\equiv \gamma \text{,} \\
\grm{\sigma \circ \delta}^\EE(\gamma)
  &:\equiv \grm{\sigma}^\EE\left(\grm{\delta}^\EE(\gamma)\right) \text{,} \\
\grm{\epsilon}^\EE(\gamma)
  &:\equiv \star \text{,} \\
\grm{(\sigma,\, t)}^\EE(\gamma)
  &:\equiv \grm{\sigma}^\EE(\gamma)
  & \text{ for \grm{\Gamma \vdash t : B[\sigma] :: \Sc },} \\
\grm{(\sigma,\, t)}^\EE(\gamma)
  &:\equiv \left(\grm{\sigma}^\EE, \grm{t}^\EE\right)
  & \text{ for \grm{\Gamma \vdash t : A[\sigma] :: \Pc},} \\
\grm{\pi_1(\sigma)}^\EE(\gamma)
  &:\equiv \grm{\sigma}^\EE(\gamma)
  & \text{ for \grm{\IISub{\sigma}{\Gamma}{(\Delta,\, B :: \Sc)}},} \\
\grm{\pi_1(\sigma)}^\EE(\gamma, \alpha)
  &:\equiv \grm{\sigma}^\EE(\gamma)
  & \text{ for \grm{\IISub{\sigma}{\Gamma}{(\Delta,\, A :: \Pc)}}.}
\end{align*}
\end{defn}

\section{The Wellformedness Predicate}

To remove the ambiguity created by the type erasure we will now have to find
a way to select those instances of the types which are ``wellformed'' in the
sense that the lie in the correct fibers of dependent sorts.
This predicate will be a proposition dependent on a realization of the erased
signature, i.\,e. on contexts, it will be a function on the type of algebras
of the erasure.
It is important keep this dependencies and not only to use the initial such
algebra, since when we will recursively define this wellformedness predicate,
the corresponding piece of signature will not always be initial
-- in the same way in which a projection of an initial algebra is not necessarily
initial anymore.

\begin{example}[Natural Numbers]\label{ex:red-w-nat}
Taking up the example of \grm{\Gamma_{nat}} from \ref{ex:red-e-nat},
we observe that algebras of \tqm{\grm{\Gamma_{nat}}^\EE_\Sc} take the form of
$(\star, N)$ with $N : \UU$ and, given $N$, those of
\tqm{\grm{\Gamma_{nat}}^\EE} are of the form $(\star, z, s)$ with
$z : N$ and $s : N \to N$.
Our wellformedness predicate in this case will encode a type family on $N$, inductively
populated by elements ``over'' $z$ and $n$.
The code for its sort and point constructors looks as follows:
\begin{align*}
\tqm{\grm{\Gamma_{nat}}^\WW_\Sc}(\star, z, s)
  &= \tqm{\left( \cdot_\Sc,\, \ExtPiS{N}{\UU} \right)} \text{ and} \\
\tqm{\grm{\Gamma_{nat}}^\WW}(\star, z, s)
  &= \tqm{\left( \cdot,\,  \El(\var(\vz)(\blm{z})),\,
    \ExtPiP{n : N}{\var(\vz)(\blm{n}) \Rightarrow_\Pc \El(\var(\vz)(\blm{s(n)}))} \right)}
\end{align*} %TODO
It's easy to see that the initial algebra of this signature is nothing more than
the trivial (final) type family on $N$.
\end{example}

\begin{example}[Vectors]\label{ex:red-w-vec}
For vectors on a type $A : \UU$, the duties of the wellformednes predicate are less trivial:
We have to add back the length information which we erased, as described in
\ref{ex:red-e-vec}:
Empty vectors should have length zero and appending an element should increase its
length by one.
This can be achieved by, given the data from an erasure algebra in the form of
$V : \UU$, $n: V$, and $c : A \to \N \to V \to V$,
having a predicate encoded by
\begin{equation*}
\tqm{\grm{\Gamma_{vec}}^\WW_\Sc}(\star, n, c)
  = \tqm{\left(\cdot_\Sc,\, \ExtPiS{n : \N}{\ExtPiS{v : V}{\UU}} \right)} \text{,}
\end{equation*}
with point constructors that ensure the correct lengh by setting $\tqm{\grm{\Gamma_{vec}}^\WW}(\star, n, c)$ to be the point context
\begin{equation*}
\begin{gathered}
\tqm{\cdot,\, \El(\var(\vz)(\blm{0}, \blm{n})),\,} \\
\tqm{\ExtPiP{a: A}{\ExtPiP{n : \N}{\ExtPiP{v : V}{\var(\vz)(\blm{n}, \blm{v}) \Rightarrow_\Pc 
  \El(\var(\vz)(\blm{n + 1}, \blm{c(a, n, v)}) ) } } } } \text{.}
\end{gathered}
\end{equation*}
\end{example}

Like for the type erasure, we will no proceed to generalize this to arbitrary
inductive-inductive types.

\begin{defn}[Wellformedness Predicates]
Again, we start by considering what the resulting type on context.
Clearly, we want the operation to result in the sort context and the point context
of another signature of an inductive family.
As we have alredy seen in the previous exapmles,
there needs to be a dependency on an erasure algebra which leads to the following
rules:
\begin{equation*}
\begin{gathered}
\inferrule{\grm{\vdash \Gamma} \\
  \gamma_\Sc : \tqm{\grm{\Gamma}^\EE_\Sc }^\CC \\
  \gamma : \tqm{\grm{\Gamma}^\EE }^\CC(\gamma_\Sc) }
  {\tqm{\SCon \grm{\Gamma}^\WW_\Sc(\blm{\gamma})} }
\\[.7em]
\inferrule{\grm{\vdash \Gamma} \\
  \gamma_\Sc : \tqm{\grm{\Gamma}^\EE_\Sc }^\CC \\
  \gamma : \tqm{\grm{\Gamma}^\EE }^\CC(\gamma_\Sc) }
  {\tqm{\vdash_{\grm{\Gamma}^\WW_\Sc(\blm{\gamma})} \grm{\Gamma}^\WW(\blm{\gamma}) }}
\end{gathered}
\end{equation*}

To be able to do recursion we will again need to provide a suitable operation
on types.
We need to distinguish between sort and point types.
For sort types, note that we don't have an erasure operation of which we could take
an algebra, but since, implicitly, every input sort turns into the inductive-family
universe token \tqm{\UU}, we know that we can act as if its universe is a plain type.
Also, we need to know the interpretation of the erasure of the context the type is based on.
\begin{equation*}
\begin{gathered}
\inferrule{\grm{\Gamma \vdash B :: \Sc} \\
  \gamma_\Sc : \tqm{\grm{\Gamma}^\EE_\Sc }^\CC \\
  \gamma : \tqm{\grm{\Gamma}^\EE }^\CC(\gamma_\Sc) \\
  \alpha : \UU }
  {\tqm{\SCon \grm{B}^\WW(\blgamma, \blalpha) :: \Sc }}
\\[.7em]
\inferrule{\grm{\Gamma \vdash A :: \Pc} \\
  \gamma_\Sc : \tqm{\grm{\Gamma}^\EE_\Sc }^\CC \\
  \gamma : \tqm{\grm{\Gamma}^\EE }^\CC(\gamma_\Sc) \\
  \alpha : \tqm{\grm{A}^\EE}^\CC(\gamma_\Sc) }
  {\tqm{\grm{\Gamma}^\WW_\Sc(\blm{\gamma}) \SCon \grm{A}^\WW(\blgamma, \blalpha) :: \Pc}}
\end{gathered}
\end{equation*}
The recursion of the context then looks very much like the on in the definition
of type erasure:
Extending the sort context whenever we encounter a sort type in the 
inductive-inductive signature and extending the point case for each point type.
Again, we can not leave the point context fixed ``on the nose'' when encountering
a sort type since we need to weaken it to account for the new sort:
\begin{align*}
\tqm{\grm{\cdot}^\WW_\Sc(\blgamma)}
  &:\equiv \tqm{\cdot_\Sc} \\
\tqm{\grm{(\Gamma,\, B :: \Sc)}^\WW_\Sc\{\gamma_\Sc, \alpha\}(\blgamma)}
  &:\equiv \tqm{\left( \grm{\Gamma}^\WW_\Sc(\blgamma)
    ,\, \grm{B}^\WW(\blgamma, \blalpha)\right)} \\
\tqm{\grm{(\Gamma,\, A :: \Pc)}^\WW_\Sc(\blgamma, \blalpha)}
  &:\equiv \tqm{\grm{\Gamma}^\WW_\Sc(\blgamma) } \\[.7em]
\tqm{\grm{\cdot}^\WW(\blgamma)}
  &:\equiv \tqm{\cdot} \\
\tqm{\grm{(\Gamma,\, B :: \Sc)}^\WW(\blgamma)}
  &:\equiv \tqm{\grm{\Gamma}^\WW(\blgamma)[\wk_\id]} \\
\tqm{\grm{(\Gamma,\, A :: \Pc)}^\WW(\blgamma, \blalpha)}
  &:\equiv \tqm{\left(\grm{\Gamma}^\WW(\blgamma),\, \grm{A}^\WW(\blgamma, \blalpha)\right) }
\end{align*}

Like in the definition of type erasure, recursing on \grm{\El(a)} makes it
necessary to extend the definition at least to sort types.
So we will also give an operation producing the following data:
\begin{equation*}
\inferrule{\grm{\Gamma \vdash t : B :: \Sc} \\
  \gamma_\Sc : \tqm{\grm{\Gamma}^\EE_\Sc }^\CC \\
  \gamma : \tqm{\grm{\Gamma}^\EE }^\CC(\gamma_\Sc)}
  {\tqm{\grm{\Gamma}^\WW_\Sc(\blgamma) \SCon \grm{t}^\WW(\blgamma)
    : \grm{B}^\WW(\blgamma, \blm{\tqm{\grm{t}^\EE}^\CC(\gamma_\Sc)})}}
\end{equation*}

Let us now proceed to give the definition on all type formers.
The each sort of the input signature should become a predicate.
Since a predicate is the same as a type family with propositional values,
we set the wellformedness on the universe to be a type family, the domain of which
is given by the set we obtain from the algebra of the erased context.
Not that this type family is a non-dependent, \emph{non-recursive} $\Pi$-type.
The interpretation of \grm{\El(a)} has to make up for this shift by applying
to the wellformedness predicate corresponding the sort term \grm{a} the
element we get from the erasure of \grm{\El(a)}: %TODO this still reads horrible
\begin{align*}
\tqm{\grm{\UU}^\WW(\blgamma, \blalpha)}
  &:\equiv \tqm{\ExtPiS{x : \blalpha}{\UU}} \text{ and} \\
\tqm{\grm{\El(a)}^\WW(\blgamma, \blalpha)}
  &:\equiv \tqm{\El\left(\grm{a}^\WW(\blgamma)(\blalpha)\right)} \text{.}
\end{align*}

For sort-kinded, recursive $\Pi$-types, we again need to remember that in the
definition of type erasure, we turned them into instances of \tqm{\UU}, so to
add the information back which we erased, the wellformedness has to turn them into
non-recursive $\Pi$-types over the erasure of sort term which is the domain of the
$\Pi$-type we started with.
The iterpretation of application terms has to follow this step accordingly:
\begin{align*}
\tqm{\grm{\Pi(a, B :: \Sc)}^\WW\{\blm{\gamma_\Sc}\}(\blgamma, \blphi)}
  &:\equiv \tqm{\ExtPiS{\blalpha : \tqm{\grm{a}^\EE}^\CC(\gamma_\Sc)}
    {\grm{B}^\WW((\blgamma, \blalpha), \blphi)}} \text { and} \\
\tqm{\grm{\IIapp(f)}^\WW(\blgamma, \blalpha)}
  &:\equiv \tqm{\grm{f}^\WW(\blgamma)(\blalpha)}
  \text{ for \grm{\Gamma \vdash f : \Pi(a, B :: \Sc)}.}
\end{align*}

The treatment of $\Pi$-types in point constructors is arguably the trickiest part
of the definition.
A non-technical description of the effect of the wellformedness operation on
these $\Pi$-types is the following:
For each bit of input data from an algebra of the erasure, wellformedness of this
input data should imply wellformedness of the result.
\begin{equation*}
\tqm{\grm{\Pi(a, A :: \Pc)}^\WW\{\blm{\gamma_\Sc}\}(\blgamma, \blphi)}
  :\equiv \tqm{\ExtPi{\blalpha : \tqm{\grm{a}^\EE}^\CC(\gamma_\Sc)}
    { \grm{a}^\WW(\blgamma)(\blalpha)
      \Rightarrow_\Pc \grm{A}^\WW((\gamma, \alpha), \phi(\alpha))}}
%\tqm{\grm{\IIapp(f)}^\WW\{\blm{\gamma_\Sc}\}
%  \{\blgamma, \blphi\}(\blm{\gamma^\WW}, \blm{\phi^\WW}) }
%  &:\equiv \tqm{\grm{f}^\WW(\blm{\gamma^\WW})(\blphi)(\blm{\phi^\WW}) }
%  \text{ for \grm{\Gamma \vdash f : \Pi(a, A :: \Pc)}.} TODO this belongs to extra constr
\end{equation*}

Let us next look at the non-recursive function types.
Since we erased them just like the recursive ones, they are processed similar to
the definitions above, with the difference that for point constructors, there is
no wellformedness of the domain that we have to presuppose to infer wellformedness
of the codomain:
\begin{align*}
\tqm{\grm{\ExtPi{T}{B :: \Sc}}^\WW(\blgamma, \blphi)}
  &:\equiv \tqm{\ExtPiS{\tau : T}{\grm{B(\bltau)}^\WW(\blgamma, \blphi)}} \text{,} \\
\tqm{\grm{\ExtPi{T}{A :: \Pc}}^\WW(\blgamma, \blphi)}
  &:\equiv \tqm{\ExtPiP{\tau : T}{\grm{A(\bltau)}^\WW(\blgamma, \blm{\phi(\tau)})}}
  \text{, and} \\
\tqm{\grm{f(\bltau)}^\WW(\blgamma)}
  &:\equiv \tqm{\grm{f}^\WW(\blgamma)(\bltau) } \text{.}
\end{align*}

\end{defn}












