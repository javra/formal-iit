\section{(OLD) --The Source Type Theory}

Since for some examples of inductive-inductive types, especially ones which are
\emph{infinitely branching}, we need functions with external domain as the domains of
other functions, we will also add another function type which is itself small, and
has external domain and small codomain:
\begin{equation*}
\begin{gathered}
\inferrule{\grm{\vdash \Gamma} \\ A : \UU_i \\ (x : A) \to (\grm{\Gamma \vdash b : \UU})}
	{\grm{\Gamma \vdash ((\blm{x : A}) \to b) : \UU}}
\qquad
\inferrule{\grm{\Gamma \vdash t : \underline{(\blm{x : A}) \to b}} \\ u : A}
	{\grm{\Gamma \vdash (t \app \blm{u}) : \underline{b[\blm{x} \mapsto \blm{u}]}}}
\end{gathered}
\end{equation*}

\section{(OLD) --Motives and Methods}

TODO: Add definition of $\grm{-}^\MM$.

TODO: Add examples.

\section{(OLD) --Recursion and Computation}

TODO: Add definition of $\grm{-}^\EE$.

TODO: Add examples.

\section{(OLD) --Existence of HIITs}

\begin{defn}[Dependent Eliminator]
An algebra $c : \grm{\Gamma}^\CC$ is said to admit \emph{dependent elimination}
if for each motive $m : \grm{\Gamma}^\MM(c)$ there is a dependent eliminator
$\elim_{\grm{\Gamma}}(m) : \grm{\Gamma}^\EE(c, m)$.
\end{defn}

\begin{thm}[Admissibility of Inductive-Inductive Types]
Our type theory admits inductive-inductive types if for each wellformed context
$\grm{\Gamma}$ we can find a constructor $\con{\grm{\Gamma}}$ which admits dependent
elimination $\elim_{\grm{\Gamma}}$.
\end{thm}

\section{(OLD) --Morphisms of Algebras}

In the following, it will be useful to regard the algebras of a given context
\grm{\Gamma} as a category.
To this end, we need to make clear what a morphism between \grm{\Gamma}-algebras
is.
Intutitively a morphism is given by maps between the interpretations of the sorts,
together with evidence that those maps preserve the interpretation of point
constructors.

\begin{defn}[Morphisms of Algebras]
We want to define the following by mutual recursion on contexts, types and terms:
\begin{equation*}
\begin{gathered}
\inferrule{\grm{\vdash \Gamma} \\ \gamma_0, \gamma_1 : \grm{\Gamma}^\AA}
	{\grm{\Gamma}^\mm(\gamma_0,\gamma_1) : \UU} \\[.7em]
\inferrule{\grm{\Gamma \vdash A} %\\ \gamma_0, \gamma_1 : \grm{\Gamma}^\AA TODO decide if show
		\\ g : \grm{\Gamma}^\mm(\gamma_0, \gamma_1) \\
		\\ \alpha_0 : \grm{A}^\AA(\gamma_0) \\ \alpha_1 : \grm{A}^\AA(\gamma_1)}
	{\grm{A}^\mm(g, \alpha_0, \alpha_1) : \UU} \\[.7em]
\inferrule{\grm{\Gamma \vdash t : A} \\ \gamma_0, \gamma_1 : \grm{\Gamma}^\AA 
		\\ g : \grm{\Gamma}^\mm(\gamma_0, \gamma_1)}
	{\grm{t}^\mm(g) : \grm{A}^\mm(g, \grm{t}^\AA(\gamma_0), \grm{t}^\AA(\gamma_1))}
\end{gathered}
\end{equation*}

Like we did for the definition of algebras, we want morphisms of contexts to be
just iterated $\Sigma$-types of the respective interpretation of types:
\begin{align*}
\grm{\cdot}^\mm(\gamma_0, \gamma_1) &:\equiv \unit \text{ and} \\
\grm{(\Gamma, x : A)}^\mm(\gamma_0, \gamma_1) &:\equiv
	(g : \grm{\Gamma}^\mm(\pr_1(\gamma_0), \pr_1(\gamma_1))) \times \grm{A}^\mm(g, \pr_2(\gamma_0), \pr_2(\gamma_1)) \text{.}
\end{align*}
The core of the definition on sort types is that the universe is interpreted as
a function space:
\begin{align*}
\grm{\UU}^\mm(g, \alpha_0, \alpha_1)  			&:\equiv \alpha_0 \to \alpha_1 \text{,} \\
\grm{(\underline{a})}^\mm(g, \alpha_0, \alpha_1)	&:\equiv (\grm{a}^\mm(g, \alpha_0) = \alpha_1) \text{,} \\
\grm{((x : a) \to B)}^\mm(g, \alpha_0, \alpha_1)	&:\equiv (x : \grm{a}^\CC(\gamma_0))
							\to \grm{B}^\mm((g, \refl), \alpha_0(x), \alpha_1(\grm{a}^\mm(g, x))) \text{,} \\
\grm{((\blm{x : A}) \to B)}^\mm(g, \alpha_0, \alpha_1)  &:\equiv (x : A)
							\to \grm{(B(\blm{x}))}^\mm(g, \alpha_0(x), \alpha_1(x)) \text{.}
\end{align*}
On terms, consider the foo
\begin{align*}
\grm{x}^\mm(g) 				&:\equiv g.\grm{x} \qquad \text{for variables \grm{x},} \\ %???
\grm{(t(u))}^\mm(g)			&:\equiv (\grm{u}^\mm(g))_* (\grm{t}^\mm(g)(\grm{u}^\AA(\gamma_0))) \text{,} \\ %TODO check this
\grm{((\blm{x : A}) \to b)}^\mm(g)	&:\equiv (x : A) \to \grm{b}^\mm(g) \text{,} \\
\grm{(t(\blm{u}))}^\mm(g)		&:\equiv \grm{t}^\mm(g)(u) \text{, and} \\
\grm{(t \app \blm{u})}^\mm(g)		&:\equiv (\grm{t}^\mm(g))(u) \text{.} %TODO use happly here sometimes!!! uargh
\end{align*}

\end{defn}

TODO: Add examples.

